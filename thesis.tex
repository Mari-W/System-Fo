\documentclass[runningheads]{llncs} 
\usepackage[utf8]{inputenc}
\usepackage{amsmath}
\usepackage{amsfonts}

\newcommand{\snip}[1]{\footnotesize{\ttfamily{#1}}}
\newcommand{\with}[2]{\begin{#1}#2\end{#1}}

\begin{document}
\title{Formal Proof for Type Preservation of the Dictionary Passing Transform from System O to Hindley Milner}
\author{Marius Weidner}
\institute{Chair of Programming Languages, University of Freiburg}
\maketitle

\begin{abstract}
  Abstract.
\end{abstract}

\section{Introduction}
\section{Preliminary}
\subsection{Agda}
\subsubsection{Curry Howard Isomorphism}
\subsection{Sorts / ???}
\subsection{}
\section{Hindley-Milner System}
\subsection{Syntax, Typing and Semantics}
\subsection{Formalization}
\subsection{Soundness Proof}
\section{Overloading System}
\subsection{Syntax, Typing and Semantics}
\subsection{Formalization}
\section{Dictionary Passing Transform}
\subsection{Type Preservation Proof}
\section{Conclusion and Further Work}

\begin{thebibliography}{8}

\bibitem{system-o}Odersky, M., Wadler, P. \& Wehr, M. A Second Look at Overloading. 
{\em Proceedings Of The Seventh International Conference On Functional Programming Languages And Computer Architecture}. 
pp. 135-146 (1995), https://doi.org/10.1145/224164.224195
\bibitem{hindley-milner}Milner, R. A theory of type polymorphism in programming. 
{\em Journal Of Computer And System Sciences}. 
\textbf{17}, 348-375 (1978), https://www.sciencedirect.com/science/article/pii/0022000078900144

\end{thebibliography}

\end{document}