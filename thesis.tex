\documentclass[runningheads]{llncs} 
\usepackage[utf8]{inputenc}
\usepackage{amsmath}
\usepackage{amsfonts}
\usepackage{upgreek}

\newcommand{\cx}{\Upgamma}
\newcommand{\poly}{\upsigma}

\begin{document}
\title{Type Preservation of the Dictionary Passing Transform}
\author{Marius Weidner}
\institute{Chair for Programming Languages, University of Freiburg \email{weidner@cs.uni-freiburg.de}}
\date\today
\maketitle
\begin{abstract}
  Abstract.
\end{abstract}

\section{Introduction}
\subsection{Motivation}
$\cx\cx\poly$
\subsection{Example}
\section{Preliminary}
\subsection{Dependently Typed Programming in Agda}
\subsection{Pure Type Systems}
\section{System F}
\subsection{Specification}
\subsubsection{Syntax.} 
\subsubsection{Renaming.}
\subsubsection{Substitution.}
\subsubsection{Context.}
\subsubsection{Typing.}
\subsubsection{Semantic.}
\subsection{Soundness}
\subsubsection{Progress.}
\subsubsection{Subject Reduction.}
\section{System F with Overloading}
\subsubsection{Syntax.} 
\subsubsection{Renaming.}
\subsubsection{Substitution.}
\subsubsection{Context.}
\subsubsection{Typing.}
\subsection{Specification}
\section{Dictionary Passing Transform}
\subsection{Type Preservation Proof}
\section{Conclusion and Further Work}
\subsection{Hindley Milner}
\subsection{Semantic Preservation}

\begin{thebibliography}{8}

\bibitem{system-o}Odersky, M., Wadler, P. \& Wehr, M. A Second Look at Overloading. 
{\em Proceedings Of The Seventh International Conference On Functional Programming Languages And Computer Architecture}. 
pp. 135-146 (1995), https://doi.org/10.1145/224164.224195
\bibitem{hindley-milner}Milner, R. A theory of type polymorphism in programming. 
{\em Journal Of Computer And System Sciences}. 
\textbf{17}, 348-375 (1978), https://www.sciencedirect.com/science/article/pii/0022000078900144

\end{thebibliography}

\end{document}