\documentclass{beamer}
\usetheme{Boadilla}
\useoutertheme{infolines}
\setbeamertemplate{navigation symbols}{} 
\usepackage[utf8]{inputenc}
\usepackage[T1]{fontenc}
\usepackage{alphabeta}
\usepackage{graphicx}
\usepackage{mathtools}
\usepackage{amssymb}
\usepackage{amsmath}
\usepackage{amsfonts}
\usepackage[conor,links]{agda}
\usepackage{bbm}
\usepackage{newunicodechar}
\usepackage{minted}
\usepackage{xcolor}

\renewcommand{\fcolorbox}[4][]{#4}
\definecolor{mgreen}{HTML}{008000}
\newcommand{\inl}[1]{\mintinline[escapeinside=||]{haskell}{#1}}
\newcommand{\mono}[1]{\footnotesize{\ttfamily{#1}}}
\newcommand{\Fo}[0]{F$_\text{O}$} 
\newcommand{\Inst}[0]{\textcolor{mgreen}{inst}}
\newcommand{\Decl}[0]{\textcolor{mgreen}{decl}}
\newcommand{\Blk}[1]{\textcolor{black}{#1}}
\makeatletter
\newcommand{\lambdabar}{\ensuremath{\mathchoice
  {\smash@bar\textfont\displaystyle{0.25}{1.2}\lambda}
  {\smash@bar\textfont\textstyle{0.25}{1.2}\lambda}
  {\smash@bar\scriptfont\scriptstyle{0.25}{1.2}\lambda}
  {\smash@bar\scriptscriptfont\scriptscriptstyle{0.25}{1.2}\lambda}
}}
\newcommand{\smash@bar}[4]{%
  \smash{\rlap{\raisebox{-#3\fontdimen5#10}{$\m@th#2\mkern#4mu\mathchar'26$}}}%
}
\newcommand{\Sym}[1]{\AgdaGeneralizable{#1}}
\newcommand{\Data}[1]{\AgdaDatatype{#1}}
\newcommand{\Constr}[1]{\AgdaInductiveConstructor{#1}}
\newcommand{\Prim}[1]{\AgdaPrimitive{#1}}
\makeatletter
\newcommand{\Scalecenter}[2][1]{\mathpalette\Scalecenter@{{#1}{#2}}}
\newcommand{\Scalecenter@}[2]{\Scalecenter@@#1#2}
\newcommand{\Scalecenter@@}[3]{%
  \vcenter{\hbox{\scalebox{#2}{$\m@th#1#3$}}}%
}
\makeatother

\newcommand{\smallblacktriangleright}{%
  \Scalecenter[0.75]{\blacktriangleright}%
}



\newunicodechar{λ}{\ensuremath{\mathnormal\lambda}}
\newunicodechar{Λ}{\ensuremath{\mathnormal\Lambda}}
\newunicodechar{ƛ}{\ensuremath{\lbar}}
\newunicodechar{τ}{\ensuremath{\mathnormal\tau}}
\newunicodechar{ℕ}{\ensuremath{\mathbb{N}}}
\newunicodechar{∶}{\ensuremath{:}}
\newunicodechar{≡}{\ensuremath{\equiv}}
\newunicodechar{∀}{\ensuremath{\forall}}
\newunicodechar{⊤}{\ensuremath{\top}}
\newunicodechar{⊥}{\ensuremath{\bot}}
\newunicodechar{₁}{\ensuremath{_1}}
\newunicodechar{₂}{\ensuremath{_2}}
\newunicodechar{∈}{\ensuremath{\in}}
\newunicodechar{′}{\ensuremath{'}}
\newunicodechar{·}{\ensuremath{\cdot}}
\newunicodechar{⊎}{\ensuremath{\uplus}}
\newunicodechar{∷}{\ensuremath{::}}
\newunicodechar{▷}{\ensuremath{\triangleright}}
\newunicodechar{ᶜ}{\ensuremath{^c}}
\newunicodechar{⊎}{\ensuremath{\uplus}}
\newunicodechar{×}{\ensuremath{\times}}
\newunicodechar{Σ}{\ensuremath{\Sigma}}
\newunicodechar{∃}{\ensuremath{\exists}}
\newunicodechar{≢}{\ensuremath{\not\equiv}}
\newunicodechar{∘}{\ensuremath{\circ}}
\newunicodechar{α}{\ensuremath{\mathnormal\alpha}}
\newunicodechar{⇒}{\ensuremath{\Rightarrow}}
\newunicodechar{→}{\ensuremath{\rightarrow}}
\newunicodechar{·}{\ensuremath{\cdot}}
\newunicodechar{•}{\ensuremath{\bullet}}
\newunicodechar{∶}{\ensuremath{:}}
\newunicodechar{∀}{\ensuremath{\forall}}
\newunicodechar{ρ}{\ensuremath{\mathnormal\rho}}
\newunicodechar{σ}{\ensuremath{\mathnormal\sigma}}
\newunicodechar{∅}{\ensuremath{\emptyset}}
\newunicodechar{▶}{\ensuremath{\blacktriangleright}}
\newunicodechar{▸}{\ensuremath{\smallblacktriangleright}}
\newunicodechar{⊘}{\ensuremath{\oslash}}
\newunicodechar{Γ}{\ensuremath{\mathnormal\Gamma}}
\newunicodechar{⊢}{\ensuremath{\vdash}}
\newunicodechar{ᵣ}{\ensuremath{_r}}
\newunicodechar{ₛ}{\ensuremath{_s}}
\newunicodechar{ᴼ}{\ensuremath{^\text{O}}}
\newunicodechar{⇝}{\ensuremath{\rightsquigarrow}}
\newunicodechar{≐}{\ensuremath{\doteq}}


\title[Formal Dictionary Passing Transform]{Formal Proof of Type Preservation of the\\Dictionary Passing Transform for System F}
\institute[Uni Freiburg]{Chair of Programming Languages, University of Freiburg}
\author{Marius Weidner}


\begin{document}

\begin{code}[hide]%
\>[0]\AgdaSymbol{\{-\#}\AgdaSpace{}%
\AgdaKeyword{OPTIONS}\AgdaSpace{}%
\AgdaPragma{--allow-unsolved-metas}\AgdaSpace{}%
\AgdaSymbol{\#-\}}\<%
\\
\>[0]\AgdaKeyword{open}\AgdaSpace{}%
\AgdaKeyword{import}\AgdaSpace{}%
\AgdaModule{Data.Unit}\AgdaSpace{}%
\AgdaKeyword{using}\AgdaSpace{}%
\AgdaSymbol{(}\AgdaRecord{⊤}\AgdaSymbol{;}\AgdaSpace{}%
\AgdaInductiveConstructor{tt}\AgdaSymbol{)}\<%
\\
\>[0]\AgdaKeyword{open}\AgdaSpace{}%
\AgdaKeyword{import}\AgdaSpace{}%
\AgdaModule{Data.Nat}\AgdaSpace{}%
\AgdaKeyword{using}\AgdaSpace{}%
\AgdaSymbol{(}\AgdaDatatype{ℕ}\AgdaSymbol{;}\AgdaSpace{}%
\AgdaInductiveConstructor{zero}\AgdaSymbol{;}\AgdaSpace{}%
\AgdaInductiveConstructor{suc}\AgdaSymbol{)}\<%
\\
\>[0]\AgdaKeyword{open}\AgdaSpace{}%
\AgdaKeyword{import}\AgdaSpace{}%
\AgdaModule{Data.List}\AgdaSpace{}%
\AgdaKeyword{using}\AgdaSpace{}%
\AgdaSymbol{(}\AgdaDatatype{List}\AgdaSymbol{;}\AgdaSpace{}%
\AgdaInductiveConstructor{[]}\AgdaSymbol{;}\AgdaSpace{}%
\AgdaOperator{\AgdaInductiveConstructor{\AgdaUnderscore{}∷\AgdaUnderscore{}}}\AgdaSymbol{;}\AgdaSpace{}%
\AgdaOperator{\AgdaFunction{\AgdaUnderscore{}++\AgdaUnderscore{}}}\AgdaSymbol{;}\AgdaSpace{}%
\AgdaFunction{drop}\AgdaSymbol{)}\<%
\\
\>[0]\AgdaKeyword{open}\AgdaSpace{}%
\AgdaKeyword{import}\AgdaSpace{}%
\AgdaModule{Data.List.Relation.Unary.Any}\AgdaSpace{}%
\AgdaKeyword{using}\AgdaSpace{}%
\AgdaSymbol{(}\AgdaInductiveConstructor{here}\AgdaSymbol{;}\AgdaSpace{}%
\AgdaInductiveConstructor{there}\AgdaSymbol{)}\<%
\\
\>[0]\AgdaKeyword{open}\AgdaSpace{}%
\AgdaKeyword{import}\AgdaSpace{}%
\AgdaModule{Data.List.Membership.Propositional}\AgdaSpace{}%
\AgdaKeyword{using}\AgdaSpace{}%
\AgdaSymbol{(}\AgdaOperator{\AgdaFunction{\AgdaUnderscore{}∈\AgdaUnderscore{}}}\AgdaSymbol{)}\<%
\\
\>[0]\AgdaKeyword{open}\AgdaSpace{}%
\AgdaKeyword{import}\AgdaSpace{}%
\AgdaModule{Data.Sum.Base}\AgdaSpace{}%
\AgdaKeyword{using}\AgdaSpace{}%
\AgdaSymbol{(}\AgdaOperator{\AgdaDatatype{\AgdaUnderscore{}⊎\AgdaUnderscore{}}}\AgdaSymbol{;}\AgdaSpace{}%
\AgdaInductiveConstructor{inj₁}\AgdaSymbol{;}\AgdaSpace{}%
\AgdaInductiveConstructor{inj₂}\AgdaSymbol{)}\<%
\\
\>[0]\AgdaKeyword{open}\AgdaSpace{}%
\AgdaKeyword{import}\AgdaSpace{}%
\AgdaModule{Data.Product}\AgdaSpace{}%
\AgdaKeyword{using}\AgdaSpace{}%
\AgdaSymbol{(}\AgdaOperator{\AgdaFunction{\AgdaUnderscore{}×\AgdaUnderscore{}}}\AgdaSymbol{;}\AgdaSpace{}%
\AgdaOperator{\AgdaInductiveConstructor{\AgdaUnderscore{},\AgdaUnderscore{}}}\AgdaSymbol{;}\AgdaSpace{}%
\AgdaFunction{Σ-syntax}\AgdaSymbol{;}\AgdaSpace{}%
\AgdaFunction{∃-syntax}\AgdaSymbol{)}\<%
\\
\>[0]\AgdaKeyword{open}\AgdaSpace{}%
\AgdaKeyword{import}\AgdaSpace{}%
\AgdaModule{Relation.Binary.PropositionalEquality}\AgdaSpace{}%
\AgdaKeyword{using}\AgdaSpace{}%
\AgdaSymbol{(}\AgdaOperator{\AgdaDatatype{\AgdaUnderscore{}≡\AgdaUnderscore{}}}\AgdaSymbol{;}\AgdaSpace{}%
\AgdaInductiveConstructor{refl}\AgdaSymbol{;}\AgdaSpace{}%
\AgdaFunction{subst}\AgdaSymbol{;}\AgdaSpace{}%
\AgdaFunction{sym}\AgdaSymbol{;}\AgdaSpace{}%
\AgdaFunction{cong}\AgdaSymbol{;}\AgdaSpace{}%
\AgdaFunction{cong₂}\AgdaSymbol{;}\AgdaSpace{}%
\AgdaFunction{trans}\AgdaSymbol{;}\AgdaSpace{}%
\AgdaKeyword{module}\AgdaSpace{}%
\AgdaModule{≡-Reasoning}\AgdaSymbol{)}\<%
\\
\>[0]\AgdaKeyword{open}\AgdaSpace{}%
\AgdaKeyword{import}\AgdaSpace{}%
\AgdaModule{Function}\AgdaSpace{}%
\AgdaKeyword{using}\AgdaSpace{}%
\AgdaSymbol{(}\AgdaFunction{id}\AgdaSymbol{;}\AgdaSpace{}%
\AgdaOperator{\AgdaFunction{\AgdaUnderscore{}∘\AgdaUnderscore{}}}\AgdaSymbol{)}\<%
\\
\>[0]\AgdaKeyword{open}\AgdaSpace{}%
\AgdaModule{≡-Reasoning}\<%
\\
%
\\[\AgdaEmptyExtraSkip]%
\>[0]\AgdaKeyword{module}\AgdaSpace{}%
\AgdaModule{SystemF}\AgdaSpace{}%
\AgdaKeyword{where}\<%
\\
%
\\[\AgdaEmptyExtraSkip]%
\>[0]\AgdaComment{--\ Sorts\ --------------------------------------------------------------------------------}\<%
\\
%
\\[\AgdaEmptyExtraSkip]%
\>[0]\AgdaKeyword{data}\AgdaSpace{}%
\AgdaDatatype{Bindable}\AgdaSpace{}%
\AgdaSymbol{:}\AgdaSpace{}%
\AgdaPrimitive{Set}\AgdaSpace{}%
\AgdaKeyword{where}\<%
\\
\>[0][@{}l@{\AgdaIndent{0}}]%
\>[2]\AgdaInductiveConstructor{B}\AgdaSpace{}%
\AgdaSymbol{:}\AgdaSpace{}%
\AgdaDatatype{Bindable}\<%
\\
%
\>[2]\AgdaInductiveConstructor{¬B}\AgdaSpace{}%
\AgdaSymbol{:}\AgdaSpace{}%
\AgdaDatatype{Bindable}\<%
\end{code}
\newcommand{\FSort}[0]{\begin{code}%
\>[0]\AgdaKeyword{data}\AgdaSpace{}%
\AgdaDatatype{Sort}\AgdaSpace{}%
\AgdaSymbol{:}\AgdaSpace{}%
\AgdaDatatype{Bindable}\AgdaSpace{}%
\AgdaSymbol{→}\AgdaSpace{}%
\AgdaPrimitive{Set}\AgdaSpace{}%
\AgdaKeyword{where}\<%
\\
\>[0][@{}l@{\AgdaIndent{0}}]%
\>[2]\AgdaInductiveConstructor{eₛ}%
\>[6]\AgdaSymbol{:}\AgdaSpace{}%
\AgdaDatatype{Sort}\AgdaSpace{}%
\AgdaInductiveConstructor{B}\<%
\\
%
\>[2]\AgdaInductiveConstructor{τₛ}%
\>[6]\AgdaSymbol{:}\AgdaSpace{}%
\AgdaDatatype{Sort}\AgdaSpace{}%
\AgdaInductiveConstructor{B}\<%
\\
%
\>[2]\AgdaInductiveConstructor{κₛ}%
\>[6]\AgdaSymbol{:}\AgdaSpace{}%
\AgdaDatatype{Sort}\AgdaSpace{}%
\AgdaInductiveConstructor{¬B}\<%
\\
%
\\[\AgdaEmptyExtraSkip]%
\>[0]\AgdaFunction{Sorts}\AgdaSpace{}%
\AgdaSymbol{:}\AgdaSpace{}%
\AgdaPrimitive{Set}\<%
\\
\>[0]\AgdaFunction{Sorts}\AgdaSpace{}%
\AgdaSymbol{=}\AgdaSpace{}%
\AgdaDatatype{List}\AgdaSpace{}%
\AgdaSymbol{(}\AgdaDatatype{Sort}\AgdaSpace{}%
\AgdaInductiveConstructor{B}\AgdaSymbol{)}\<%
\end{code}}
\begin{code}[hide]%
\>[0]\AgdaKeyword{infix}\AgdaSpace{}%
\AgdaNumber{25}\AgdaSpace{}%
\AgdaOperator{\AgdaInductiveConstructor{\AgdaUnderscore{}▷\AgdaUnderscore{}}}\AgdaSpace{}%
\AgdaOperator{\AgdaFunction{\AgdaUnderscore{}▷▷\AgdaUnderscore{}}}\<%
\\
\>[0]\AgdaKeyword{pattern}\AgdaSpace{}%
\AgdaOperator{\AgdaInductiveConstructor{\AgdaUnderscore{}▷\AgdaUnderscore{}}}\AgdaSpace{}%
\AgdaBound{xs}\AgdaSpace{}%
\AgdaBound{x}\AgdaSpace{}%
\AgdaSymbol{=}\AgdaSpace{}%
\AgdaBound{x}\AgdaSpace{}%
\AgdaOperator{\AgdaInductiveConstructor{∷}}\AgdaSpace{}%
\AgdaBound{xs}\<%
\\
\>[0]\AgdaOperator{\AgdaFunction{\AgdaUnderscore{}▷▷\AgdaUnderscore{}}}\AgdaSpace{}%
\AgdaSymbol{:}\AgdaSpace{}%
\AgdaSymbol{\{}\AgdaBound{A}\AgdaSpace{}%
\AgdaSymbol{:}\AgdaSpace{}%
\AgdaPrimitive{Set}\AgdaSymbol{\}}\AgdaSpace{}%
\AgdaSymbol{→}\AgdaSpace{}%
\AgdaDatatype{List}\AgdaSpace{}%
\AgdaBound{A}\AgdaSpace{}%
\AgdaSymbol{→}\AgdaSpace{}%
\AgdaDatatype{List}\AgdaSpace{}%
\AgdaBound{A}\AgdaSpace{}%
\AgdaSymbol{→}\AgdaSpace{}%
\AgdaDatatype{List}\AgdaSpace{}%
\AgdaBound{A}\<%
\\
\>[0]\AgdaBound{xs}\AgdaSpace{}%
\AgdaOperator{\AgdaFunction{▷▷}}\AgdaSpace{}%
\AgdaBound{ys}\AgdaSpace{}%
\AgdaSymbol{=}\AgdaSpace{}%
\AgdaBound{ys}\AgdaSpace{}%
\AgdaOperator{\AgdaFunction{++}}\AgdaSpace{}%
\AgdaBound{xs}\<%
\\
%
\\[\AgdaEmptyExtraSkip]%
\>[0]\AgdaKeyword{variable}\<%
\\
\>[0][@{}l@{\AgdaIndent{0}}]%
\>[2]\AgdaGeneralizable{r}\AgdaSpace{}%
\AgdaGeneralizable{r'}\AgdaSpace{}%
\AgdaGeneralizable{r''}\AgdaSpace{}%
\AgdaGeneralizable{r₁}\AgdaSpace{}%
\AgdaGeneralizable{r₂}\AgdaSpace{}%
\AgdaSymbol{:}\AgdaSpace{}%
\AgdaDatatype{Bindable}\<%
\\
%
\>[2]\AgdaGeneralizable{s}\AgdaSpace{}%
\AgdaGeneralizable{s'}\AgdaSpace{}%
\AgdaGeneralizable{s''}\AgdaSpace{}%
\AgdaGeneralizable{s₁}\AgdaSpace{}%
\AgdaGeneralizable{s₂}\AgdaSpace{}%
\AgdaSymbol{:}\AgdaSpace{}%
\AgdaDatatype{Sort}\AgdaSpace{}%
\AgdaGeneralizable{r}\<%
\\
%
\>[2]\AgdaGeneralizable{S}\AgdaSpace{}%
\AgdaGeneralizable{S'}\AgdaSpace{}%
\AgdaGeneralizable{S''}\AgdaSpace{}%
\AgdaGeneralizable{S₁}\AgdaSpace{}%
\AgdaGeneralizable{S₂}\AgdaSpace{}%
\AgdaSymbol{:}\AgdaSpace{}%
\AgdaFunction{Sorts}\<%
\\
%
\>[2]\AgdaGeneralizable{x}\AgdaSpace{}%
\AgdaGeneralizable{x'}\AgdaSpace{}%
\AgdaGeneralizable{x''}\AgdaSpace{}%
\AgdaGeneralizable{x₁}\AgdaSpace{}%
\AgdaGeneralizable{x₂}\AgdaSpace{}%
\AgdaSymbol{:}\AgdaSpace{}%
\AgdaInductiveConstructor{eₛ}\AgdaSpace{}%
\AgdaOperator{\AgdaFunction{∈}}\AgdaSpace{}%
\AgdaGeneralizable{S}\<%
\\
%
\>[2]\AgdaGeneralizable{α}\AgdaSpace{}%
\AgdaGeneralizable{α'}\AgdaSpace{}%
\AgdaGeneralizable{α''}\AgdaSpace{}%
\AgdaGeneralizable{α₁}\AgdaSpace{}%
\AgdaGeneralizable{α₂}\AgdaSpace{}%
\AgdaSymbol{:}\AgdaSpace{}%
\AgdaInductiveConstructor{τₛ}\AgdaSpace{}%
\AgdaOperator{\AgdaFunction{∈}}\AgdaSpace{}%
\AgdaGeneralizable{S}\<%
\\
%
\\[\AgdaEmptyExtraSkip]%
\>[0]\AgdaComment{--\ Syntax\ -------------------------------------------------------------------------------}\<%
\\
%
\\[\AgdaEmptyExtraSkip]%
\>[0]\AgdaKeyword{infixr}\AgdaSpace{}%
\AgdaNumber{4}\AgdaSpace{}%
\AgdaOperator{\AgdaInductiveConstructor{λ`x→\AgdaUnderscore{}}}\AgdaSpace{}%
\AgdaOperator{\AgdaInductiveConstructor{Λ`α→\AgdaUnderscore{}}}\AgdaSpace{}%
\AgdaOperator{\AgdaInductiveConstructor{let`x=\AgdaUnderscore{}`in\AgdaUnderscore{}}}\AgdaSpace{}%
\AgdaOperator{\AgdaInductiveConstructor{∀`α\AgdaUnderscore{}}}\<%
\\
\>[0]\AgdaKeyword{infixr}\AgdaSpace{}%
\AgdaNumber{5}\AgdaSpace{}%
\AgdaOperator{\AgdaInductiveConstructor{\AgdaUnderscore{}⇒\AgdaUnderscore{}}}\AgdaSpace{}%
\AgdaOperator{\AgdaInductiveConstructor{\AgdaUnderscore{}·\AgdaUnderscore{}}}\AgdaSpace{}%
\AgdaOperator{\AgdaInductiveConstructor{\AgdaUnderscore{}•\AgdaUnderscore{}}}\<%
\\
\>[0]\AgdaKeyword{infix}%
\>[7]\AgdaNumber{6}\AgdaSpace{}%
\AgdaOperator{\AgdaInductiveConstructor{`\AgdaUnderscore{}}}\<%
\end{code}
\newcommand{\FTerm}[0]{\begin{code}%
\>[0]\AgdaKeyword{data}\AgdaSpace{}%
\AgdaDatatype{Term}\AgdaSpace{}%
\AgdaSymbol{:}\AgdaSpace{}%
\AgdaFunction{Sorts}\AgdaSpace{}%
\AgdaSymbol{→}\AgdaSpace{}%
\AgdaDatatype{Sort}\AgdaSpace{}%
\AgdaGeneralizable{r}\AgdaSpace{}%
\AgdaSymbol{→}\AgdaSpace{}%
\AgdaPrimitive{Set}\AgdaSpace{}%
\AgdaKeyword{where}\<%
\\
\>[0][@{}l@{\AgdaIndent{0}}]%
\>[2]\AgdaOperator{\AgdaInductiveConstructor{`\AgdaUnderscore{}}}%
\>[15]\AgdaSymbol{:}\AgdaSpace{}%
\AgdaGeneralizable{s}\AgdaSpace{}%
\AgdaOperator{\AgdaFunction{∈}}\AgdaSpace{}%
\AgdaGeneralizable{S}\AgdaSpace{}%
\AgdaSymbol{→}\AgdaSpace{}%
\AgdaDatatype{Term}\AgdaSpace{}%
\AgdaGeneralizable{S}\AgdaSpace{}%
\AgdaGeneralizable{s}\<%
\\
%
\>[2]\AgdaInductiveConstructor{tt}%
\>[15]\AgdaSymbol{:}\AgdaSpace{}%
\AgdaDatatype{Term}\AgdaSpace{}%
\AgdaGeneralizable{S}\AgdaSpace{}%
\AgdaInductiveConstructor{eₛ}\<%
\\
%
\>[2]\AgdaOperator{\AgdaInductiveConstructor{λ`x→\AgdaUnderscore{}}}%
\>[15]\AgdaSymbol{:}\AgdaSpace{}%
\AgdaDatatype{Term}\AgdaSpace{}%
\AgdaSymbol{(}\AgdaGeneralizable{S}\AgdaSpace{}%
\AgdaOperator{\AgdaInductiveConstructor{▷}}\AgdaSpace{}%
\AgdaInductiveConstructor{eₛ}\AgdaSymbol{)}\AgdaSpace{}%
\AgdaInductiveConstructor{eₛ}\AgdaSpace{}%
\AgdaSymbol{→}\AgdaSpace{}%
\AgdaDatatype{Term}\AgdaSpace{}%
\AgdaGeneralizable{S}\AgdaSpace{}%
\AgdaInductiveConstructor{eₛ}\<%
\\
%
\>[2]\AgdaOperator{\AgdaInductiveConstructor{Λ`α→\AgdaUnderscore{}}}%
\>[15]\AgdaSymbol{:}\AgdaSpace{}%
\AgdaDatatype{Term}\AgdaSpace{}%
\AgdaSymbol{(}\AgdaGeneralizable{S}\AgdaSpace{}%
\AgdaOperator{\AgdaInductiveConstructor{▷}}\AgdaSpace{}%
\AgdaInductiveConstructor{τₛ}\AgdaSymbol{)}\AgdaSpace{}%
\AgdaInductiveConstructor{eₛ}\AgdaSpace{}%
\AgdaSymbol{→}\AgdaSpace{}%
\AgdaDatatype{Term}\AgdaSpace{}%
\AgdaGeneralizable{S}\AgdaSpace{}%
\AgdaInductiveConstructor{eₛ}\<%
\\
%
\>[2]\AgdaOperator{\AgdaInductiveConstructor{\AgdaUnderscore{}·\AgdaUnderscore{}}}%
\>[15]\AgdaSymbol{:}\AgdaSpace{}%
\AgdaDatatype{Term}\AgdaSpace{}%
\AgdaGeneralizable{S}\AgdaSpace{}%
\AgdaInductiveConstructor{eₛ}\AgdaSpace{}%
\AgdaSymbol{→}\AgdaSpace{}%
\AgdaDatatype{Term}\AgdaSpace{}%
\AgdaGeneralizable{S}\AgdaSpace{}%
\AgdaInductiveConstructor{eₛ}\AgdaSpace{}%
\AgdaSymbol{→}\AgdaSpace{}%
\AgdaDatatype{Term}\AgdaSpace{}%
\AgdaGeneralizable{S}\AgdaSpace{}%
\AgdaInductiveConstructor{eₛ}\<%
\\
%
\>[2]\AgdaOperator{\AgdaInductiveConstructor{\AgdaUnderscore{}•\AgdaUnderscore{}}}%
\>[15]\AgdaSymbol{:}\AgdaSpace{}%
\AgdaDatatype{Term}\AgdaSpace{}%
\AgdaGeneralizable{S}\AgdaSpace{}%
\AgdaInductiveConstructor{eₛ}\AgdaSpace{}%
\AgdaSymbol{→}\AgdaSpace{}%
\AgdaDatatype{Term}\AgdaSpace{}%
\AgdaGeneralizable{S}\AgdaSpace{}%
\AgdaInductiveConstructor{τₛ}\AgdaSpace{}%
\AgdaSymbol{→}\AgdaSpace{}%
\AgdaDatatype{Term}\AgdaSpace{}%
\AgdaGeneralizable{S}\AgdaSpace{}%
\AgdaInductiveConstructor{eₛ}\<%
\\
%
\>[2]\AgdaOperator{\AgdaInductiveConstructor{let`x=\AgdaUnderscore{}`in\AgdaUnderscore{}}}%
\>[15]\AgdaSymbol{:}\AgdaSpace{}%
\AgdaDatatype{Term}\AgdaSpace{}%
\AgdaGeneralizable{S}\AgdaSpace{}%
\AgdaInductiveConstructor{eₛ}\AgdaSpace{}%
\AgdaSymbol{→}\AgdaSpace{}%
\AgdaDatatype{Term}\AgdaSpace{}%
\AgdaSymbol{(}\AgdaGeneralizable{S}\AgdaSpace{}%
\AgdaOperator{\AgdaInductiveConstructor{▷}}\AgdaSpace{}%
\AgdaInductiveConstructor{eₛ}\AgdaSymbol{)}\AgdaSpace{}%
\AgdaInductiveConstructor{eₛ}\AgdaSpace{}%
\AgdaSymbol{→}\AgdaSpace{}%
\AgdaDatatype{Term}\AgdaSpace{}%
\AgdaGeneralizable{S}\AgdaSpace{}%
\AgdaInductiveConstructor{eₛ}\<%
\\
%
\>[2]\AgdaInductiveConstructor{`⊤}%
\>[15]\AgdaSymbol{:}\AgdaSpace{}%
\AgdaDatatype{Term}\AgdaSpace{}%
\AgdaGeneralizable{S}\AgdaSpace{}%
\AgdaInductiveConstructor{τₛ}\<%
\\
%
\>[2]\AgdaOperator{\AgdaInductiveConstructor{\AgdaUnderscore{}⇒\AgdaUnderscore{}}}%
\>[15]\AgdaSymbol{:}\AgdaSpace{}%
\AgdaDatatype{Term}\AgdaSpace{}%
\AgdaGeneralizable{S}\AgdaSpace{}%
\AgdaInductiveConstructor{τₛ}\AgdaSpace{}%
\AgdaSymbol{→}\AgdaSpace{}%
\AgdaDatatype{Term}\AgdaSpace{}%
\AgdaGeneralizable{S}\AgdaSpace{}%
\AgdaInductiveConstructor{τₛ}\AgdaSpace{}%
\AgdaSymbol{→}\AgdaSpace{}%
\AgdaDatatype{Term}\AgdaSpace{}%
\AgdaGeneralizable{S}\AgdaSpace{}%
\AgdaInductiveConstructor{τₛ}\<%
\\
%
\>[2]\AgdaOperator{\AgdaInductiveConstructor{∀`α\AgdaUnderscore{}}}%
\>[15]\AgdaSymbol{:}\AgdaSpace{}%
\AgdaDatatype{Term}\AgdaSpace{}%
\AgdaSymbol{(}\AgdaGeneralizable{S}\AgdaSpace{}%
\AgdaOperator{\AgdaInductiveConstructor{▷}}\AgdaSpace{}%
\AgdaInductiveConstructor{τₛ}\AgdaSymbol{)}\AgdaSpace{}%
\AgdaInductiveConstructor{τₛ}\AgdaSpace{}%
\AgdaSymbol{→}\AgdaSpace{}%
\AgdaDatatype{Term}\AgdaSpace{}%
\AgdaGeneralizable{S}\AgdaSpace{}%
\AgdaInductiveConstructor{τₛ}\<%
\\
%
\>[2]\AgdaInductiveConstructor{⋆}%
\>[15]\AgdaSymbol{:}\AgdaSpace{}%
\AgdaDatatype{Term}\AgdaSpace{}%
\AgdaGeneralizable{S}\AgdaSpace{}%
\AgdaInductiveConstructor{κₛ}\<%
\end{code}}
\begin{code}[hide]%
\>[0]\AgdaFunction{Var}\AgdaSpace{}%
\AgdaSymbol{:}\AgdaSpace{}%
\AgdaFunction{Sorts}\AgdaSpace{}%
\AgdaSymbol{→}\AgdaSpace{}%
\AgdaDatatype{Sort}\AgdaSpace{}%
\AgdaInductiveConstructor{B}\AgdaSpace{}%
\AgdaSymbol{→}\AgdaSpace{}%
\AgdaPrimitive{Set}\<%
\end{code}
\newcommand{\FVar}[0]{\begin{code}[inline]%
\>[0]\AgdaFunction{Var}\AgdaSpace{}%
\AgdaBound{S}\AgdaSpace{}%
\AgdaBound{s}\AgdaSpace{}%
\AgdaSymbol{=}\AgdaSpace{}%
\AgdaBound{s}\AgdaSpace{}%
\AgdaOperator{\AgdaFunction{∈}}\AgdaSpace{}%
\AgdaBound{S}\<%
\end{code}}
\begin{code}[hide]%
\>[0]\AgdaFunction{Expr}\AgdaSpace{}%
\AgdaSymbol{:}\AgdaSpace{}%
\AgdaFunction{Sorts}\AgdaSpace{}%
\AgdaSymbol{→}\AgdaSpace{}%
\AgdaPrimitive{Set}\<%
\end{code}
\newcommand{\FExpr}[0]{\begin{code}[inline]%
\>[0]\AgdaFunction{Expr}\AgdaSpace{}%
\AgdaBound{S}\AgdaSpace{}%
\AgdaSymbol{=}\AgdaSpace{}%
\AgdaDatatype{Term}\AgdaSpace{}%
\AgdaBound{S}\AgdaSpace{}%
\AgdaInductiveConstructor{eₛ}\<%
\end{code}}
\begin{code}[hide]%
\>[0]\AgdaFunction{Type}\AgdaSpace{}%
\AgdaSymbol{:}\AgdaSpace{}%
\AgdaFunction{Sorts}\AgdaSpace{}%
\AgdaSymbol{→}\AgdaSpace{}%
\AgdaPrimitive{Set}\<%
\end{code}
\newcommand{\FType}[0]{\begin{code}[inline]%
\>[0]\AgdaFunction{Type}\AgdaSpace{}%
\AgdaBound{S}\AgdaSpace{}%
\AgdaSymbol{=}\AgdaSpace{}%
\AgdaDatatype{Term}\AgdaSpace{}%
\AgdaBound{S}\AgdaSpace{}%
\AgdaInductiveConstructor{τₛ}\<%
\end{code}}
\begin{code}[hide]%
\>[0]\AgdaKeyword{variable}\<%
\\
\>[0][@{}l@{\AgdaIndent{0}}]%
\>[2]\AgdaGeneralizable{t}\AgdaSpace{}%
\AgdaGeneralizable{t'}\AgdaSpace{}%
\AgdaGeneralizable{t''}\AgdaSpace{}%
\AgdaGeneralizable{t₁}\AgdaSpace{}%
\AgdaGeneralizable{t₂}\AgdaSpace{}%
\AgdaSymbol{:}\AgdaSpace{}%
\AgdaDatatype{Term}\AgdaSpace{}%
\AgdaGeneralizable{S}\AgdaSpace{}%
\AgdaGeneralizable{s}\<%
\\
%
\>[2]\AgdaGeneralizable{e}\AgdaSpace{}%
\AgdaGeneralizable{e'}\AgdaSpace{}%
\AgdaGeneralizable{e''}\AgdaSpace{}%
\AgdaGeneralizable{e₁}\AgdaSpace{}%
\AgdaGeneralizable{e₂}\AgdaSpace{}%
\AgdaSymbol{:}\AgdaSpace{}%
\AgdaFunction{Expr}\AgdaSpace{}%
\AgdaGeneralizable{S}\<%
\\
%
\>[2]\AgdaGeneralizable{τ}\AgdaSpace{}%
\AgdaGeneralizable{τ'}\AgdaSpace{}%
\AgdaGeneralizable{τ''}\AgdaSpace{}%
\AgdaGeneralizable{τ₁}\AgdaSpace{}%
\AgdaGeneralizable{τ₂}\AgdaSpace{}%
\AgdaSymbol{:}\AgdaSpace{}%
\AgdaFunction{Type}\AgdaSpace{}%
\AgdaGeneralizable{S}\<%
\\
%
\\[\AgdaEmptyExtraSkip]%
\>[0]\AgdaComment{--\ Renaming\ -----------------------------------------------------------------------------}\<%
\end{code}
\newcommand{\FRen}[0]{\begin{code}%
\>[0]\AgdaFunction{Ren}\AgdaSpace{}%
\AgdaSymbol{:}\AgdaSpace{}%
\AgdaFunction{Sorts}\AgdaSpace{}%
\AgdaSymbol{→}\AgdaSpace{}%
\AgdaFunction{Sorts}\AgdaSpace{}%
\AgdaSymbol{→}\AgdaSpace{}%
\AgdaPrimitive{Set}\<%
\\
\>[0]\AgdaFunction{Ren}\AgdaSpace{}%
\AgdaBound{S₁}\AgdaSpace{}%
\AgdaBound{S₂}\AgdaSpace{}%
\AgdaSymbol{=}\AgdaSpace{}%
\AgdaSymbol{∀}\AgdaSpace{}%
\AgdaSymbol{\{}\AgdaBound{s}\AgdaSymbol{\}}\AgdaSpace{}%
\AgdaSymbol{→}\AgdaSpace{}%
\AgdaFunction{Var}\AgdaSpace{}%
\AgdaBound{S₁}\AgdaSpace{}%
\AgdaBound{s}\AgdaSpace{}%
\AgdaSymbol{→}\AgdaSpace{}%
\AgdaFunction{Var}\AgdaSpace{}%
\AgdaBound{S₂}\AgdaSpace{}%
\AgdaBound{s}\<%
\end{code}}
\begin{code}[hide]%
\>[0]\AgdaFunction{idᵣ}\AgdaSpace{}%
\AgdaSymbol{:}\AgdaSpace{}%
\AgdaFunction{Ren}\AgdaSpace{}%
\AgdaGeneralizable{S}\AgdaSpace{}%
\AgdaGeneralizable{S}\<%
\\
\>[0]\AgdaFunction{idᵣ}\AgdaSpace{}%
\AgdaSymbol{=}\AgdaSpace{}%
\AgdaFunction{id}\<%
\\
%
\\[\AgdaEmptyExtraSkip]%
\>[0]\AgdaFunction{wkᵣ}\AgdaSpace{}%
\AgdaSymbol{:}\AgdaSpace{}%
\AgdaFunction{Ren}\AgdaSpace{}%
\AgdaGeneralizable{S}\AgdaSpace{}%
\AgdaSymbol{(}\AgdaGeneralizable{S}\AgdaSpace{}%
\AgdaOperator{\AgdaInductiveConstructor{▷}}\AgdaSpace{}%
\AgdaGeneralizable{s}\AgdaSymbol{)}\<%
\\
\>[0]\AgdaFunction{wkᵣ}\AgdaSpace{}%
\AgdaSymbol{=}\AgdaSpace{}%
\AgdaInductiveConstructor{there}\<%
\end{code}
\newcommand{\Frenext}[0]{\begin{code}[inline]%
\>[0]\AgdaFunction{extᵣ}\AgdaSpace{}%
\AgdaSymbol{:}\AgdaSpace{}%
\AgdaFunction{Ren}\AgdaSpace{}%
\AgdaGeneralizable{S₁}\AgdaSpace{}%
\AgdaGeneralizable{S₂}\AgdaSpace{}%
\AgdaSymbol{→}\AgdaSpace{}%
\AgdaFunction{Ren}\AgdaSpace{}%
\AgdaSymbol{(}\AgdaGeneralizable{S₁}\AgdaSpace{}%
\AgdaOperator{\AgdaInductiveConstructor{▷}}\AgdaSpace{}%
\AgdaGeneralizable{s}\AgdaSymbol{)}\AgdaSpace{}%
\AgdaSymbol{(}\AgdaGeneralizable{S₂}\AgdaSpace{}%
\AgdaOperator{\AgdaInductiveConstructor{▷}}\AgdaSpace{}%
\AgdaGeneralizable{s}\AgdaSymbol{)}\<%
\end{code}}
\begin{code}[hide]%
\>[0]\AgdaFunction{extᵣ}\AgdaSpace{}%
\AgdaBound{ρ}\AgdaSpace{}%
\AgdaSymbol{(}\AgdaInductiveConstructor{here}\AgdaSpace{}%
\AgdaInductiveConstructor{refl}\AgdaSymbol{)}\AgdaSpace{}%
\AgdaSymbol{=}\AgdaSpace{}%
\AgdaInductiveConstructor{here}\AgdaSpace{}%
\AgdaInductiveConstructor{refl}\<%
\\
\>[0]\AgdaFunction{extᵣ}\AgdaSpace{}%
\AgdaBound{ρ}\AgdaSpace{}%
\AgdaSymbol{(}\AgdaInductiveConstructor{there}\AgdaSpace{}%
\AgdaBound{x}\AgdaSymbol{)}\AgdaSpace{}%
\AgdaSymbol{=}\AgdaSpace{}%
\AgdaInductiveConstructor{there}\AgdaSpace{}%
\AgdaSymbol{(}\AgdaBound{ρ}\AgdaSpace{}%
\AgdaBound{x}\AgdaSymbol{)}\<%
\\
%
\\[\AgdaEmptyExtraSkip]%
\>[0]\AgdaFunction{dropᵣ}\AgdaSpace{}%
\AgdaSymbol{:}\AgdaSpace{}%
\AgdaFunction{Ren}\AgdaSpace{}%
\AgdaGeneralizable{S₁}\AgdaSpace{}%
\AgdaGeneralizable{S₂}\AgdaSpace{}%
\AgdaSymbol{→}\AgdaSpace{}%
\AgdaFunction{Ren}\AgdaSpace{}%
\AgdaGeneralizable{S₁}\AgdaSpace{}%
\AgdaSymbol{(}\AgdaGeneralizable{S₂}\AgdaSpace{}%
\AgdaOperator{\AgdaInductiveConstructor{▷}}\AgdaSpace{}%
\AgdaGeneralizable{s}\AgdaSymbol{)}\<%
\\
\>[0]\AgdaFunction{dropᵣ}\AgdaSpace{}%
\AgdaBound{ρ}\AgdaSpace{}%
\AgdaBound{x}\AgdaSpace{}%
\AgdaSymbol{=}\AgdaSpace{}%
\AgdaInductiveConstructor{there}\AgdaSpace{}%
\AgdaSymbol{(}\AgdaBound{ρ}\AgdaSpace{}%
\AgdaBound{x}\AgdaSymbol{)}\<%
\end{code}
\newcommand{\Fren}[0]{\begin{code}%
\>[0]\AgdaFunction{ren}\AgdaSpace{}%
\AgdaSymbol{:}\AgdaSpace{}%
\AgdaFunction{Ren}\AgdaSpace{}%
\AgdaGeneralizable{S₁}\AgdaSpace{}%
\AgdaGeneralizable{S₂}\AgdaSpace{}%
\AgdaSymbol{→}\AgdaSpace{}%
\AgdaSymbol{(}\AgdaDatatype{Term}\AgdaSpace{}%
\AgdaGeneralizable{S₁}\AgdaSpace{}%
\AgdaGeneralizable{s}\AgdaSpace{}%
\AgdaSymbol{→}\AgdaSpace{}%
\AgdaDatatype{Term}\AgdaSpace{}%
\AgdaGeneralizable{S₂}\AgdaSpace{}%
\AgdaGeneralizable{s}\AgdaSymbol{)}\<%
\\
\>[0]\AgdaFunction{ren}\AgdaSpace{}%
\AgdaBound{ρ}\AgdaSpace{}%
\AgdaSymbol{(}\AgdaOperator{\AgdaInductiveConstructor{`}}\AgdaSpace{}%
\AgdaBound{x}\AgdaSymbol{)}\AgdaSpace{}%
\AgdaSymbol{=}\AgdaSpace{}%
\AgdaOperator{\AgdaInductiveConstructor{`}}\AgdaSpace{}%
\AgdaSymbol{(}\AgdaBound{ρ}\AgdaSpace{}%
\AgdaBound{x}\AgdaSymbol{)}\<%
\\
\>[0]\AgdaFunction{ren}\AgdaSpace{}%
\AgdaBound{ρ}\AgdaSpace{}%
\AgdaInductiveConstructor{tt}\AgdaSpace{}%
\AgdaSymbol{=}\AgdaSpace{}%
\AgdaInductiveConstructor{tt}\<%
\\
\>[0]\AgdaFunction{ren}\AgdaSpace{}%
\AgdaBound{ρ}\AgdaSpace{}%
\AgdaSymbol{(}\AgdaOperator{\AgdaInductiveConstructor{λ`x→}}\AgdaSpace{}%
\AgdaBound{e}\AgdaSymbol{)}\AgdaSpace{}%
\AgdaSymbol{=}\AgdaSpace{}%
\AgdaOperator{\AgdaInductiveConstructor{λ`x→}}\AgdaSpace{}%
\AgdaSymbol{(}\AgdaFunction{ren}\AgdaSpace{}%
\AgdaSymbol{(}\AgdaFunction{extᵣ}\AgdaSpace{}%
\AgdaBound{ρ}\AgdaSymbol{)}\AgdaSpace{}%
\AgdaBound{e}\AgdaSymbol{)}\<%
\\
\>[0]\AgdaFunction{ren}\AgdaSpace{}%
\AgdaBound{ρ}\AgdaSpace{}%
\AgdaSymbol{(}\AgdaOperator{\AgdaInductiveConstructor{Λ`α→}}\AgdaSpace{}%
\AgdaBound{e}\AgdaSymbol{)}\AgdaSpace{}%
\AgdaSymbol{=}\AgdaSpace{}%
\AgdaOperator{\AgdaInductiveConstructor{Λ`α→}}\AgdaSpace{}%
\AgdaSymbol{(}\AgdaFunction{ren}\AgdaSpace{}%
\AgdaSymbol{(}\AgdaFunction{extᵣ}\AgdaSpace{}%
\AgdaBound{ρ}\AgdaSymbol{)}\AgdaSpace{}%
\AgdaBound{e}\AgdaSymbol{)}\<%
\\
\>[0]\AgdaFunction{ren}\AgdaSpace{}%
\AgdaBound{ρ}\AgdaSpace{}%
\AgdaSymbol{(}\AgdaBound{e₁}\AgdaSpace{}%
\AgdaOperator{\AgdaInductiveConstructor{·}}\AgdaSpace{}%
\AgdaBound{e₂}\AgdaSymbol{)}\AgdaSpace{}%
\AgdaSymbol{=}\AgdaSpace{}%
\AgdaSymbol{(}\AgdaFunction{ren}\AgdaSpace{}%
\AgdaBound{ρ}\AgdaSpace{}%
\AgdaBound{e₁}\AgdaSymbol{)}\AgdaSpace{}%
\AgdaOperator{\AgdaInductiveConstructor{·}}\AgdaSpace{}%
\AgdaSymbol{(}\AgdaFunction{ren}\AgdaSpace{}%
\AgdaBound{ρ}\AgdaSpace{}%
\AgdaBound{e₂}\AgdaSymbol{)}\<%
\\
\>[0]\AgdaFunction{ren}\AgdaSpace{}%
\AgdaBound{ρ}\AgdaSpace{}%
\AgdaSymbol{(}\AgdaBound{e}\AgdaSpace{}%
\AgdaOperator{\AgdaInductiveConstructor{•}}\AgdaSpace{}%
\AgdaBound{τ}\AgdaSymbol{)}\AgdaSpace{}%
\AgdaSymbol{=}\AgdaSpace{}%
\AgdaSymbol{(}\AgdaFunction{ren}\AgdaSpace{}%
\AgdaBound{ρ}\AgdaSpace{}%
\AgdaBound{e}\AgdaSymbol{)}\AgdaSpace{}%
\AgdaOperator{\AgdaInductiveConstructor{•}}\AgdaSpace{}%
\AgdaSymbol{(}\AgdaFunction{ren}\AgdaSpace{}%
\AgdaBound{ρ}\AgdaSpace{}%
\AgdaBound{τ}\AgdaSymbol{)}\<%
\\
\>[0]\AgdaFunction{ren}\AgdaSpace{}%
\AgdaBound{ρ}\AgdaSpace{}%
\AgdaSymbol{(}\AgdaOperator{\AgdaInductiveConstructor{let`x=}}\AgdaSpace{}%
\AgdaBound{e₂}\AgdaSpace{}%
\AgdaOperator{\AgdaInductiveConstructor{`in}}\AgdaSpace{}%
\AgdaBound{e₁}\AgdaSymbol{)}\AgdaSpace{}%
\AgdaSymbol{=}\AgdaSpace{}%
\AgdaOperator{\AgdaInductiveConstructor{let`x=}}\AgdaSpace{}%
\AgdaSymbol{(}\AgdaFunction{ren}\AgdaSpace{}%
\AgdaBound{ρ}\AgdaSpace{}%
\AgdaBound{e₂}\AgdaSymbol{)}\AgdaSpace{}%
\AgdaOperator{\AgdaInductiveConstructor{`in}}\AgdaSpace{}%
\AgdaFunction{ren}\AgdaSpace{}%
\AgdaSymbol{(}\AgdaFunction{extᵣ}\AgdaSpace{}%
\AgdaBound{ρ}\AgdaSymbol{)}\AgdaSpace{}%
\AgdaBound{e₁}\<%
\\
\>[0]\AgdaFunction{ren}\AgdaSpace{}%
\AgdaBound{ρ}\AgdaSpace{}%
\AgdaInductiveConstructor{`⊤}\AgdaSpace{}%
\AgdaSymbol{=}\AgdaSpace{}%
\AgdaInductiveConstructor{`⊤}\<%
\\
\>[0]\AgdaFunction{ren}\AgdaSpace{}%
\AgdaBound{ρ}\AgdaSpace{}%
\AgdaSymbol{(}\AgdaBound{τ₁}\AgdaSpace{}%
\AgdaOperator{\AgdaInductiveConstructor{⇒}}\AgdaSpace{}%
\AgdaBound{τ₂}\AgdaSymbol{)}\AgdaSpace{}%
\AgdaSymbol{=}\AgdaSpace{}%
\AgdaFunction{ren}\AgdaSpace{}%
\AgdaBound{ρ}\AgdaSpace{}%
\AgdaBound{τ₁}\AgdaSpace{}%
\AgdaOperator{\AgdaInductiveConstructor{⇒}}\AgdaSpace{}%
\AgdaFunction{ren}\AgdaSpace{}%
\AgdaBound{ρ}\AgdaSpace{}%
\AgdaBound{τ₂}\<%
\\
\>[0]\AgdaFunction{ren}\AgdaSpace{}%
\AgdaBound{ρ}\AgdaSpace{}%
\AgdaSymbol{(}\AgdaOperator{\AgdaInductiveConstructor{∀`α}}\AgdaSpace{}%
\AgdaBound{τ}\AgdaSymbol{)}\AgdaSpace{}%
\AgdaSymbol{=}\AgdaSpace{}%
\AgdaOperator{\AgdaInductiveConstructor{∀`α}}\AgdaSpace{}%
\AgdaSymbol{(}\AgdaFunction{ren}\AgdaSpace{}%
\AgdaSymbol{(}\AgdaFunction{extᵣ}\AgdaSpace{}%
\AgdaBound{ρ}\AgdaSymbol{)}\AgdaSpace{}%
\AgdaBound{τ}\AgdaSymbol{)}\<%
\\
\>[0]\AgdaFunction{ren}\AgdaSpace{}%
\AgdaBound{ρ}\AgdaSpace{}%
\AgdaInductiveConstructor{⋆}\AgdaSpace{}%
\AgdaSymbol{=}\AgdaSpace{}%
\AgdaInductiveConstructor{⋆}\<%
\end{code}}
\newcommand{\Fwk}[0]{\begin{code}%
\>[0]\AgdaFunction{wk}\AgdaSpace{}%
\AgdaSymbol{:}\AgdaSpace{}%
\AgdaDatatype{Term}\AgdaSpace{}%
\AgdaGeneralizable{S}\AgdaSpace{}%
\AgdaGeneralizable{s}\AgdaSpace{}%
\AgdaSymbol{→}\AgdaSpace{}%
\AgdaDatatype{Term}\AgdaSpace{}%
\AgdaSymbol{(}\AgdaGeneralizable{S}\AgdaSpace{}%
\AgdaOperator{\AgdaInductiveConstructor{▷}}\AgdaSpace{}%
\AgdaGeneralizable{s'}\AgdaSymbol{)}\AgdaSpace{}%
\AgdaGeneralizable{s}\<%
\\
\>[0]\AgdaFunction{wk}\AgdaSpace{}%
\AgdaSymbol{=}\AgdaSpace{}%
\AgdaFunction{ren}\AgdaSpace{}%
\AgdaInductiveConstructor{there}\<%
\end{code}}
\begin{code}[hide]%
\>[0]\AgdaKeyword{variable}\<%
\\
\>[0][@{}l@{\AgdaIndent{0}}]%
\>[2]\AgdaGeneralizable{ρ}\AgdaSpace{}%
\AgdaGeneralizable{ρ'}\AgdaSpace{}%
\AgdaGeneralizable{ρ''}\AgdaSpace{}%
\AgdaGeneralizable{ρ₁}\AgdaSpace{}%
\AgdaGeneralizable{ρ₂}\AgdaSpace{}%
\AgdaSymbol{:}\AgdaSpace{}%
\AgdaFunction{Ren}\AgdaSpace{}%
\AgdaGeneralizable{S₁}\AgdaSpace{}%
\AgdaGeneralizable{S₂}\<%
\\
%
\\[\AgdaEmptyExtraSkip]%
\>[0]\AgdaComment{--\ Substitution\ -------------------------------------------------------------------------}\<%
\end{code}
\newcommand{\FSub}[0]{\begin{code}%
\>[0]\AgdaFunction{Sub}\AgdaSpace{}%
\AgdaSymbol{:}\AgdaSpace{}%
\AgdaFunction{Sorts}\AgdaSpace{}%
\AgdaSymbol{→}\AgdaSpace{}%
\AgdaFunction{Sorts}\AgdaSpace{}%
\AgdaSymbol{→}\AgdaSpace{}%
\AgdaPrimitive{Set}\<%
\\
\>[0]\AgdaFunction{Sub}\AgdaSpace{}%
\AgdaBound{S₁}\AgdaSpace{}%
\AgdaBound{S₂}\AgdaSpace{}%
\AgdaSymbol{=}\AgdaSpace{}%
\AgdaSymbol{∀}\AgdaSpace{}%
\AgdaSymbol{\{}\AgdaBound{s}\AgdaSymbol{\}}\AgdaSpace{}%
\AgdaSymbol{→}\AgdaSpace{}%
\AgdaFunction{Var}\AgdaSpace{}%
\AgdaBound{S₁}\AgdaSpace{}%
\AgdaBound{s}\AgdaSpace{}%
\AgdaSymbol{→}\AgdaSpace{}%
\AgdaDatatype{Term}\AgdaSpace{}%
\AgdaBound{S₂}\AgdaSpace{}%
\AgdaBound{s}\<%
\end{code}}
\newcommand{\Fidsub}[0]{\begin{code}[inline]%
\>[0]\AgdaFunction{idₛ}\AgdaSpace{}%
\AgdaSymbol{:}\AgdaSpace{}%
\AgdaFunction{Sub}\AgdaSpace{}%
\AgdaGeneralizable{S}\AgdaSpace{}%
\AgdaGeneralizable{S}\<%
\end{code}}
\begin{code}[hide]%
\>[0]\AgdaFunction{idₛ}\AgdaSpace{}%
\AgdaSymbol{=}\AgdaSpace{}%
\AgdaOperator{\AgdaInductiveConstructor{`\AgdaUnderscore{}}}\<%
\\
%
\\[\AgdaEmptyExtraSkip]%
\>[0]\AgdaFunction{extₛ}\AgdaSpace{}%
\AgdaSymbol{:}\AgdaSpace{}%
\AgdaFunction{Sub}\AgdaSpace{}%
\AgdaGeneralizable{S₁}\AgdaSpace{}%
\AgdaGeneralizable{S₂}\AgdaSpace{}%
\AgdaSymbol{→}\AgdaSpace{}%
\AgdaFunction{Sub}\AgdaSpace{}%
\AgdaSymbol{(}\AgdaGeneralizable{S₁}\AgdaSpace{}%
\AgdaOperator{\AgdaInductiveConstructor{▷}}\AgdaSpace{}%
\AgdaGeneralizable{s}\AgdaSymbol{)}\AgdaSpace{}%
\AgdaSymbol{(}\AgdaGeneralizable{S₂}\AgdaSpace{}%
\AgdaOperator{\AgdaInductiveConstructor{▷}}\AgdaSpace{}%
\AgdaGeneralizable{s}\AgdaSymbol{)}\<%
\\
\>[0]\AgdaFunction{extₛ}\AgdaSpace{}%
\AgdaBound{σ}\AgdaSpace{}%
\AgdaSymbol{(}\AgdaInductiveConstructor{here}\AgdaSpace{}%
\AgdaInductiveConstructor{refl}\AgdaSymbol{)}\AgdaSpace{}%
\AgdaSymbol{=}\AgdaSpace{}%
\AgdaOperator{\AgdaInductiveConstructor{`}}\AgdaSpace{}%
\AgdaInductiveConstructor{here}\AgdaSpace{}%
\AgdaInductiveConstructor{refl}\<%
\\
\>[0]\AgdaFunction{extₛ}\AgdaSpace{}%
\AgdaBound{σ}\AgdaSpace{}%
\AgdaSymbol{(}\AgdaInductiveConstructor{there}\AgdaSpace{}%
\AgdaBound{x}\AgdaSymbol{)}\AgdaSpace{}%
\AgdaSymbol{=}\AgdaSpace{}%
\AgdaFunction{ren}\AgdaSpace{}%
\AgdaFunction{wkᵣ}\AgdaSpace{}%
\AgdaSymbol{(}\AgdaBound{σ}\AgdaSpace{}%
\AgdaBound{x}\AgdaSymbol{)}\<%
\\
%
\\[\AgdaEmptyExtraSkip]%
\>[0]\AgdaFunction{dropₛ}\AgdaSpace{}%
\AgdaSymbol{:}\AgdaSpace{}%
\AgdaFunction{Sub}\AgdaSpace{}%
\AgdaGeneralizable{S₁}\AgdaSpace{}%
\AgdaGeneralizable{S₂}\AgdaSpace{}%
\AgdaSymbol{→}\AgdaSpace{}%
\AgdaFunction{Sub}\AgdaSpace{}%
\AgdaGeneralizable{S₁}\AgdaSpace{}%
\AgdaSymbol{(}\AgdaGeneralizable{S₂}\AgdaSpace{}%
\AgdaOperator{\AgdaInductiveConstructor{▷}}\AgdaSpace{}%
\AgdaGeneralizable{s}\AgdaSymbol{)}\<%
\\
\>[0]\AgdaFunction{dropₛ}\AgdaSpace{}%
\AgdaBound{σ}\AgdaSpace{}%
\AgdaBound{x}\AgdaSpace{}%
\AgdaSymbol{=}\AgdaSpace{}%
\AgdaFunction{wk}\AgdaSpace{}%
\AgdaSymbol{(}\AgdaBound{σ}\AgdaSpace{}%
\AgdaBound{x}\AgdaSymbol{)}\<%
\end{code}
\newcommand{\Fsinglesub}[0]{\begin{code}[inline]%
\>[0]\AgdaFunction{singleₛ}\AgdaSpace{}%
\AgdaSymbol{:}\AgdaSpace{}%
\AgdaFunction{Sub}\AgdaSpace{}%
\AgdaGeneralizable{S₁}\AgdaSpace{}%
\AgdaGeneralizable{S₂}\AgdaSpace{}%
\AgdaSymbol{→}\AgdaSpace{}%
\AgdaDatatype{Term}\AgdaSpace{}%
\AgdaGeneralizable{S₂}\AgdaSpace{}%
\AgdaGeneralizable{s}\AgdaSpace{}%
\AgdaSymbol{→}\AgdaSpace{}%
\AgdaFunction{Sub}\AgdaSpace{}%
\AgdaSymbol{(}\AgdaGeneralizable{S₁}\AgdaSpace{}%
\AgdaOperator{\AgdaInductiveConstructor{▷}}\AgdaSpace{}%
\AgdaGeneralizable{s}\AgdaSymbol{)}\AgdaSpace{}%
\AgdaGeneralizable{S₂}\<%
\end{code}}
\begin{code}[hide]%
\>[0]\AgdaFunction{singleₛ}\AgdaSpace{}%
\AgdaBound{σ}\AgdaSpace{}%
\AgdaBound{t}\AgdaSpace{}%
\AgdaSymbol{(}\AgdaInductiveConstructor{here}\AgdaSpace{}%
\AgdaInductiveConstructor{refl}\AgdaSymbol{)}\AgdaSpace{}%
\AgdaSymbol{=}\AgdaSpace{}%
\AgdaBound{t}\<%
\\
\>[0]\AgdaFunction{singleₛ}\AgdaSpace{}%
\AgdaBound{σ}\AgdaSpace{}%
\AgdaBound{t}\AgdaSpace{}%
\AgdaSymbol{(}\AgdaInductiveConstructor{there}\AgdaSpace{}%
\AgdaBound{x}\AgdaSymbol{)}\AgdaSpace{}%
\AgdaSymbol{=}\AgdaSpace{}%
\AgdaBound{σ}\AgdaSpace{}%
\AgdaBound{x}\<%
\end{code}
\newcommand{\Fsub}[0]{\begin{code}[inline]%
\>[0]\AgdaFunction{sub}\AgdaSpace{}%
\AgdaSymbol{:}\AgdaSpace{}%
\AgdaFunction{Sub}\AgdaSpace{}%
\AgdaGeneralizable{S₁}\AgdaSpace{}%
\AgdaGeneralizable{S₂}\AgdaSpace{}%
\AgdaSymbol{→}\AgdaSpace{}%
\AgdaSymbol{(}\AgdaDatatype{Term}\AgdaSpace{}%
\AgdaGeneralizable{S₁}\AgdaSpace{}%
\AgdaGeneralizable{s}\AgdaSpace{}%
\AgdaSymbol{→}\AgdaSpace{}%
\AgdaDatatype{Term}\AgdaSpace{}%
\AgdaGeneralizable{S₂}\AgdaSpace{}%
\AgdaGeneralizable{s}\AgdaSymbol{)}\<%
\end{code}}
\begin{code}[hide]%
\>[0]\AgdaFunction{sub}\AgdaSpace{}%
\AgdaBound{σ}\AgdaSpace{}%
\AgdaSymbol{(}\AgdaOperator{\AgdaInductiveConstructor{`}}\AgdaSpace{}%
\AgdaBound{x}\AgdaSymbol{)}\AgdaSpace{}%
\AgdaSymbol{=}\AgdaSpace{}%
\AgdaSymbol{(}\AgdaBound{σ}\AgdaSpace{}%
\AgdaBound{x}\AgdaSymbol{)}\<%
\\
\>[0]\AgdaFunction{sub}\AgdaSpace{}%
\AgdaBound{σ}\AgdaSpace{}%
\AgdaInductiveConstructor{tt}\AgdaSpace{}%
\AgdaSymbol{=}\AgdaSpace{}%
\AgdaInductiveConstructor{tt}\<%
\\
\>[0]\AgdaFunction{sub}\AgdaSpace{}%
\AgdaBound{σ}\AgdaSpace{}%
\AgdaSymbol{(}\AgdaOperator{\AgdaInductiveConstructor{λ`x→}}\AgdaSpace{}%
\AgdaBound{e}\AgdaSymbol{)}\AgdaSpace{}%
\AgdaSymbol{=}\AgdaSpace{}%
\AgdaOperator{\AgdaInductiveConstructor{λ`x→}}\AgdaSpace{}%
\AgdaSymbol{(}\AgdaFunction{sub}\AgdaSpace{}%
\AgdaSymbol{(}\AgdaFunction{extₛ}\AgdaSpace{}%
\AgdaBound{σ}\AgdaSymbol{)}\AgdaSpace{}%
\AgdaBound{e}\AgdaSymbol{)}\<%
\\
\>[0]\AgdaFunction{sub}\AgdaSpace{}%
\AgdaBound{σ}\AgdaSpace{}%
\AgdaSymbol{(}\AgdaOperator{\AgdaInductiveConstructor{Λ`α→}}\AgdaSpace{}%
\AgdaBound{e}\AgdaSymbol{)}\AgdaSpace{}%
\AgdaSymbol{=}\AgdaSpace{}%
\AgdaOperator{\AgdaInductiveConstructor{Λ`α→}}\AgdaSpace{}%
\AgdaSymbol{(}\AgdaFunction{sub}\AgdaSpace{}%
\AgdaSymbol{(}\AgdaFunction{extₛ}\AgdaSpace{}%
\AgdaBound{σ}\AgdaSymbol{)}\AgdaSpace{}%
\AgdaBound{e}\AgdaSymbol{)}\<%
\\
\>[0]\AgdaFunction{sub}\AgdaSpace{}%
\AgdaBound{σ}\AgdaSpace{}%
\AgdaSymbol{(}\AgdaBound{e₁}\AgdaSpace{}%
\AgdaOperator{\AgdaInductiveConstructor{·}}\AgdaSpace{}%
\AgdaBound{e₂}\AgdaSymbol{)}\AgdaSpace{}%
\AgdaSymbol{=}\AgdaSpace{}%
\AgdaFunction{sub}\AgdaSpace{}%
\AgdaBound{σ}\AgdaSpace{}%
\AgdaBound{e₁}\AgdaSpace{}%
\AgdaOperator{\AgdaInductiveConstructor{·}}\AgdaSpace{}%
\AgdaFunction{sub}\AgdaSpace{}%
\AgdaBound{σ}\AgdaSpace{}%
\AgdaBound{e₂}\<%
\\
\>[0]\AgdaFunction{sub}\AgdaSpace{}%
\AgdaBound{σ}\AgdaSpace{}%
\AgdaSymbol{(}\AgdaBound{e}\AgdaSpace{}%
\AgdaOperator{\AgdaInductiveConstructor{•}}\AgdaSpace{}%
\AgdaBound{τ}\AgdaSymbol{)}\AgdaSpace{}%
\AgdaSymbol{=}\AgdaSpace{}%
\AgdaFunction{sub}\AgdaSpace{}%
\AgdaBound{σ}\AgdaSpace{}%
\AgdaBound{e}\AgdaSpace{}%
\AgdaOperator{\AgdaInductiveConstructor{•}}\AgdaSpace{}%
\AgdaFunction{sub}\AgdaSpace{}%
\AgdaBound{σ}\AgdaSpace{}%
\AgdaBound{τ}\<%
\\
\>[0]\AgdaFunction{sub}\AgdaSpace{}%
\AgdaBound{σ}\AgdaSpace{}%
\AgdaSymbol{(}\AgdaOperator{\AgdaInductiveConstructor{let`x=}}\AgdaSpace{}%
\AgdaBound{e₂}\AgdaSpace{}%
\AgdaOperator{\AgdaInductiveConstructor{`in}}\AgdaSpace{}%
\AgdaBound{e₁}\AgdaSymbol{)}\AgdaSpace{}%
\AgdaSymbol{=}\AgdaSpace{}%
\AgdaOperator{\AgdaInductiveConstructor{let`x=}}\AgdaSpace{}%
\AgdaFunction{sub}\AgdaSpace{}%
\AgdaBound{σ}\AgdaSpace{}%
\AgdaBound{e₂}\AgdaSpace{}%
\AgdaOperator{\AgdaInductiveConstructor{`in}}\AgdaSpace{}%
\AgdaSymbol{(}\AgdaFunction{sub}\AgdaSpace{}%
\AgdaSymbol{(}\AgdaFunction{extₛ}\AgdaSpace{}%
\AgdaBound{σ}\AgdaSymbol{)}\AgdaSpace{}%
\AgdaBound{e₁}\AgdaSymbol{)}\<%
\\
\>[0]\AgdaFunction{sub}\AgdaSpace{}%
\AgdaBound{σ}\AgdaSpace{}%
\AgdaInductiveConstructor{`⊤}\AgdaSpace{}%
\AgdaSymbol{=}\AgdaSpace{}%
\AgdaInductiveConstructor{`⊤}\<%
\\
\>[0]\AgdaFunction{sub}\AgdaSpace{}%
\AgdaBound{σ}\AgdaSpace{}%
\AgdaSymbol{(}\AgdaBound{τ₁}\AgdaSpace{}%
\AgdaOperator{\AgdaInductiveConstructor{⇒}}\AgdaSpace{}%
\AgdaBound{τ₂}\AgdaSymbol{)}\AgdaSpace{}%
\AgdaSymbol{=}\AgdaSpace{}%
\AgdaFunction{sub}\AgdaSpace{}%
\AgdaBound{σ}\AgdaSpace{}%
\AgdaBound{τ₁}\AgdaSpace{}%
\AgdaOperator{\AgdaInductiveConstructor{⇒}}\AgdaSpace{}%
\AgdaFunction{sub}\AgdaSpace{}%
\AgdaBound{σ}\AgdaSpace{}%
\AgdaBound{τ₂}\<%
\\
\>[0]\AgdaFunction{sub}\AgdaSpace{}%
\AgdaBound{σ}\AgdaSpace{}%
\AgdaSymbol{(}\AgdaOperator{\AgdaInductiveConstructor{∀`α}}\AgdaSpace{}%
\AgdaBound{τ}\AgdaSymbol{)}\AgdaSpace{}%
\AgdaSymbol{=}\AgdaSpace{}%
\AgdaOperator{\AgdaInductiveConstructor{∀`α}}\AgdaSpace{}%
\AgdaSymbol{(}\AgdaFunction{sub}\AgdaSpace{}%
\AgdaSymbol{(}\AgdaFunction{extₛ}\AgdaSpace{}%
\AgdaBound{σ}\AgdaSymbol{)}\AgdaSpace{}%
\AgdaBound{τ}\AgdaSymbol{)}\<%
\\
\>[0]\AgdaFunction{sub}\AgdaSpace{}%
\AgdaBound{σ}\AgdaSpace{}%
\AgdaInductiveConstructor{⋆}\AgdaSpace{}%
\AgdaSymbol{=}\AgdaSpace{}%
\AgdaInductiveConstructor{⋆}\<%
\end{code}
\newcommand{\Fsubs}[0]{\begin{code}%
\>[0]\AgdaOperator{\AgdaFunction{\AgdaUnderscore{}[\AgdaUnderscore{}]}}\AgdaSpace{}%
\AgdaSymbol{:}\AgdaSpace{}%
\AgdaDatatype{Term}\AgdaSpace{}%
\AgdaSymbol{(}\AgdaGeneralizable{S}\AgdaSpace{}%
\AgdaOperator{\AgdaInductiveConstructor{▷}}\AgdaSpace{}%
\AgdaGeneralizable{s'}\AgdaSymbol{)}\AgdaSpace{}%
\AgdaGeneralizable{s}\AgdaSpace{}%
\AgdaSymbol{→}\AgdaSpace{}%
\AgdaDatatype{Term}\AgdaSpace{}%
\AgdaGeneralizable{S}\AgdaSpace{}%
\AgdaGeneralizable{s'}\AgdaSpace{}%
\AgdaSymbol{→}\AgdaSpace{}%
\AgdaDatatype{Term}\AgdaSpace{}%
\AgdaGeneralizable{S}\AgdaSpace{}%
\AgdaGeneralizable{s}\<%
\\
\>[0]\AgdaBound{t}\AgdaSpace{}%
\AgdaOperator{\AgdaFunction{[}}\AgdaSpace{}%
\AgdaBound{t'}\AgdaSpace{}%
\AgdaOperator{\AgdaFunction{]}}\AgdaSpace{}%
\AgdaSymbol{=}\AgdaSpace{}%
\AgdaFunction{sub}\AgdaSpace{}%
\AgdaSymbol{(}\AgdaFunction{singleₛ}\AgdaSpace{}%
\AgdaFunction{idₛ}\AgdaSpace{}%
\AgdaBound{t'}\AgdaSymbol{)}\AgdaSpace{}%
\AgdaBound{t}\<%
\end{code}}
\newcommand{\Fhide}[0]{\begin{code}%
\>[0]\AgdaKeyword{variable}\<%
\\
\>[0][@{}l@{\AgdaIndent{0}}]%
\>[2]\AgdaGeneralizable{σ}\AgdaSpace{}%
\AgdaGeneralizable{σ'}\AgdaSpace{}%
\AgdaGeneralizable{σ''}\AgdaSpace{}%
\AgdaGeneralizable{σ₁}\AgdaSpace{}%
\AgdaGeneralizable{σ₂}\AgdaSpace{}%
\AgdaSymbol{:}\AgdaSpace{}%
\AgdaFunction{Sub}\AgdaSpace{}%
\AgdaGeneralizable{S₁}\AgdaSpace{}%
\AgdaGeneralizable{S₂}\<%
\\
%
\\[\AgdaEmptyExtraSkip]%
\>[0]\AgdaComment{--\ Context\ ------------------------------------------------------------------------------}\<%
\\
%
\\[\AgdaEmptyExtraSkip]%
\>[0]\AgdaFunction{kind-Bindable}\AgdaSpace{}%
\AgdaSymbol{:}\AgdaSpace{}%
\AgdaDatatype{Sort}\AgdaSpace{}%
\AgdaInductiveConstructor{B}\AgdaSpace{}%
\AgdaSymbol{→}\AgdaSpace{}%
\AgdaDatatype{Bindable}\<%
\\
\>[0]\AgdaFunction{kind-Bindable}\AgdaSpace{}%
\AgdaInductiveConstructor{eₛ}\AgdaSpace{}%
\AgdaSymbol{=}\AgdaSpace{}%
\AgdaInductiveConstructor{B}\<%
\\
\>[0]\AgdaFunction{kind-Bindable}\AgdaSpace{}%
\AgdaInductiveConstructor{τₛ}\AgdaSpace{}%
\AgdaSymbol{=}\AgdaSpace{}%
\AgdaInductiveConstructor{¬B}\<%
\\
%
\\[\AgdaEmptyExtraSkip]%
%
\\[\AgdaEmptyExtraSkip]%
\>[0]\AgdaFunction{type-of}\AgdaSpace{}%
\AgdaSymbol{:}\AgdaSpace{}%
\AgdaSymbol{(}\AgdaBound{s}\AgdaSpace{}%
\AgdaSymbol{:}\AgdaSpace{}%
\AgdaDatatype{Sort}\AgdaSpace{}%
\AgdaInductiveConstructor{B}\AgdaSymbol{)}\AgdaSpace{}%
\AgdaSymbol{→}\AgdaSpace{}%
\AgdaDatatype{Sort}\AgdaSpace{}%
\AgdaSymbol{(}\AgdaFunction{kind-Bindable}\AgdaSpace{}%
\AgdaBound{s}\AgdaSymbol{)}\<%
\end{code}}
\newcommand{\Fkind}[0]{\begin{code}%
\>[0]\AgdaFunction{type-of}\AgdaSpace{}%
\AgdaInductiveConstructor{eₛ}\AgdaSpace{}%
\AgdaSymbol{=}\AgdaSpace{}%
\AgdaInductiveConstructor{τₛ}\<%
\\
\>[0]\AgdaFunction{type-of}\AgdaSpace{}%
\AgdaInductiveConstructor{τₛ}\AgdaSpace{}%
\AgdaSymbol{=}\AgdaSpace{}%
\AgdaInductiveConstructor{κₛ}\<%
\end{code}}
\begin{code}[hide]%
\>[0]\AgdaKeyword{variable}\<%
\\
\>[0][@{}l@{\AgdaIndent{0}}]%
\>[2]\AgdaGeneralizable{T}\AgdaSpace{}%
\AgdaGeneralizable{T'}\AgdaSpace{}%
\AgdaGeneralizable{T''}\AgdaSpace{}%
\AgdaGeneralizable{T₁}\AgdaSpace{}%
\AgdaGeneralizable{T₂}\AgdaSpace{}%
\AgdaSymbol{:}\AgdaSpace{}%
\AgdaDatatype{Term}\AgdaSpace{}%
\AgdaGeneralizable{S}\AgdaSpace{}%
\AgdaSymbol{(}\AgdaFunction{type-of}\AgdaSpace{}%
\AgdaGeneralizable{s}\AgdaSymbol{)}\<%
\end{code}
\newcommand{\FCtx}[0]{\begin{code}%
\>[0]\AgdaKeyword{data}\AgdaSpace{}%
\AgdaDatatype{Ctx}\AgdaSpace{}%
\AgdaSymbol{:}\AgdaSpace{}%
\AgdaFunction{Sorts}\AgdaSpace{}%
\AgdaSymbol{→}\AgdaSpace{}%
\AgdaPrimitive{Set}\AgdaSpace{}%
\AgdaKeyword{where}\<%
\\
\>[0][@{}l@{\AgdaIndent{0}}]%
\>[2]\AgdaInductiveConstructor{∅}%
\>[6]\AgdaSymbol{:}\AgdaSpace{}%
\AgdaDatatype{Ctx}\AgdaSpace{}%
\AgdaInductiveConstructor{[]}\<%
\\
%
\>[2]\AgdaOperator{\AgdaInductiveConstructor{\AgdaUnderscore{}▶\AgdaUnderscore{}}}\AgdaSpace{}%
\AgdaSymbol{:}\AgdaSpace{}%
\AgdaDatatype{Ctx}\AgdaSpace{}%
\AgdaGeneralizable{S}\AgdaSpace{}%
\AgdaSymbol{→}\AgdaSpace{}%
\AgdaDatatype{Term}\AgdaSpace{}%
\AgdaGeneralizable{S}\AgdaSpace{}%
\AgdaSymbol{(}\AgdaFunction{type-of}\AgdaSpace{}%
\AgdaGeneralizable{s}\AgdaSymbol{)}\AgdaSpace{}%
\AgdaSymbol{→}\AgdaSpace{}%
\AgdaDatatype{Ctx}\AgdaSpace{}%
\AgdaSymbol{(}\AgdaGeneralizable{S}\AgdaSpace{}%
\AgdaOperator{\AgdaInductiveConstructor{▷}}\AgdaSpace{}%
\AgdaGeneralizable{s}\AgdaSymbol{)}\<%
\end{code}}
\newcommand{\Flookup}[0]{\begin{code}[inline]%
\>[0]\AgdaFunction{lookup}\AgdaSpace{}%
\AgdaSymbol{:}\AgdaSpace{}%
\AgdaDatatype{Ctx}\AgdaSpace{}%
\AgdaGeneralizable{S}\AgdaSpace{}%
\AgdaSymbol{→}\AgdaSpace{}%
\AgdaFunction{Var}\AgdaSpace{}%
\AgdaGeneralizable{S}\AgdaSpace{}%
\AgdaGeneralizable{s}\AgdaSpace{}%
\AgdaSymbol{→}\AgdaSpace{}%
\AgdaDatatype{Term}\AgdaSpace{}%
\AgdaGeneralizable{S}\AgdaSpace{}%
\AgdaSymbol{(}\AgdaFunction{type-of}\AgdaSpace{}%
\AgdaGeneralizable{s}\AgdaSymbol{)}\<%
\end{code}}
\begin{code}[hide]%
\>[0]\AgdaFunction{lookup}\AgdaSpace{}%
\AgdaSymbol{(}\AgdaBound{Γ}\AgdaSpace{}%
\AgdaOperator{\AgdaInductiveConstructor{▶}}\AgdaSpace{}%
\AgdaBound{T}\AgdaSymbol{)}\AgdaSpace{}%
\AgdaSymbol{(}\AgdaInductiveConstructor{here}\AgdaSpace{}%
\AgdaInductiveConstructor{refl}\AgdaSymbol{)}\AgdaSpace{}%
\AgdaSymbol{=}\AgdaSpace{}%
\AgdaFunction{wk}\AgdaSpace{}%
\AgdaBound{T}\<%
\\
\>[0]\AgdaFunction{lookup}\AgdaSpace{}%
\AgdaSymbol{(}\AgdaBound{Γ}\AgdaSpace{}%
\AgdaOperator{\AgdaInductiveConstructor{▶}}\AgdaSpace{}%
\AgdaBound{T}\AgdaSymbol{)}\AgdaSpace{}%
\AgdaSymbol{(}\AgdaInductiveConstructor{there}\AgdaSpace{}%
\AgdaBound{x}\AgdaSymbol{)}\AgdaSpace{}%
\AgdaSymbol{=}\AgdaSpace{}%
\AgdaFunction{wk}\AgdaSpace{}%
\AgdaSymbol{(}\AgdaFunction{lookup}\AgdaSpace{}%
\AgdaBound{Γ}\AgdaSpace{}%
\AgdaBound{x}\AgdaSymbol{)}\<%
\end{code}
\begin{code}[hide]%
\>[0]\AgdaKeyword{variable}\<%
\\
\>[0][@{}l@{\AgdaIndent{0}}]%
\>[2]\AgdaGeneralizable{Γ}\AgdaSpace{}%
\AgdaGeneralizable{Γ'}\AgdaSpace{}%
\AgdaGeneralizable{Γ''}\AgdaSpace{}%
\AgdaGeneralizable{Γ₁}\AgdaSpace{}%
\AgdaGeneralizable{Γ₂}\AgdaSpace{}%
\AgdaSymbol{:}\AgdaSpace{}%
\AgdaDatatype{Ctx}\AgdaSpace{}%
\AgdaGeneralizable{S}\<%
\\
%
\\[\AgdaEmptyExtraSkip]%
\>[0]\AgdaComment{--\ Typing\ -------------------------------------------------------------------------------}\<%
\\
%
\\[\AgdaEmptyExtraSkip]%
\>[0]\AgdaComment{--\ Expression\ Typing}\<%
\\
%
\\[\AgdaEmptyExtraSkip]%
\>[0]\AgdaKeyword{infix}\AgdaSpace{}%
\AgdaNumber{3}\AgdaSpace{}%
\AgdaOperator{\AgdaDatatype{\AgdaUnderscore{}⊢\AgdaUnderscore{}∶\AgdaUnderscore{}}}\<%
\end{code}
\newcommand{\FTyping}[0]{\begin{code}%
\>[0]\AgdaKeyword{data}\AgdaSpace{}%
\AgdaOperator{\AgdaDatatype{\AgdaUnderscore{}⊢\AgdaUnderscore{}∶\AgdaUnderscore{}}}\AgdaSpace{}%
\AgdaSymbol{:}\AgdaSpace{}%
\AgdaDatatype{Ctx}\AgdaSpace{}%
\AgdaGeneralizable{S}\AgdaSpace{}%
\AgdaSymbol{→}\AgdaSpace{}%
\AgdaDatatype{Term}\AgdaSpace{}%
\AgdaGeneralizable{S}\AgdaSpace{}%
\AgdaGeneralizable{s}\AgdaSpace{}%
\AgdaSymbol{→}\AgdaSpace{}%
\AgdaDatatype{Term}\AgdaSpace{}%
\AgdaGeneralizable{S}\AgdaSpace{}%
\AgdaSymbol{(}\AgdaFunction{type-of}\AgdaSpace{}%
\AgdaGeneralizable{s}\AgdaSymbol{)}\AgdaSpace{}%
\AgdaSymbol{→}\AgdaSpace{}%
\AgdaPrimitive{Set}\AgdaSpace{}%
\AgdaKeyword{where}\<%
\\
\>[0][@{}l@{\AgdaIndent{0}}]%
\>[2]\AgdaInductiveConstructor{⊢`x}\AgdaSpace{}%
\AgdaSymbol{:}\<%
\\
\>[2][@{}l@{\AgdaIndent{0}}]%
\>[4]\AgdaFunction{lookup}\AgdaSpace{}%
\AgdaGeneralizable{Γ}\AgdaSpace{}%
\AgdaGeneralizable{x}\AgdaSpace{}%
\AgdaOperator{\AgdaDatatype{≡}}\AgdaSpace{}%
\AgdaGeneralizable{τ}\AgdaSpace{}%
\AgdaSymbol{→}\<%
\\
%
\>[4]\AgdaGeneralizable{Γ}\AgdaSpace{}%
\AgdaOperator{\AgdaDatatype{⊢}}\AgdaSpace{}%
\AgdaOperator{\AgdaInductiveConstructor{`}}\AgdaSpace{}%
\AgdaGeneralizable{x}\AgdaSpace{}%
\AgdaOperator{\AgdaDatatype{∶}}\AgdaSpace{}%
\AgdaGeneralizable{τ}\<%
\\
%
\>[2]\AgdaInductiveConstructor{⊢⊤}\AgdaSpace{}%
\AgdaSymbol{:}\<%
\\
\>[2][@{}l@{\AgdaIndent{0}}]%
\>[4]\AgdaGeneralizable{Γ}\AgdaSpace{}%
\AgdaOperator{\AgdaDatatype{⊢}}\AgdaSpace{}%
\AgdaInductiveConstructor{tt}\AgdaSpace{}%
\AgdaOperator{\AgdaDatatype{∶}}\AgdaSpace{}%
\AgdaInductiveConstructor{`⊤}\<%
\\
%
\>[2]\AgdaInductiveConstructor{⊢λ}\AgdaSpace{}%
\AgdaSymbol{:}\<%
\\
\>[2][@{}l@{\AgdaIndent{0}}]%
\>[4]\AgdaGeneralizable{Γ}\AgdaSpace{}%
\AgdaOperator{\AgdaInductiveConstructor{▶}}\AgdaSpace{}%
\AgdaGeneralizable{τ}\AgdaSpace{}%
\AgdaOperator{\AgdaDatatype{⊢}}\AgdaSpace{}%
\AgdaGeneralizable{e}\AgdaSpace{}%
\AgdaOperator{\AgdaDatatype{∶}}\AgdaSpace{}%
\AgdaFunction{wk}\AgdaSpace{}%
\AgdaGeneralizable{τ'}\AgdaSpace{}%
\AgdaSymbol{→}\<%
\\
%
\>[4]\AgdaGeneralizable{Γ}\AgdaSpace{}%
\AgdaOperator{\AgdaDatatype{⊢}}\AgdaSpace{}%
\AgdaOperator{\AgdaInductiveConstructor{λ`x→}}\AgdaSpace{}%
\AgdaGeneralizable{e}\AgdaSpace{}%
\AgdaOperator{\AgdaDatatype{∶}}\AgdaSpace{}%
\AgdaGeneralizable{τ}\AgdaSpace{}%
\AgdaOperator{\AgdaInductiveConstructor{⇒}}\AgdaSpace{}%
\AgdaGeneralizable{τ'}\<%
\\
%
\>[2]\AgdaInductiveConstructor{⊢Λ}\AgdaSpace{}%
\AgdaSymbol{:}\<%
\\
\>[2][@{}l@{\AgdaIndent{0}}]%
\>[4]\AgdaGeneralizable{Γ}\AgdaSpace{}%
\AgdaOperator{\AgdaInductiveConstructor{▶}}\AgdaSpace{}%
\AgdaInductiveConstructor{⋆}\AgdaSpace{}%
\AgdaOperator{\AgdaDatatype{⊢}}\AgdaSpace{}%
\AgdaGeneralizable{e}\AgdaSpace{}%
\AgdaOperator{\AgdaDatatype{∶}}\AgdaSpace{}%
\AgdaGeneralizable{τ}\AgdaSpace{}%
\AgdaSymbol{→}\<%
\\
%
\>[4]\AgdaGeneralizable{Γ}\AgdaSpace{}%
\AgdaOperator{\AgdaDatatype{⊢}}\AgdaSpace{}%
\AgdaOperator{\AgdaInductiveConstructor{Λ`α→}}\AgdaSpace{}%
\AgdaGeneralizable{e}\AgdaSpace{}%
\AgdaOperator{\AgdaDatatype{∶}}\AgdaSpace{}%
\AgdaOperator{\AgdaInductiveConstructor{∀`α}}\AgdaSpace{}%
\AgdaGeneralizable{τ}\<%
\\
%
\>[2]\AgdaInductiveConstructor{⊢·}\AgdaSpace{}%
\AgdaSymbol{:}\<%
\\
\>[2][@{}l@{\AgdaIndent{0}}]%
\>[4]\AgdaGeneralizable{Γ}\AgdaSpace{}%
\AgdaOperator{\AgdaDatatype{⊢}}\AgdaSpace{}%
\AgdaGeneralizable{e₁}\AgdaSpace{}%
\AgdaOperator{\AgdaDatatype{∶}}\AgdaSpace{}%
\AgdaGeneralizable{τ₁}\AgdaSpace{}%
\AgdaOperator{\AgdaInductiveConstructor{⇒}}\AgdaSpace{}%
\AgdaGeneralizable{τ₂}\AgdaSpace{}%
\AgdaSymbol{→}\<%
\\
%
\>[4]\AgdaGeneralizable{Γ}\AgdaSpace{}%
\AgdaOperator{\AgdaDatatype{⊢}}\AgdaSpace{}%
\AgdaGeneralizable{e₂}\AgdaSpace{}%
\AgdaOperator{\AgdaDatatype{∶}}\AgdaSpace{}%
\AgdaGeneralizable{τ₁}\AgdaSpace{}%
\AgdaSymbol{→}\<%
\\
%
\>[4]\AgdaGeneralizable{Γ}\AgdaSpace{}%
\AgdaOperator{\AgdaDatatype{⊢}}\AgdaSpace{}%
\AgdaGeneralizable{e₁}\AgdaSpace{}%
\AgdaOperator{\AgdaInductiveConstructor{·}}\AgdaSpace{}%
\AgdaGeneralizable{e₂}\AgdaSpace{}%
\AgdaOperator{\AgdaDatatype{∶}}\AgdaSpace{}%
\AgdaGeneralizable{τ₂}\<%
\\
%
\>[2]\AgdaInductiveConstructor{⊢•}\AgdaSpace{}%
\AgdaSymbol{:}\<%
\\
\>[2][@{}l@{\AgdaIndent{0}}]%
\>[4]\AgdaGeneralizable{Γ}\AgdaSpace{}%
\AgdaOperator{\AgdaDatatype{⊢}}\AgdaSpace{}%
\AgdaGeneralizable{e}\AgdaSpace{}%
\AgdaOperator{\AgdaDatatype{∶}}\AgdaSpace{}%
\AgdaOperator{\AgdaInductiveConstructor{∀`α}}\AgdaSpace{}%
\AgdaGeneralizable{τ'}\AgdaSpace{}%
\AgdaSymbol{→}\<%
\\
%
\>[4]\AgdaGeneralizable{Γ}\AgdaSpace{}%
\AgdaOperator{\AgdaDatatype{⊢}}\AgdaSpace{}%
\AgdaGeneralizable{e}\AgdaSpace{}%
\AgdaOperator{\AgdaInductiveConstructor{•}}\AgdaSpace{}%
\AgdaGeneralizable{τ}\AgdaSpace{}%
\AgdaOperator{\AgdaDatatype{∶}}\AgdaSpace{}%
\AgdaGeneralizable{τ'}\AgdaSpace{}%
\AgdaOperator{\AgdaFunction{[}}\AgdaSpace{}%
\AgdaGeneralizable{τ}\AgdaSpace{}%
\AgdaOperator{\AgdaFunction{]}}\<%
\\
%
\>[2]\AgdaInductiveConstructor{⊢let}\AgdaSpace{}%
\AgdaSymbol{:}\<%
\\
\>[2][@{}l@{\AgdaIndent{0}}]%
\>[4]\AgdaGeneralizable{Γ}\AgdaSpace{}%
\AgdaOperator{\AgdaDatatype{⊢}}\AgdaSpace{}%
\AgdaGeneralizable{e₂}\AgdaSpace{}%
\AgdaOperator{\AgdaDatatype{∶}}\AgdaSpace{}%
\AgdaGeneralizable{τ}\AgdaSpace{}%
\AgdaSymbol{→}\<%
\\
%
\>[4]\AgdaGeneralizable{Γ}\AgdaSpace{}%
\AgdaOperator{\AgdaInductiveConstructor{▶}}\AgdaSpace{}%
\AgdaGeneralizable{τ}\AgdaSpace{}%
\AgdaOperator{\AgdaDatatype{⊢}}\AgdaSpace{}%
\AgdaGeneralizable{e₁}\AgdaSpace{}%
\AgdaOperator{\AgdaDatatype{∶}}\AgdaSpace{}%
\AgdaFunction{wk}\AgdaSpace{}%
\AgdaGeneralizable{τ'}\AgdaSpace{}%
\AgdaSymbol{→}\<%
\\
%
\>[4]\AgdaGeneralizable{Γ}\AgdaSpace{}%
\AgdaOperator{\AgdaDatatype{⊢}}\AgdaSpace{}%
\AgdaOperator{\AgdaInductiveConstructor{let`x=}}\AgdaSpace{}%
\AgdaGeneralizable{e₂}\AgdaSpace{}%
\AgdaOperator{\AgdaInductiveConstructor{`in}}\AgdaSpace{}%
\AgdaGeneralizable{e₁}\AgdaSpace{}%
\AgdaOperator{\AgdaDatatype{∶}}\AgdaSpace{}%
\AgdaGeneralizable{τ'}\<%
\\
%
\>[2]\AgdaInductiveConstructor{⊢τ}\AgdaSpace{}%
\AgdaSymbol{:}\<%
\\
\>[2][@{}l@{\AgdaIndent{0}}]%
\>[4]\AgdaGeneralizable{Γ}\AgdaSpace{}%
\AgdaOperator{\AgdaDatatype{⊢}}\AgdaSpace{}%
\AgdaGeneralizable{τ}\AgdaSpace{}%
\AgdaOperator{\AgdaDatatype{∶}}\AgdaSpace{}%
\AgdaInductiveConstructor{⋆}\<%
\end{code}}
\begin{code}[hide]%
\>[0]\AgdaComment{--\ Renaming\ Typing}\<%
\\
%
\\[\AgdaEmptyExtraSkip]%
\>[0]\AgdaKeyword{infix}\AgdaSpace{}%
\AgdaNumber{3}\AgdaSpace{}%
\AgdaOperator{\AgdaDatatype{\AgdaUnderscore{}∶\AgdaUnderscore{}⇒ᵣ\AgdaUnderscore{}}}\<%
\end{code}
\newcommand{\FRenTyping}[0]{\begin{code}%
\>[0]\AgdaKeyword{data}\AgdaSpace{}%
\AgdaOperator{\AgdaDatatype{\AgdaUnderscore{}∶\AgdaUnderscore{}⇒ᵣ\AgdaUnderscore{}}}\AgdaSpace{}%
\AgdaSymbol{:}\AgdaSpace{}%
\AgdaFunction{Ren}\AgdaSpace{}%
\AgdaGeneralizable{S₁}\AgdaSpace{}%
\AgdaGeneralizable{S₂}\AgdaSpace{}%
\AgdaSymbol{→}\AgdaSpace{}%
\AgdaDatatype{Ctx}\AgdaSpace{}%
\AgdaGeneralizable{S₁}\AgdaSpace{}%
\AgdaSymbol{→}\AgdaSpace{}%
\AgdaDatatype{Ctx}\AgdaSpace{}%
\AgdaGeneralizable{S₂}\AgdaSpace{}%
\AgdaSymbol{→}\AgdaSpace{}%
\AgdaPrimitive{Set}\AgdaSpace{}%
\AgdaKeyword{where}\<%
\\
\>[0][@{}l@{\AgdaIndent{0}}]%
\>[2]\AgdaInductiveConstructor{⊢idᵣ}\AgdaSpace{}%
\AgdaSymbol{:}\AgdaSpace{}%
\AgdaSymbol{∀}\AgdaSpace{}%
\AgdaSymbol{\{}\AgdaBound{Γ}\AgdaSymbol{\}}\AgdaSpace{}%
\AgdaSymbol{→}\AgdaSpace{}%
\AgdaOperator{\AgdaDatatype{\AgdaUnderscore{}∶\AgdaUnderscore{}⇒ᵣ\AgdaUnderscore{}}}\AgdaSpace{}%
\AgdaSymbol{\{}\AgdaArgument{S₁}\AgdaSpace{}%
\AgdaSymbol{=}\AgdaSpace{}%
\AgdaGeneralizable{S}\AgdaSymbol{\}}\AgdaSpace{}%
\AgdaSymbol{\{}\AgdaArgument{S₂}\AgdaSpace{}%
\AgdaSymbol{=}\AgdaSpace{}%
\AgdaGeneralizable{S}\AgdaSymbol{\}}\AgdaSpace{}%
\AgdaFunction{idᵣ}\AgdaSpace{}%
\AgdaBound{Γ}\AgdaSpace{}%
\AgdaBound{Γ}\<%
\\
%
\>[2]\AgdaInductiveConstructor{⊢extᵣ}\AgdaSpace{}%
\AgdaSymbol{:}\AgdaSpace{}%
\AgdaSymbol{∀}\AgdaSpace{}%
\AgdaSymbol{\{}\AgdaBound{ρ}\AgdaSpace{}%
\AgdaSymbol{:}\AgdaSpace{}%
\AgdaFunction{Ren}\AgdaSpace{}%
\AgdaGeneralizable{S₁}\AgdaSpace{}%
\AgdaGeneralizable{S₂}\AgdaSymbol{\}}\AgdaSpace{}%
\AgdaSymbol{\{}\AgdaBound{Γ₁}\AgdaSpace{}%
\AgdaSymbol{:}\AgdaSpace{}%
\AgdaDatatype{Ctx}\AgdaSpace{}%
\AgdaGeneralizable{S₁}\AgdaSymbol{\}}\AgdaSpace{}%
\AgdaSymbol{\{}\AgdaBound{Γ₂}\AgdaSpace{}%
\AgdaSymbol{:}\AgdaSpace{}%
\AgdaDatatype{Ctx}\AgdaSpace{}%
\AgdaGeneralizable{S₂}\AgdaSymbol{\}}\AgdaSpace{}%
\AgdaSymbol{\{}\AgdaBound{T'}\AgdaSpace{}%
\AgdaSymbol{:}\AgdaSpace{}%
\AgdaDatatype{Term}\AgdaSpace{}%
\AgdaGeneralizable{S₁}\AgdaSpace{}%
\AgdaSymbol{(}\AgdaFunction{type-of}\AgdaSpace{}%
\AgdaGeneralizable{s}\AgdaSymbol{)\}}\AgdaSpace{}%
\AgdaSymbol{→}\<%
\\
\>[2][@{}l@{\AgdaIndent{0}}]%
\>[4]\AgdaBound{ρ}\AgdaSpace{}%
\AgdaOperator{\AgdaDatatype{∶}}\AgdaSpace{}%
\AgdaBound{Γ₁}\AgdaSpace{}%
\AgdaOperator{\AgdaDatatype{⇒ᵣ}}\AgdaSpace{}%
\AgdaBound{Γ₂}\AgdaSpace{}%
\AgdaSymbol{→}\<%
\\
%
\>[4]\AgdaSymbol{(}\AgdaFunction{extᵣ}\AgdaSpace{}%
\AgdaBound{ρ}\AgdaSymbol{)}\AgdaSpace{}%
\AgdaOperator{\AgdaDatatype{∶}}\AgdaSpace{}%
\AgdaSymbol{(}\AgdaBound{Γ₁}\AgdaSpace{}%
\AgdaOperator{\AgdaInductiveConstructor{▶}}\AgdaSpace{}%
\AgdaBound{T'}\AgdaSymbol{)}\AgdaSpace{}%
\AgdaOperator{\AgdaDatatype{⇒ᵣ}}\AgdaSpace{}%
\AgdaSymbol{(}\AgdaBound{Γ₂}\AgdaSpace{}%
\AgdaOperator{\AgdaInductiveConstructor{▶}}\AgdaSpace{}%
\AgdaFunction{ren}\AgdaSpace{}%
\AgdaBound{ρ}\AgdaSpace{}%
\AgdaBound{T'}\AgdaSymbol{)}\<%
\\
%
\>[2]\AgdaInductiveConstructor{⊢dropᵣ}\AgdaSpace{}%
\AgdaSymbol{:}\AgdaSpace{}%
\AgdaSymbol{∀}\AgdaSpace{}%
\AgdaSymbol{\{}\AgdaBound{ρ}\AgdaSpace{}%
\AgdaSymbol{:}\AgdaSpace{}%
\AgdaFunction{Ren}\AgdaSpace{}%
\AgdaGeneralizable{S₁}\AgdaSpace{}%
\AgdaGeneralizable{S₂}\AgdaSymbol{\}}\AgdaSpace{}%
\AgdaSymbol{\{}\AgdaBound{Γ₁}\AgdaSpace{}%
\AgdaSymbol{:}\AgdaSpace{}%
\AgdaDatatype{Ctx}\AgdaSpace{}%
\AgdaGeneralizable{S₁}\AgdaSymbol{\}}\AgdaSpace{}%
\AgdaSymbol{\{}\AgdaBound{Γ₂}\AgdaSpace{}%
\AgdaSymbol{:}\AgdaSpace{}%
\AgdaDatatype{Ctx}\AgdaSpace{}%
\AgdaGeneralizable{S₂}\AgdaSymbol{\}}\AgdaSpace{}%
\AgdaSymbol{\{}\AgdaBound{T'}\AgdaSpace{}%
\AgdaSymbol{:}\AgdaSpace{}%
\AgdaDatatype{Term}\AgdaSpace{}%
\AgdaGeneralizable{S₂}\AgdaSpace{}%
\AgdaSymbol{(}\AgdaFunction{type-of}\AgdaSpace{}%
\AgdaGeneralizable{s}\AgdaSymbol{)\}}\AgdaSpace{}%
\AgdaSymbol{→}\<%
\\
\>[2][@{}l@{\AgdaIndent{0}}]%
\>[4]\AgdaBound{ρ}\AgdaSpace{}%
\AgdaOperator{\AgdaDatatype{∶}}\AgdaSpace{}%
\AgdaBound{Γ₁}%
\>[12]\AgdaOperator{\AgdaDatatype{⇒ᵣ}}\AgdaSpace{}%
\AgdaBound{Γ₂}\AgdaSpace{}%
\AgdaSymbol{→}\<%
\\
%
\>[4]\AgdaSymbol{(}\AgdaFunction{dropᵣ}\AgdaSpace{}%
\AgdaBound{ρ}\AgdaSymbol{)}\AgdaSpace{}%
\AgdaOperator{\AgdaDatatype{∶}}\AgdaSpace{}%
\AgdaBound{Γ₁}\AgdaSpace{}%
\AgdaOperator{\AgdaDatatype{⇒ᵣ}}\AgdaSpace{}%
\AgdaSymbol{(}\AgdaBound{Γ₂}\AgdaSpace{}%
\AgdaOperator{\AgdaInductiveConstructor{▶}}\AgdaSpace{}%
\AgdaBound{T'}\AgdaSymbol{)}\<%
\end{code}}
\begin{code}[hide]%
\>[0]\AgdaFunction{⊢wkᵣ}\AgdaSpace{}%
\AgdaSymbol{:}\AgdaSpace{}%
\AgdaSymbol{∀}\AgdaSpace{}%
\AgdaSymbol{\{}\AgdaBound{T}\AgdaSpace{}%
\AgdaSymbol{:}\AgdaSpace{}%
\AgdaDatatype{Term}\AgdaSpace{}%
\AgdaGeneralizable{S}\AgdaSpace{}%
\AgdaSymbol{(}\AgdaFunction{type-of}\AgdaSpace{}%
\AgdaGeneralizable{s}\AgdaSymbol{)\}}\AgdaSpace{}%
\AgdaSymbol{→}\AgdaSpace{}%
\AgdaSymbol{(}\AgdaFunction{dropᵣ}\AgdaSpace{}%
\AgdaFunction{idᵣ}\AgdaSymbol{)}\AgdaSpace{}%
\AgdaOperator{\AgdaDatatype{∶}}\AgdaSpace{}%
\AgdaGeneralizable{Γ}\AgdaSpace{}%
\AgdaOperator{\AgdaDatatype{⇒ᵣ}}\AgdaSpace{}%
\AgdaSymbol{(}\AgdaGeneralizable{Γ}\AgdaSpace{}%
\AgdaOperator{\AgdaInductiveConstructor{▶}}\AgdaSpace{}%
\AgdaBound{T}\AgdaSymbol{)}\<%
\\
\>[0]\AgdaFunction{⊢wkᵣ}\AgdaSpace{}%
\AgdaSymbol{=}\AgdaSpace{}%
\AgdaInductiveConstructor{⊢dropᵣ}\AgdaSpace{}%
\AgdaInductiveConstructor{⊢idᵣ}\<%
\end{code}
\newcommand{\FSubTyping}[0]{\begin{code}%
\>[0]\AgdaOperator{\AgdaFunction{\AgdaUnderscore{}∶\AgdaUnderscore{}⇒ₛ\AgdaUnderscore{}}}\AgdaSpace{}%
\AgdaSymbol{:}\AgdaSpace{}%
\AgdaFunction{Sub}\AgdaSpace{}%
\AgdaGeneralizable{S₁}\AgdaSpace{}%
\AgdaGeneralizable{S₂}\AgdaSpace{}%
\AgdaSymbol{→}\AgdaSpace{}%
\AgdaDatatype{Ctx}\AgdaSpace{}%
\AgdaGeneralizable{S₁}\AgdaSpace{}%
\AgdaSymbol{→}\AgdaSpace{}%
\AgdaDatatype{Ctx}\AgdaSpace{}%
\AgdaGeneralizable{S₂}\AgdaSpace{}%
\AgdaSymbol{→}\AgdaSpace{}%
\AgdaPrimitive{Set}\<%
\\
\>[0]\AgdaOperator{\AgdaFunction{\AgdaUnderscore{}∶\AgdaUnderscore{}⇒ₛ\AgdaUnderscore{}}}\AgdaSpace{}%
\AgdaSymbol{\{}\AgdaArgument{S₁}\AgdaSpace{}%
\AgdaSymbol{=}\AgdaSpace{}%
\AgdaBound{S₁}\AgdaSymbol{\}}\AgdaSpace{}%
\AgdaBound{σ}\AgdaSpace{}%
\AgdaBound{Γ₁}\AgdaSpace{}%
\AgdaBound{Γ₂}\AgdaSpace{}%
\AgdaSymbol{=}\AgdaSpace{}%
\AgdaSymbol{∀}\AgdaSpace{}%
\AgdaSymbol{\{}\AgdaBound{s}\AgdaSymbol{\}}\AgdaSpace{}%
\AgdaSymbol{(}\AgdaBound{x}\AgdaSpace{}%
\AgdaSymbol{:}\AgdaSpace{}%
\AgdaFunction{Var}\AgdaSpace{}%
\AgdaBound{S₁}\AgdaSpace{}%
\AgdaBound{s}\AgdaSymbol{)}\AgdaSpace{}%
\AgdaSymbol{→}\AgdaSpace{}%
\AgdaBound{Γ₂}\AgdaSpace{}%
\AgdaOperator{\AgdaDatatype{⊢}}\AgdaSpace{}%
\AgdaBound{σ}\AgdaSpace{}%
\AgdaBound{x}\AgdaSpace{}%
\AgdaOperator{\AgdaDatatype{∶}}\AgdaSpace{}%
\AgdaSymbol{(}\AgdaFunction{sub}\AgdaSpace{}%
\AgdaBound{σ}\AgdaSpace{}%
\AgdaSymbol{(}\AgdaFunction{lookup}\AgdaSpace{}%
\AgdaBound{Γ₁}\AgdaSpace{}%
\AgdaBound{x}\AgdaSymbol{))}\<%
\end{code}}
\begin{code}[hide]%
\>[0]\AgdaFunction{extₛidₛ≡idₛ}\AgdaSpace{}%
\AgdaSymbol{:}\AgdaSpace{}%
\AgdaSymbol{∀}\AgdaSpace{}%
\AgdaSymbol{(}\AgdaBound{x}\AgdaSpace{}%
\AgdaSymbol{:}\AgdaSpace{}%
\AgdaFunction{Var}\AgdaSpace{}%
\AgdaSymbol{(}\AgdaGeneralizable{S}\AgdaSpace{}%
\AgdaOperator{\AgdaInductiveConstructor{▷}}\AgdaSpace{}%
\AgdaGeneralizable{s'}\AgdaSymbol{)}\AgdaSpace{}%
\AgdaGeneralizable{s}\AgdaSymbol{)}\AgdaSpace{}%
\AgdaSymbol{→}\AgdaSpace{}%
\AgdaFunction{extₛ}\AgdaSpace{}%
\AgdaFunction{idₛ}\AgdaSpace{}%
\AgdaBound{x}\AgdaSpace{}%
\AgdaOperator{\AgdaDatatype{≡}}\AgdaSpace{}%
\AgdaFunction{idₛ}\AgdaSpace{}%
\AgdaBound{x}\<%
\\
\>[0]\AgdaFunction{extₛidₛ≡idₛ}\AgdaSpace{}%
\AgdaSymbol{(}\AgdaInductiveConstructor{here}\AgdaSpace{}%
\AgdaInductiveConstructor{refl}\AgdaSymbol{)}\AgdaSpace{}%
\AgdaSymbol{=}\AgdaSpace{}%
\AgdaInductiveConstructor{refl}\<%
\\
\>[0]\AgdaFunction{extₛidₛ≡idₛ}\AgdaSpace{}%
\AgdaSymbol{(}\AgdaInductiveConstructor{there}\AgdaSpace{}%
\AgdaBound{x}\AgdaSymbol{)}\AgdaSpace{}%
\AgdaSymbol{=}\AgdaSpace{}%
\AgdaInductiveConstructor{refl}\<%
\\
%
\\[\AgdaEmptyExtraSkip]%
\>[0]\AgdaFunction{⊢ext-σ₁≡ext-σ₂}\AgdaSpace{}%
\AgdaSymbol{:}\AgdaSpace{}%
\AgdaSymbol{∀}\AgdaSpace{}%
\AgdaSymbol{\{}\AgdaBound{σ₁}\AgdaSpace{}%
\AgdaBound{σ₂}\AgdaSpace{}%
\AgdaSymbol{:}\AgdaSpace{}%
\AgdaFunction{Sub}\AgdaSpace{}%
\AgdaGeneralizable{S₁}\AgdaSpace{}%
\AgdaGeneralizable{S₂}\AgdaSymbol{\}}\AgdaSpace{}%
\AgdaSymbol{→}\<%
\\
\>[0][@{}l@{\AgdaIndent{0}}]%
\>[1]\AgdaSymbol{(∀}\AgdaSpace{}%
\AgdaSymbol{\{}\AgdaBound{s}\AgdaSymbol{\}}\AgdaSpace{}%
\AgdaSymbol{(}\AgdaBound{x}\AgdaSpace{}%
\AgdaSymbol{:}\AgdaSpace{}%
\AgdaFunction{Var}\AgdaSpace{}%
\AgdaGeneralizable{S₁}\AgdaSpace{}%
\AgdaBound{s}\AgdaSymbol{)}\AgdaSpace{}%
\AgdaSymbol{→}\AgdaSpace{}%
\AgdaBound{σ₁}\AgdaSpace{}%
\AgdaBound{x}\AgdaSpace{}%
\AgdaOperator{\AgdaDatatype{≡}}\AgdaSpace{}%
\AgdaBound{σ₂}\AgdaSpace{}%
\AgdaBound{x}\AgdaSymbol{)}\AgdaSpace{}%
\AgdaSymbol{→}\<%
\\
%
\>[1]\AgdaSymbol{(∀}\AgdaSpace{}%
\AgdaSymbol{\{}\AgdaBound{s}\AgdaSymbol{\}}\AgdaSpace{}%
\AgdaSymbol{(}\AgdaBound{x}\AgdaSpace{}%
\AgdaSymbol{:}\AgdaSpace{}%
\AgdaFunction{Var}\AgdaSpace{}%
\AgdaSymbol{(}\AgdaGeneralizable{S₁}\AgdaSpace{}%
\AgdaOperator{\AgdaInductiveConstructor{▷}}\AgdaSpace{}%
\AgdaGeneralizable{s'}\AgdaSymbol{)}\AgdaSpace{}%
\AgdaBound{s}\AgdaSymbol{)}\AgdaSpace{}%
\AgdaSymbol{→}\AgdaSpace{}%
\AgdaSymbol{(}\AgdaFunction{extₛ}\AgdaSpace{}%
\AgdaBound{σ₁}\AgdaSymbol{)}\AgdaSpace{}%
\AgdaBound{x}\AgdaSpace{}%
\AgdaOperator{\AgdaDatatype{≡}}\AgdaSpace{}%
\AgdaSymbol{(}\AgdaFunction{extₛ}\AgdaSpace{}%
\AgdaBound{σ₂}\AgdaSymbol{)}\AgdaSpace{}%
\AgdaBound{x}\AgdaSymbol{)}\<%
\\
\>[0]\AgdaFunction{⊢ext-σ₁≡ext-σ₂}\AgdaSpace{}%
\AgdaBound{σ₁≡σ₂}\AgdaSpace{}%
\AgdaSymbol{(}\AgdaInductiveConstructor{here}\AgdaSpace{}%
\AgdaInductiveConstructor{refl}\AgdaSymbol{)}\AgdaSpace{}%
\AgdaSymbol{=}\AgdaSpace{}%
\AgdaInductiveConstructor{refl}\<%
\\
\>[0]\AgdaFunction{⊢ext-σ₁≡ext-σ₂}\AgdaSpace{}%
\AgdaBound{σ₁≡σ₂}\AgdaSpace{}%
\AgdaSymbol{(}\AgdaInductiveConstructor{there}\AgdaSpace{}%
\AgdaBound{x}\AgdaSymbol{)}\AgdaSpace{}%
\AgdaSymbol{=}\AgdaSpace{}%
\AgdaFunction{cong}\AgdaSpace{}%
\AgdaFunction{wk}\AgdaSpace{}%
\AgdaSymbol{(}\AgdaBound{σ₁≡σ₂}\AgdaSpace{}%
\AgdaBound{x}\AgdaSymbol{)}\<%
\\
%
\\[\AgdaEmptyExtraSkip]%
\>[0]\AgdaFunction{σ₁≡σ₂→σ₁τ≡σ₂τ}\AgdaSpace{}%
\AgdaSymbol{:}\AgdaSpace{}%
\AgdaSymbol{∀}\AgdaSpace{}%
\AgdaSymbol{\{}\AgdaBound{σ₁}\AgdaSpace{}%
\AgdaBound{σ₂}\AgdaSpace{}%
\AgdaSymbol{:}\AgdaSpace{}%
\AgdaFunction{Sub}\AgdaSpace{}%
\AgdaGeneralizable{S₁}\AgdaSpace{}%
\AgdaGeneralizable{S₂}\AgdaSymbol{\}}\AgdaSpace{}%
\AgdaSymbol{(}\AgdaBound{τ}\AgdaSpace{}%
\AgdaSymbol{:}\AgdaSpace{}%
\AgdaFunction{Type}\AgdaSpace{}%
\AgdaGeneralizable{S₁}\AgdaSymbol{)}\AgdaSpace{}%
\AgdaSymbol{→}\<%
\\
\>[0][@{}l@{\AgdaIndent{0}}]%
\>[2]\AgdaSymbol{(∀}\AgdaSpace{}%
\AgdaSymbol{\{}\AgdaBound{s}\AgdaSymbol{\}}\AgdaSpace{}%
\AgdaSymbol{(}\AgdaBound{x}\AgdaSpace{}%
\AgdaSymbol{:}\AgdaSpace{}%
\AgdaFunction{Var}\AgdaSpace{}%
\AgdaGeneralizable{S₁}\AgdaSpace{}%
\AgdaBound{s}\AgdaSymbol{)}\AgdaSpace{}%
\AgdaSymbol{→}\AgdaSpace{}%
\AgdaBound{σ₁}\AgdaSpace{}%
\AgdaBound{x}\AgdaSpace{}%
\AgdaOperator{\AgdaDatatype{≡}}\AgdaSpace{}%
\AgdaBound{σ₂}\AgdaSpace{}%
\AgdaBound{x}\AgdaSymbol{)}\AgdaSpace{}%
\AgdaSymbol{→}\<%
\\
%
\>[2]\AgdaFunction{sub}\AgdaSpace{}%
\AgdaBound{σ₁}\AgdaSpace{}%
\AgdaBound{τ}\AgdaSpace{}%
\AgdaOperator{\AgdaDatatype{≡}}\AgdaSpace{}%
\AgdaFunction{sub}\AgdaSpace{}%
\AgdaBound{σ₂}\AgdaSpace{}%
\AgdaBound{τ}\<%
\\
\>[0]\AgdaFunction{σ₁≡σ₂→σ₁τ≡σ₂τ}\AgdaSpace{}%
\AgdaSymbol{(}\AgdaOperator{\AgdaInductiveConstructor{`}}\AgdaSpace{}%
\AgdaBound{x}\AgdaSymbol{)}\AgdaSpace{}%
\AgdaBound{σ₁≡σ₂}\AgdaSpace{}%
\AgdaSymbol{=}\AgdaSpace{}%
\AgdaBound{σ₁≡σ₂}\AgdaSpace{}%
\AgdaBound{x}\<%
\\
\>[0]\AgdaFunction{σ₁≡σ₂→σ₁τ≡σ₂τ}\AgdaSpace{}%
\AgdaInductiveConstructor{`⊤}\AgdaSpace{}%
\AgdaBound{σ₁≡σ₂}\AgdaSpace{}%
\AgdaSymbol{=}\AgdaSpace{}%
\AgdaInductiveConstructor{refl}\<%
\\
\>[0]\AgdaFunction{σ₁≡σ₂→σ₁τ≡σ₂τ}\AgdaSpace{}%
\AgdaSymbol{(}\AgdaBound{τ₁}\AgdaSpace{}%
\AgdaOperator{\AgdaInductiveConstructor{⇒}}\AgdaSpace{}%
\AgdaBound{τ₂}\AgdaSymbol{)}\AgdaSpace{}%
\AgdaBound{σ₁≡σ₂}\AgdaSpace{}%
\AgdaSymbol{=}\AgdaSpace{}%
\AgdaFunction{cong₂}\AgdaSpace{}%
\AgdaOperator{\AgdaInductiveConstructor{\AgdaUnderscore{}⇒\AgdaUnderscore{}}}\AgdaSpace{}%
\AgdaSymbol{(}\AgdaFunction{σ₁≡σ₂→σ₁τ≡σ₂τ}\AgdaSpace{}%
\AgdaBound{τ₁}\AgdaSpace{}%
\AgdaBound{σ₁≡σ₂}\AgdaSymbol{)}\AgdaSpace{}%
\AgdaSymbol{(}\AgdaFunction{σ₁≡σ₂→σ₁τ≡σ₂τ}\AgdaSpace{}%
\AgdaBound{τ₂}\AgdaSpace{}%
\AgdaBound{σ₁≡σ₂}\AgdaSymbol{)}\<%
\\
\>[0]\AgdaFunction{σ₁≡σ₂→σ₁τ≡σ₂τ}\AgdaSpace{}%
\AgdaSymbol{(}\AgdaOperator{\AgdaInductiveConstructor{∀`α}}\AgdaSpace{}%
\AgdaBound{τ}\AgdaSymbol{)}\AgdaSpace{}%
\AgdaBound{σ₁≡σ₂}\AgdaSpace{}%
\AgdaSymbol{=}\AgdaSpace{}%
\AgdaFunction{cong}\AgdaSpace{}%
\AgdaOperator{\AgdaInductiveConstructor{∀`α\AgdaUnderscore{}}}\AgdaSpace{}%
\AgdaSymbol{(}\AgdaFunction{σ₁≡σ₂→σ₁τ≡σ₂τ}\AgdaSpace{}%
\AgdaBound{τ}\AgdaSpace{}%
\AgdaSymbol{(}\AgdaFunction{⊢ext-σ₁≡ext-σ₂}\AgdaSpace{}%
\AgdaBound{σ₁≡σ₂}\AgdaSymbol{))}\<%
\\
%
\\[\AgdaEmptyExtraSkip]%
\>[0]\AgdaFunction{idₛτ≡τ}\AgdaSpace{}%
\AgdaSymbol{:}\AgdaSpace{}%
\AgdaSymbol{(}\AgdaBound{τ}\AgdaSpace{}%
\AgdaSymbol{:}\AgdaSpace{}%
\AgdaFunction{Type}\AgdaSpace{}%
\AgdaGeneralizable{S}\AgdaSymbol{)}\AgdaSpace{}%
\AgdaSymbol{→}\<%
\\
\>[0][@{}l@{\AgdaIndent{0}}]%
\>[2]\AgdaFunction{sub}\AgdaSpace{}%
\AgdaFunction{idₛ}\AgdaSpace{}%
\AgdaBound{τ}\AgdaSpace{}%
\AgdaOperator{\AgdaDatatype{≡}}\AgdaSpace{}%
\AgdaBound{τ}\<%
\\
\>[0]\AgdaFunction{idₛτ≡τ}\AgdaSpace{}%
\AgdaSymbol{(}\AgdaOperator{\AgdaInductiveConstructor{`}}\AgdaSpace{}%
\AgdaBound{x}\AgdaSymbol{)}\AgdaSpace{}%
\AgdaSymbol{=}\AgdaSpace{}%
\AgdaInductiveConstructor{refl}\<%
\\
\>[0]\AgdaFunction{idₛτ≡τ}\AgdaSpace{}%
\AgdaInductiveConstructor{`⊤}\AgdaSpace{}%
\AgdaSymbol{=}\AgdaSpace{}%
\AgdaInductiveConstructor{refl}\<%
\\
\>[0]\AgdaFunction{idₛτ≡τ}\AgdaSpace{}%
\AgdaSymbol{(}\AgdaBound{τ₁}\AgdaSpace{}%
\AgdaOperator{\AgdaInductiveConstructor{⇒}}\AgdaSpace{}%
\AgdaBound{τ₂}\AgdaSymbol{)}\AgdaSpace{}%
\AgdaSymbol{=}\AgdaSpace{}%
\AgdaFunction{cong₂}\AgdaSpace{}%
\AgdaOperator{\AgdaInductiveConstructor{\AgdaUnderscore{}⇒\AgdaUnderscore{}}}\AgdaSpace{}%
\AgdaSymbol{(}\AgdaFunction{idₛτ≡τ}\AgdaSpace{}%
\AgdaBound{τ₁}\AgdaSymbol{)}\AgdaSpace{}%
\AgdaSymbol{(}\AgdaFunction{idₛτ≡τ}\AgdaSpace{}%
\AgdaBound{τ₂}\AgdaSymbol{)}\<%
\\
\>[0]\AgdaFunction{idₛτ≡τ}\AgdaSpace{}%
\AgdaSymbol{(}\AgdaOperator{\AgdaInductiveConstructor{∀`α}}\AgdaSpace{}%
\AgdaBound{τ}\AgdaSymbol{)}\AgdaSpace{}%
\AgdaSymbol{=}\AgdaSpace{}%
\AgdaFunction{cong}\AgdaSpace{}%
\AgdaOperator{\AgdaInductiveConstructor{∀`α\AgdaUnderscore{}}}\AgdaSpace{}%
\AgdaSymbol{(}\AgdaFunction{trans}\AgdaSpace{}%
\AgdaSymbol{(}\AgdaFunction{σ₁≡σ₂→σ₁τ≡σ₂τ}\AgdaSpace{}%
\AgdaBound{τ}\AgdaSpace{}%
\AgdaFunction{extₛidₛ≡idₛ}\AgdaSymbol{)}\AgdaSpace{}%
\AgdaSymbol{(}\AgdaFunction{idₛτ≡τ}\AgdaSpace{}%
\AgdaBound{τ}\AgdaSymbol{))}\<%
\\
%
\\[\AgdaEmptyExtraSkip]%
\>[0]\AgdaFunction{⊢idₛ}\AgdaSpace{}%
\AgdaSymbol{:}\AgdaSpace{}%
\AgdaSymbol{∀}\AgdaSpace{}%
\AgdaSymbol{\{}\AgdaBound{Γ}\AgdaSpace{}%
\AgdaSymbol{:}\AgdaSpace{}%
\AgdaDatatype{Ctx}\AgdaSpace{}%
\AgdaGeneralizable{S}\AgdaSymbol{\}}\AgdaSpace{}%
\AgdaSymbol{\{}\AgdaBound{t}\AgdaSpace{}%
\AgdaSymbol{:}\AgdaSpace{}%
\AgdaDatatype{Term}\AgdaSpace{}%
\AgdaGeneralizable{S}\AgdaSpace{}%
\AgdaGeneralizable{s}\AgdaSymbol{\}}\AgdaSpace{}%
\AgdaSymbol{\{}\AgdaBound{T}\AgdaSpace{}%
\AgdaSymbol{:}\AgdaSpace{}%
\AgdaDatatype{Term}\AgdaSpace{}%
\AgdaGeneralizable{S}\AgdaSpace{}%
\AgdaSymbol{(}\AgdaFunction{type-of}\AgdaSpace{}%
\AgdaGeneralizable{s}\AgdaSymbol{)\}}\AgdaSpace{}%
\AgdaSymbol{(}\AgdaBound{⊢t}\AgdaSpace{}%
\AgdaSymbol{:}\AgdaSpace{}%
\AgdaBound{Γ}\AgdaSpace{}%
\AgdaOperator{\AgdaDatatype{⊢}}\AgdaSpace{}%
\AgdaBound{t}\AgdaSpace{}%
\AgdaOperator{\AgdaDatatype{∶}}\AgdaSpace{}%
\AgdaBound{T}\AgdaSymbol{)}\AgdaSpace{}%
\AgdaSymbol{→}\AgdaSpace{}%
\AgdaFunction{idₛ}\AgdaSpace{}%
\AgdaOperator{\AgdaFunction{∶}}\AgdaSpace{}%
\AgdaBound{Γ}\AgdaSpace{}%
\AgdaOperator{\AgdaFunction{⇒ₛ}}\AgdaSpace{}%
\AgdaBound{Γ}\<%
\\
\>[0]\AgdaFunction{⊢idₛ}\AgdaSpace{}%
\AgdaSymbol{\{}\AgdaArgument{Γ}\AgdaSpace{}%
\AgdaSymbol{=}\AgdaSpace{}%
\AgdaBound{Γ}\AgdaSymbol{\}}\AgdaSpace{}%
\AgdaBound{⊢t}\AgdaSpace{}%
\AgdaSymbol{\{}\AgdaInductiveConstructor{eₛ}\AgdaSymbol{\}}\AgdaSpace{}%
\AgdaBound{x}\AgdaSpace{}%
\AgdaSymbol{=}\AgdaSpace{}%
\AgdaInductiveConstructor{⊢`x}\AgdaSpace{}%
\AgdaSymbol{(}\AgdaFunction{sym}\AgdaSpace{}%
\AgdaSymbol{(}\AgdaFunction{idₛτ≡τ}\AgdaSpace{}%
\AgdaSymbol{(}\AgdaFunction{lookup}\AgdaSpace{}%
\AgdaBound{Γ}\AgdaSpace{}%
\AgdaBound{x}\AgdaSymbol{)))}\<%
\\
\>[0]\AgdaFunction{⊢idₛ}\AgdaSpace{}%
\AgdaSymbol{\{}\AgdaArgument{Γ}\AgdaSpace{}%
\AgdaSymbol{=}\AgdaSpace{}%
\AgdaBound{Γ}\AgdaSymbol{\}}\AgdaSpace{}%
\AgdaBound{⊢t}\AgdaSpace{}%
\AgdaSymbol{\{}\AgdaInductiveConstructor{τₛ}\AgdaSymbol{\}}\AgdaSpace{}%
\AgdaBound{x}\AgdaSpace{}%
\AgdaKeyword{with}\AgdaSpace{}%
\AgdaFunction{lookup}\AgdaSpace{}%
\AgdaBound{Γ}\AgdaSpace{}%
\AgdaBound{x}\<%
\\
\>[0]\AgdaSymbol{...}\AgdaSpace{}%
\AgdaSymbol{|}\AgdaSpace{}%
\AgdaInductiveConstructor{⋆}\AgdaSpace{}%
\AgdaSymbol{=}\AgdaSpace{}%
\AgdaInductiveConstructor{⊢τ}\<%
\\
%
\\[\AgdaEmptyExtraSkip]%
\>[0]\AgdaFunction{⊢singleₛ}\AgdaSpace{}%
\AgdaSymbol{:}\AgdaSpace{}%
\AgdaSymbol{∀}\AgdaSpace{}%
\AgdaSymbol{\{}\AgdaBound{T'}\AgdaSpace{}%
\AgdaSymbol{:}\AgdaSpace{}%
\AgdaDatatype{Term}\AgdaSpace{}%
\AgdaGeneralizable{S}\AgdaSpace{}%
\AgdaSymbol{(}\AgdaFunction{type-of}\AgdaSpace{}%
\AgdaGeneralizable{s}\AgdaSymbol{)\}}\AgdaSpace{}%
\AgdaSymbol{(}\AgdaBound{⊢t}\AgdaSpace{}%
\AgdaSymbol{:}\AgdaSpace{}%
\AgdaGeneralizable{Γ}\AgdaSpace{}%
\AgdaOperator{\AgdaDatatype{⊢}}\AgdaSpace{}%
\AgdaGeneralizable{t}\AgdaSpace{}%
\AgdaOperator{\AgdaDatatype{∶}}\AgdaSpace{}%
\AgdaGeneralizable{T}\AgdaSymbol{)}\AgdaSpace{}%
\AgdaSymbol{→}\AgdaSpace{}%
\AgdaFunction{singleₛ}\AgdaSpace{}%
\AgdaFunction{idₛ}\AgdaSpace{}%
\AgdaGeneralizable{t}\AgdaSpace{}%
\AgdaOperator{\AgdaFunction{∶}}\AgdaSpace{}%
\AgdaSymbol{(}\AgdaGeneralizable{Γ}\AgdaSpace{}%
\AgdaOperator{\AgdaInductiveConstructor{▶}}\AgdaSpace{}%
\AgdaBound{T'}\AgdaSymbol{)}\AgdaSpace{}%
\AgdaOperator{\AgdaFunction{⇒ₛ}}\AgdaSpace{}%
\AgdaGeneralizable{Γ}\<%
\\
\>[0]\AgdaFunction{⊢singleₛ}\AgdaSpace{}%
\AgdaBound{⊢t}\AgdaSpace{}%
\AgdaSymbol{\{}\AgdaInductiveConstructor{eₛ}\AgdaSymbol{\}}\AgdaSpace{}%
\AgdaBound{x}\AgdaSpace{}%
\AgdaSymbol{=}\AgdaSpace{}%
\AgdaHole{\{!\ \ \ !\}}\<%
\\
\>[0]\AgdaFunction{⊢singleₛ}\AgdaSpace{}%
\AgdaBound{⊢t}\AgdaSpace{}%
\AgdaSymbol{\{}\AgdaInductiveConstructor{τₛ}\AgdaSymbol{\}}\AgdaSpace{}%
\AgdaBound{x}\AgdaSpace{}%
\AgdaSymbol{=}\AgdaSpace{}%
\AgdaHole{\{!\ \ \ !\}}\<%
\\
%
\\[\AgdaEmptyExtraSkip]%
\>[0]\AgdaComment{--\ Semantics\ ----------------------------------------------------------------------------}\<%
\end{code}
\newcommand{\FVal}[0]{\begin{code}%
\>[0]\AgdaKeyword{data}\AgdaSpace{}%
\AgdaDatatype{Val}\AgdaSpace{}%
\AgdaSymbol{:}\AgdaSpace{}%
\AgdaFunction{Expr}\AgdaSpace{}%
\AgdaGeneralizable{S}\AgdaSpace{}%
\AgdaSymbol{→}\AgdaSpace{}%
\AgdaPrimitive{Set}\AgdaSpace{}%
\AgdaKeyword{where}\<%
\\
\>[0][@{}l@{\AgdaIndent{0}}]%
\>[2]\AgdaInductiveConstructor{v-λ}\AgdaSpace{}%
\AgdaSymbol{:}\AgdaSpace{}%
\AgdaDatatype{Val}\AgdaSpace{}%
\AgdaSymbol{(}\AgdaOperator{\AgdaInductiveConstructor{λ`x→}}\AgdaSpace{}%
\AgdaGeneralizable{e}\AgdaSymbol{)}\<%
\\
%
\>[2]\AgdaInductiveConstructor{v-Λ}\AgdaSpace{}%
\AgdaSymbol{:}\AgdaSpace{}%
\AgdaDatatype{Val}\AgdaSpace{}%
\AgdaSymbol{(}\AgdaOperator{\AgdaInductiveConstructor{Λ`α→}}\AgdaSpace{}%
\AgdaGeneralizable{e}\AgdaSymbol{)}\<%
\\
%
\>[2]\AgdaInductiveConstructor{v-tt}\AgdaSpace{}%
\AgdaSymbol{:}\AgdaSpace{}%
\AgdaSymbol{∀}\AgdaSpace{}%
\AgdaSymbol{\{}\AgdaBound{S}\AgdaSymbol{\}}\AgdaSpace{}%
\AgdaSymbol{→}\AgdaSpace{}%
\AgdaDatatype{Val}\AgdaSpace{}%
\AgdaSymbol{(}\AgdaInductiveConstructor{tt}\AgdaSpace{}%
\AgdaSymbol{\{}\AgdaArgument{S}\AgdaSpace{}%
\AgdaSymbol{=}\AgdaSpace{}%
\AgdaBound{S}\AgdaSymbol{\})}\<%
\end{code}}
\begin{code}[hide]%
\>[0]\AgdaKeyword{infixr}\AgdaSpace{}%
\AgdaNumber{3}\AgdaSpace{}%
\AgdaOperator{\AgdaDatatype{\AgdaUnderscore{}↪\AgdaUnderscore{}}}\<%
\end{code}
\newcommand{\FSemantics}[0]{\begin{code}%
\>[0]\AgdaKeyword{data}\AgdaSpace{}%
\AgdaOperator{\AgdaDatatype{\AgdaUnderscore{}↪\AgdaUnderscore{}}}\AgdaSpace{}%
\AgdaSymbol{:}\AgdaSpace{}%
\AgdaFunction{Expr}\AgdaSpace{}%
\AgdaGeneralizable{S}\AgdaSpace{}%
\AgdaSymbol{→}\AgdaSpace{}%
\AgdaFunction{Expr}\AgdaSpace{}%
\AgdaGeneralizable{S}\AgdaSpace{}%
\AgdaSymbol{→}\AgdaSpace{}%
\AgdaPrimitive{Set}\AgdaSpace{}%
\AgdaKeyword{where}\<%
\\
\>[0][@{}l@{\AgdaIndent{0}}]%
\>[2]\AgdaInductiveConstructor{β-λ}\AgdaSpace{}%
\AgdaSymbol{:}\<%
\\
\>[2][@{}l@{\AgdaIndent{0}}]%
\>[4]\AgdaDatatype{Val}\AgdaSpace{}%
\AgdaGeneralizable{e₂}\AgdaSpace{}%
\AgdaSymbol{→}\<%
\\
%
\>[4]\AgdaSymbol{(}\AgdaOperator{\AgdaInductiveConstructor{λ`x→}}\AgdaSpace{}%
\AgdaGeneralizable{e₁}\AgdaSymbol{)}\AgdaSpace{}%
\AgdaOperator{\AgdaInductiveConstructor{·}}\AgdaSpace{}%
\AgdaGeneralizable{e₂}\AgdaSpace{}%
\AgdaOperator{\AgdaDatatype{↪}}\AgdaSpace{}%
\AgdaSymbol{(}\AgdaGeneralizable{e₁}\AgdaSpace{}%
\AgdaOperator{\AgdaFunction{[}}\AgdaSpace{}%
\AgdaGeneralizable{e₂}\AgdaSpace{}%
\AgdaOperator{\AgdaFunction{]}}\AgdaSymbol{)}\<%
\\
%
\>[2]\AgdaInductiveConstructor{β-Λ}\AgdaSpace{}%
\AgdaSymbol{:}\<%
\\
\>[2][@{}l@{\AgdaIndent{0}}]%
\>[4]\AgdaSymbol{(}\AgdaOperator{\AgdaInductiveConstructor{Λ`α→}}\AgdaSpace{}%
\AgdaGeneralizable{e}\AgdaSymbol{)}\AgdaSpace{}%
\AgdaOperator{\AgdaInductiveConstructor{•}}\AgdaSpace{}%
\AgdaGeneralizable{τ}\AgdaSpace{}%
\AgdaOperator{\AgdaDatatype{↪}}\AgdaSpace{}%
\AgdaGeneralizable{e}\AgdaSpace{}%
\AgdaOperator{\AgdaFunction{[}}\AgdaSpace{}%
\AgdaGeneralizable{τ}\AgdaSpace{}%
\AgdaOperator{\AgdaFunction{]}}\<%
\\
%
\>[2]\AgdaInductiveConstructor{β-let}\AgdaSpace{}%
\AgdaSymbol{:}\<%
\\
\>[2][@{}l@{\AgdaIndent{0}}]%
\>[4]\AgdaDatatype{Val}\AgdaSpace{}%
\AgdaGeneralizable{e₂}\AgdaSpace{}%
\AgdaSymbol{→}\<%
\\
%
\>[4]\AgdaOperator{\AgdaInductiveConstructor{let`x=}}\AgdaSpace{}%
\AgdaGeneralizable{e₂}\AgdaSpace{}%
\AgdaOperator{\AgdaInductiveConstructor{`in}}\AgdaSpace{}%
\AgdaGeneralizable{e₁}\AgdaSpace{}%
\AgdaOperator{\AgdaDatatype{↪}}\AgdaSpace{}%
\AgdaSymbol{(}\AgdaGeneralizable{e₁}\AgdaSpace{}%
\AgdaOperator{\AgdaFunction{[}}\AgdaSpace{}%
\AgdaGeneralizable{e₂}\AgdaSpace{}%
\AgdaOperator{\AgdaFunction{]}}\AgdaSymbol{)}\<%
\\
%
\>[2]\AgdaInductiveConstructor{ξ-·₁}\AgdaSpace{}%
\AgdaSymbol{:}\<%
\\
\>[2][@{}l@{\AgdaIndent{0}}]%
\>[4]\AgdaGeneralizable{e₁}\AgdaSpace{}%
\AgdaOperator{\AgdaDatatype{↪}}\AgdaSpace{}%
\AgdaGeneralizable{e}\AgdaSpace{}%
\AgdaSymbol{→}\<%
\\
%
\>[4]\AgdaComment{----------------}\<%
\\
%
\>[4]\AgdaGeneralizable{e₁}\AgdaSpace{}%
\AgdaOperator{\AgdaInductiveConstructor{·}}\AgdaSpace{}%
\AgdaGeneralizable{e₂}\AgdaSpace{}%
\AgdaOperator{\AgdaDatatype{↪}}\AgdaSpace{}%
\AgdaGeneralizable{e}\AgdaSpace{}%
\AgdaOperator{\AgdaInductiveConstructor{·}}\AgdaSpace{}%
\AgdaGeneralizable{e₂}\<%
\\
%
\>[2]\AgdaInductiveConstructor{ξ-·₂}\AgdaSpace{}%
\AgdaSymbol{:}\<%
\\
\>[2][@{}l@{\AgdaIndent{0}}]%
\>[4]\AgdaGeneralizable{e₂}\AgdaSpace{}%
\AgdaOperator{\AgdaDatatype{↪}}\AgdaSpace{}%
\AgdaGeneralizable{e}\AgdaSpace{}%
\AgdaSymbol{→}\<%
\\
%
\>[4]\AgdaDatatype{Val}\AgdaSpace{}%
\AgdaGeneralizable{e₁}\AgdaSpace{}%
\AgdaSymbol{→}\<%
\\
%
\>[4]\AgdaGeneralizable{e₁}\AgdaSpace{}%
\AgdaOperator{\AgdaInductiveConstructor{·}}\AgdaSpace{}%
\AgdaGeneralizable{e₂}\AgdaSpace{}%
\AgdaOperator{\AgdaDatatype{↪}}\AgdaSpace{}%
\AgdaGeneralizable{e₁}\AgdaSpace{}%
\AgdaOperator{\AgdaInductiveConstructor{·}}\AgdaSpace{}%
\AgdaGeneralizable{e}\<%
\\
%
\>[2]\AgdaInductiveConstructor{ξ-•}\AgdaSpace{}%
\AgdaSymbol{:}\<%
\\
\>[2][@{}l@{\AgdaIndent{0}}]%
\>[4]\AgdaGeneralizable{e}\AgdaSpace{}%
\AgdaOperator{\AgdaDatatype{↪}}\AgdaSpace{}%
\AgdaGeneralizable{e'}\AgdaSpace{}%
\AgdaSymbol{→}\<%
\\
%
\>[4]\AgdaComment{----------------}\<%
\\
%
\>[4]\AgdaGeneralizable{e}\AgdaSpace{}%
\AgdaOperator{\AgdaInductiveConstructor{•}}\AgdaSpace{}%
\AgdaGeneralizable{τ}\AgdaSpace{}%
\AgdaOperator{\AgdaDatatype{↪}}\AgdaSpace{}%
\AgdaGeneralizable{e'}\AgdaSpace{}%
\AgdaOperator{\AgdaInductiveConstructor{•}}\AgdaSpace{}%
\AgdaGeneralizable{τ}\<%
\\
%
\>[2]\AgdaInductiveConstructor{ξ-let}\AgdaSpace{}%
\AgdaSymbol{:}\<%
\\
\>[2][@{}l@{\AgdaIndent{0}}]%
\>[4]\AgdaGeneralizable{e₂}\AgdaSpace{}%
\AgdaOperator{\AgdaDatatype{↪}}\AgdaSpace{}%
\AgdaGeneralizable{e}\AgdaSpace{}%
\AgdaSymbol{→}\<%
\\
%
\>[4]\AgdaOperator{\AgdaInductiveConstructor{let`x=}}\AgdaSpace{}%
\AgdaGeneralizable{e₂}\AgdaSpace{}%
\AgdaOperator{\AgdaInductiveConstructor{`in}}\AgdaSpace{}%
\AgdaGeneralizable{e₁}\AgdaSpace{}%
\AgdaOperator{\AgdaDatatype{↪}}\AgdaSpace{}%
\AgdaOperator{\AgdaInductiveConstructor{let`x=}}\AgdaSpace{}%
\AgdaGeneralizable{e}\AgdaSpace{}%
\AgdaOperator{\AgdaInductiveConstructor{`in}}\AgdaSpace{}%
\AgdaGeneralizable{e₁}\<%
\end{code}}
\begin{code}[hide]%
\>[0]\AgdaComment{--\ Soundness\ ----------------------------------------------------------------------------\ }\<%
\\
%
\\[\AgdaEmptyExtraSkip]%
\>[0]\AgdaComment{--\ Progress}\<%
\end{code}
\newcommand{\FProgress}[0]{\begin{code}%
\>[0]\AgdaFunction{progress}\AgdaSpace{}%
\AgdaSymbol{:}\<%
\\
\>[0][@{}l@{\AgdaIndent{0}}]%
\>[2]\AgdaInductiveConstructor{∅}\AgdaSpace{}%
\AgdaOperator{\AgdaDatatype{⊢}}\AgdaSpace{}%
\AgdaGeneralizable{e}\AgdaSpace{}%
\AgdaOperator{\AgdaDatatype{∶}}\AgdaSpace{}%
\AgdaGeneralizable{τ}\AgdaSpace{}%
\AgdaSymbol{→}\<%
\\
%
\>[2]\AgdaSymbol{(}\AgdaFunction{∃[}\AgdaSpace{}%
\AgdaBound{e'}\AgdaSpace{}%
\AgdaFunction{]}\AgdaSpace{}%
\AgdaSymbol{(}\AgdaGeneralizable{e}\AgdaSpace{}%
\AgdaOperator{\AgdaDatatype{↪}}\AgdaSpace{}%
\AgdaBound{e'}\AgdaSymbol{))}\AgdaSpace{}%
\AgdaOperator{\AgdaDatatype{⊎}}\AgdaSpace{}%
\AgdaDatatype{Val}\AgdaSpace{}%
\AgdaGeneralizable{e}\<%
\\
\>[0]\AgdaFunction{progress}\AgdaSpace{}%
\AgdaInductiveConstructor{⊢⊤}\AgdaSpace{}%
\AgdaSymbol{=}\AgdaSpace{}%
\AgdaInductiveConstructor{inj₂}\AgdaSpace{}%
\AgdaInductiveConstructor{v-tt}\<%
\\
\>[0]\AgdaFunction{progress}\AgdaSpace{}%
\AgdaSymbol{(}\AgdaInductiveConstructor{⊢λ}\AgdaSpace{}%
\AgdaSymbol{\AgdaUnderscore{})}\AgdaSpace{}%
\AgdaSymbol{=}\AgdaSpace{}%
\AgdaInductiveConstructor{inj₂}\AgdaSpace{}%
\AgdaInductiveConstructor{v-λ}\<%
\\
\>[0]\AgdaFunction{progress}\AgdaSpace{}%
\AgdaSymbol{(}\AgdaInductiveConstructor{⊢Λ}\AgdaSpace{}%
\AgdaSymbol{\AgdaUnderscore{})}\AgdaSpace{}%
\AgdaSymbol{=}\AgdaSpace{}%
\AgdaInductiveConstructor{inj₂}\AgdaSpace{}%
\AgdaInductiveConstructor{v-Λ}\<%
\\
\>[0]\AgdaFunction{progress}\AgdaSpace{}%
\AgdaSymbol{(}\AgdaInductiveConstructor{⊢·}\AgdaSpace{}%
\AgdaSymbol{\{}\AgdaArgument{e₁}\AgdaSpace{}%
\AgdaSymbol{=}\AgdaSpace{}%
\AgdaBound{e₁}\AgdaSymbol{\}}\AgdaSpace{}%
\AgdaSymbol{\{}\AgdaArgument{e₂}\AgdaSpace{}%
\AgdaSymbol{=}\AgdaSpace{}%
\AgdaBound{e₂}\AgdaSymbol{\}}\AgdaSpace{}%
\AgdaBound{⊢e₁}%
\>[38]\AgdaBound{⊢e₂}\AgdaSymbol{)}\AgdaSpace{}%
\AgdaKeyword{with}\AgdaSpace{}%
\AgdaFunction{progress}\AgdaSpace{}%
\AgdaBound{⊢e₁}\AgdaSpace{}%
\AgdaSymbol{|}\AgdaSpace{}%
\AgdaFunction{progress}\AgdaSpace{}%
\AgdaBound{⊢e₂}\<%
\\
\>[0]\AgdaSymbol{...}\AgdaSpace{}%
\AgdaSymbol{|}\AgdaSpace{}%
\AgdaInductiveConstructor{inj₁}\AgdaSpace{}%
\AgdaSymbol{(}\AgdaBound{e₁'}\AgdaSpace{}%
\AgdaOperator{\AgdaInductiveConstructor{,}}\AgdaSpace{}%
\AgdaBound{e₁↪e₁'}\AgdaSymbol{)}\AgdaSpace{}%
\AgdaSymbol{|}\AgdaSpace{}%
\AgdaSymbol{\AgdaUnderscore{}}\AgdaSpace{}%
\AgdaSymbol{=}\AgdaSpace{}%
\AgdaInductiveConstructor{inj₁}\AgdaSpace{}%
\AgdaSymbol{(}\AgdaBound{e₁'}\AgdaSpace{}%
\AgdaOperator{\AgdaInductiveConstructor{·}}\AgdaSpace{}%
\AgdaBound{e₂}\AgdaSpace{}%
\AgdaOperator{\AgdaInductiveConstructor{,}}\AgdaSpace{}%
\AgdaInductiveConstructor{ξ-·₁}\AgdaSpace{}%
\AgdaBound{e₁↪e₁'}\AgdaSymbol{)}\<%
\\
\>[0]\AgdaSymbol{...}\AgdaSpace{}%
\AgdaSymbol{|}\AgdaSpace{}%
\AgdaInductiveConstructor{inj₂}\AgdaSpace{}%
\AgdaBound{v}\AgdaSpace{}%
\AgdaSymbol{|}\AgdaSpace{}%
\AgdaInductiveConstructor{inj₁}\AgdaSpace{}%
\AgdaSymbol{(}\AgdaBound{e₂'}\AgdaSpace{}%
\AgdaOperator{\AgdaInductiveConstructor{,}}\AgdaSpace{}%
\AgdaBound{e₂↪e₂'}\AgdaSymbol{)}\AgdaSpace{}%
\AgdaSymbol{=}\AgdaSpace{}%
\AgdaInductiveConstructor{inj₁}\AgdaSpace{}%
\AgdaSymbol{(}\AgdaBound{e₁}\AgdaSpace{}%
\AgdaOperator{\AgdaInductiveConstructor{·}}\AgdaSpace{}%
\AgdaBound{e₂'}\AgdaSpace{}%
\AgdaOperator{\AgdaInductiveConstructor{,}}\AgdaSpace{}%
\AgdaInductiveConstructor{ξ-·₂}\AgdaSpace{}%
\AgdaBound{e₂↪e₂'}\AgdaSpace{}%
\AgdaBound{v}\AgdaSymbol{)}\<%
\\
\>[0]\AgdaSymbol{...}\AgdaSpace{}%
\AgdaSymbol{|}\AgdaSpace{}%
\AgdaInductiveConstructor{inj₂}\AgdaSpace{}%
\AgdaSymbol{(}\AgdaInductiveConstructor{v-λ}\AgdaSpace{}%
\AgdaSymbol{\{}\AgdaArgument{e}\AgdaSpace{}%
\AgdaSymbol{=}\AgdaSpace{}%
\AgdaBound{e₁}\AgdaSymbol{\})}\AgdaSpace{}%
\AgdaSymbol{|}\AgdaSpace{}%
\AgdaInductiveConstructor{inj₂}\AgdaSpace{}%
\AgdaBound{v}\AgdaSpace{}%
\AgdaSymbol{=}\AgdaSpace{}%
\AgdaInductiveConstructor{inj₁}\AgdaSpace{}%
\AgdaSymbol{(}\AgdaBound{e₁}\AgdaSpace{}%
\AgdaOperator{\AgdaFunction{[}}\AgdaSpace{}%
\AgdaBound{e₂}\AgdaSpace{}%
\AgdaOperator{\AgdaFunction{]}}\AgdaSpace{}%
\AgdaOperator{\AgdaInductiveConstructor{,}}\AgdaSpace{}%
\AgdaInductiveConstructor{β-λ}\AgdaSpace{}%
\AgdaBound{v}\AgdaSymbol{)}\<%
\\
\>[0]\AgdaFunction{progress}\AgdaSpace{}%
\AgdaSymbol{(}\AgdaInductiveConstructor{⊢•}\AgdaSpace{}%
\AgdaSymbol{\{}\AgdaArgument{τ}\AgdaSpace{}%
\AgdaSymbol{=}\AgdaSpace{}%
\AgdaBound{τ}\AgdaSymbol{\}}\AgdaSpace{}%
\AgdaBound{⊢e}\AgdaSymbol{)}\AgdaSpace{}%
\AgdaKeyword{with}\AgdaSpace{}%
\AgdaFunction{progress}\AgdaSpace{}%
\AgdaBound{⊢e}\<%
\\
\>[0]\AgdaSymbol{...}\AgdaSpace{}%
\AgdaSymbol{|}\AgdaSpace{}%
\AgdaInductiveConstructor{inj₁}\AgdaSpace{}%
\AgdaSymbol{(}\AgdaBound{e'}\AgdaSpace{}%
\AgdaOperator{\AgdaInductiveConstructor{,}}\AgdaSpace{}%
\AgdaBound{e↪e'}\AgdaSymbol{)}\AgdaSpace{}%
\AgdaSymbol{=}\AgdaSpace{}%
\AgdaInductiveConstructor{inj₁}\AgdaSpace{}%
\AgdaSymbol{(}\AgdaBound{e'}\AgdaSpace{}%
\AgdaOperator{\AgdaInductiveConstructor{•}}\AgdaSpace{}%
\AgdaBound{τ}\AgdaSpace{}%
\AgdaOperator{\AgdaInductiveConstructor{,}}\AgdaSpace{}%
\AgdaInductiveConstructor{ξ-•}\AgdaSpace{}%
\AgdaBound{e↪e'}\AgdaSymbol{)}\<%
\\
\>[0]\AgdaSymbol{...}\AgdaSpace{}%
\AgdaSymbol{|}\AgdaSpace{}%
\AgdaInductiveConstructor{inj₂}\AgdaSpace{}%
\AgdaSymbol{(}\AgdaInductiveConstructor{v-Λ}\AgdaSpace{}%
\AgdaSymbol{\{}\AgdaArgument{e}\AgdaSpace{}%
\AgdaSymbol{=}\AgdaSpace{}%
\AgdaBound{e}\AgdaSymbol{\})}\AgdaSpace{}%
\AgdaSymbol{=}\AgdaSpace{}%
\AgdaInductiveConstructor{inj₁}\AgdaSpace{}%
\AgdaSymbol{(}\AgdaBound{e}\AgdaSpace{}%
\AgdaOperator{\AgdaFunction{[}}\AgdaSpace{}%
\AgdaBound{τ}\AgdaSpace{}%
\AgdaOperator{\AgdaFunction{]}}\AgdaSpace{}%
\AgdaOperator{\AgdaInductiveConstructor{,}}\AgdaSpace{}%
\AgdaInductiveConstructor{β-Λ}\AgdaSymbol{)}\<%
\\
\>[0]\AgdaFunction{progress}\AgdaSpace{}%
\AgdaSymbol{(}\AgdaInductiveConstructor{⊢let}%
\>[16]\AgdaSymbol{\{}\AgdaArgument{e₂}\AgdaSpace{}%
\AgdaSymbol{=}\AgdaSpace{}%
\AgdaBound{e₂}\AgdaSymbol{\}}\AgdaSpace{}%
\AgdaSymbol{\{}\AgdaArgument{e₁}\AgdaSpace{}%
\AgdaSymbol{=}\AgdaSpace{}%
\AgdaBound{e₁}\AgdaSymbol{\}}\AgdaSpace{}%
\AgdaBound{⊢e₂}\AgdaSpace{}%
\AgdaBound{⊢e₁}\AgdaSymbol{)}\AgdaSpace{}%
\AgdaKeyword{with}\AgdaSpace{}%
\AgdaFunction{progress}\AgdaSpace{}%
\AgdaBound{⊢e₂}\<%
\\
\>[0]\AgdaSymbol{...}\AgdaSpace{}%
\AgdaSymbol{|}\AgdaSpace{}%
\AgdaInductiveConstructor{inj₁}\AgdaSpace{}%
\AgdaSymbol{(}\AgdaBound{e₂'}\AgdaSpace{}%
\AgdaOperator{\AgdaInductiveConstructor{,}}\AgdaSpace{}%
\AgdaBound{e₂↪e₂'}\AgdaSymbol{)}\AgdaSpace{}%
\AgdaSymbol{=}\AgdaSpace{}%
\AgdaInductiveConstructor{inj₁}\AgdaSpace{}%
\AgdaSymbol{((}\AgdaOperator{\AgdaInductiveConstructor{let`x=}}\AgdaSpace{}%
\AgdaBound{e₂'}\AgdaSpace{}%
\AgdaOperator{\AgdaInductiveConstructor{`in}}\AgdaSpace{}%
\AgdaBound{e₁}\AgdaSymbol{)}\AgdaSpace{}%
\AgdaOperator{\AgdaInductiveConstructor{,}}\AgdaSpace{}%
\AgdaInductiveConstructor{ξ-let}\AgdaSpace{}%
\AgdaBound{e₂↪e₂'}\AgdaSymbol{)}\<%
\\
\>[0]\AgdaSymbol{...}\AgdaSpace{}%
\AgdaSymbol{|}\AgdaSpace{}%
\AgdaInductiveConstructor{inj₂}\AgdaSpace{}%
\AgdaBound{v}\AgdaSpace{}%
\AgdaSymbol{=}\AgdaSpace{}%
\AgdaInductiveConstructor{inj₁}\AgdaSpace{}%
\AgdaSymbol{(}\AgdaBound{e₁}\AgdaSpace{}%
\AgdaOperator{\AgdaFunction{[}}\AgdaSpace{}%
\AgdaBound{e₂}\AgdaSpace{}%
\AgdaOperator{\AgdaFunction{]}}\AgdaSpace{}%
\AgdaOperator{\AgdaInductiveConstructor{,}}\AgdaSpace{}%
\AgdaInductiveConstructor{β-let}\AgdaSpace{}%
\AgdaBound{v}\AgdaSymbol{)}\<%
\end{code}}
\begin{code}[hide]%
\>[0]\AgdaComment{--\ Subject\ Reduction}\<%
\\
%
\\[\AgdaEmptyExtraSkip]%
\>[0]\AgdaFunction{⊢ρ-preserves-Γ}\AgdaSpace{}%
\AgdaSymbol{:}\AgdaSpace{}%
\AgdaSymbol{∀}\AgdaSpace{}%
\AgdaSymbol{\{}\AgdaBound{Γ₁}\AgdaSpace{}%
\AgdaSymbol{:}\AgdaSpace{}%
\AgdaDatatype{Ctx}\AgdaSpace{}%
\AgdaGeneralizable{S₁}\AgdaSymbol{\}}\AgdaSpace{}%
\AgdaSymbol{\{}\AgdaBound{Γ₂}\AgdaSpace{}%
\AgdaSymbol{:}\AgdaSpace{}%
\AgdaDatatype{Ctx}\AgdaSpace{}%
\AgdaGeneralizable{S₂}\AgdaSymbol{\}}\AgdaSpace{}%
\AgdaSymbol{(}\AgdaBound{x}\AgdaSpace{}%
\AgdaSymbol{:}\AgdaSpace{}%
\AgdaFunction{Var}\AgdaSpace{}%
\AgdaGeneralizable{S₁}\AgdaSpace{}%
\AgdaGeneralizable{s}\AgdaSymbol{)}\AgdaSpace{}%
\AgdaSymbol{→}\<%
\\
\>[0][@{}l@{\AgdaIndent{0}}]%
\>[2]\AgdaGeneralizable{ρ}\AgdaSpace{}%
\AgdaOperator{\AgdaDatatype{∶}}\AgdaSpace{}%
\AgdaBound{Γ₁}\AgdaSpace{}%
\AgdaOperator{\AgdaDatatype{⇒ᵣ}}\AgdaSpace{}%
\AgdaBound{Γ₂}\AgdaSpace{}%
\AgdaSymbol{→}\<%
\\
%
\>[2]\AgdaFunction{ren}\AgdaSpace{}%
\AgdaGeneralizable{ρ}\AgdaSpace{}%
\AgdaSymbol{(}\AgdaFunction{lookup}\AgdaSpace{}%
\AgdaBound{Γ₁}\AgdaSpace{}%
\AgdaBound{x}\AgdaSymbol{)}\AgdaSpace{}%
\AgdaOperator{\AgdaDatatype{≡}}\AgdaSpace{}%
\AgdaFunction{lookup}\AgdaSpace{}%
\AgdaBound{Γ₂}\AgdaSpace{}%
\AgdaSymbol{(}\AgdaGeneralizable{ρ}\AgdaSpace{}%
\AgdaBound{x}\AgdaSymbol{)}\<%
\\
\>[0]\AgdaFunction{⊢ρ-preserves-Γ}\AgdaSpace{}%
\AgdaBound{x}\AgdaSpace{}%
\AgdaBound{⊢ρ}\AgdaSpace{}%
\AgdaSymbol{=}\AgdaSpace{}%
\AgdaHole{\{!\ \ \ \ \ \ \ !\}}\<%
\\
%
\\[\AgdaEmptyExtraSkip]%
\>[0]\AgdaFunction{⊢ρ-preserves}\AgdaSpace{}%
\AgdaSymbol{:}\AgdaSpace{}%
\AgdaSymbol{∀}\AgdaSpace{}%
\AgdaSymbol{\{}\AgdaBound{ρ}\AgdaSpace{}%
\AgdaSymbol{:}\AgdaSpace{}%
\AgdaFunction{Ren}\AgdaSpace{}%
\AgdaGeneralizable{S₁}\AgdaSpace{}%
\AgdaGeneralizable{S₂}\AgdaSymbol{\}}\AgdaSpace{}%
\AgdaSymbol{\{}\AgdaBound{Γ₁}\AgdaSpace{}%
\AgdaSymbol{:}\AgdaSpace{}%
\AgdaDatatype{Ctx}\AgdaSpace{}%
\AgdaGeneralizable{S₁}\AgdaSymbol{\}}\AgdaSpace{}%
\AgdaSymbol{\{}\AgdaBound{Γ₂}\AgdaSpace{}%
\AgdaSymbol{:}\AgdaSpace{}%
\AgdaDatatype{Ctx}\AgdaSpace{}%
\AgdaGeneralizable{S₂}\AgdaSymbol{\}}\AgdaSpace{}%
\AgdaSymbol{\{}\AgdaBound{t}\AgdaSpace{}%
\AgdaSymbol{:}\AgdaSpace{}%
\AgdaDatatype{Term}\AgdaSpace{}%
\AgdaGeneralizable{S₁}\AgdaSpace{}%
\AgdaGeneralizable{s}\AgdaSymbol{\}}\AgdaSpace{}%
\AgdaSymbol{\{}\AgdaBound{T}\AgdaSpace{}%
\AgdaSymbol{:}\AgdaSpace{}%
\AgdaDatatype{Term}\AgdaSpace{}%
\AgdaGeneralizable{S₁}\AgdaSpace{}%
\AgdaSymbol{(}\AgdaFunction{type-of}\AgdaSpace{}%
\AgdaGeneralizable{s}\AgdaSymbol{)\}}\AgdaSpace{}%
\AgdaSymbol{→}\<%
\\
\>[0][@{}l@{\AgdaIndent{0}}]%
\>[2]\AgdaBound{ρ}\AgdaSpace{}%
\AgdaOperator{\AgdaDatatype{∶}}\AgdaSpace{}%
\AgdaBound{Γ₁}\AgdaSpace{}%
\AgdaOperator{\AgdaDatatype{⇒ᵣ}}\AgdaSpace{}%
\AgdaBound{Γ₂}\AgdaSpace{}%
\AgdaSymbol{→}\<%
\\
%
\>[2]\AgdaBound{Γ₁}\AgdaSpace{}%
\AgdaOperator{\AgdaDatatype{⊢}}\AgdaSpace{}%
\AgdaBound{t}\AgdaSpace{}%
\AgdaOperator{\AgdaDatatype{∶}}\AgdaSpace{}%
\AgdaBound{T}\AgdaSpace{}%
\AgdaSymbol{→}\<%
\\
%
\>[2]\AgdaBound{Γ₂}\AgdaSpace{}%
\AgdaOperator{\AgdaDatatype{⊢}}\AgdaSpace{}%
\AgdaSymbol{(}\AgdaFunction{ren}\AgdaSpace{}%
\AgdaBound{ρ}\AgdaSpace{}%
\AgdaBound{t}\AgdaSymbol{)}\AgdaSpace{}%
\AgdaOperator{\AgdaDatatype{∶}}\AgdaSpace{}%
\AgdaSymbol{(}\AgdaFunction{ren}\AgdaSpace{}%
\AgdaBound{ρ}\AgdaSpace{}%
\AgdaBound{T}\AgdaSymbol{)}\<%
\\
\>[0]\AgdaFunction{⊢ρ-preserves}\AgdaSpace{}%
\AgdaBound{⊢ρ}\AgdaSpace{}%
\AgdaSymbol{(}\AgdaInductiveConstructor{⊢`x}\AgdaSpace{}%
\AgdaSymbol{\{}\AgdaArgument{x}\AgdaSpace{}%
\AgdaSymbol{=}\AgdaSpace{}%
\AgdaBound{x}\AgdaSymbol{\}}\AgdaSpace{}%
\AgdaInductiveConstructor{refl}\AgdaSymbol{)}\AgdaSpace{}%
\AgdaSymbol{=}\AgdaSpace{}%
\AgdaInductiveConstructor{⊢`x}\AgdaSpace{}%
\AgdaSymbol{(}\AgdaFunction{sym}\AgdaSpace{}%
\AgdaSymbol{(}\AgdaFunction{⊢ρ-preserves-Γ}\AgdaSpace{}%
\AgdaBound{x}\AgdaSpace{}%
\AgdaBound{⊢ρ}\AgdaSymbol{))}\<%
\\
\>[0]\AgdaFunction{⊢ρ-preserves}\AgdaSpace{}%
\AgdaBound{⊢ρ}\AgdaSpace{}%
\AgdaInductiveConstructor{⊢⊤}\AgdaSpace{}%
\AgdaSymbol{=}\AgdaSpace{}%
\AgdaInductiveConstructor{⊢⊤}\<%
\\
\>[0]\AgdaFunction{⊢ρ-preserves}\AgdaSpace{}%
\AgdaBound{⊢ρ}\AgdaSpace{}%
\AgdaSymbol{(}\AgdaInductiveConstructor{⊢λ}\AgdaSpace{}%
\AgdaBound{⊢e}\AgdaSymbol{)}\AgdaSpace{}%
\AgdaSymbol{=}\AgdaSpace{}%
\AgdaHole{\{!\ \ \ !\}}\AgdaSpace{}%
\AgdaComment{--\ ⊢λ\ (subst\ (\AgdaUnderscore{}\ ⊢\ \AgdaUnderscore{}\ ∶\AgdaUnderscore{})\ \{!\ \ \ !\}\ (⊢•\ (⊢ρ-preserves\ (⊢extᵣ\ ⊢ρ)\ ⊢e)))}\<%
\\
\>[0]\AgdaFunction{⊢ρ-preserves}\AgdaSpace{}%
\AgdaBound{⊢ρ}\AgdaSpace{}%
\AgdaSymbol{(}\AgdaInductiveConstructor{⊢Λ}\AgdaSpace{}%
\AgdaBound{⊢e}\AgdaSymbol{)}\AgdaSpace{}%
\AgdaSymbol{=}\AgdaSpace{}%
\AgdaInductiveConstructor{⊢Λ}\AgdaSpace{}%
\AgdaSymbol{(}\AgdaFunction{⊢ρ-preserves}\AgdaSpace{}%
\AgdaSymbol{(}\AgdaInductiveConstructor{⊢extᵣ}\AgdaSpace{}%
\AgdaBound{⊢ρ}\AgdaSymbol{)}\AgdaSpace{}%
\AgdaBound{⊢e}\AgdaSymbol{)}\<%
\\
\>[0]\AgdaFunction{⊢ρ-preserves}\AgdaSpace{}%
\AgdaBound{⊢ρ}\AgdaSpace{}%
\AgdaSymbol{(}\AgdaInductiveConstructor{⊢·}\AgdaSpace{}%
\AgdaBound{⊢e₁}\AgdaSpace{}%
\AgdaBound{⊢e₂}\AgdaSymbol{)}\AgdaSpace{}%
\AgdaSymbol{=}\AgdaSpace{}%
\AgdaInductiveConstructor{⊢·}\AgdaSpace{}%
\AgdaSymbol{(}\AgdaFunction{⊢ρ-preserves}\AgdaSpace{}%
\AgdaBound{⊢ρ}\AgdaSpace{}%
\AgdaBound{⊢e₁}\AgdaSymbol{)}\AgdaSpace{}%
\AgdaSymbol{(}\AgdaFunction{⊢ρ-preserves}\AgdaSpace{}%
\AgdaBound{⊢ρ}\AgdaSpace{}%
\AgdaBound{⊢e₂}\AgdaSymbol{)}\<%
\\
\>[0]\AgdaFunction{⊢ρ-preserves}\AgdaSpace{}%
\AgdaBound{⊢ρ}\AgdaSpace{}%
\AgdaSymbol{(}\AgdaInductiveConstructor{⊢•}\AgdaSpace{}%
\AgdaBound{⊢e}\AgdaSymbol{)}\AgdaSpace{}%
\AgdaSymbol{=}\AgdaSpace{}%
\AgdaHole{\{!\ \ \ !\}}\AgdaSpace{}%
\AgdaComment{--\ subst\ (\AgdaUnderscore{}\ ⊢\ \AgdaUnderscore{}\ ∶\AgdaUnderscore{})\ \{!\ \ \ !\}\ (⊢•\ (⊢ρ-preserves\ ⊢ρ\ ⊢e))}\<%
\\
\>[0]\AgdaFunction{⊢ρ-preserves}\AgdaSpace{}%
\AgdaBound{⊢ρ}\AgdaSpace{}%
\AgdaSymbol{(}\AgdaInductiveConstructor{⊢let}\AgdaSpace{}%
\AgdaBound{⊢e₂}\AgdaSpace{}%
\AgdaBound{⊢e₁}\AgdaSymbol{)}\AgdaSpace{}%
\AgdaSymbol{=}\AgdaSpace{}%
\AgdaInductiveConstructor{⊢let}\AgdaSpace{}%
\AgdaSymbol{(}\AgdaFunction{⊢ρ-preserves}\AgdaSpace{}%
\AgdaBound{⊢ρ}\AgdaSpace{}%
\AgdaBound{⊢e₂}\AgdaSymbol{)}\AgdaSpace{}%
\AgdaHole{\{!\ \ \ !\}}\<%
\\
\>[0]\AgdaFunction{⊢ρ-preserves}\AgdaSpace{}%
\AgdaBound{⊢ρ}\AgdaSpace{}%
\AgdaInductiveConstructor{⊢τ}\AgdaSpace{}%
\AgdaSymbol{=}\AgdaSpace{}%
\AgdaInductiveConstructor{⊢τ}\<%
\\
%
\\[\AgdaEmptyExtraSkip]%
\>[0]\AgdaFunction{⊢wk-preserves}\AgdaSpace{}%
\AgdaSymbol{:}\AgdaSpace{}%
\AgdaSymbol{∀}\AgdaSpace{}%
\AgdaSymbol{\{}\AgdaBound{Γ}\AgdaSpace{}%
\AgdaSymbol{:}\AgdaSpace{}%
\AgdaDatatype{Ctx}\AgdaSpace{}%
\AgdaGeneralizable{S}\AgdaSymbol{\}}\AgdaSpace{}%
\AgdaSymbol{\{}\AgdaBound{t}\AgdaSpace{}%
\AgdaSymbol{:}\AgdaSpace{}%
\AgdaDatatype{Term}\AgdaSpace{}%
\AgdaGeneralizable{S}\AgdaSpace{}%
\AgdaGeneralizable{s}\AgdaSymbol{\}}\AgdaSpace{}%
\AgdaSymbol{\{}\AgdaBound{T}\AgdaSpace{}%
\AgdaSymbol{:}\AgdaSpace{}%
\AgdaDatatype{Term}\AgdaSpace{}%
\AgdaGeneralizable{S}\AgdaSpace{}%
\AgdaSymbol{(}\AgdaFunction{type-of}\AgdaSpace{}%
\AgdaGeneralizable{s}\AgdaSymbol{)\}}\AgdaSpace{}%
\AgdaSymbol{\{}\AgdaBound{T'}\AgdaSpace{}%
\AgdaSymbol{:}\AgdaSpace{}%
\AgdaDatatype{Term}\AgdaSpace{}%
\AgdaGeneralizable{S}\AgdaSpace{}%
\AgdaSymbol{(}\AgdaFunction{type-of}\AgdaSpace{}%
\AgdaGeneralizable{s'}\AgdaSymbol{)\}}\AgdaSpace{}%
\AgdaSymbol{→}\<%
\\
\>[0][@{}l@{\AgdaIndent{0}}]%
\>[2]\AgdaBound{Γ}\AgdaSpace{}%
\AgdaOperator{\AgdaDatatype{⊢}}\AgdaSpace{}%
\AgdaBound{t}\AgdaSpace{}%
\AgdaOperator{\AgdaDatatype{∶}}\AgdaSpace{}%
\AgdaBound{T}\AgdaSpace{}%
\AgdaSymbol{→}\<%
\\
%
\>[2]\AgdaBound{Γ}\AgdaSpace{}%
\AgdaOperator{\AgdaInductiveConstructor{▶}}\AgdaSpace{}%
\AgdaBound{T'}\AgdaSpace{}%
\AgdaOperator{\AgdaDatatype{⊢}}\AgdaSpace{}%
\AgdaFunction{wk}\AgdaSpace{}%
\AgdaBound{t}\AgdaSpace{}%
\AgdaOperator{\AgdaDatatype{∶}}\AgdaSpace{}%
\AgdaFunction{wk}\AgdaSpace{}%
\AgdaBound{T}\<%
\\
\>[0]\AgdaFunction{⊢wk-preserves}\AgdaSpace{}%
\AgdaBound{⊢e}\AgdaSpace{}%
\AgdaSymbol{=}\AgdaSpace{}%
\AgdaFunction{⊢ρ-preserves}\AgdaSpace{}%
\AgdaSymbol{(}\AgdaInductiveConstructor{⊢dropᵣ}\AgdaSpace{}%
\AgdaInductiveConstructor{⊢idᵣ}\AgdaSymbol{)}\AgdaSpace{}%
\AgdaBound{⊢e}\<%
\\
%
\\[\AgdaEmptyExtraSkip]%
\>[0]\AgdaFunction{σ↑idₛ≡σ}\AgdaSpace{}%
\AgdaSymbol{:}\AgdaSpace{}%
\AgdaSymbol{∀}\AgdaSpace{}%
\AgdaSymbol{(}\AgdaBound{t}\AgdaSpace{}%
\AgdaSymbol{:}\AgdaSpace{}%
\AgdaDatatype{Term}\AgdaSpace{}%
\AgdaGeneralizable{S₁}\AgdaSpace{}%
\AgdaGeneralizable{s}\AgdaSymbol{)}\AgdaSpace{}%
\AgdaSymbol{(}\AgdaBound{t'}\AgdaSpace{}%
\AgdaSymbol{:}\AgdaSpace{}%
\AgdaDatatype{Term}\AgdaSpace{}%
\AgdaGeneralizable{S₂}\AgdaSpace{}%
\AgdaGeneralizable{s'}\AgdaSymbol{)}\AgdaSpace{}%
\AgdaSymbol{(}\AgdaBound{σ}\AgdaSpace{}%
\AgdaSymbol{:}\AgdaSpace{}%
\AgdaFunction{Sub}\AgdaSpace{}%
\AgdaGeneralizable{S₁}\AgdaSpace{}%
\AgdaGeneralizable{S₂}\AgdaSymbol{)}\AgdaSpace{}%
\AgdaSymbol{→}\<%
\\
\>[0][@{}l@{\AgdaIndent{0}}]%
\>[2]\AgdaFunction{sub}\AgdaSpace{}%
\AgdaSymbol{(}\AgdaFunction{singleₛ}\AgdaSpace{}%
\AgdaBound{σ}\AgdaSpace{}%
\AgdaBound{t'}\AgdaSymbol{)}\AgdaSpace{}%
\AgdaSymbol{(}\AgdaFunction{wk}\AgdaSpace{}%
\AgdaBound{t}\AgdaSymbol{)}\AgdaSpace{}%
\AgdaOperator{\AgdaDatatype{≡}}\AgdaSpace{}%
\AgdaFunction{sub}\AgdaSpace{}%
\AgdaBound{σ}\AgdaSpace{}%
\AgdaBound{t}\<%
\\
\>[0]\AgdaFunction{σ↑idₛ≡σ}\AgdaSpace{}%
\AgdaBound{t}\AgdaSpace{}%
\AgdaBound{t'}\AgdaSpace{}%
\AgdaBound{σ}\AgdaSpace{}%
\AgdaSymbol{=}\AgdaSpace{}%
\AgdaHole{\{!\ \ \ !\}}\<%
\\
%
\\[\AgdaEmptyExtraSkip]%
\>[0]\AgdaFunction{⊢extₛ}\AgdaSpace{}%
\AgdaSymbol{:}\AgdaSpace{}%
\AgdaSymbol{∀}\AgdaSpace{}%
\AgdaSymbol{\{}\AgdaBound{σ}\AgdaSpace{}%
\AgdaSymbol{:}\AgdaSpace{}%
\AgdaFunction{Sub}\AgdaSpace{}%
\AgdaGeneralizable{S₁}\AgdaSpace{}%
\AgdaGeneralizable{S₂}\AgdaSymbol{\}}\AgdaSpace{}%
\AgdaSymbol{\{}\AgdaBound{Γ₁}\AgdaSpace{}%
\AgdaSymbol{:}\AgdaSpace{}%
\AgdaDatatype{Ctx}\AgdaSpace{}%
\AgdaGeneralizable{S₁}\AgdaSymbol{\}}\AgdaSpace{}%
\AgdaSymbol{\{}\AgdaBound{Γ₂}\AgdaSpace{}%
\AgdaSymbol{:}\AgdaSpace{}%
\AgdaDatatype{Ctx}\AgdaSpace{}%
\AgdaGeneralizable{S₂}\AgdaSymbol{\}}\AgdaSpace{}%
\AgdaSymbol{\{}\AgdaBound{t}\AgdaSpace{}%
\AgdaSymbol{:}\AgdaSpace{}%
\AgdaFunction{Expr}\AgdaSpace{}%
\AgdaGeneralizable{S₂}\AgdaSymbol{\}}\AgdaSpace{}%
\AgdaSymbol{\{}\AgdaBound{τ}\AgdaSpace{}%
\AgdaSymbol{:}\AgdaSpace{}%
\AgdaFunction{Type}\AgdaSpace{}%
\AgdaGeneralizable{S₁}\AgdaSymbol{\}}\AgdaSpace{}%
\AgdaSymbol{→}\<%
\\
\>[0][@{}l@{\AgdaIndent{0}}]%
\>[2]\AgdaBound{σ}\AgdaSpace{}%
\AgdaOperator{\AgdaFunction{∶}}\AgdaSpace{}%
\AgdaBound{Γ₁}\AgdaSpace{}%
\AgdaOperator{\AgdaFunction{⇒ₛ}}\AgdaSpace{}%
\AgdaBound{Γ₂}\AgdaSpace{}%
\AgdaSymbol{→}\<%
\\
%
\>[2]\AgdaBound{Γ₂}\AgdaSpace{}%
\AgdaOperator{\AgdaDatatype{⊢}}\AgdaSpace{}%
\AgdaBound{t}\AgdaSpace{}%
\AgdaOperator{\AgdaDatatype{∶}}\AgdaSpace{}%
\AgdaFunction{sub}\AgdaSpace{}%
\AgdaBound{σ}\AgdaSpace{}%
\AgdaBound{τ}\AgdaSpace{}%
\AgdaSymbol{→}\<%
\\
%
\>[2]\AgdaFunction{singleₛ}\AgdaSpace{}%
\AgdaBound{σ}\AgdaSpace{}%
\AgdaBound{t}\AgdaSpace{}%
\AgdaOperator{\AgdaFunction{∶}}\AgdaSpace{}%
\AgdaBound{Γ₁}\AgdaSpace{}%
\AgdaOperator{\AgdaInductiveConstructor{▶}}\AgdaSpace{}%
\AgdaBound{τ}\AgdaSpace{}%
\AgdaOperator{\AgdaFunction{⇒ₛ}}\AgdaSpace{}%
\AgdaBound{Γ₂}\<%
\\
\>[0]\AgdaFunction{⊢extₛ}\AgdaSpace{}%
\AgdaSymbol{\{}\AgdaArgument{σ}\AgdaSpace{}%
\AgdaSymbol{=}\AgdaSpace{}%
\AgdaBound{σ}\AgdaSymbol{\}}\AgdaSpace{}%
\AgdaSymbol{\{}\AgdaArgument{t}\AgdaSpace{}%
\AgdaSymbol{=}\AgdaSpace{}%
\AgdaBound{t}\AgdaSymbol{\}}\AgdaSpace{}%
\AgdaSymbol{\{}\AgdaArgument{τ}\AgdaSpace{}%
\AgdaSymbol{=}\AgdaSpace{}%
\AgdaBound{τ}\AgdaSymbol{\}}\AgdaSpace{}%
\AgdaBound{⊢σ}\AgdaSpace{}%
\AgdaBound{⊢e}\AgdaSpace{}%
\AgdaSymbol{(}\AgdaInductiveConstructor{here}\AgdaSpace{}%
\AgdaInductiveConstructor{refl}\AgdaSymbol{)}\AgdaSpace{}%
\AgdaSymbol{=}\AgdaSpace{}%
\AgdaFunction{subst}\AgdaSpace{}%
\AgdaSymbol{(\AgdaUnderscore{}}\AgdaSpace{}%
\AgdaOperator{\AgdaDatatype{⊢}}\AgdaSpace{}%
\AgdaBound{t}\AgdaSpace{}%
\AgdaOperator{\AgdaDatatype{∶\AgdaUnderscore{}}}\AgdaSymbol{)}\AgdaSpace{}%
\AgdaSymbol{(}\AgdaFunction{sym}\AgdaSpace{}%
\AgdaSymbol{(}\AgdaFunction{σ↑idₛ≡σ}\AgdaSpace{}%
\AgdaBound{τ}\AgdaSpace{}%
\AgdaBound{t}\AgdaSpace{}%
\AgdaBound{σ}\AgdaSymbol{))}\AgdaSpace{}%
\AgdaBound{⊢e}\<%
\\
\>[0]\AgdaFunction{⊢extₛ}\AgdaSpace{}%
\AgdaSymbol{\{}\AgdaArgument{σ}\AgdaSpace{}%
\AgdaSymbol{=}\AgdaSpace{}%
\AgdaBound{σ}\AgdaSymbol{\}}\AgdaSpace{}%
\AgdaSymbol{\{}\AgdaArgument{Γ₁}\AgdaSpace{}%
\AgdaSymbol{=}\AgdaSpace{}%
\AgdaBound{Γ₁}\AgdaSymbol{\}}\AgdaSpace{}%
\AgdaSymbol{\{}\AgdaArgument{t}\AgdaSpace{}%
\AgdaSymbol{=}\AgdaSpace{}%
\AgdaBound{t}\AgdaSymbol{\}}\AgdaSpace{}%
\AgdaSymbol{\{}\AgdaArgument{τ}\AgdaSpace{}%
\AgdaSymbol{=}\AgdaSpace{}%
\AgdaBound{τ}\AgdaSymbol{\}}\AgdaSpace{}%
\AgdaBound{⊢σ}\AgdaSpace{}%
\AgdaBound{⊢e}\AgdaSpace{}%
\AgdaSymbol{\{}\AgdaInductiveConstructor{eₛ}\AgdaSymbol{\}}\AgdaSpace{}%
\AgdaSymbol{(}\AgdaInductiveConstructor{there}\AgdaSpace{}%
\AgdaBound{x}\AgdaSymbol{)}\AgdaSpace{}%
\AgdaSymbol{=}\AgdaSpace{}%
\AgdaFunction{subst}\AgdaSpace{}%
\AgdaSymbol{(\AgdaUnderscore{}}\AgdaSpace{}%
\AgdaOperator{\AgdaDatatype{⊢}}\AgdaSpace{}%
\AgdaBound{σ}\AgdaSpace{}%
\AgdaBound{x}\AgdaSpace{}%
\AgdaOperator{\AgdaDatatype{∶\AgdaUnderscore{}}}\AgdaSymbol{)}\AgdaSpace{}%
\AgdaSymbol{(}\AgdaFunction{sym}\AgdaSpace{}%
\AgdaSymbol{(}\AgdaFunction{σ↑idₛ≡σ}\AgdaSpace{}%
\AgdaSymbol{(}\AgdaFunction{lookup}\AgdaSpace{}%
\AgdaBound{Γ₁}\AgdaSpace{}%
\AgdaBound{x}\AgdaSymbol{)}\AgdaSpace{}%
\AgdaBound{t}\AgdaSpace{}%
\AgdaBound{σ}\AgdaSymbol{))}\AgdaSpace{}%
\AgdaSymbol{(}\AgdaBound{⊢σ}\AgdaSpace{}%
\AgdaBound{x}\AgdaSymbol{)}\<%
\\
\>[0]\AgdaFunction{⊢extₛ}\AgdaSpace{}%
\AgdaSymbol{\{}\AgdaArgument{σ}\AgdaSpace{}%
\AgdaSymbol{=}\AgdaSpace{}%
\AgdaBound{σ}\AgdaSymbol{\}}\AgdaSpace{}%
\AgdaSymbol{\{}\AgdaArgument{t}\AgdaSpace{}%
\AgdaSymbol{=}\AgdaSpace{}%
\AgdaBound{t}\AgdaSymbol{\}}\AgdaSpace{}%
\AgdaSymbol{\{}\AgdaArgument{τ}\AgdaSpace{}%
\AgdaSymbol{=}\AgdaSpace{}%
\AgdaBound{τ}\AgdaSymbol{\}}\AgdaSpace{}%
\AgdaBound{⊢σ}\AgdaSpace{}%
\AgdaBound{⊢e}\AgdaSpace{}%
\AgdaSymbol{\{}\AgdaInductiveConstructor{τₛ}\AgdaSymbol{\}}\AgdaSpace{}%
\AgdaSymbol{(}\AgdaInductiveConstructor{there}\AgdaSpace{}%
\AgdaBound{x}\AgdaSymbol{)}\AgdaSpace{}%
\AgdaSymbol{=}\AgdaSpace{}%
\AgdaHole{\{!\ \ \ !\}}\<%
\\
%
\\[\AgdaEmptyExtraSkip]%
\>[0]\AgdaFunction{τ[e]≡τ}\AgdaSpace{}%
\AgdaSymbol{:}\AgdaSpace{}%
\AgdaSymbol{∀}\AgdaSpace{}%
\AgdaSymbol{\{}\AgdaBound{τ}\AgdaSpace{}%
\AgdaSymbol{:}\AgdaSpace{}%
\AgdaFunction{Type}\AgdaSpace{}%
\AgdaGeneralizable{S}\AgdaSymbol{\}}\AgdaSpace{}%
\AgdaSymbol{\{}\AgdaBound{e}\AgdaSpace{}%
\AgdaSymbol{:}\AgdaSpace{}%
\AgdaFunction{Expr}\AgdaSpace{}%
\AgdaGeneralizable{S}\AgdaSymbol{\}}\AgdaSpace{}%
\AgdaSymbol{→}\AgdaSpace{}%
\AgdaFunction{wk}\AgdaSpace{}%
\AgdaBound{τ}\AgdaSpace{}%
\AgdaOperator{\AgdaFunction{[}}\AgdaSpace{}%
\AgdaBound{e}\AgdaSpace{}%
\AgdaOperator{\AgdaFunction{]}}\AgdaSpace{}%
\AgdaOperator{\AgdaDatatype{≡}}\AgdaSpace{}%
\AgdaBound{τ}\<%
\\
\>[0]\AgdaFunction{τ[e]≡τ}\AgdaSpace{}%
\AgdaSymbol{\{}\AgdaArgument{τ}\AgdaSpace{}%
\AgdaSymbol{=}\AgdaSpace{}%
\AgdaBound{τ}\AgdaSymbol{\}}\AgdaSpace{}%
\AgdaSymbol{\{}\AgdaArgument{e}\AgdaSpace{}%
\AgdaSymbol{=}\AgdaSpace{}%
\AgdaBound{e}\AgdaSymbol{\}}\AgdaSpace{}%
\AgdaSymbol{=}\<%
\\
\>[0][@{}l@{\AgdaIndent{0}}]%
\>[2]\AgdaOperator{\AgdaFunction{begin}}\<%
\\
\>[2][@{}l@{\AgdaIndent{0}}]%
\>[4]\AgdaFunction{wk}\AgdaSpace{}%
\AgdaBound{τ}\AgdaSpace{}%
\AgdaOperator{\AgdaFunction{[}}\AgdaSpace{}%
\AgdaBound{e}\AgdaSpace{}%
\AgdaOperator{\AgdaFunction{]}}\<%
\\
%
\>[2]\AgdaFunction{≡⟨}\AgdaSpace{}%
\AgdaHole{\{!\ \ !\}}\AgdaSpace{}%
\AgdaFunction{⟩}\<%
\\
\>[2][@{}l@{\AgdaIndent{0}}]%
\>[4]\AgdaBound{τ}\<%
\\
%
\>[2]\AgdaOperator{\AgdaFunction{∎}}\<%
\\
%
\\[\AgdaEmptyExtraSkip]%
\>[0]\AgdaFunction{σ↑·wkt≡wk·σt}\AgdaSpace{}%
\AgdaSymbol{:}\AgdaSpace{}%
\AgdaSymbol{∀}\AgdaSpace{}%
\AgdaSymbol{\{}\AgdaBound{s'}\AgdaSymbol{\}}\AgdaSpace{}%
\AgdaSymbol{(}\AgdaBound{σ}\AgdaSpace{}%
\AgdaSymbol{:}\AgdaSpace{}%
\AgdaFunction{Sub}\AgdaSpace{}%
\AgdaGeneralizable{S₁}\AgdaSpace{}%
\AgdaGeneralizable{S₂}\AgdaSymbol{)}\AgdaSpace{}%
\AgdaSymbol{(}\AgdaBound{t}\AgdaSpace{}%
\AgdaSymbol{:}\AgdaSpace{}%
\AgdaDatatype{Term}\AgdaSpace{}%
\AgdaGeneralizable{S₁}\AgdaSpace{}%
\AgdaGeneralizable{s}\AgdaSymbol{)}\AgdaSpace{}%
\AgdaSymbol{→}\<%
\\
\>[0][@{}l@{\AgdaIndent{0}}]%
\>[2]\AgdaFunction{sub}\AgdaSpace{}%
\AgdaSymbol{(}\AgdaFunction{extₛ}\AgdaSpace{}%
\AgdaSymbol{\{}\AgdaArgument{s}\AgdaSpace{}%
\AgdaSymbol{=}\AgdaSpace{}%
\AgdaBound{s'}\AgdaSymbol{\}}\AgdaSpace{}%
\AgdaBound{σ}\AgdaSymbol{)}\AgdaSpace{}%
\AgdaSymbol{(}\AgdaFunction{wk}\AgdaSpace{}%
\AgdaSymbol{\{}\AgdaArgument{s'}\AgdaSpace{}%
\AgdaSymbol{=}\AgdaSpace{}%
\AgdaBound{s'}\AgdaSymbol{\}}\AgdaSpace{}%
\AgdaBound{t}\AgdaSymbol{)}\AgdaSpace{}%
\AgdaOperator{\AgdaDatatype{≡}}\AgdaSpace{}%
\AgdaFunction{wk}\AgdaSpace{}%
\AgdaSymbol{(}\AgdaFunction{sub}\AgdaSpace{}%
\AgdaBound{σ}\AgdaSpace{}%
\AgdaBound{t}\AgdaSymbol{)}\<%
\\
\>[0]\AgdaFunction{σ↑·wkt≡wk·σt}\AgdaSpace{}%
\AgdaBound{σ}\AgdaSpace{}%
\AgdaBound{t}\AgdaSpace{}%
\AgdaSymbol{=}\AgdaSpace{}%
\AgdaHole{\{!\ \ \ !\}}\<%
\\
%
\\[\AgdaEmptyExtraSkip]%
\>[0]\AgdaFunction{σ·t[t']≡σ↑·t[σ·t']}\AgdaSpace{}%
\AgdaSymbol{:}\AgdaSpace{}%
\AgdaSymbol{∀}\AgdaSpace{}%
\AgdaSymbol{\{}\AgdaBound{s'}\AgdaSymbol{\}}\AgdaSpace{}%
\AgdaSymbol{(}\AgdaBound{σ}\AgdaSpace{}%
\AgdaSymbol{:}\AgdaSpace{}%
\AgdaFunction{Sub}\AgdaSpace{}%
\AgdaGeneralizable{S₁}\AgdaSpace{}%
\AgdaGeneralizable{S₂}\AgdaSymbol{)}\AgdaSpace{}%
\AgdaSymbol{(}\AgdaBound{t}\AgdaSpace{}%
\AgdaSymbol{:}\AgdaSpace{}%
\AgdaDatatype{Term}\AgdaSpace{}%
\AgdaSymbol{(}\AgdaGeneralizable{S₁}\AgdaSpace{}%
\AgdaOperator{\AgdaInductiveConstructor{▷}}\AgdaSpace{}%
\AgdaBound{s'}\AgdaSymbol{)}\AgdaSpace{}%
\AgdaGeneralizable{s}\AgdaSymbol{)}\AgdaSpace{}%
\AgdaSymbol{(}\AgdaBound{t'}\AgdaSpace{}%
\AgdaSymbol{:}\AgdaSpace{}%
\AgdaDatatype{Term}\AgdaSpace{}%
\AgdaGeneralizable{S₁}\AgdaSpace{}%
\AgdaBound{s'}\AgdaSymbol{)}\AgdaSpace{}%
\AgdaSymbol{→}\<%
\\
\>[0][@{}l@{\AgdaIndent{0}}]%
\>[2]\AgdaFunction{sub}\AgdaSpace{}%
\AgdaBound{σ}\AgdaSpace{}%
\AgdaSymbol{(}\AgdaBound{t}\AgdaSpace{}%
\AgdaOperator{\AgdaFunction{[}}\AgdaSpace{}%
\AgdaBound{t'}\AgdaSpace{}%
\AgdaOperator{\AgdaFunction{]}}\AgdaSymbol{)}\AgdaSpace{}%
\AgdaOperator{\AgdaDatatype{≡}}\AgdaSpace{}%
\AgdaSymbol{(}\AgdaFunction{sub}\AgdaSpace{}%
\AgdaSymbol{(}\AgdaFunction{extₛ}\AgdaSpace{}%
\AgdaBound{σ}\AgdaSymbol{)}\AgdaSpace{}%
\AgdaBound{t}\AgdaSymbol{)}\AgdaSpace{}%
\AgdaOperator{\AgdaFunction{[}}\AgdaSpace{}%
\AgdaFunction{sub}\AgdaSpace{}%
\AgdaBound{σ}\AgdaSpace{}%
\AgdaBound{t'}\AgdaSpace{}%
\AgdaOperator{\AgdaFunction{]}}\<%
\\
\>[0]\AgdaFunction{σ·t[t']≡σ↑·t[σ·t']}\AgdaSpace{}%
\AgdaSymbol{=}\AgdaSpace{}%
\AgdaHole{\{!\ \ \ !\}}\<%
\\
%
\\[\AgdaEmptyExtraSkip]%
\>[0]\AgdaFunction{⊢σ↑}\AgdaSpace{}%
\AgdaSymbol{:}\AgdaSpace{}%
\AgdaSymbol{∀}\AgdaSpace{}%
\AgdaSymbol{\{}\AgdaBound{σ}\AgdaSpace{}%
\AgdaSymbol{:}\AgdaSpace{}%
\AgdaFunction{Sub}\AgdaSpace{}%
\AgdaGeneralizable{S₁}\AgdaSpace{}%
\AgdaGeneralizable{S₂}\AgdaSymbol{\}}\AgdaSpace{}%
\AgdaSymbol{\{}\AgdaBound{Γ₁}\AgdaSpace{}%
\AgdaSymbol{:}\AgdaSpace{}%
\AgdaDatatype{Ctx}\AgdaSpace{}%
\AgdaGeneralizable{S₁}\AgdaSymbol{\}}\AgdaSpace{}%
\AgdaSymbol{\{}\AgdaBound{Γ₂}\AgdaSpace{}%
\AgdaSymbol{:}\AgdaSpace{}%
\AgdaDatatype{Ctx}\AgdaSpace{}%
\AgdaGeneralizable{S₂}\AgdaSymbol{\}}\AgdaSpace{}%
\AgdaSymbol{\{}\AgdaBound{T}\AgdaSpace{}%
\AgdaSymbol{:}\AgdaSpace{}%
\AgdaDatatype{Term}\AgdaSpace{}%
\AgdaGeneralizable{S₁}\AgdaSpace{}%
\AgdaSymbol{(}\AgdaFunction{type-of}\AgdaSpace{}%
\AgdaGeneralizable{s}\AgdaSymbol{)\}}\AgdaSpace{}%
\AgdaSymbol{→}\<%
\\
\>[0][@{}l@{\AgdaIndent{0}}]%
\>[2]\AgdaBound{σ}\AgdaSpace{}%
\AgdaOperator{\AgdaFunction{∶}}\AgdaSpace{}%
\AgdaBound{Γ₁}\AgdaSpace{}%
\AgdaOperator{\AgdaFunction{⇒ₛ}}\AgdaSpace{}%
\AgdaBound{Γ₂}\AgdaSpace{}%
\AgdaSymbol{→}\<%
\\
%
\>[2]\AgdaFunction{extₛ}\AgdaSpace{}%
\AgdaBound{σ}\AgdaSpace{}%
\AgdaOperator{\AgdaFunction{∶}}\AgdaSpace{}%
\AgdaBound{Γ₁}\AgdaSpace{}%
\AgdaOperator{\AgdaInductiveConstructor{▶}}\AgdaSpace{}%
\AgdaBound{T}\AgdaSpace{}%
\AgdaOperator{\AgdaFunction{⇒ₛ}}\AgdaSpace{}%
\AgdaSymbol{(}\AgdaBound{Γ₂}\AgdaSpace{}%
\AgdaOperator{\AgdaInductiveConstructor{▶}}\AgdaSpace{}%
\AgdaFunction{sub}\AgdaSpace{}%
\AgdaBound{σ}\AgdaSpace{}%
\AgdaBound{T}\AgdaSymbol{)}\<%
\\
\>[0]\AgdaFunction{⊢σ↑}\AgdaSpace{}%
\AgdaSymbol{\{}\AgdaArgument{σ}\AgdaSpace{}%
\AgdaSymbol{=}\AgdaSpace{}%
\AgdaBound{σ}\AgdaSymbol{\}}\AgdaSpace{}%
\AgdaSymbol{\{}\AgdaArgument{T}\AgdaSpace{}%
\AgdaSymbol{=}\AgdaSpace{}%
\AgdaBound{τ}\AgdaSymbol{\}}\AgdaSpace{}%
\AgdaBound{⊢σ}\AgdaSpace{}%
\AgdaSymbol{\{}\AgdaInductiveConstructor{eₛ}\AgdaSymbol{\}}\AgdaSpace{}%
\AgdaSymbol{(}\AgdaInductiveConstructor{here}\AgdaSpace{}%
\AgdaInductiveConstructor{refl}\AgdaSymbol{)}\AgdaSpace{}%
\AgdaSymbol{=}\AgdaSpace{}%
\AgdaInductiveConstructor{⊢`x}\AgdaSpace{}%
\AgdaSymbol{(}\AgdaFunction{sym}\AgdaSpace{}%
\AgdaSymbol{(}\AgdaFunction{σ↑·wkt≡wk·σt}\AgdaSpace{}%
\AgdaBound{σ}\AgdaSpace{}%
\AgdaBound{τ}\AgdaSymbol{))}\<%
\\
\>[0]\AgdaFunction{⊢σ↑}\AgdaSpace{}%
\AgdaBound{⊢σ}\AgdaSpace{}%
\AgdaSymbol{\{}\AgdaInductiveConstructor{τₛ}\AgdaSymbol{\}}\AgdaSpace{}%
\AgdaSymbol{(}\AgdaInductiveConstructor{here}\AgdaSpace{}%
\AgdaInductiveConstructor{refl}\AgdaSymbol{)}\AgdaSpace{}%
\AgdaSymbol{=}\AgdaSpace{}%
\AgdaHole{\{!\ \ \ \ !\}}\<%
\\
\>[0]\AgdaCatchallClause{\AgdaFunction{⊢σ↑}}\AgdaSpace{}%
\AgdaCatchallClause{\AgdaBound{⊢σ}}\AgdaSpace{}%
\AgdaCatchallClause{\AgdaSymbol{(}}\AgdaCatchallClause{\AgdaInductiveConstructor{there}}\AgdaSpace{}%
\AgdaCatchallClause{\AgdaBound{x}}\AgdaCatchallClause{\AgdaSymbol{)}}\AgdaSpace{}%
\AgdaSymbol{=}\AgdaSpace{}%
\AgdaHole{\{!\ \ \ !\}}\<%
\end{code}
\newcommand{\Fpreserves}[0]{\begin{code}%
\>[0]\AgdaFunction{⊢σ-preserves}\AgdaSpace{}%
\AgdaSymbol{:}\AgdaSpace{}%
\AgdaSymbol{∀}%
\>[2178I]\AgdaSymbol{\{}\AgdaBound{σ}\AgdaSpace{}%
\AgdaSymbol{:}\AgdaSpace{}%
\AgdaFunction{Sub}\AgdaSpace{}%
\AgdaGeneralizable{S₁}\AgdaSpace{}%
\AgdaGeneralizable{S₂}\AgdaSymbol{\}}\AgdaSpace{}%
\AgdaSymbol{\{}\AgdaBound{Γ₁}\AgdaSpace{}%
\AgdaSymbol{:}\AgdaSpace{}%
\AgdaDatatype{Ctx}\AgdaSpace{}%
\AgdaGeneralizable{S₁}\AgdaSymbol{\}}\AgdaSpace{}%
\AgdaSymbol{\{}\AgdaBound{Γ₂}\AgdaSpace{}%
\AgdaSymbol{:}\AgdaSpace{}%
\AgdaDatatype{Ctx}\AgdaSpace{}%
\AgdaGeneralizable{S₂}\AgdaSymbol{\}}\<%
\\
\>[.][@{}l@{}]\<[2178I]%
\>[17]\AgdaSymbol{\{}\AgdaBound{t}\AgdaSpace{}%
\AgdaSymbol{:}\AgdaSpace{}%
\AgdaDatatype{Term}\AgdaSpace{}%
\AgdaGeneralizable{S₁}\AgdaSpace{}%
\AgdaGeneralizable{s}\AgdaSymbol{\}}\AgdaSpace{}%
\AgdaSymbol{\{}\AgdaBound{T}\AgdaSpace{}%
\AgdaSymbol{:}\AgdaSpace{}%
\AgdaDatatype{Term}\AgdaSpace{}%
\AgdaGeneralizable{S₁}\AgdaSpace{}%
\AgdaSymbol{(}\AgdaFunction{type-of}\AgdaSpace{}%
\AgdaGeneralizable{s}\AgdaSymbol{)\}}\AgdaSpace{}%
\AgdaSymbol{→}\<%
\\
\>[0][@{}l@{\AgdaIndent{0}}]%
\>[2]\AgdaBound{σ}\AgdaSpace{}%
\AgdaOperator{\AgdaFunction{∶}}\AgdaSpace{}%
\AgdaBound{Γ₁}\AgdaSpace{}%
\AgdaOperator{\AgdaFunction{⇒ₛ}}\AgdaSpace{}%
\AgdaBound{Γ₂}\AgdaSpace{}%
\AgdaSymbol{→}\<%
\\
%
\>[2]\AgdaBound{Γ₁}\AgdaSpace{}%
\AgdaOperator{\AgdaDatatype{⊢}}\AgdaSpace{}%
\AgdaBound{t}\AgdaSpace{}%
\AgdaOperator{\AgdaDatatype{∶}}\AgdaSpace{}%
\AgdaBound{T}\AgdaSpace{}%
\AgdaSymbol{→}\<%
\\
%
\>[2]\AgdaBound{Γ₂}\AgdaSpace{}%
\AgdaOperator{\AgdaDatatype{⊢}}\AgdaSpace{}%
\AgdaSymbol{(}\AgdaFunction{sub}\AgdaSpace{}%
\AgdaBound{σ}\AgdaSpace{}%
\AgdaBound{t}\AgdaSymbol{)}\AgdaSpace{}%
\AgdaOperator{\AgdaDatatype{∶}}\AgdaSpace{}%
\AgdaSymbol{(}\AgdaFunction{sub}\AgdaSpace{}%
\AgdaBound{σ}\AgdaSpace{}%
\AgdaBound{T}\AgdaSymbol{)}\<%
\end{code}}
\begin{code}[hide]%
\>[0]\AgdaFunction{⊢σ-preserves}\AgdaSpace{}%
\AgdaBound{⊢σ}\AgdaSpace{}%
\AgdaSymbol{(}\AgdaInductiveConstructor{⊢`x}\AgdaSpace{}%
\AgdaSymbol{\{}\AgdaArgument{x}\AgdaSpace{}%
\AgdaSymbol{=}\AgdaSpace{}%
\AgdaBound{x}\AgdaSymbol{\}}\AgdaSpace{}%
\AgdaInductiveConstructor{refl}\AgdaSymbol{)}\AgdaSpace{}%
\AgdaSymbol{=}\AgdaSpace{}%
\AgdaBound{⊢σ}\AgdaSpace{}%
\AgdaBound{x}\<%
\\
\>[0]\AgdaFunction{⊢σ-preserves}\AgdaSpace{}%
\AgdaBound{⊢σ}\AgdaSpace{}%
\AgdaInductiveConstructor{⊢⊤}\AgdaSpace{}%
\AgdaSymbol{=}\AgdaSpace{}%
\AgdaInductiveConstructor{⊢⊤}\<%
\\
\>[0]\AgdaFunction{⊢σ-preserves}\AgdaSpace{}%
\AgdaSymbol{\{}\AgdaArgument{σ}\AgdaSpace{}%
\AgdaSymbol{=}\AgdaSpace{}%
\AgdaBound{σ}\AgdaSymbol{\}}\AgdaSpace{}%
\AgdaBound{⊢σ}\AgdaSpace{}%
\AgdaSymbol{(}\AgdaInductiveConstructor{⊢λ}\AgdaSpace{}%
\AgdaSymbol{\{}\AgdaArgument{τ'}\AgdaSpace{}%
\AgdaSymbol{=}\AgdaSpace{}%
\AgdaBound{τ'}\AgdaSymbol{\}}\AgdaSpace{}%
\AgdaBound{⊢e}\AgdaSymbol{)}\AgdaSpace{}%
\AgdaSymbol{=}\AgdaSpace{}%
\AgdaInductiveConstructor{⊢λ}\<%
\\
\>[0][@{}l@{\AgdaIndent{0}}]%
\>[2]\AgdaSymbol{(}\AgdaFunction{subst}\AgdaSpace{}%
\AgdaSymbol{(\AgdaUnderscore{}}\AgdaSpace{}%
\AgdaOperator{\AgdaDatatype{⊢}}\AgdaSpace{}%
\AgdaSymbol{\AgdaUnderscore{}}\AgdaSpace{}%
\AgdaOperator{\AgdaDatatype{∶\AgdaUnderscore{}}}\AgdaSymbol{)}\AgdaSpace{}%
\AgdaSymbol{(}\AgdaFunction{σ↑·wkt≡wk·σt}\AgdaSpace{}%
\AgdaBound{σ}\AgdaSpace{}%
\AgdaBound{τ'}\AgdaSymbol{)}\AgdaSpace{}%
\AgdaSymbol{(}\AgdaFunction{⊢σ-preserves}\AgdaSpace{}%
\AgdaSymbol{(}\AgdaFunction{⊢σ↑}\AgdaSpace{}%
\AgdaBound{⊢σ}\AgdaSymbol{)}\AgdaSpace{}%
\AgdaBound{⊢e}\AgdaSymbol{))}\<%
\\
\>[0]\AgdaFunction{⊢σ-preserves}\AgdaSpace{}%
\AgdaBound{⊢σ}\AgdaSpace{}%
\AgdaSymbol{(}\AgdaInductiveConstructor{⊢Λ}\AgdaSpace{}%
\AgdaBound{⊢e}\AgdaSymbol{)}\AgdaSpace{}%
\AgdaSymbol{=}\AgdaSpace{}%
\AgdaInductiveConstructor{⊢Λ}\AgdaSpace{}%
\AgdaSymbol{(}\AgdaFunction{⊢σ-preserves}\AgdaSpace{}%
\AgdaSymbol{(}\AgdaFunction{⊢σ↑}\AgdaSpace{}%
\AgdaBound{⊢σ}\AgdaSymbol{)}\AgdaSpace{}%
\AgdaBound{⊢e}\AgdaSymbol{)}\<%
\\
\>[0]\AgdaFunction{⊢σ-preserves}\AgdaSpace{}%
\AgdaBound{⊢σ}\AgdaSpace{}%
\AgdaSymbol{(}\AgdaInductiveConstructor{⊢·}\AgdaSpace{}%
\AgdaBound{⊢e₁}\AgdaSpace{}%
\AgdaBound{⊢e₂}\AgdaSymbol{)}\AgdaSpace{}%
\AgdaSymbol{=}\AgdaSpace{}%
\AgdaInductiveConstructor{⊢·}\AgdaSpace{}%
\AgdaSymbol{(}\AgdaFunction{⊢σ-preserves}\AgdaSpace{}%
\AgdaBound{⊢σ}\AgdaSpace{}%
\AgdaBound{⊢e₁}\AgdaSymbol{)}\AgdaSpace{}%
\AgdaSymbol{(}\AgdaFunction{⊢σ-preserves}\AgdaSpace{}%
\AgdaBound{⊢σ}\AgdaSpace{}%
\AgdaBound{⊢e₂}\AgdaSymbol{)}\<%
\\
\>[0]\AgdaFunction{⊢σ-preserves}\AgdaSpace{}%
\AgdaSymbol{\{}\AgdaArgument{σ}\AgdaSpace{}%
\AgdaSymbol{=}\AgdaSpace{}%
\AgdaBound{σ}\AgdaSymbol{\}}\AgdaSpace{}%
\AgdaBound{⊢σ}\AgdaSpace{}%
\AgdaSymbol{(}\AgdaInductiveConstructor{⊢•}\AgdaSpace{}%
\AgdaSymbol{\{}\AgdaArgument{e}\AgdaSpace{}%
\AgdaSymbol{=}\AgdaSpace{}%
\AgdaBound{e}\AgdaSymbol{\}}\AgdaSpace{}%
\AgdaSymbol{\{}\AgdaArgument{τ'}\AgdaSpace{}%
\AgdaSymbol{=}\AgdaSpace{}%
\AgdaBound{τ'}\AgdaSymbol{\}}\AgdaSpace{}%
\AgdaSymbol{\{}\AgdaArgument{τ}\AgdaSpace{}%
\AgdaSymbol{=}\AgdaSpace{}%
\AgdaBound{τ}\AgdaSymbol{\}}\AgdaSpace{}%
\AgdaBound{⊢e}\AgdaSymbol{)}\AgdaSpace{}%
\AgdaSymbol{=}\<%
\\
\>[0][@{}l@{\AgdaIndent{0}}]%
\>[2]\AgdaFunction{subst}\AgdaSpace{}%
\AgdaSymbol{(\AgdaUnderscore{}}\AgdaSpace{}%
\AgdaOperator{\AgdaDatatype{⊢}}\AgdaSpace{}%
\AgdaFunction{sub}\AgdaSpace{}%
\AgdaBound{σ}\AgdaSpace{}%
\AgdaSymbol{(}\AgdaBound{e}\AgdaSpace{}%
\AgdaOperator{\AgdaInductiveConstructor{•}}\AgdaSpace{}%
\AgdaBound{τ}\AgdaSymbol{)}\AgdaSpace{}%
\AgdaOperator{\AgdaDatatype{∶\AgdaUnderscore{}}}\AgdaSymbol{)}\AgdaSpace{}%
\AgdaSymbol{(}\AgdaFunction{sym}\AgdaSpace{}%
\AgdaSymbol{(}\AgdaFunction{σ·t[t']≡σ↑·t[σ·t']}\AgdaSpace{}%
\AgdaBound{σ}\AgdaSpace{}%
\AgdaBound{τ'}\AgdaSpace{}%
\AgdaBound{τ}\AgdaSymbol{))}\AgdaSpace{}%
\AgdaSymbol{(}\AgdaInductiveConstructor{⊢•}\AgdaSpace{}%
\AgdaSymbol{(}\AgdaFunction{⊢σ-preserves}\AgdaSpace{}%
\AgdaBound{⊢σ}\AgdaSpace{}%
\AgdaBound{⊢e}\AgdaSymbol{))}\<%
\\
\>[0]\AgdaFunction{⊢σ-preserves}\AgdaSpace{}%
\AgdaSymbol{\{}\AgdaArgument{σ}\AgdaSpace{}%
\AgdaSymbol{=}\AgdaSpace{}%
\AgdaBound{σ}\AgdaSymbol{\}}\AgdaSpace{}%
\AgdaBound{⊢σ}\AgdaSpace{}%
\AgdaSymbol{(}\AgdaInductiveConstructor{⊢let}\AgdaSpace{}%
\AgdaSymbol{\{}\AgdaArgument{τ'}\AgdaSpace{}%
\AgdaSymbol{=}\AgdaSpace{}%
\AgdaBound{τ'}\AgdaSymbol{\}}\AgdaSpace{}%
\AgdaBound{⊢e₂}\AgdaSpace{}%
\AgdaBound{⊢e₁}\AgdaSymbol{)}\AgdaSpace{}%
\AgdaSymbol{=}\AgdaSpace{}%
\AgdaInductiveConstructor{⊢let}\AgdaSpace{}%
\AgdaSymbol{(}\AgdaFunction{⊢σ-preserves}\AgdaSpace{}%
\AgdaBound{⊢σ}\AgdaSpace{}%
\AgdaBound{⊢e₂}\AgdaSymbol{)}\<%
\\
\>[0][@{}l@{\AgdaIndent{0}}]%
\>[2]\AgdaSymbol{(}\AgdaFunction{subst}\AgdaSpace{}%
\AgdaSymbol{(\AgdaUnderscore{}}\AgdaSpace{}%
\AgdaOperator{\AgdaDatatype{⊢}}\AgdaSpace{}%
\AgdaSymbol{\AgdaUnderscore{}}\AgdaSpace{}%
\AgdaOperator{\AgdaDatatype{∶\AgdaUnderscore{}}}\AgdaSymbol{)}\AgdaSpace{}%
\AgdaSymbol{(}\AgdaFunction{σ↑·wkt≡wk·σt}\AgdaSpace{}%
\AgdaBound{σ}\AgdaSpace{}%
\AgdaBound{τ'}\AgdaSymbol{)}\AgdaSpace{}%
\AgdaSymbol{(}\AgdaFunction{⊢σ-preserves}\AgdaSpace{}%
\AgdaSymbol{(}\AgdaFunction{⊢σ↑}\AgdaSpace{}%
\AgdaBound{⊢σ}\AgdaSymbol{)}\AgdaSpace{}%
\AgdaBound{⊢e₁}\AgdaSymbol{))}\<%
\\
\>[0]\<%
\\
\>[0]\AgdaFunction{⊢σ-preserves}\AgdaSpace{}%
\AgdaBound{⊢σ}\AgdaSpace{}%
\AgdaInductiveConstructor{⊢τ}\AgdaSpace{}%
\AgdaSymbol{=}\AgdaSpace{}%
\AgdaInductiveConstructor{⊢τ}\<%
\end{code}
\newcommand{\Feepreserves}[0]{\begin{code}%
\>[0]\AgdaFunction{e[e]-preserves}\AgdaSpace{}%
\AgdaSymbol{:}%
\>[18]\AgdaSymbol{∀}\AgdaSpace{}%
\AgdaSymbol{\{}\AgdaBound{Γ}\AgdaSpace{}%
\AgdaSymbol{:}\AgdaSpace{}%
\AgdaDatatype{Ctx}\AgdaSpace{}%
\AgdaGeneralizable{S}\AgdaSymbol{\}}\AgdaSpace{}%
\AgdaSymbol{\{}\AgdaBound{e₁}\AgdaSpace{}%
\AgdaSymbol{:}\AgdaSpace{}%
\AgdaFunction{Expr}\AgdaSpace{}%
\AgdaSymbol{(}\AgdaGeneralizable{S}\AgdaSpace{}%
\AgdaOperator{\AgdaInductiveConstructor{▷}}\AgdaSpace{}%
\AgdaInductiveConstructor{eₛ}\AgdaSymbol{)\}}\AgdaSpace{}%
\AgdaSymbol{\{}\AgdaBound{e₂}\AgdaSpace{}%
\AgdaSymbol{:}\AgdaSpace{}%
\AgdaFunction{Expr}\AgdaSpace{}%
\AgdaGeneralizable{S}\AgdaSymbol{\}}\AgdaSpace{}%
\AgdaSymbol{\{}\AgdaBound{τ}\AgdaSpace{}%
\AgdaBound{τ'}\AgdaSpace{}%
\AgdaSymbol{:}\AgdaSpace{}%
\AgdaFunction{Type}\AgdaSpace{}%
\AgdaGeneralizable{S}\AgdaSymbol{\}}\AgdaSpace{}%
\AgdaSymbol{→}\<%
\\
\>[0][@{}l@{\AgdaIndent{0}}]%
\>[2]\AgdaBound{Γ}\AgdaSpace{}%
\AgdaOperator{\AgdaInductiveConstructor{▶}}\AgdaSpace{}%
\AgdaBound{τ}\AgdaSpace{}%
\AgdaOperator{\AgdaDatatype{⊢}}\AgdaSpace{}%
\AgdaBound{e₁}\AgdaSpace{}%
\AgdaOperator{\AgdaDatatype{∶}}\AgdaSpace{}%
\AgdaFunction{wk}\AgdaSpace{}%
\AgdaBound{τ'}\AgdaSpace{}%
\AgdaSymbol{→}\<%
\\
%
\>[2]\AgdaBound{Γ}\AgdaSpace{}%
\AgdaOperator{\AgdaDatatype{⊢}}\AgdaSpace{}%
\AgdaBound{e₂}\AgdaSpace{}%
\AgdaOperator{\AgdaDatatype{∶}}\AgdaSpace{}%
\AgdaBound{τ}\AgdaSpace{}%
\AgdaSymbol{→}\<%
\\
%
\>[2]\AgdaBound{Γ}\AgdaSpace{}%
\AgdaOperator{\AgdaDatatype{⊢}}\AgdaSpace{}%
\AgdaBound{e₁}\AgdaSpace{}%
\AgdaOperator{\AgdaFunction{[}}\AgdaSpace{}%
\AgdaBound{e₂}\AgdaSpace{}%
\AgdaOperator{\AgdaFunction{]}}\AgdaSpace{}%
\AgdaOperator{\AgdaDatatype{∶}}\AgdaSpace{}%
\AgdaBound{τ'}\<%
\end{code}}
\begin{code}[hide]%
\>[0]\AgdaFunction{e[e]-preserves}\AgdaSpace{}%
\AgdaSymbol{\{}\AgdaArgument{τ}\AgdaSpace{}%
\AgdaSymbol{=}\AgdaSpace{}%
\AgdaBound{τ}\AgdaSymbol{\}}\AgdaSpace{}%
\AgdaBound{⊢e₁}\AgdaSpace{}%
\AgdaBound{⊢e₂}\AgdaSpace{}%
\AgdaSymbol{=}\AgdaSpace{}%
\AgdaFunction{subst}\AgdaSpace{}%
\AgdaSymbol{(\AgdaUnderscore{}}\AgdaSpace{}%
\AgdaOperator{\AgdaDatatype{⊢}}\AgdaSpace{}%
\AgdaSymbol{\AgdaUnderscore{}}\AgdaSpace{}%
\AgdaOperator{\AgdaDatatype{∶\AgdaUnderscore{}}}\AgdaSymbol{)}\AgdaSpace{}%
\AgdaFunction{τ[e]≡τ}\<%
\\
\>[0][@{}l@{\AgdaIndent{0}}]%
\>[2]\AgdaSymbol{(}\AgdaFunction{⊢σ-preserves}\AgdaSpace{}%
\AgdaSymbol{(}\AgdaFunction{⊢extₛ}\AgdaSpace{}%
\AgdaSymbol{(}\AgdaFunction{⊢idₛ}\AgdaSpace{}%
\AgdaBound{⊢e₂}\AgdaSymbol{)}\AgdaSpace{}%
\AgdaSymbol{(}\AgdaFunction{subst}\AgdaSpace{}%
\AgdaSymbol{(\AgdaUnderscore{}}\AgdaSpace{}%
\AgdaOperator{\AgdaDatatype{⊢}}\AgdaSpace{}%
\AgdaSymbol{\AgdaUnderscore{}}\AgdaSpace{}%
\AgdaOperator{\AgdaDatatype{∶\AgdaUnderscore{}}}\AgdaSymbol{)}\AgdaSpace{}%
\AgdaSymbol{(}\AgdaFunction{sym}\AgdaSpace{}%
\AgdaSymbol{(}\AgdaFunction{idₛτ≡τ}\AgdaSpace{}%
\AgdaBound{τ}\AgdaSymbol{))}\AgdaSpace{}%
\AgdaBound{⊢e₂}\AgdaSymbol{))}\AgdaSpace{}%
\AgdaBound{⊢e₁}\AgdaSymbol{)}\<%
\end{code}
\newcommand{\Fetpreserves}[0]{\begin{code}%
\>[0]\AgdaFunction{e[τ]-preserves}\AgdaSpace{}%
\AgdaSymbol{:}%
\>[18]\AgdaSymbol{∀}\AgdaSpace{}%
\AgdaSymbol{\{}\AgdaBound{Γ}\AgdaSpace{}%
\AgdaSymbol{:}\AgdaSpace{}%
\AgdaDatatype{Ctx}\AgdaSpace{}%
\AgdaGeneralizable{S}\AgdaSymbol{\}}\AgdaSpace{}%
\AgdaSymbol{\{}\AgdaBound{e}\AgdaSpace{}%
\AgdaSymbol{:}\AgdaSpace{}%
\AgdaFunction{Expr}\AgdaSpace{}%
\AgdaSymbol{(}\AgdaGeneralizable{S}\AgdaSpace{}%
\AgdaOperator{\AgdaInductiveConstructor{▷}}\AgdaSpace{}%
\AgdaInductiveConstructor{τₛ}\AgdaSymbol{)\}}\AgdaSpace{}%
\AgdaSymbol{\{}\AgdaBound{τ}\AgdaSpace{}%
\AgdaSymbol{:}\AgdaSpace{}%
\AgdaFunction{Type}\AgdaSpace{}%
\AgdaGeneralizable{S}\AgdaSymbol{\}}\AgdaSpace{}%
\AgdaSymbol{\{}\AgdaBound{τ'}\AgdaSpace{}%
\AgdaSymbol{:}\AgdaSpace{}%
\AgdaFunction{Type}\AgdaSpace{}%
\AgdaSymbol{(}\AgdaGeneralizable{S}\AgdaSpace{}%
\AgdaOperator{\AgdaInductiveConstructor{▷}}\AgdaSpace{}%
\AgdaInductiveConstructor{τₛ}\AgdaSymbol{)\}}\AgdaSpace{}%
\AgdaSymbol{→}\<%
\\
\>[0][@{}l@{\AgdaIndent{0}}]%
\>[2]\AgdaBound{Γ}\AgdaSpace{}%
\AgdaOperator{\AgdaInductiveConstructor{▶}}\AgdaSpace{}%
\AgdaInductiveConstructor{⋆}\AgdaSpace{}%
\AgdaOperator{\AgdaDatatype{⊢}}\AgdaSpace{}%
\AgdaBound{e}\AgdaSpace{}%
\AgdaOperator{\AgdaDatatype{∶}}\AgdaSpace{}%
\AgdaBound{τ'}\AgdaSpace{}%
\AgdaSymbol{→}\<%
\\
%
\>[2]\AgdaBound{Γ}\AgdaSpace{}%
\AgdaOperator{\AgdaDatatype{⊢}}\AgdaSpace{}%
\AgdaBound{τ}\AgdaSpace{}%
\AgdaOperator{\AgdaDatatype{∶}}\AgdaSpace{}%
\AgdaInductiveConstructor{⋆}\AgdaSpace{}%
\AgdaSymbol{→}\<%
\\
%
\>[2]\AgdaBound{Γ}\AgdaSpace{}%
\AgdaOperator{\AgdaDatatype{⊢}}\AgdaSpace{}%
\AgdaBound{e}\AgdaSpace{}%
\AgdaOperator{\AgdaFunction{[}}\AgdaSpace{}%
\AgdaBound{τ}\AgdaSpace{}%
\AgdaOperator{\AgdaFunction{]}}\AgdaSpace{}%
\AgdaOperator{\AgdaDatatype{∶}}\AgdaSpace{}%
\AgdaBound{τ'}\AgdaSpace{}%
\AgdaOperator{\AgdaFunction{[}}\AgdaSpace{}%
\AgdaBound{τ}\AgdaSpace{}%
\AgdaOperator{\AgdaFunction{]}}\<%
\end{code}}
\begin{code}[hide]%
\>[0]\AgdaFunction{e[τ]-preserves}\AgdaSpace{}%
\AgdaBound{⊢e}\AgdaSpace{}%
\AgdaInductiveConstructor{⊢τ}\AgdaSpace{}%
\AgdaSymbol{=}\AgdaSpace{}%
\AgdaFunction{⊢σ-preserves}\AgdaSpace{}%
\AgdaSymbol{(}\AgdaFunction{⊢singleₛ}\AgdaSpace{}%
\AgdaInductiveConstructor{⊢τ}\AgdaSymbol{)}\AgdaSpace{}%
\AgdaBound{⊢e}\<%
\end{code}
\newcommand{\FSubjectReduction}[0]{\begin{code}%
\>[0]\AgdaFunction{subject-reduction}\AgdaSpace{}%
\AgdaSymbol{:}\AgdaSpace{}%
\AgdaSymbol{∀}\AgdaSpace{}%
\AgdaSymbol{\{}\AgdaBound{Γ}\AgdaSpace{}%
\AgdaSymbol{:}\AgdaSpace{}%
\AgdaDatatype{Ctx}\AgdaSpace{}%
\AgdaGeneralizable{S}\AgdaSymbol{\}}\AgdaSpace{}%
\AgdaSymbol{→}\<%
\\
\>[0][@{}l@{\AgdaIndent{0}}]%
\>[2]\AgdaBound{Γ}\AgdaSpace{}%
\AgdaOperator{\AgdaDatatype{⊢}}\AgdaSpace{}%
\AgdaGeneralizable{e}\AgdaSpace{}%
\AgdaOperator{\AgdaDatatype{∶}}\AgdaSpace{}%
\AgdaGeneralizable{τ}\AgdaSpace{}%
\AgdaSymbol{→}\<%
\\
%
\>[2]\AgdaGeneralizable{e}\AgdaSpace{}%
\AgdaOperator{\AgdaDatatype{↪}}\AgdaSpace{}%
\AgdaGeneralizable{e'}\AgdaSpace{}%
\AgdaSymbol{→}\<%
\\
%
\>[2]\AgdaBound{Γ}\AgdaSpace{}%
\AgdaOperator{\AgdaDatatype{⊢}}\AgdaSpace{}%
\AgdaGeneralizable{e'}\AgdaSpace{}%
\AgdaOperator{\AgdaDatatype{∶}}\AgdaSpace{}%
\AgdaGeneralizable{τ}\<%
\\
\>[0]\AgdaFunction{subject-reduction}\AgdaSpace{}%
\AgdaSymbol{(}\AgdaInductiveConstructor{⊢·}\AgdaSpace{}%
\AgdaSymbol{(}\AgdaInductiveConstructor{⊢λ}\AgdaSpace{}%
\AgdaBound{⊢e₁}\AgdaSymbol{)}\AgdaSpace{}%
\AgdaBound{⊢e₂}\AgdaSymbol{)}\AgdaSpace{}%
\AgdaSymbol{(}\AgdaInductiveConstructor{β-λ}\AgdaSpace{}%
\AgdaBound{v₂}\AgdaSymbol{)}\AgdaSpace{}%
\AgdaSymbol{=}\AgdaSpace{}%
\AgdaFunction{e[e]-preserves}\AgdaSpace{}%
\AgdaBound{⊢e₁}\AgdaSpace{}%
\AgdaBound{⊢e₂}\<%
\\
\>[0]\AgdaFunction{subject-reduction}\AgdaSpace{}%
\AgdaSymbol{(}\AgdaInductiveConstructor{⊢·}\AgdaSpace{}%
\AgdaBound{⊢e₁}\AgdaSpace{}%
\AgdaBound{⊢e₂}\AgdaSymbol{)}\AgdaSpace{}%
\AgdaSymbol{(}\AgdaInductiveConstructor{ξ-·₁}\AgdaSpace{}%
\AgdaBound{e₁↪e}\AgdaSymbol{)}\AgdaSpace{}%
\AgdaSymbol{=}\AgdaSpace{}%
\AgdaInductiveConstructor{⊢·}\AgdaSpace{}%
\AgdaSymbol{(}\AgdaFunction{subject-reduction}\AgdaSpace{}%
\AgdaBound{⊢e₁}\AgdaSpace{}%
\AgdaBound{e₁↪e}\AgdaSymbol{)}\AgdaSpace{}%
\AgdaBound{⊢e₂}\<%
\\
\>[0]\AgdaFunction{subject-reduction}\AgdaSpace{}%
\AgdaSymbol{(}\AgdaInductiveConstructor{⊢·}\AgdaSpace{}%
\AgdaBound{⊢e₁}\AgdaSpace{}%
\AgdaBound{⊢e₂}\AgdaSymbol{)}\AgdaSpace{}%
\AgdaSymbol{(}\AgdaInductiveConstructor{ξ-·₂}\AgdaSpace{}%
\AgdaBound{e₂↪e}\AgdaSpace{}%
\AgdaBound{x}\AgdaSymbol{)}\AgdaSpace{}%
\AgdaSymbol{=}\AgdaSpace{}%
\AgdaInductiveConstructor{⊢·}\AgdaSpace{}%
\AgdaBound{⊢e₁}\AgdaSpace{}%
\AgdaSymbol{(}\AgdaFunction{subject-reduction}\AgdaSpace{}%
\AgdaBound{⊢e₂}\AgdaSpace{}%
\AgdaBound{e₂↪e}\AgdaSymbol{)}\<%
\\
\>[0]\AgdaFunction{subject-reduction}\AgdaSpace{}%
\AgdaSymbol{(}\AgdaInductiveConstructor{⊢•}\AgdaSpace{}%
\AgdaSymbol{(}\AgdaInductiveConstructor{⊢Λ}\AgdaSpace{}%
\AgdaBound{⊢e}\AgdaSymbol{))}\AgdaSpace{}%
\AgdaInductiveConstructor{β-Λ}\AgdaSpace{}%
\AgdaSymbol{=}\AgdaSpace{}%
\AgdaFunction{e[τ]-preserves}\AgdaSpace{}%
\AgdaBound{⊢e}\AgdaSpace{}%
\AgdaInductiveConstructor{⊢τ}\<%
\\
\>[0]\AgdaFunction{subject-reduction}\AgdaSpace{}%
\AgdaSymbol{(}\AgdaInductiveConstructor{⊢•}\AgdaSpace{}%
\AgdaBound{⊢e}\AgdaSymbol{)}\AgdaSpace{}%
\AgdaSymbol{(}\AgdaInductiveConstructor{ξ-•}\AgdaSpace{}%
\AgdaBound{e↪e'}\AgdaSymbol{)}\AgdaSpace{}%
\AgdaSymbol{=}\AgdaSpace{}%
\AgdaInductiveConstructor{⊢•}\AgdaSpace{}%
\AgdaSymbol{(}\AgdaFunction{subject-reduction}\AgdaSpace{}%
\AgdaBound{⊢e}\AgdaSpace{}%
\AgdaBound{e↪e'}\AgdaSymbol{)}\<%
\\
\>[0]\AgdaFunction{subject-reduction}\AgdaSpace{}%
\AgdaSymbol{(}\AgdaInductiveConstructor{⊢let}\AgdaSpace{}%
\AgdaBound{⊢e₂}\AgdaSpace{}%
\AgdaBound{⊢e₁}\AgdaSymbol{)}\AgdaSpace{}%
\AgdaSymbol{(}\AgdaInductiveConstructor{β-let}\AgdaSpace{}%
\AgdaBound{v₂}\AgdaSymbol{)}\AgdaSpace{}%
\AgdaSymbol{=}\AgdaSpace{}%
\AgdaFunction{e[e]-preserves}\AgdaSpace{}%
\AgdaBound{⊢e₁}\AgdaSpace{}%
\AgdaBound{⊢e₂}\<%
\\
\>[0]\AgdaFunction{subject-reduction}\AgdaSpace{}%
\AgdaSymbol{(}\AgdaInductiveConstructor{⊢let}\AgdaSpace{}%
\AgdaBound{⊢e₂}\AgdaSpace{}%
\AgdaBound{⊢e₁}\AgdaSymbol{)}\AgdaSpace{}%
\AgdaSymbol{(}\AgdaInductiveConstructor{ξ-let}\AgdaSpace{}%
\AgdaBound{e₂↪e'}\AgdaSymbol{)}\AgdaSpace{}%
\AgdaSymbol{=}\AgdaSpace{}%
\AgdaInductiveConstructor{⊢let}\AgdaSpace{}%
\AgdaSymbol{(}\AgdaFunction{subject-reduction}\AgdaSpace{}%
\AgdaBound{⊢e₂}\AgdaSpace{}%
\AgdaBound{e₂↪e'}\AgdaSymbol{)}\AgdaSpace{}%
\AgdaBound{⊢e₁}\<%
\end{code}}

\begin{code}[hide]%
\>[0]\AgdaKeyword{open}\AgdaSpace{}%
\AgdaKeyword{import}\AgdaSpace{}%
\AgdaModule{Data.Unit}\AgdaSpace{}%
\AgdaKeyword{using}\AgdaSpace{}%
\AgdaSymbol{(}\AgdaRecord{⊤}\AgdaSymbol{;}\AgdaSpace{}%
\AgdaInductiveConstructor{tt}\AgdaSymbol{)}\<%
\\
\>[0]\AgdaKeyword{open}\AgdaSpace{}%
\AgdaKeyword{import}\AgdaSpace{}%
\AgdaModule{Data.Nat}\AgdaSpace{}%
\AgdaKeyword{using}\AgdaSpace{}%
\AgdaSymbol{(}\AgdaDatatype{ℕ}\AgdaSymbol{;}\AgdaSpace{}%
\AgdaInductiveConstructor{zero}\AgdaSymbol{;}\AgdaSpace{}%
\AgdaInductiveConstructor{suc}\AgdaSymbol{)}\<%
\\
\>[0]\AgdaKeyword{open}\AgdaSpace{}%
\AgdaKeyword{import}\AgdaSpace{}%
\AgdaModule{Data.List}\AgdaSpace{}%
\AgdaKeyword{using}\AgdaSpace{}%
\AgdaSymbol{(}\AgdaDatatype{List}\AgdaSymbol{;}\AgdaSpace{}%
\AgdaInductiveConstructor{[]}\AgdaSymbol{;}\AgdaSpace{}%
\AgdaOperator{\AgdaInductiveConstructor{\AgdaUnderscore{}∷\AgdaUnderscore{}}}\AgdaSymbol{;}\AgdaSpace{}%
\AgdaOperator{\AgdaFunction{\AgdaUnderscore{}++\AgdaUnderscore{}}}\AgdaSymbol{;}\AgdaSpace{}%
\AgdaFunction{drop}\AgdaSymbol{)}\<%
\\
\>[0]\AgdaKeyword{open}\AgdaSpace{}%
\AgdaKeyword{import}\AgdaSpace{}%
\AgdaModule{Data.List.Relation.Unary.Any}\AgdaSpace{}%
\AgdaKeyword{using}\AgdaSpace{}%
\AgdaSymbol{(}\AgdaInductiveConstructor{here}\AgdaSymbol{;}\AgdaSpace{}%
\AgdaInductiveConstructor{there}\AgdaSymbol{)}\<%
\\
\>[0]\AgdaKeyword{open}\AgdaSpace{}%
\AgdaKeyword{import}\AgdaSpace{}%
\AgdaModule{Data.List.Membership.Propositional}\AgdaSpace{}%
\AgdaKeyword{using}\AgdaSpace{}%
\AgdaSymbol{(}\AgdaOperator{\AgdaFunction{\AgdaUnderscore{}∈\AgdaUnderscore{}}}\AgdaSymbol{)}\<%
\\
\>[0]\AgdaKeyword{open}\AgdaSpace{}%
\AgdaKeyword{import}\AgdaSpace{}%
\AgdaModule{Data.Sum.Base}\AgdaSpace{}%
\AgdaKeyword{using}\AgdaSpace{}%
\AgdaSymbol{(}\AgdaOperator{\AgdaDatatype{\AgdaUnderscore{}⊎\AgdaUnderscore{}}}\AgdaSymbol{;}\AgdaSpace{}%
\AgdaInductiveConstructor{inj₁}\AgdaSymbol{;}\AgdaSpace{}%
\AgdaInductiveConstructor{inj₂}\AgdaSymbol{)}\<%
\\
\>[0]\AgdaKeyword{open}\AgdaSpace{}%
\AgdaKeyword{import}\AgdaSpace{}%
\AgdaModule{Data.Product}\AgdaSpace{}%
\AgdaKeyword{using}\AgdaSpace{}%
\AgdaSymbol{(}\AgdaOperator{\AgdaFunction{\AgdaUnderscore{}×\AgdaUnderscore{}}}\AgdaSymbol{;}\AgdaSpace{}%
\AgdaOperator{\AgdaInductiveConstructor{\AgdaUnderscore{},\AgdaUnderscore{}}}\AgdaSymbol{;}\AgdaSpace{}%
\AgdaFunction{Σ-syntax}\AgdaSymbol{;}\AgdaSpace{}%
\AgdaFunction{∃-syntax}\AgdaSymbol{)}\<%
\\
\>[0]\AgdaKeyword{open}\AgdaSpace{}%
\AgdaKeyword{import}\AgdaSpace{}%
\AgdaModule{Relation.Binary.PropositionalEquality}\AgdaSpace{}%
\AgdaKeyword{using}\AgdaSpace{}%
\AgdaSymbol{(}\AgdaOperator{\AgdaDatatype{\AgdaUnderscore{}≡\AgdaUnderscore{}}}\AgdaSymbol{;}\AgdaSpace{}%
\AgdaOperator{\AgdaFunction{\AgdaUnderscore{}≢\AgdaUnderscore{}}}\AgdaSymbol{;}\AgdaSpace{}%
\AgdaInductiveConstructor{refl}\AgdaSymbol{;}\AgdaSpace{}%
\AgdaFunction{cong}\AgdaSymbol{;}\AgdaSpace{}%
\AgdaFunction{cong₂}\AgdaSymbol{;}\AgdaSpace{}%
\AgdaFunction{trans}\AgdaSymbol{)}\<%
\\
\>[0]\AgdaKeyword{open}\AgdaSpace{}%
\AgdaKeyword{import}\AgdaSpace{}%
\AgdaModule{Function}\AgdaSpace{}%
\AgdaKeyword{using}\AgdaSpace{}%
\AgdaSymbol{(}\AgdaFunction{id}\AgdaSymbol{;}\AgdaSpace{}%
\AgdaOperator{\AgdaFunction{\AgdaUnderscore{}∘\AgdaUnderscore{}}}\AgdaSymbol{)}\<%
\\
%
\\[\AgdaEmptyExtraSkip]%
\>[0]\AgdaKeyword{module}\AgdaSpace{}%
\AgdaModule{SystemFo}\AgdaSpace{}%
\AgdaKeyword{where}\<%
\\
%
\\[\AgdaEmptyExtraSkip]%
\>[0]\AgdaComment{--\ Sorts\ --------------------------------------------------------------------------------}\<%
\\
\>[0]\AgdaKeyword{data}\AgdaSpace{}%
\AgdaDatatype{Bindable}\AgdaSpace{}%
\AgdaSymbol{:}\AgdaSpace{}%
\AgdaPrimitive{Set}\AgdaSpace{}%
\AgdaKeyword{where}\<%
\\
\>[0][@{}l@{\AgdaIndent{0}}]%
\>[2]\AgdaInductiveConstructor{⊤ᴮ}\AgdaSpace{}%
\AgdaSymbol{:}\AgdaSpace{}%
\AgdaDatatype{Bindable}\<%
\\
%
\>[2]\AgdaInductiveConstructor{⊥ᴮ}\AgdaSpace{}%
\AgdaSymbol{:}\AgdaSpace{}%
\AgdaDatatype{Bindable}\<%
\\
%
\\[\AgdaEmptyExtraSkip]%
\>[0]\AgdaKeyword{variable}\<%
\\
\>[0][@{}l@{\AgdaIndent{0}}]%
\>[2]\AgdaGeneralizable{r}\AgdaSpace{}%
\AgdaGeneralizable{r'}\AgdaSpace{}%
\AgdaGeneralizable{r''}\AgdaSpace{}%
\AgdaGeneralizable{r₁}\AgdaSpace{}%
\AgdaGeneralizable{r₂}\AgdaSpace{}%
\AgdaSymbol{:}\AgdaSpace{}%
\AgdaDatatype{Bindable}\<%
\end{code}
\newcommand{\FoSort}[0]{\begin{code}%
\>[0]\AgdaKeyword{data}\AgdaSpace{}%
\AgdaDatatype{Sort}\AgdaSpace{}%
\AgdaSymbol{:}\AgdaSpace{}%
\AgdaDatatype{Bindable}\AgdaSpace{}%
\AgdaSymbol{→}\AgdaSpace{}%
\AgdaPrimitive{Set}\AgdaSpace{}%
\AgdaKeyword{where}\<%
\\
\>[0][@{}l@{\AgdaIndent{0}}]%
\>[2]\AgdaInductiveConstructor{oₛ}%
\>[6]\AgdaSymbol{:}\AgdaSpace{}%
\AgdaDatatype{Sort}\AgdaSpace{}%
\AgdaInductiveConstructor{⊤ᴮ}\<%
\\
%
\>[2]\AgdaInductiveConstructor{cₛ}%
\>[6]\AgdaSymbol{:}\AgdaSpace{}%
\AgdaDatatype{Sort}\AgdaSpace{}%
\AgdaInductiveConstructor{⊥ᴮ}\<%
\\
%
\>[2]\AgdaComment{--\ ...}\<%
\end{code}}
\begin{code}[hide]%
%
\>[2]\AgdaInductiveConstructor{eₛ}%
\>[6]\AgdaSymbol{:}\AgdaSpace{}%
\AgdaDatatype{Sort}\AgdaSpace{}%
\AgdaInductiveConstructor{⊤ᴮ}\<%
\\
%
\>[2]\AgdaInductiveConstructor{τₛ}%
\>[6]\AgdaSymbol{:}\AgdaSpace{}%
\AgdaDatatype{Sort}\AgdaSpace{}%
\AgdaInductiveConstructor{⊤ᴮ}\<%
\\
%
\>[2]\AgdaInductiveConstructor{κₛ}%
\>[6]\AgdaSymbol{:}\AgdaSpace{}%
\AgdaDatatype{Sort}\AgdaSpace{}%
\AgdaInductiveConstructor{⊥ᴮ}\<%
\end{code}
\begin{code}[hide]%
\>[0]\AgdaFunction{Sorts}\AgdaSpace{}%
\AgdaSymbol{:}\AgdaSpace{}%
\AgdaPrimitive{Set}\<%
\\
\>[0]\AgdaFunction{Sorts}\AgdaSpace{}%
\AgdaSymbol{=}\AgdaSpace{}%
\AgdaDatatype{List}\AgdaSpace{}%
\AgdaSymbol{(}\AgdaDatatype{Sort}\AgdaSpace{}%
\AgdaInductiveConstructor{⊤ᴮ}\AgdaSymbol{)}\<%
\\
%
\\[\AgdaEmptyExtraSkip]%
\>[0]\AgdaKeyword{infix}\AgdaSpace{}%
\AgdaNumber{25}\AgdaSpace{}%
\AgdaOperator{\AgdaInductiveConstructor{\AgdaUnderscore{}▷\AgdaUnderscore{}}}\AgdaSpace{}%
\AgdaOperator{\AgdaFunction{\AgdaUnderscore{}▷▷\AgdaUnderscore{}}}\<%
\\
\>[0]\AgdaKeyword{pattern}\AgdaSpace{}%
\AgdaOperator{\AgdaInductiveConstructor{\AgdaUnderscore{}▷\AgdaUnderscore{}}}\AgdaSpace{}%
\AgdaBound{xs}\AgdaSpace{}%
\AgdaBound{x}\AgdaSpace{}%
\AgdaSymbol{=}\AgdaSpace{}%
\AgdaBound{x}\AgdaSpace{}%
\AgdaOperator{\AgdaInductiveConstructor{∷}}\AgdaSpace{}%
\AgdaBound{xs}\<%
\\
\>[0]\AgdaOperator{\AgdaFunction{\AgdaUnderscore{}▷▷\AgdaUnderscore{}}}\AgdaSpace{}%
\AgdaSymbol{:}\AgdaSpace{}%
\AgdaSymbol{\{}\AgdaBound{A}\AgdaSpace{}%
\AgdaSymbol{:}\AgdaSpace{}%
\AgdaPrimitive{Set}\AgdaSymbol{\}}\AgdaSpace{}%
\AgdaSymbol{→}\AgdaSpace{}%
\AgdaDatatype{List}\AgdaSpace{}%
\AgdaBound{A}\AgdaSpace{}%
\AgdaSymbol{→}\AgdaSpace{}%
\AgdaDatatype{List}\AgdaSpace{}%
\AgdaBound{A}\AgdaSpace{}%
\AgdaSymbol{→}\AgdaSpace{}%
\AgdaDatatype{List}\AgdaSpace{}%
\AgdaBound{A}\<%
\\
\>[0]\AgdaBound{xs}\AgdaSpace{}%
\AgdaOperator{\AgdaFunction{▷▷}}\AgdaSpace{}%
\AgdaBound{ys}\AgdaSpace{}%
\AgdaSymbol{=}\AgdaSpace{}%
\AgdaBound{ys}\AgdaSpace{}%
\AgdaOperator{\AgdaFunction{++}}\AgdaSpace{}%
\AgdaBound{xs}\<%
\\
%
\\[\AgdaEmptyExtraSkip]%
\>[0]\AgdaKeyword{variable}\<%
\\
\>[0][@{}l@{\AgdaIndent{0}}]%
\>[2]\AgdaGeneralizable{s}\AgdaSpace{}%
\AgdaGeneralizable{s'}\AgdaSpace{}%
\AgdaGeneralizable{s''}\AgdaSpace{}%
\AgdaGeneralizable{s₁}\AgdaSpace{}%
\AgdaGeneralizable{s₂}\AgdaSpace{}%
\AgdaSymbol{:}\AgdaSpace{}%
\AgdaDatatype{Sort}\AgdaSpace{}%
\AgdaGeneralizable{r}\<%
\\
%
\>[2]\AgdaGeneralizable{S}\AgdaSpace{}%
\AgdaGeneralizable{S'}\AgdaSpace{}%
\AgdaGeneralizable{S''}\AgdaSpace{}%
\AgdaGeneralizable{S₁}\AgdaSpace{}%
\AgdaGeneralizable{S₂}\AgdaSpace{}%
\AgdaSymbol{:}\AgdaSpace{}%
\AgdaFunction{Sorts}\<%
\\
%
\>[2]\AgdaGeneralizable{x}\AgdaSpace{}%
\AgdaGeneralizable{x'}\AgdaSpace{}%
\AgdaGeneralizable{x''}\AgdaSpace{}%
\AgdaGeneralizable{x₁}\AgdaSpace{}%
\AgdaGeneralizable{x₂}\AgdaSpace{}%
\AgdaSymbol{:}\AgdaSpace{}%
\AgdaInductiveConstructor{eₛ}\AgdaSpace{}%
\AgdaOperator{\AgdaFunction{∈}}\AgdaSpace{}%
\AgdaGeneralizable{S}\<%
\\
%
\>[2]\AgdaGeneralizable{o}\AgdaSpace{}%
\AgdaGeneralizable{o'}\AgdaSpace{}%
\AgdaGeneralizable{o''}\AgdaSpace{}%
\AgdaGeneralizable{o₁}\AgdaSpace{}%
\AgdaGeneralizable{o₂}\AgdaSpace{}%
\AgdaSymbol{:}\AgdaSpace{}%
\AgdaInductiveConstructor{oₛ}\AgdaSpace{}%
\AgdaOperator{\AgdaFunction{∈}}\AgdaSpace{}%
\AgdaGeneralizable{S}\<%
\\
%
\>[2]\AgdaGeneralizable{α}\AgdaSpace{}%
\AgdaGeneralizable{α'}\AgdaSpace{}%
\AgdaGeneralizable{α''}\AgdaSpace{}%
\AgdaGeneralizable{α₁}\AgdaSpace{}%
\AgdaGeneralizable{α₂}\AgdaSpace{}%
\AgdaSymbol{:}\AgdaSpace{}%
\AgdaInductiveConstructor{τₛ}\AgdaSpace{}%
\AgdaOperator{\AgdaFunction{∈}}\AgdaSpace{}%
\AgdaGeneralizable{S}\<%
\\
%
\\[\AgdaEmptyExtraSkip]%
\>[0]\AgdaFunction{Var}\AgdaSpace{}%
\AgdaSymbol{:}\AgdaSpace{}%
\AgdaFunction{Sorts}\AgdaSpace{}%
\AgdaSymbol{→}\AgdaSpace{}%
\AgdaDatatype{Sort}\AgdaSpace{}%
\AgdaInductiveConstructor{⊤ᴮ}\AgdaSpace{}%
\AgdaSymbol{→}\AgdaSpace{}%
\AgdaPrimitive{Set}\<%
\\
\>[0]\AgdaFunction{Var}\AgdaSpace{}%
\AgdaBound{S}\AgdaSpace{}%
\AgdaBound{s}\AgdaSpace{}%
\AgdaSymbol{=}\AgdaSpace{}%
\AgdaBound{s}\AgdaSpace{}%
\AgdaOperator{\AgdaFunction{∈}}\AgdaSpace{}%
\AgdaBound{S}\<%
\\
%
\\[\AgdaEmptyExtraSkip]%
\>[0]\AgdaComment{--\ Syntax\ -------------------------------------------------------------------------------}\<%
\\
%
\\[\AgdaEmptyExtraSkip]%
\>[0]\AgdaKeyword{infixr}\AgdaSpace{}%
\AgdaNumber{4}\AgdaSpace{}%
\AgdaOperator{\AgdaInductiveConstructor{λ`x→\AgdaUnderscore{}}}\AgdaSpace{}%
\AgdaOperator{\AgdaInductiveConstructor{Λ`α→\AgdaUnderscore{}}}\AgdaSpace{}%
\AgdaOperator{\AgdaInductiveConstructor{let`x=\AgdaUnderscore{}`in\AgdaUnderscore{}}}\AgdaSpace{}%
\AgdaOperator{\AgdaInductiveConstructor{inst`\AgdaUnderscore{}`=\AgdaUnderscore{}`in\AgdaUnderscore{}}}\AgdaSpace{}%
\AgdaOperator{\AgdaInductiveConstructor{∀`α\AgdaUnderscore{}}}\AgdaSpace{}%
\AgdaOperator{\AgdaInductiveConstructor{\AgdaUnderscore{}∶\AgdaUnderscore{}}}\<%
\\
\>[0]\AgdaKeyword{infixr}\AgdaSpace{}%
\AgdaNumber{5}\AgdaSpace{}%
\AgdaOperator{\AgdaInductiveConstructor{\AgdaUnderscore{}⇒\AgdaUnderscore{}}}\AgdaSpace{}%
\AgdaOperator{\AgdaInductiveConstructor{\AgdaUnderscore{}·\AgdaUnderscore{}}}\AgdaSpace{}%
\AgdaOperator{\AgdaInductiveConstructor{\AgdaUnderscore{}•\AgdaUnderscore{}}}\<%
\\
\>[0]\AgdaKeyword{infix}%
\>[7]\AgdaNumber{6}\AgdaSpace{}%
\AgdaOperator{\AgdaInductiveConstructor{`\AgdaUnderscore{}}}\AgdaSpace{}%
\AgdaOperator{\AgdaInductiveConstructor{decl`o`in\AgdaUnderscore{}}}\<%
\end{code}
\newcommand{\FoTerm}[0]{\begin{code}%
\>[0]\AgdaKeyword{data}\AgdaSpace{}%
\AgdaDatatype{Term}\AgdaSpace{}%
\AgdaSymbol{:}\AgdaSpace{}%
\AgdaFunction{Sorts}\AgdaSpace{}%
\AgdaSymbol{→}\AgdaSpace{}%
\AgdaDatatype{Sort}\AgdaSpace{}%
\AgdaGeneralizable{r}\AgdaSpace{}%
\AgdaSymbol{→}\AgdaSpace{}%
\AgdaPrimitive{Set}\AgdaSpace{}%
\AgdaKeyword{where}\<%
\\
\>[0][@{}l@{\AgdaIndent{0}}]%
\>[2]\AgdaOperator{\AgdaInductiveConstructor{decl`o`in\AgdaUnderscore{}}}%
\>[18]\AgdaSymbol{:}\AgdaSpace{}%
\AgdaDatatype{Term}\AgdaSpace{}%
\AgdaSymbol{(}\AgdaGeneralizable{S}\AgdaSpace{}%
\AgdaOperator{\AgdaInductiveConstructor{▷}}\AgdaSpace{}%
\AgdaInductiveConstructor{oₛ}\AgdaSymbol{)}\AgdaSpace{}%
\AgdaInductiveConstructor{eₛ}\AgdaSpace{}%
\AgdaSymbol{→}\AgdaSpace{}%
\AgdaDatatype{Term}\AgdaSpace{}%
\AgdaGeneralizable{S}\AgdaSpace{}%
\AgdaInductiveConstructor{eₛ}\<%
\\
%
\>[2]\AgdaOperator{\AgdaInductiveConstructor{inst`\AgdaUnderscore{}`=\AgdaUnderscore{}`in\AgdaUnderscore{}}}%
\>[18]\AgdaSymbol{:}\AgdaSpace{}%
\AgdaDatatype{Term}\AgdaSpace{}%
\AgdaGeneralizable{S}\AgdaSpace{}%
\AgdaInductiveConstructor{oₛ}\AgdaSpace{}%
\AgdaSymbol{→}\AgdaSpace{}%
\AgdaDatatype{Term}\AgdaSpace{}%
\AgdaGeneralizable{S}\AgdaSpace{}%
\AgdaInductiveConstructor{eₛ}\AgdaSpace{}%
\AgdaSymbol{→}\AgdaSpace{}%
\AgdaDatatype{Term}\AgdaSpace{}%
\AgdaGeneralizable{S}\AgdaSpace{}%
\AgdaInductiveConstructor{eₛ}\AgdaSpace{}%
\AgdaSymbol{→}\AgdaSpace{}%
\AgdaDatatype{Term}\AgdaSpace{}%
\AgdaGeneralizable{S}\AgdaSpace{}%
\AgdaInductiveConstructor{eₛ}\<%
\\
%
\>[2]\AgdaOperator{\AgdaInductiveConstructor{\AgdaUnderscore{}∶\AgdaUnderscore{}}}%
\>[18]\AgdaSymbol{:}\AgdaSpace{}%
\AgdaDatatype{Term}\AgdaSpace{}%
\AgdaGeneralizable{S}\AgdaSpace{}%
\AgdaInductiveConstructor{oₛ}\AgdaSpace{}%
\AgdaSymbol{→}\AgdaSpace{}%
\AgdaDatatype{Term}\AgdaSpace{}%
\AgdaGeneralizable{S}\AgdaSpace{}%
\AgdaInductiveConstructor{τₛ}\AgdaSpace{}%
\AgdaSymbol{→}\AgdaSpace{}%
\AgdaDatatype{Term}\AgdaSpace{}%
\AgdaGeneralizable{S}\AgdaSpace{}%
\AgdaInductiveConstructor{cₛ}\<%
\\
%
\>[2]\AgdaOperator{\AgdaInductiveConstructor{ƛ\AgdaUnderscore{}⇒\AgdaUnderscore{}}}%
\>[18]\AgdaSymbol{:}\AgdaSpace{}%
\AgdaDatatype{Term}\AgdaSpace{}%
\AgdaGeneralizable{S}\AgdaSpace{}%
\AgdaInductiveConstructor{cₛ}\AgdaSpace{}%
\AgdaSymbol{→}\AgdaSpace{}%
\AgdaDatatype{Term}\AgdaSpace{}%
\AgdaGeneralizable{S}\AgdaSpace{}%
\AgdaInductiveConstructor{eₛ}\AgdaSpace{}%
\AgdaSymbol{→}\AgdaSpace{}%
\AgdaDatatype{Term}\AgdaSpace{}%
\AgdaGeneralizable{S}\AgdaSpace{}%
\AgdaInductiveConstructor{eₛ}\<%
\\
%
\>[2]\AgdaOperator{\AgdaInductiveConstructor{[\AgdaUnderscore{}]⇒\AgdaUnderscore{}}}%
\>[18]\AgdaSymbol{:}\AgdaSpace{}%
\AgdaDatatype{Term}\AgdaSpace{}%
\AgdaGeneralizable{S}\AgdaSpace{}%
\AgdaInductiveConstructor{cₛ}\AgdaSpace{}%
\AgdaSymbol{→}\AgdaSpace{}%
\AgdaDatatype{Term}\AgdaSpace{}%
\AgdaGeneralizable{S}\AgdaSpace{}%
\AgdaInductiveConstructor{τₛ}\AgdaSpace{}%
\AgdaSymbol{→}\AgdaSpace{}%
\AgdaDatatype{Term}\AgdaSpace{}%
\AgdaGeneralizable{S}\AgdaSpace{}%
\AgdaInductiveConstructor{τₛ}\<%
\\
%
\>[2]\AgdaComment{--\ ...}\<%
\end{code}}
\begin{code}[hide]%
%
\>[2]\AgdaOperator{\AgdaInductiveConstructor{`\AgdaUnderscore{}}}%
\>[18]\AgdaSymbol{:}\AgdaSpace{}%
\AgdaGeneralizable{s}\AgdaSpace{}%
\AgdaOperator{\AgdaFunction{∈}}\AgdaSpace{}%
\AgdaGeneralizable{S}\AgdaSpace{}%
\AgdaSymbol{→}\AgdaSpace{}%
\AgdaDatatype{Term}\AgdaSpace{}%
\AgdaGeneralizable{S}\AgdaSpace{}%
\AgdaGeneralizable{s}\<%
\\
%
\>[2]\AgdaInductiveConstructor{tt}%
\>[18]\AgdaSymbol{:}\AgdaSpace{}%
\AgdaDatatype{Term}\AgdaSpace{}%
\AgdaGeneralizable{S}\AgdaSpace{}%
\AgdaInductiveConstructor{eₛ}\<%
\\
%
\>[2]\AgdaOperator{\AgdaInductiveConstructor{λ`x→\AgdaUnderscore{}}}%
\>[18]\AgdaSymbol{:}\AgdaSpace{}%
\AgdaDatatype{Term}\AgdaSpace{}%
\AgdaSymbol{(}\AgdaGeneralizable{S}\AgdaSpace{}%
\AgdaOperator{\AgdaInductiveConstructor{▷}}\AgdaSpace{}%
\AgdaInductiveConstructor{eₛ}\AgdaSymbol{)}\AgdaSpace{}%
\AgdaInductiveConstructor{eₛ}\AgdaSpace{}%
\AgdaSymbol{→}\AgdaSpace{}%
\AgdaDatatype{Term}\AgdaSpace{}%
\AgdaGeneralizable{S}\AgdaSpace{}%
\AgdaInductiveConstructor{eₛ}\<%
\\
%
\>[2]\AgdaOperator{\AgdaInductiveConstructor{Λ`α→\AgdaUnderscore{}}}%
\>[18]\AgdaSymbol{:}\AgdaSpace{}%
\AgdaDatatype{Term}\AgdaSpace{}%
\AgdaSymbol{(}\AgdaGeneralizable{S}\AgdaSpace{}%
\AgdaOperator{\AgdaInductiveConstructor{▷}}\AgdaSpace{}%
\AgdaInductiveConstructor{τₛ}\AgdaSymbol{)}\AgdaSpace{}%
\AgdaInductiveConstructor{eₛ}\AgdaSpace{}%
\AgdaSymbol{→}\AgdaSpace{}%
\AgdaDatatype{Term}\AgdaSpace{}%
\AgdaGeneralizable{S}\AgdaSpace{}%
\AgdaInductiveConstructor{eₛ}\<%
\\
%
\>[2]\AgdaOperator{\AgdaInductiveConstructor{\AgdaUnderscore{}·\AgdaUnderscore{}}}%
\>[18]\AgdaSymbol{:}\AgdaSpace{}%
\AgdaDatatype{Term}\AgdaSpace{}%
\AgdaGeneralizable{S}\AgdaSpace{}%
\AgdaInductiveConstructor{eₛ}\AgdaSpace{}%
\AgdaSymbol{→}\AgdaSpace{}%
\AgdaDatatype{Term}\AgdaSpace{}%
\AgdaGeneralizable{S}\AgdaSpace{}%
\AgdaInductiveConstructor{eₛ}\AgdaSpace{}%
\AgdaSymbol{→}\AgdaSpace{}%
\AgdaDatatype{Term}\AgdaSpace{}%
\AgdaGeneralizable{S}\AgdaSpace{}%
\AgdaInductiveConstructor{eₛ}\<%
\\
%
\>[2]\AgdaOperator{\AgdaInductiveConstructor{\AgdaUnderscore{}•\AgdaUnderscore{}}}%
\>[18]\AgdaSymbol{:}\AgdaSpace{}%
\AgdaDatatype{Term}\AgdaSpace{}%
\AgdaGeneralizable{S}\AgdaSpace{}%
\AgdaInductiveConstructor{eₛ}\AgdaSpace{}%
\AgdaSymbol{→}\AgdaSpace{}%
\AgdaDatatype{Term}\AgdaSpace{}%
\AgdaGeneralizable{S}\AgdaSpace{}%
\AgdaInductiveConstructor{τₛ}\AgdaSpace{}%
\AgdaSymbol{→}\AgdaSpace{}%
\AgdaDatatype{Term}\AgdaSpace{}%
\AgdaGeneralizable{S}\AgdaSpace{}%
\AgdaInductiveConstructor{eₛ}\<%
\\
%
\>[2]\AgdaOperator{\AgdaInductiveConstructor{let`x=\AgdaUnderscore{}`in\AgdaUnderscore{}}}%
\>[18]\AgdaSymbol{:}\AgdaSpace{}%
\AgdaDatatype{Term}\AgdaSpace{}%
\AgdaGeneralizable{S}\AgdaSpace{}%
\AgdaInductiveConstructor{eₛ}\AgdaSpace{}%
\AgdaSymbol{→}\AgdaSpace{}%
\AgdaDatatype{Term}\AgdaSpace{}%
\AgdaSymbol{(}\AgdaGeneralizable{S}\AgdaSpace{}%
\AgdaOperator{\AgdaInductiveConstructor{▷}}\AgdaSpace{}%
\AgdaInductiveConstructor{eₛ}\AgdaSymbol{)}\AgdaSpace{}%
\AgdaInductiveConstructor{eₛ}\AgdaSpace{}%
\AgdaSymbol{→}\AgdaSpace{}%
\AgdaDatatype{Term}\AgdaSpace{}%
\AgdaGeneralizable{S}\AgdaSpace{}%
\AgdaInductiveConstructor{eₛ}\<%
\\
%
\>[2]\AgdaInductiveConstructor{`⊤}%
\>[18]\AgdaSymbol{:}\AgdaSpace{}%
\AgdaDatatype{Term}\AgdaSpace{}%
\AgdaGeneralizable{S}\AgdaSpace{}%
\AgdaInductiveConstructor{τₛ}\<%
\\
%
\>[2]\AgdaOperator{\AgdaInductiveConstructor{\AgdaUnderscore{}⇒\AgdaUnderscore{}}}%
\>[18]\AgdaSymbol{:}\AgdaSpace{}%
\AgdaDatatype{Term}\AgdaSpace{}%
\AgdaGeneralizable{S}\AgdaSpace{}%
\AgdaInductiveConstructor{τₛ}\AgdaSpace{}%
\AgdaSymbol{→}\AgdaSpace{}%
\AgdaDatatype{Term}\AgdaSpace{}%
\AgdaGeneralizable{S}\AgdaSpace{}%
\AgdaInductiveConstructor{τₛ}\AgdaSpace{}%
\AgdaSymbol{→}\AgdaSpace{}%
\AgdaDatatype{Term}\AgdaSpace{}%
\AgdaGeneralizable{S}\AgdaSpace{}%
\AgdaInductiveConstructor{τₛ}\<%
\\
%
\>[2]\AgdaOperator{\AgdaInductiveConstructor{∀`α\AgdaUnderscore{}}}%
\>[18]\AgdaSymbol{:}\AgdaSpace{}%
\AgdaDatatype{Term}\AgdaSpace{}%
\AgdaSymbol{(}\AgdaGeneralizable{S}\AgdaSpace{}%
\AgdaOperator{\AgdaInductiveConstructor{▷}}\AgdaSpace{}%
\AgdaInductiveConstructor{τₛ}\AgdaSymbol{)}\AgdaSpace{}%
\AgdaInductiveConstructor{τₛ}\AgdaSpace{}%
\AgdaSymbol{→}\AgdaSpace{}%
\AgdaDatatype{Term}\AgdaSpace{}%
\AgdaGeneralizable{S}\AgdaSpace{}%
\AgdaInductiveConstructor{τₛ}\<%
\\
%
\>[2]\AgdaInductiveConstructor{⋆}%
\>[18]\AgdaSymbol{:}\AgdaSpace{}%
\AgdaDatatype{Term}\AgdaSpace{}%
\AgdaGeneralizable{S}\AgdaSpace{}%
\AgdaInductiveConstructor{κₛ}\<%
\\
%
\\[\AgdaEmptyExtraSkip]%
\>[0]\AgdaFunction{Expr}\AgdaSpace{}%
\AgdaSymbol{:}\AgdaSpace{}%
\AgdaFunction{Sorts}\AgdaSpace{}%
\AgdaSymbol{→}\AgdaSpace{}%
\AgdaPrimitive{Set}\<%
\\
\>[0]\AgdaFunction{Expr}\AgdaSpace{}%
\AgdaBound{S}\AgdaSpace{}%
\AgdaSymbol{=}\AgdaSpace{}%
\AgdaDatatype{Term}\AgdaSpace{}%
\AgdaBound{S}\AgdaSpace{}%
\AgdaInductiveConstructor{eₛ}\<%
\\
\>[0]\AgdaFunction{Cstr}\AgdaSpace{}%
\AgdaSymbol{:}\AgdaSpace{}%
\AgdaFunction{Sorts}\AgdaSpace{}%
\AgdaSymbol{→}\AgdaSpace{}%
\AgdaPrimitive{Set}\<%
\end{code}
\newcommand{\FoCstr}[0]{\begin{code}[inline]%
\>[0]\AgdaFunction{Cstr}\AgdaSpace{}%
\AgdaBound{S}\AgdaSpace{}%
\AgdaSymbol{=}\AgdaSpace{}%
\AgdaDatatype{Term}\AgdaSpace{}%
\AgdaBound{S}\AgdaSpace{}%
\AgdaInductiveConstructor{cₛ}\<%
\end{code}}
\begin{code}[hide]%
\>[0]\AgdaFunction{Type}\AgdaSpace{}%
\AgdaSymbol{:}\AgdaSpace{}%
\AgdaFunction{Sorts}\AgdaSpace{}%
\AgdaSymbol{→}\AgdaSpace{}%
\AgdaPrimitive{Set}\<%
\\
\>[0]\AgdaFunction{Type}\AgdaSpace{}%
\AgdaBound{S}\AgdaSpace{}%
\AgdaSymbol{=}\AgdaSpace{}%
\AgdaDatatype{Term}\AgdaSpace{}%
\AgdaBound{S}\AgdaSpace{}%
\AgdaInductiveConstructor{τₛ}\<%
\\
%
\\[\AgdaEmptyExtraSkip]%
\>[0]\AgdaKeyword{variable}\<%
\\
\>[0][@{}l@{\AgdaIndent{0}}]%
\>[2]\AgdaGeneralizable{t}\AgdaSpace{}%
\AgdaGeneralizable{t'}\AgdaSpace{}%
\AgdaGeneralizable{t''}\AgdaSpace{}%
\AgdaGeneralizable{t₁}\AgdaSpace{}%
\AgdaGeneralizable{t₂}\AgdaSpace{}%
\AgdaSymbol{:}\AgdaSpace{}%
\AgdaDatatype{Term}\AgdaSpace{}%
\AgdaGeneralizable{S}\AgdaSpace{}%
\AgdaGeneralizable{s}\<%
\\
%
\>[2]\AgdaGeneralizable{e}\AgdaSpace{}%
\AgdaGeneralizable{e'}\AgdaSpace{}%
\AgdaGeneralizable{e''}\AgdaSpace{}%
\AgdaGeneralizable{e₁}\AgdaSpace{}%
\AgdaGeneralizable{e₂}\AgdaSpace{}%
\AgdaSymbol{:}\AgdaSpace{}%
\AgdaFunction{Expr}\AgdaSpace{}%
\AgdaGeneralizable{S}\<%
\\
%
\>[2]\AgdaGeneralizable{c}\AgdaSpace{}%
\AgdaGeneralizable{c'}\AgdaSpace{}%
\AgdaGeneralizable{c''}\AgdaSpace{}%
\AgdaGeneralizable{c₁}\AgdaSpace{}%
\AgdaGeneralizable{c₂}\AgdaSpace{}%
\AgdaSymbol{:}\AgdaSpace{}%
\AgdaFunction{Cstr}\AgdaSpace{}%
\AgdaGeneralizable{S}\<%
\\
%
\>[2]\AgdaGeneralizable{τ}\AgdaSpace{}%
\AgdaGeneralizable{τ'}\AgdaSpace{}%
\AgdaGeneralizable{τ''}\AgdaSpace{}%
\AgdaGeneralizable{τ₁}\AgdaSpace{}%
\AgdaGeneralizable{τ₂}\AgdaSpace{}%
\AgdaSymbol{:}\AgdaSpace{}%
\AgdaFunction{Type}\AgdaSpace{}%
\AgdaGeneralizable{S}\<%
\\
\>[0]\<%
\\
\>[0]\AgdaComment{--\ Renaming\ -----------------------------------------------------------------------------}\<%
\\
%
\\[\AgdaEmptyExtraSkip]%
\>[0]\AgdaFunction{Ren}\AgdaSpace{}%
\AgdaSymbol{:}\AgdaSpace{}%
\AgdaFunction{Sorts}\AgdaSpace{}%
\AgdaSymbol{→}\AgdaSpace{}%
\AgdaFunction{Sorts}\AgdaSpace{}%
\AgdaSymbol{→}\AgdaSpace{}%
\AgdaPrimitive{Set}\<%
\\
\>[0]\AgdaFunction{Ren}\AgdaSpace{}%
\AgdaBound{S₁}\AgdaSpace{}%
\AgdaBound{S₂}\AgdaSpace{}%
\AgdaSymbol{=}\AgdaSpace{}%
\AgdaSymbol{∀}\AgdaSpace{}%
\AgdaSymbol{\{}\AgdaBound{s}\AgdaSymbol{\}}\AgdaSpace{}%
\AgdaSymbol{→}\AgdaSpace{}%
\AgdaFunction{Var}\AgdaSpace{}%
\AgdaBound{S₁}\AgdaSpace{}%
\AgdaBound{s}\AgdaSpace{}%
\AgdaSymbol{→}\AgdaSpace{}%
\AgdaFunction{Var}\AgdaSpace{}%
\AgdaBound{S₂}\AgdaSpace{}%
\AgdaBound{s}\<%
\\
%
\\[\AgdaEmptyExtraSkip]%
\>[0]\AgdaFunction{idᵣ}\AgdaSpace{}%
\AgdaSymbol{:}\AgdaSpace{}%
\AgdaFunction{Ren}\AgdaSpace{}%
\AgdaGeneralizable{S}\AgdaSpace{}%
\AgdaGeneralizable{S}\<%
\\
\>[0]\AgdaFunction{idᵣ}\AgdaSpace{}%
\AgdaSymbol{=}\AgdaSpace{}%
\AgdaFunction{id}\<%
\\
%
\\[\AgdaEmptyExtraSkip]%
\>[0]\AgdaFunction{wkᵣ}\AgdaSpace{}%
\AgdaSymbol{:}\AgdaSpace{}%
\AgdaFunction{Ren}\AgdaSpace{}%
\AgdaGeneralizable{S}\AgdaSpace{}%
\AgdaSymbol{(}\AgdaGeneralizable{S}\AgdaSpace{}%
\AgdaOperator{\AgdaInductiveConstructor{▷}}\AgdaSpace{}%
\AgdaGeneralizable{s}\AgdaSymbol{)}\<%
\\
\>[0]\AgdaFunction{wkᵣ}\AgdaSpace{}%
\AgdaSymbol{=}\AgdaSpace{}%
\AgdaInductiveConstructor{there}\<%
\\
%
\\[\AgdaEmptyExtraSkip]%
\>[0]\AgdaFunction{extᵣ}\AgdaSpace{}%
\AgdaSymbol{:}\AgdaSpace{}%
\AgdaFunction{Ren}\AgdaSpace{}%
\AgdaGeneralizable{S₁}\AgdaSpace{}%
\AgdaGeneralizable{S₂}\AgdaSpace{}%
\AgdaSymbol{→}\AgdaSpace{}%
\AgdaFunction{Ren}\AgdaSpace{}%
\AgdaSymbol{(}\AgdaGeneralizable{S₁}\AgdaSpace{}%
\AgdaOperator{\AgdaInductiveConstructor{▷}}\AgdaSpace{}%
\AgdaGeneralizable{s}\AgdaSymbol{)}\AgdaSpace{}%
\AgdaSymbol{(}\AgdaGeneralizable{S₂}\AgdaSpace{}%
\AgdaOperator{\AgdaInductiveConstructor{▷}}\AgdaSpace{}%
\AgdaGeneralizable{s}\AgdaSymbol{)}\<%
\\
\>[0]\AgdaFunction{extᵣ}\AgdaSpace{}%
\AgdaBound{ρ}\AgdaSpace{}%
\AgdaSymbol{(}\AgdaInductiveConstructor{here}\AgdaSpace{}%
\AgdaInductiveConstructor{refl}\AgdaSymbol{)}\AgdaSpace{}%
\AgdaSymbol{=}\AgdaSpace{}%
\AgdaInductiveConstructor{here}\AgdaSpace{}%
\AgdaInductiveConstructor{refl}\<%
\\
\>[0]\AgdaFunction{extᵣ}\AgdaSpace{}%
\AgdaBound{ρ}\AgdaSpace{}%
\AgdaSymbol{(}\AgdaInductiveConstructor{there}\AgdaSpace{}%
\AgdaBound{x}\AgdaSymbol{)}\AgdaSpace{}%
\AgdaSymbol{=}\AgdaSpace{}%
\AgdaInductiveConstructor{there}\AgdaSpace{}%
\AgdaSymbol{(}\AgdaBound{ρ}\AgdaSpace{}%
\AgdaBound{x}\AgdaSymbol{)}\<%
\\
%
\\[\AgdaEmptyExtraSkip]%
\>[0]\AgdaFunction{dropᵣ}\AgdaSpace{}%
\AgdaSymbol{:}\AgdaSpace{}%
\AgdaFunction{Ren}\AgdaSpace{}%
\AgdaGeneralizable{S₁}\AgdaSpace{}%
\AgdaGeneralizable{S₂}\AgdaSpace{}%
\AgdaSymbol{→}\AgdaSpace{}%
\AgdaFunction{Ren}\AgdaSpace{}%
\AgdaGeneralizable{S₁}\AgdaSpace{}%
\AgdaSymbol{(}\AgdaGeneralizable{S₂}\AgdaSpace{}%
\AgdaOperator{\AgdaInductiveConstructor{▷}}\AgdaSpace{}%
\AgdaGeneralizable{s}\AgdaSymbol{)}\<%
\\
\>[0]\AgdaFunction{dropᵣ}\AgdaSpace{}%
\AgdaBound{ρ}\AgdaSpace{}%
\AgdaBound{x}\AgdaSpace{}%
\AgdaSymbol{=}\AgdaSpace{}%
\AgdaInductiveConstructor{there}\AgdaSpace{}%
\AgdaSymbol{(}\AgdaBound{ρ}\AgdaSpace{}%
\AgdaBound{x}\AgdaSymbol{)}\<%
\\
%
\\[\AgdaEmptyExtraSkip]%
\>[0]\AgdaFunction{ren}\AgdaSpace{}%
\AgdaSymbol{:}\AgdaSpace{}%
\AgdaFunction{Ren}\AgdaSpace{}%
\AgdaGeneralizable{S₁}\AgdaSpace{}%
\AgdaGeneralizable{S₂}\AgdaSpace{}%
\AgdaSymbol{→}\AgdaSpace{}%
\AgdaSymbol{(}\AgdaDatatype{Term}\AgdaSpace{}%
\AgdaGeneralizable{S₁}\AgdaSpace{}%
\AgdaGeneralizable{s}\AgdaSpace{}%
\AgdaSymbol{→}\AgdaSpace{}%
\AgdaDatatype{Term}\AgdaSpace{}%
\AgdaGeneralizable{S₂}\AgdaSpace{}%
\AgdaGeneralizable{s}\AgdaSymbol{)}\<%
\\
\>[0]\AgdaFunction{ren}\AgdaSpace{}%
\AgdaBound{ρ}\AgdaSpace{}%
\AgdaSymbol{(}\AgdaOperator{\AgdaInductiveConstructor{`}}\AgdaSpace{}%
\AgdaBound{x}\AgdaSymbol{)}\AgdaSpace{}%
\AgdaSymbol{=}\AgdaSpace{}%
\AgdaOperator{\AgdaInductiveConstructor{`}}\AgdaSpace{}%
\AgdaSymbol{(}\AgdaBound{ρ}\AgdaSpace{}%
\AgdaBound{x}\AgdaSymbol{)}\<%
\\
\>[0]\AgdaFunction{ren}\AgdaSpace{}%
\AgdaBound{ρ}\AgdaSpace{}%
\AgdaInductiveConstructor{tt}\AgdaSpace{}%
\AgdaSymbol{=}\AgdaSpace{}%
\AgdaInductiveConstructor{tt}\<%
\\
\>[0]\AgdaFunction{ren}\AgdaSpace{}%
\AgdaBound{ρ}\AgdaSpace{}%
\AgdaSymbol{(}\AgdaOperator{\AgdaInductiveConstructor{λ`x→}}\AgdaSpace{}%
\AgdaBound{e}\AgdaSymbol{)}\AgdaSpace{}%
\AgdaSymbol{=}\AgdaSpace{}%
\AgdaOperator{\AgdaInductiveConstructor{λ`x→}}\AgdaSpace{}%
\AgdaSymbol{(}\AgdaFunction{ren}\AgdaSpace{}%
\AgdaSymbol{(}\AgdaFunction{extᵣ}\AgdaSpace{}%
\AgdaBound{ρ}\AgdaSymbol{)}\AgdaSpace{}%
\AgdaBound{e}\AgdaSymbol{)}\<%
\\
\>[0]\AgdaFunction{ren}\AgdaSpace{}%
\AgdaBound{ρ}\AgdaSpace{}%
\AgdaSymbol{(}\AgdaOperator{\AgdaInductiveConstructor{Λ`α→}}\AgdaSpace{}%
\AgdaBound{e}\AgdaSymbol{)}\AgdaSpace{}%
\AgdaSymbol{=}\AgdaSpace{}%
\AgdaOperator{\AgdaInductiveConstructor{Λ`α→}}\AgdaSpace{}%
\AgdaSymbol{(}\AgdaFunction{ren}\AgdaSpace{}%
\AgdaSymbol{(}\AgdaFunction{extᵣ}\AgdaSpace{}%
\AgdaBound{ρ}\AgdaSymbol{)}\AgdaSpace{}%
\AgdaBound{e}\AgdaSymbol{)}\<%
\\
\>[0]\AgdaFunction{ren}\AgdaSpace{}%
\AgdaBound{ρ}\AgdaSpace{}%
\AgdaSymbol{(}\AgdaOperator{\AgdaInductiveConstructor{ƛ}}\AgdaSpace{}%
\AgdaBound{c}\AgdaSpace{}%
\AgdaOperator{\AgdaInductiveConstructor{⇒}}\AgdaSpace{}%
\AgdaBound{e}\AgdaSymbol{)}\AgdaSpace{}%
\AgdaSymbol{=}\AgdaSpace{}%
\AgdaOperator{\AgdaInductiveConstructor{ƛ}}\AgdaSpace{}%
\AgdaFunction{ren}\AgdaSpace{}%
\AgdaBound{ρ}\AgdaSpace{}%
\AgdaBound{c}\AgdaSpace{}%
\AgdaOperator{\AgdaInductiveConstructor{⇒}}\AgdaSpace{}%
\AgdaFunction{ren}\AgdaSpace{}%
\AgdaBound{ρ}\AgdaSpace{}%
\AgdaBound{e}\<%
\\
\>[0]\AgdaFunction{ren}\AgdaSpace{}%
\AgdaBound{ρ}\AgdaSpace{}%
\AgdaSymbol{(}\AgdaBound{e₁}\AgdaSpace{}%
\AgdaOperator{\AgdaInductiveConstructor{·}}\AgdaSpace{}%
\AgdaBound{e₂}\AgdaSymbol{)}\AgdaSpace{}%
\AgdaSymbol{=}\AgdaSpace{}%
\AgdaSymbol{(}\AgdaFunction{ren}\AgdaSpace{}%
\AgdaBound{ρ}\AgdaSpace{}%
\AgdaBound{e₁}\AgdaSymbol{)}\AgdaSpace{}%
\AgdaOperator{\AgdaInductiveConstructor{·}}\AgdaSpace{}%
\AgdaSymbol{(}\AgdaFunction{ren}\AgdaSpace{}%
\AgdaBound{ρ}\AgdaSpace{}%
\AgdaBound{e₂}\AgdaSymbol{)}\<%
\\
\>[0]\AgdaFunction{ren}\AgdaSpace{}%
\AgdaBound{ρ}\AgdaSpace{}%
\AgdaSymbol{(}\AgdaBound{e}\AgdaSpace{}%
\AgdaOperator{\AgdaInductiveConstructor{•}}\AgdaSpace{}%
\AgdaBound{τ}\AgdaSymbol{)}\AgdaSpace{}%
\AgdaSymbol{=}\AgdaSpace{}%
\AgdaSymbol{(}\AgdaFunction{ren}\AgdaSpace{}%
\AgdaBound{ρ}\AgdaSpace{}%
\AgdaBound{e}\AgdaSymbol{)}\AgdaSpace{}%
\AgdaOperator{\AgdaInductiveConstructor{•}}\AgdaSpace{}%
\AgdaSymbol{(}\AgdaFunction{ren}\AgdaSpace{}%
\AgdaBound{ρ}\AgdaSpace{}%
\AgdaBound{τ}\AgdaSymbol{)}\<%
\\
\>[0]\AgdaFunction{ren}\AgdaSpace{}%
\AgdaBound{ρ}\AgdaSpace{}%
\AgdaSymbol{(}\AgdaOperator{\AgdaInductiveConstructor{let`x=}}\AgdaSpace{}%
\AgdaBound{e₂}\AgdaSpace{}%
\AgdaOperator{\AgdaInductiveConstructor{`in}}\AgdaSpace{}%
\AgdaBound{e₁}\AgdaSymbol{)}\AgdaSpace{}%
\AgdaSymbol{=}\AgdaSpace{}%
\AgdaOperator{\AgdaInductiveConstructor{let`x=}}\AgdaSpace{}%
\AgdaSymbol{(}\AgdaFunction{ren}\AgdaSpace{}%
\AgdaBound{ρ}\AgdaSpace{}%
\AgdaBound{e₂}\AgdaSymbol{)}\AgdaSpace{}%
\AgdaOperator{\AgdaInductiveConstructor{`in}}\AgdaSpace{}%
\AgdaFunction{ren}\AgdaSpace{}%
\AgdaSymbol{(}\AgdaFunction{extᵣ}\AgdaSpace{}%
\AgdaBound{ρ}\AgdaSymbol{)}\AgdaSpace{}%
\AgdaBound{e₁}\<%
\\
\>[0]\AgdaFunction{ren}\AgdaSpace{}%
\AgdaBound{ρ}\AgdaSpace{}%
\AgdaSymbol{(}\AgdaOperator{\AgdaInductiveConstructor{decl`o`in}}\AgdaSpace{}%
\AgdaBound{e}\AgdaSymbol{)}\AgdaSpace{}%
\AgdaSymbol{=}\AgdaSpace{}%
\AgdaOperator{\AgdaInductiveConstructor{decl`o`in}}\AgdaSpace{}%
\AgdaFunction{ren}\AgdaSpace{}%
\AgdaSymbol{(}\AgdaFunction{extᵣ}\AgdaSpace{}%
\AgdaBound{ρ}\AgdaSymbol{)}\AgdaSpace{}%
\AgdaBound{e}\<%
\\
\>[0]\AgdaFunction{ren}\AgdaSpace{}%
\AgdaBound{ρ}\AgdaSpace{}%
\AgdaSymbol{(}\AgdaOperator{\AgdaInductiveConstructor{inst`}}\AgdaSpace{}%
\AgdaBound{o}\AgdaSpace{}%
\AgdaOperator{\AgdaInductiveConstructor{`=}}\AgdaSpace{}%
\AgdaBound{e₂}\AgdaSpace{}%
\AgdaOperator{\AgdaInductiveConstructor{`in}}\AgdaSpace{}%
\AgdaBound{e₁}\AgdaSymbol{)}\AgdaSpace{}%
\AgdaSymbol{=}\AgdaSpace{}%
\AgdaOperator{\AgdaInductiveConstructor{inst`}}\AgdaSpace{}%
\AgdaSymbol{(}\AgdaFunction{ren}\AgdaSpace{}%
\AgdaBound{ρ}\AgdaSpace{}%
\AgdaBound{o}\AgdaSymbol{)}\AgdaSpace{}%
\AgdaOperator{\AgdaInductiveConstructor{`=}}%
\>[51]\AgdaFunction{ren}\AgdaSpace{}%
\AgdaBound{ρ}\AgdaSpace{}%
\AgdaBound{e₂}\AgdaSpace{}%
\AgdaOperator{\AgdaInductiveConstructor{`in}}\AgdaSpace{}%
\AgdaFunction{ren}\AgdaSpace{}%
\AgdaBound{ρ}\AgdaSpace{}%
\AgdaBound{e₁}\<%
\\
\>[0]\AgdaFunction{ren}\AgdaSpace{}%
\AgdaBound{ρ}\AgdaSpace{}%
\AgdaSymbol{(}\AgdaBound{o}\AgdaSpace{}%
\AgdaOperator{\AgdaInductiveConstructor{∶}}\AgdaSpace{}%
\AgdaBound{τ}\AgdaSymbol{)}\AgdaSpace{}%
\AgdaSymbol{=}\AgdaSpace{}%
\AgdaFunction{ren}\AgdaSpace{}%
\AgdaBound{ρ}\AgdaSpace{}%
\AgdaBound{o}\AgdaSpace{}%
\AgdaOperator{\AgdaInductiveConstructor{∶}}\AgdaSpace{}%
\AgdaFunction{ren}\AgdaSpace{}%
\AgdaBound{ρ}\AgdaSpace{}%
\AgdaBound{τ}\<%
\\
\>[0]\AgdaFunction{ren}\AgdaSpace{}%
\AgdaBound{ρ}\AgdaSpace{}%
\AgdaInductiveConstructor{`⊤}\AgdaSpace{}%
\AgdaSymbol{=}\AgdaSpace{}%
\AgdaInductiveConstructor{`⊤}\<%
\\
\>[0]\AgdaFunction{ren}\AgdaSpace{}%
\AgdaBound{ρ}\AgdaSpace{}%
\AgdaSymbol{(}\AgdaBound{τ₁}\AgdaSpace{}%
\AgdaOperator{\AgdaInductiveConstructor{⇒}}\AgdaSpace{}%
\AgdaBound{τ₂}\AgdaSymbol{)}\AgdaSpace{}%
\AgdaSymbol{=}\AgdaSpace{}%
\AgdaFunction{ren}\AgdaSpace{}%
\AgdaBound{ρ}\AgdaSpace{}%
\AgdaBound{τ₁}\AgdaSpace{}%
\AgdaOperator{\AgdaInductiveConstructor{⇒}}\AgdaSpace{}%
\AgdaFunction{ren}\AgdaSpace{}%
\AgdaBound{ρ}\AgdaSpace{}%
\AgdaBound{τ₂}\<%
\\
\>[0]\AgdaFunction{ren}\AgdaSpace{}%
\AgdaBound{ρ}\AgdaSpace{}%
\AgdaSymbol{(}\AgdaOperator{\AgdaInductiveConstructor{∀`α}}\AgdaSpace{}%
\AgdaBound{τ}\AgdaSymbol{)}\AgdaSpace{}%
\AgdaSymbol{=}\AgdaSpace{}%
\AgdaOperator{\AgdaInductiveConstructor{∀`α}}\AgdaSpace{}%
\AgdaSymbol{(}\AgdaFunction{ren}\AgdaSpace{}%
\AgdaSymbol{(}\AgdaFunction{extᵣ}\AgdaSpace{}%
\AgdaBound{ρ}\AgdaSymbol{)}\AgdaSpace{}%
\AgdaBound{τ}\AgdaSymbol{)}\<%
\\
\>[0]\AgdaFunction{ren}\AgdaSpace{}%
\AgdaBound{ρ}\AgdaSpace{}%
\AgdaSymbol{(}\AgdaOperator{\AgdaInductiveConstructor{[}}\AgdaSpace{}%
\AgdaBound{c}\AgdaSpace{}%
\AgdaOperator{\AgdaInductiveConstructor{]⇒}}\AgdaSpace{}%
\AgdaBound{τ}\AgdaSymbol{)}\AgdaSpace{}%
\AgdaSymbol{=}\AgdaSpace{}%
\AgdaOperator{\AgdaInductiveConstructor{[}}\AgdaSpace{}%
\AgdaFunction{ren}\AgdaSpace{}%
\AgdaBound{ρ}\AgdaSpace{}%
\AgdaBound{c}\AgdaSpace{}%
\AgdaOperator{\AgdaInductiveConstructor{]⇒}}\AgdaSpace{}%
\AgdaSymbol{(}\AgdaFunction{ren}\AgdaSpace{}%
\AgdaBound{ρ}\AgdaSpace{}%
\AgdaBound{τ}\AgdaSymbol{)}\<%
\\
\>[0]\AgdaFunction{ren}\AgdaSpace{}%
\AgdaBound{ρ}\AgdaSpace{}%
\AgdaInductiveConstructor{⋆}\AgdaSpace{}%
\AgdaSymbol{=}\AgdaSpace{}%
\AgdaInductiveConstructor{⋆}\<%
\\
%
\\[\AgdaEmptyExtraSkip]%
\>[0]\AgdaFunction{wk}\AgdaSpace{}%
\AgdaSymbol{:}\AgdaSpace{}%
\AgdaDatatype{Term}\AgdaSpace{}%
\AgdaGeneralizable{S}\AgdaSpace{}%
\AgdaGeneralizable{s}\AgdaSpace{}%
\AgdaSymbol{→}\AgdaSpace{}%
\AgdaDatatype{Term}\AgdaSpace{}%
\AgdaSymbol{(}\AgdaGeneralizable{S}\AgdaSpace{}%
\AgdaOperator{\AgdaInductiveConstructor{▷}}\AgdaSpace{}%
\AgdaGeneralizable{s'}\AgdaSymbol{)}\AgdaSpace{}%
\AgdaGeneralizable{s}\<%
\\
\>[0]\AgdaFunction{wk}\AgdaSpace{}%
\AgdaSymbol{=}\AgdaSpace{}%
\AgdaFunction{ren}\AgdaSpace{}%
\AgdaInductiveConstructor{there}\<%
\\
%
\\[\AgdaEmptyExtraSkip]%
\>[0]\AgdaKeyword{variable}\<%
\\
\>[0][@{}l@{\AgdaIndent{0}}]%
\>[2]\AgdaGeneralizable{ρ}\AgdaSpace{}%
\AgdaGeneralizable{ρ'}\AgdaSpace{}%
\AgdaGeneralizable{ρ''}\AgdaSpace{}%
\AgdaGeneralizable{ρ₁}\AgdaSpace{}%
\AgdaGeneralizable{ρ₂}\AgdaSpace{}%
\AgdaSymbol{:}\AgdaSpace{}%
\AgdaFunction{Ren}\AgdaSpace{}%
\AgdaGeneralizable{S₁}\AgdaSpace{}%
\AgdaGeneralizable{S₂}\<%
\\
%
\\[\AgdaEmptyExtraSkip]%
\>[0]\AgdaComment{--\ Substitution\ -------------------------------------------------------------------------}\<%
\\
%
\\[\AgdaEmptyExtraSkip]%
\>[0]\AgdaFunction{Sub}\AgdaSpace{}%
\AgdaSymbol{:}\AgdaSpace{}%
\AgdaFunction{Sorts}\AgdaSpace{}%
\AgdaSymbol{→}\AgdaSpace{}%
\AgdaFunction{Sorts}\AgdaSpace{}%
\AgdaSymbol{→}\AgdaSpace{}%
\AgdaPrimitive{Set}\<%
\\
\>[0]\AgdaFunction{Sub}\AgdaSpace{}%
\AgdaBound{S₁}\AgdaSpace{}%
\AgdaBound{S₂}\AgdaSpace{}%
\AgdaSymbol{=}\AgdaSpace{}%
\AgdaSymbol{∀}\AgdaSpace{}%
\AgdaSymbol{\{}\AgdaBound{s}\AgdaSymbol{\}}\AgdaSpace{}%
\AgdaSymbol{→}\AgdaSpace{}%
\AgdaFunction{Var}\AgdaSpace{}%
\AgdaBound{S₁}\AgdaSpace{}%
\AgdaBound{s}\AgdaSpace{}%
\AgdaSymbol{→}\AgdaSpace{}%
\AgdaDatatype{Term}\AgdaSpace{}%
\AgdaBound{S₂}\AgdaSpace{}%
\AgdaBound{s}\<%
\\
%
\\[\AgdaEmptyExtraSkip]%
\>[0]\AgdaFunction{idₛ}\AgdaSpace{}%
\AgdaSymbol{:}\AgdaSpace{}%
\AgdaFunction{Sub}\AgdaSpace{}%
\AgdaGeneralizable{S}\AgdaSpace{}%
\AgdaGeneralizable{S}\<%
\\
\>[0]\AgdaFunction{idₛ}\AgdaSpace{}%
\AgdaSymbol{=}\AgdaSpace{}%
\AgdaOperator{\AgdaInductiveConstructor{`\AgdaUnderscore{}}}\<%
\\
%
\\[\AgdaEmptyExtraSkip]%
\>[0]\AgdaFunction{extₛ}\AgdaSpace{}%
\AgdaSymbol{:}\AgdaSpace{}%
\AgdaFunction{Sub}\AgdaSpace{}%
\AgdaGeneralizable{S₁}\AgdaSpace{}%
\AgdaGeneralizable{S₂}\AgdaSpace{}%
\AgdaSymbol{→}\AgdaSpace{}%
\AgdaFunction{Sub}\AgdaSpace{}%
\AgdaSymbol{(}\AgdaGeneralizable{S₁}\AgdaSpace{}%
\AgdaOperator{\AgdaInductiveConstructor{▷}}\AgdaSpace{}%
\AgdaGeneralizable{s}\AgdaSymbol{)}\AgdaSpace{}%
\AgdaSymbol{(}\AgdaGeneralizable{S₂}\AgdaSpace{}%
\AgdaOperator{\AgdaInductiveConstructor{▷}}\AgdaSpace{}%
\AgdaGeneralizable{s}\AgdaSymbol{)}\<%
\\
\>[0]\AgdaFunction{extₛ}\AgdaSpace{}%
\AgdaBound{σ}\AgdaSpace{}%
\AgdaSymbol{(}\AgdaInductiveConstructor{here}\AgdaSpace{}%
\AgdaInductiveConstructor{refl}\AgdaSymbol{)}\AgdaSpace{}%
\AgdaSymbol{=}\AgdaSpace{}%
\AgdaOperator{\AgdaInductiveConstructor{`}}\AgdaSpace{}%
\AgdaInductiveConstructor{here}\AgdaSpace{}%
\AgdaInductiveConstructor{refl}\<%
\\
\>[0]\AgdaFunction{extₛ}\AgdaSpace{}%
\AgdaBound{σ}\AgdaSpace{}%
\AgdaSymbol{(}\AgdaInductiveConstructor{there}\AgdaSpace{}%
\AgdaBound{x}\AgdaSymbol{)}\AgdaSpace{}%
\AgdaSymbol{=}\AgdaSpace{}%
\AgdaFunction{ren}\AgdaSpace{}%
\AgdaFunction{wkᵣ}\AgdaSpace{}%
\AgdaSymbol{(}\AgdaBound{σ}\AgdaSpace{}%
\AgdaBound{x}\AgdaSymbol{)}\<%
\\
%
\\[\AgdaEmptyExtraSkip]%
\>[0]\AgdaFunction{dropₛ}\AgdaSpace{}%
\AgdaSymbol{:}\AgdaSpace{}%
\AgdaFunction{Sub}\AgdaSpace{}%
\AgdaGeneralizable{S₁}\AgdaSpace{}%
\AgdaGeneralizable{S₂}\AgdaSpace{}%
\AgdaSymbol{→}\AgdaSpace{}%
\AgdaFunction{Sub}\AgdaSpace{}%
\AgdaGeneralizable{S₁}\AgdaSpace{}%
\AgdaSymbol{(}\AgdaGeneralizable{S₂}\AgdaSpace{}%
\AgdaOperator{\AgdaInductiveConstructor{▷}}\AgdaSpace{}%
\AgdaGeneralizable{s}\AgdaSymbol{)}\<%
\\
\>[0]\AgdaFunction{dropₛ}\AgdaSpace{}%
\AgdaBound{σ}\AgdaSpace{}%
\AgdaBound{x}\AgdaSpace{}%
\AgdaSymbol{=}\AgdaSpace{}%
\AgdaFunction{wk}\AgdaSpace{}%
\AgdaSymbol{(}\AgdaBound{σ}\AgdaSpace{}%
\AgdaBound{x}\AgdaSymbol{)}\<%
\\
%
\\[\AgdaEmptyExtraSkip]%
\>[0]\AgdaFunction{single-typeₛ}\AgdaSpace{}%
\AgdaSymbol{:}\AgdaSpace{}%
\AgdaFunction{Sub}\AgdaSpace{}%
\AgdaGeneralizable{S₁}\AgdaSpace{}%
\AgdaGeneralizable{S₂}\AgdaSpace{}%
\AgdaSymbol{→}\AgdaSpace{}%
\AgdaFunction{Type}\AgdaSpace{}%
\AgdaGeneralizable{S₂}\AgdaSpace{}%
\AgdaSymbol{→}\AgdaSpace{}%
\AgdaFunction{Sub}\AgdaSpace{}%
\AgdaSymbol{(}\AgdaGeneralizable{S₁}\AgdaSpace{}%
\AgdaOperator{\AgdaInductiveConstructor{▷}}\AgdaSpace{}%
\AgdaInductiveConstructor{τₛ}\AgdaSymbol{)}\AgdaSpace{}%
\AgdaGeneralizable{S₂}\<%
\\
\>[0]\AgdaFunction{single-typeₛ}\AgdaSpace{}%
\AgdaBound{σ}\AgdaSpace{}%
\AgdaBound{τ}\AgdaSpace{}%
\AgdaSymbol{(}\AgdaInductiveConstructor{here}\AgdaSpace{}%
\AgdaInductiveConstructor{refl}\AgdaSymbol{)}\AgdaSpace{}%
\AgdaSymbol{=}\AgdaSpace{}%
\AgdaBound{τ}\<%
\\
\>[0]\AgdaFunction{single-typeₛ}\AgdaSpace{}%
\AgdaBound{σ}\AgdaSpace{}%
\AgdaBound{τ}\AgdaSpace{}%
\AgdaSymbol{(}\AgdaInductiveConstructor{there}\AgdaSpace{}%
\AgdaBound{x}\AgdaSymbol{)}\AgdaSpace{}%
\AgdaSymbol{=}\AgdaSpace{}%
\AgdaBound{σ}\AgdaSpace{}%
\AgdaBound{x}\<%
\\
%
\\[\AgdaEmptyExtraSkip]%
\>[0]\AgdaFunction{sub}\AgdaSpace{}%
\AgdaSymbol{:}\AgdaSpace{}%
\AgdaFunction{Sub}\AgdaSpace{}%
\AgdaGeneralizable{S₁}\AgdaSpace{}%
\AgdaGeneralizable{S₂}\AgdaSpace{}%
\AgdaSymbol{→}\AgdaSpace{}%
\AgdaSymbol{(}\AgdaDatatype{Term}\AgdaSpace{}%
\AgdaGeneralizable{S₁}\AgdaSpace{}%
\AgdaGeneralizable{s}\AgdaSpace{}%
\AgdaSymbol{→}\AgdaSpace{}%
\AgdaDatatype{Term}\AgdaSpace{}%
\AgdaGeneralizable{S₂}\AgdaSpace{}%
\AgdaGeneralizable{s}\AgdaSymbol{)}\<%
\\
\>[0]\AgdaFunction{sub}\AgdaSpace{}%
\AgdaBound{σ}\AgdaSpace{}%
\AgdaSymbol{(}\AgdaOperator{\AgdaInductiveConstructor{`}}\AgdaSpace{}%
\AgdaBound{x}\AgdaSymbol{)}\AgdaSpace{}%
\AgdaSymbol{=}\AgdaSpace{}%
\AgdaSymbol{(}\AgdaBound{σ}\AgdaSpace{}%
\AgdaBound{x}\AgdaSymbol{)}\<%
\\
\>[0]\AgdaFunction{sub}\AgdaSpace{}%
\AgdaBound{σ}\AgdaSpace{}%
\AgdaInductiveConstructor{tt}\AgdaSpace{}%
\AgdaSymbol{=}\AgdaSpace{}%
\AgdaInductiveConstructor{tt}\<%
\\
\>[0]\AgdaFunction{sub}\AgdaSpace{}%
\AgdaBound{σ}\AgdaSpace{}%
\AgdaSymbol{(}\AgdaOperator{\AgdaInductiveConstructor{λ`x→}}\AgdaSpace{}%
\AgdaBound{e}\AgdaSymbol{)}\AgdaSpace{}%
\AgdaSymbol{=}\AgdaSpace{}%
\AgdaOperator{\AgdaInductiveConstructor{λ`x→}}\AgdaSpace{}%
\AgdaSymbol{(}\AgdaFunction{sub}\AgdaSpace{}%
\AgdaSymbol{(}\AgdaFunction{extₛ}\AgdaSpace{}%
\AgdaBound{σ}\AgdaSymbol{)}\AgdaSpace{}%
\AgdaBound{e}\AgdaSymbol{)}\<%
\\
\>[0]\AgdaFunction{sub}\AgdaSpace{}%
\AgdaBound{σ}\AgdaSpace{}%
\AgdaSymbol{(}\AgdaOperator{\AgdaInductiveConstructor{Λ`α→}}\AgdaSpace{}%
\AgdaBound{e}\AgdaSymbol{)}\AgdaSpace{}%
\AgdaSymbol{=}\AgdaSpace{}%
\AgdaOperator{\AgdaInductiveConstructor{Λ`α→}}\AgdaSpace{}%
\AgdaSymbol{(}\AgdaFunction{sub}\AgdaSpace{}%
\AgdaSymbol{(}\AgdaFunction{extₛ}\AgdaSpace{}%
\AgdaBound{σ}\AgdaSymbol{)}\AgdaSpace{}%
\AgdaBound{e}\AgdaSymbol{)}\<%
\\
\>[0]\AgdaFunction{sub}\AgdaSpace{}%
\AgdaBound{σ}\AgdaSpace{}%
\AgdaSymbol{(}\AgdaOperator{\AgdaInductiveConstructor{ƛ}}\AgdaSpace{}%
\AgdaBound{c}\AgdaSpace{}%
\AgdaOperator{\AgdaInductiveConstructor{⇒}}\AgdaSpace{}%
\AgdaBound{e}\AgdaSymbol{)}\AgdaSpace{}%
\AgdaSymbol{=}\AgdaSpace{}%
\AgdaOperator{\AgdaInductiveConstructor{ƛ}}\AgdaSpace{}%
\AgdaFunction{sub}\AgdaSpace{}%
\AgdaBound{σ}\AgdaSpace{}%
\AgdaBound{c}\AgdaSpace{}%
\AgdaOperator{\AgdaInductiveConstructor{⇒}}\AgdaSpace{}%
\AgdaFunction{sub}\AgdaSpace{}%
\AgdaBound{σ}\AgdaSpace{}%
\AgdaBound{e}\<%
\\
\>[0]\AgdaFunction{sub}\AgdaSpace{}%
\AgdaBound{σ}\AgdaSpace{}%
\AgdaSymbol{(}\AgdaBound{e₁}\AgdaSpace{}%
\AgdaOperator{\AgdaInductiveConstructor{·}}\AgdaSpace{}%
\AgdaBound{e₂}\AgdaSymbol{)}\AgdaSpace{}%
\AgdaSymbol{=}\AgdaSpace{}%
\AgdaFunction{sub}\AgdaSpace{}%
\AgdaBound{σ}\AgdaSpace{}%
\AgdaBound{e₁}\AgdaSpace{}%
\AgdaOperator{\AgdaInductiveConstructor{·}}\AgdaSpace{}%
\AgdaFunction{sub}\AgdaSpace{}%
\AgdaBound{σ}\AgdaSpace{}%
\AgdaBound{e₂}\<%
\\
\>[0]\AgdaFunction{sub}\AgdaSpace{}%
\AgdaBound{σ}\AgdaSpace{}%
\AgdaSymbol{(}\AgdaBound{e}\AgdaSpace{}%
\AgdaOperator{\AgdaInductiveConstructor{•}}\AgdaSpace{}%
\AgdaBound{τ}\AgdaSymbol{)}\AgdaSpace{}%
\AgdaSymbol{=}\AgdaSpace{}%
\AgdaFunction{sub}\AgdaSpace{}%
\AgdaBound{σ}\AgdaSpace{}%
\AgdaBound{e}\AgdaSpace{}%
\AgdaOperator{\AgdaInductiveConstructor{•}}\AgdaSpace{}%
\AgdaFunction{sub}\AgdaSpace{}%
\AgdaBound{σ}\AgdaSpace{}%
\AgdaBound{τ}\<%
\\
\>[0]\AgdaFunction{sub}\AgdaSpace{}%
\AgdaBound{σ}\AgdaSpace{}%
\AgdaSymbol{(}\AgdaOperator{\AgdaInductiveConstructor{let`x=}}\AgdaSpace{}%
\AgdaBound{e₂}\AgdaSpace{}%
\AgdaOperator{\AgdaInductiveConstructor{`in}}\AgdaSpace{}%
\AgdaBound{e₁}\AgdaSymbol{)}\AgdaSpace{}%
\AgdaSymbol{=}\AgdaSpace{}%
\AgdaOperator{\AgdaInductiveConstructor{let`x=}}\AgdaSpace{}%
\AgdaFunction{sub}\AgdaSpace{}%
\AgdaBound{σ}\AgdaSpace{}%
\AgdaBound{e₂}\AgdaSpace{}%
\AgdaOperator{\AgdaInductiveConstructor{`in}}\AgdaSpace{}%
\AgdaSymbol{(}\AgdaFunction{sub}\AgdaSpace{}%
\AgdaSymbol{(}\AgdaFunction{extₛ}\AgdaSpace{}%
\AgdaBound{σ}\AgdaSymbol{)}\AgdaSpace{}%
\AgdaBound{e₁}\AgdaSymbol{)}\<%
\\
\>[0]\AgdaFunction{sub}\AgdaSpace{}%
\AgdaBound{σ}\AgdaSpace{}%
\AgdaSymbol{(}\AgdaOperator{\AgdaInductiveConstructor{decl`o`in}}\AgdaSpace{}%
\AgdaBound{e}\AgdaSymbol{)}\AgdaSpace{}%
\AgdaSymbol{=}\AgdaSpace{}%
\AgdaOperator{\AgdaInductiveConstructor{decl`o`in}}\AgdaSpace{}%
\AgdaFunction{sub}\AgdaSpace{}%
\AgdaSymbol{(}\AgdaFunction{extₛ}\AgdaSpace{}%
\AgdaBound{σ}\AgdaSymbol{)}\AgdaSpace{}%
\AgdaBound{e}\<%
\\
\>[0]\AgdaFunction{sub}\AgdaSpace{}%
\AgdaBound{σ}\AgdaSpace{}%
\AgdaSymbol{(}\AgdaOperator{\AgdaInductiveConstructor{inst`}}\AgdaSpace{}%
\AgdaBound{o}\AgdaSpace{}%
\AgdaOperator{\AgdaInductiveConstructor{`=}}\AgdaSpace{}%
\AgdaBound{e₂}\AgdaSpace{}%
\AgdaOperator{\AgdaInductiveConstructor{`in}}\AgdaSpace{}%
\AgdaBound{e₁}\AgdaSymbol{)}\AgdaSpace{}%
\AgdaSymbol{=}\AgdaSpace{}%
\AgdaOperator{\AgdaInductiveConstructor{inst`}}\AgdaSpace{}%
\AgdaFunction{sub}\AgdaSpace{}%
\AgdaBound{σ}\AgdaSpace{}%
\AgdaBound{o}\AgdaSpace{}%
\AgdaOperator{\AgdaInductiveConstructor{`=}}\AgdaSpace{}%
\AgdaFunction{sub}\AgdaSpace{}%
\AgdaBound{σ}\AgdaSpace{}%
\AgdaBound{e₂}\AgdaSpace{}%
\AgdaOperator{\AgdaInductiveConstructor{`in}}\AgdaSpace{}%
\AgdaFunction{sub}\AgdaSpace{}%
\AgdaBound{σ}\AgdaSpace{}%
\AgdaBound{e₁}\<%
\\
\>[0]\AgdaFunction{sub}\AgdaSpace{}%
\AgdaBound{σ}\AgdaSpace{}%
\AgdaSymbol{(}\AgdaBound{o}\AgdaSpace{}%
\AgdaOperator{\AgdaInductiveConstructor{∶}}\AgdaSpace{}%
\AgdaBound{τ}\AgdaSymbol{)}\AgdaSpace{}%
\AgdaSymbol{=}\AgdaSpace{}%
\AgdaFunction{sub}\AgdaSpace{}%
\AgdaBound{σ}\AgdaSpace{}%
\AgdaBound{o}\AgdaSpace{}%
\AgdaOperator{\AgdaInductiveConstructor{∶}}\AgdaSpace{}%
\AgdaFunction{sub}\AgdaSpace{}%
\AgdaBound{σ}\AgdaSpace{}%
\AgdaBound{τ}\<%
\\
\>[0]\AgdaFunction{sub}\AgdaSpace{}%
\AgdaBound{σ}\AgdaSpace{}%
\AgdaInductiveConstructor{`⊤}\AgdaSpace{}%
\AgdaSymbol{=}\AgdaSpace{}%
\AgdaInductiveConstructor{`⊤}\<%
\\
\>[0]\AgdaFunction{sub}\AgdaSpace{}%
\AgdaBound{σ}\AgdaSpace{}%
\AgdaSymbol{(}\AgdaBound{τ₁}\AgdaSpace{}%
\AgdaOperator{\AgdaInductiveConstructor{⇒}}\AgdaSpace{}%
\AgdaBound{τ₂}\AgdaSymbol{)}\AgdaSpace{}%
\AgdaSymbol{=}\AgdaSpace{}%
\AgdaFunction{sub}\AgdaSpace{}%
\AgdaBound{σ}\AgdaSpace{}%
\AgdaBound{τ₁}\AgdaSpace{}%
\AgdaOperator{\AgdaInductiveConstructor{⇒}}\AgdaSpace{}%
\AgdaFunction{sub}\AgdaSpace{}%
\AgdaBound{σ}\AgdaSpace{}%
\AgdaBound{τ₂}\<%
\\
\>[0]\AgdaFunction{sub}\AgdaSpace{}%
\AgdaBound{σ}\AgdaSpace{}%
\AgdaSymbol{(}\AgdaOperator{\AgdaInductiveConstructor{∀`α}}\AgdaSpace{}%
\AgdaBound{τ}\AgdaSymbol{)}\AgdaSpace{}%
\AgdaSymbol{=}\AgdaSpace{}%
\AgdaOperator{\AgdaInductiveConstructor{∀`α}}\AgdaSpace{}%
\AgdaSymbol{(}\AgdaFunction{sub}\AgdaSpace{}%
\AgdaSymbol{(}\AgdaFunction{extₛ}\AgdaSpace{}%
\AgdaBound{σ}\AgdaSymbol{)}\AgdaSpace{}%
\AgdaBound{τ}\AgdaSymbol{)}\<%
\\
\>[0]\AgdaFunction{sub}\AgdaSpace{}%
\AgdaBound{σ}\AgdaSpace{}%
\AgdaSymbol{(}\AgdaOperator{\AgdaInductiveConstructor{[}}\AgdaSpace{}%
\AgdaBound{c}\AgdaSpace{}%
\AgdaOperator{\AgdaInductiveConstructor{]⇒}}\AgdaSpace{}%
\AgdaBound{τ}\AgdaSpace{}%
\AgdaSymbol{)}\AgdaSpace{}%
\AgdaSymbol{=}\AgdaSpace{}%
\AgdaOperator{\AgdaInductiveConstructor{[}}\AgdaSpace{}%
\AgdaFunction{sub}\AgdaSpace{}%
\AgdaBound{σ}\AgdaSpace{}%
\AgdaBound{c}\AgdaSpace{}%
\AgdaOperator{\AgdaInductiveConstructor{]⇒}}\AgdaSpace{}%
\AgdaSymbol{(}\AgdaFunction{sub}\AgdaSpace{}%
\AgdaBound{σ}\AgdaSpace{}%
\AgdaBound{τ}\AgdaSymbol{)}\<%
\\
\>[0]\AgdaFunction{sub}\AgdaSpace{}%
\AgdaBound{σ}\AgdaSpace{}%
\AgdaInductiveConstructor{⋆}\AgdaSpace{}%
\AgdaSymbol{=}\AgdaSpace{}%
\AgdaInductiveConstructor{⋆}\<%
\end{code}
\newcommand{\Fosubs}[0]{\begin{code}%
\>[0]\AgdaOperator{\AgdaFunction{\AgdaUnderscore{}[\AgdaUnderscore{}]}}\AgdaSpace{}%
\AgdaSymbol{:}\AgdaSpace{}%
\AgdaFunction{Type}\AgdaSpace{}%
\AgdaSymbol{(}\AgdaGeneralizable{S}\AgdaSpace{}%
\AgdaOperator{\AgdaInductiveConstructor{▷}}\AgdaSpace{}%
\AgdaInductiveConstructor{τₛ}\AgdaSymbol{)}\AgdaSpace{}%
\AgdaSymbol{→}\AgdaSpace{}%
\AgdaFunction{Type}\AgdaSpace{}%
\AgdaGeneralizable{S}\AgdaSpace{}%
\AgdaSymbol{→}\AgdaSpace{}%
\AgdaFunction{Type}\AgdaSpace{}%
\AgdaGeneralizable{S}\<%
\\
\>[0]\AgdaBound{τ}\AgdaSpace{}%
\AgdaOperator{\AgdaFunction{[}}\AgdaSpace{}%
\AgdaBound{τ'}\AgdaSpace{}%
\AgdaOperator{\AgdaFunction{]}}\AgdaSpace{}%
\AgdaSymbol{=}\AgdaSpace{}%
\AgdaFunction{sub}\AgdaSpace{}%
\AgdaSymbol{(}\AgdaFunction{single-typeₛ}\AgdaSpace{}%
\AgdaFunction{idₛ}\AgdaSpace{}%
\AgdaBound{τ'}\AgdaSymbol{)}\AgdaSpace{}%
\AgdaBound{τ}\<%
\end{code}}
\begin{code}[hide]%
\>[0]\AgdaKeyword{variable}\<%
\\
\>[0][@{}l@{\AgdaIndent{0}}]%
\>[2]\AgdaGeneralizable{σ}\AgdaSpace{}%
\AgdaGeneralizable{σ'}\AgdaSpace{}%
\AgdaGeneralizable{σ''}\AgdaSpace{}%
\AgdaGeneralizable{σ₁}\AgdaSpace{}%
\AgdaGeneralizable{σ₂}\AgdaSpace{}%
\AgdaSymbol{:}\AgdaSpace{}%
\AgdaFunction{Sub}\AgdaSpace{}%
\AgdaGeneralizable{S₁}\AgdaSpace{}%
\AgdaGeneralizable{S₂}\<%
\\
\>[0]\<%
\\
\>[0]\AgdaComment{--\ Context\ ------------------------------------------------------------------------------}\<%
\\
%
\\[\AgdaEmptyExtraSkip]%
\>[0]\AgdaFunction{item-Bindable}\AgdaSpace{}%
\AgdaSymbol{:}\AgdaSpace{}%
\AgdaDatatype{Sort}\AgdaSpace{}%
\AgdaInductiveConstructor{⊤ᴮ}\AgdaSpace{}%
\AgdaSymbol{→}\AgdaSpace{}%
\AgdaDatatype{Bindable}\<%
\\
\>[0]\AgdaFunction{item-Bindable}\AgdaSpace{}%
\AgdaInductiveConstructor{eₛ}\AgdaSpace{}%
\AgdaSymbol{=}\AgdaSpace{}%
\AgdaInductiveConstructor{⊤ᴮ}\<%
\\
\>[0]\AgdaFunction{item-Bindable}\AgdaSpace{}%
\AgdaInductiveConstructor{τₛ}\AgdaSpace{}%
\AgdaSymbol{=}\AgdaSpace{}%
\AgdaInductiveConstructor{⊥ᴮ}\<%
\\
\>[0]\AgdaFunction{item-Bindable}\AgdaSpace{}%
\AgdaInductiveConstructor{oₛ}\AgdaSpace{}%
\AgdaSymbol{=}\AgdaSpace{}%
\AgdaInductiveConstructor{⊥ᴮ}\<%
\\
%
\\[\AgdaEmptyExtraSkip]%
\>[0]\AgdaFunction{item-of}\AgdaSpace{}%
\AgdaSymbol{:}\AgdaSpace{}%
\AgdaSymbol{(}\AgdaBound{s}\AgdaSpace{}%
\AgdaSymbol{:}\AgdaSpace{}%
\AgdaDatatype{Sort}\AgdaSpace{}%
\AgdaInductiveConstructor{⊤ᴮ}\AgdaSymbol{)}\AgdaSpace{}%
\AgdaSymbol{→}\AgdaSpace{}%
\AgdaDatatype{Sort}\AgdaSpace{}%
\AgdaSymbol{(}\AgdaFunction{item-Bindable}\AgdaSpace{}%
\AgdaBound{s}\AgdaSymbol{)}\<%
\end{code}
\newcommand{\Foitem}[0]{\begin{code}%
\>[0]\AgdaFunction{item-of}\AgdaSpace{}%
\AgdaInductiveConstructor{eₛ}\AgdaSpace{}%
\AgdaSymbol{=}\AgdaSpace{}%
\AgdaInductiveConstructor{τₛ}\<%
\\
\>[0]\AgdaFunction{item-of}\AgdaSpace{}%
\AgdaInductiveConstructor{τₛ}\AgdaSpace{}%
\AgdaSymbol{=}\AgdaSpace{}%
\AgdaInductiveConstructor{κₛ}\<%
\\
%
\\[\AgdaEmptyExtraSkip]%
\>[0]\AgdaFunction{item-of}\AgdaSpace{}%
\AgdaInductiveConstructor{oₛ}\AgdaSpace{}%
\AgdaSymbol{=}\AgdaSpace{}%
\AgdaInductiveConstructor{κₛ}\<%
\end{code}}
\begin{code}[hide]%
\>[0]\AgdaKeyword{variable}\<%
\\
\>[0][@{}l@{\AgdaIndent{0}}]%
\>[2]\AgdaGeneralizable{I}\AgdaSpace{}%
\AgdaGeneralizable{I'}\AgdaSpace{}%
\AgdaGeneralizable{I''}\AgdaSpace{}%
\AgdaGeneralizable{I₁}\AgdaSpace{}%
\AgdaGeneralizable{I₂}\AgdaSpace{}%
\AgdaSymbol{:}\AgdaSpace{}%
\AgdaDatatype{Term}\AgdaSpace{}%
\AgdaGeneralizable{S}\AgdaSpace{}%
\AgdaSymbol{(}\AgdaFunction{item-of}\AgdaSpace{}%
\AgdaGeneralizable{s}\AgdaSymbol{)}\<%
\end{code}
\newcommand{\FoCtx}[0]{\begin{code}%
\>[0]\AgdaKeyword{data}\AgdaSpace{}%
\AgdaDatatype{Ctx}\AgdaSpace{}%
\AgdaSymbol{:}\AgdaSpace{}%
\AgdaFunction{Sorts}\AgdaSpace{}%
\AgdaSymbol{→}\AgdaSpace{}%
\AgdaPrimitive{Set}\AgdaSpace{}%
\AgdaKeyword{where}\<%
\\
\>[0][@{}l@{\AgdaIndent{0}}]%
\>[2]\AgdaOperator{\AgdaInductiveConstructor{\AgdaUnderscore{}▸\AgdaUnderscore{}}}\AgdaSpace{}%
\AgdaSymbol{:}\AgdaSpace{}%
\AgdaDatatype{Ctx}\AgdaSpace{}%
\AgdaGeneralizable{S}\AgdaSpace{}%
\AgdaSymbol{→}\AgdaSpace{}%
\AgdaFunction{Cstr}\AgdaSpace{}%
\AgdaGeneralizable{S}\AgdaSpace{}%
\AgdaSymbol{→}\AgdaSpace{}%
\AgdaDatatype{Ctx}\AgdaSpace{}%
\AgdaGeneralizable{S}\<%
\\
%
\>[2]\AgdaComment{--\ ...}\<%
\end{code}}
\begin{code}[hide]%
%
\>[2]\AgdaInductiveConstructor{∅}%
\>[6]\AgdaSymbol{:}\AgdaSpace{}%
\AgdaDatatype{Ctx}\AgdaSpace{}%
\AgdaInductiveConstructor{[]}\<%
\\
%
\>[2]\AgdaOperator{\AgdaInductiveConstructor{\AgdaUnderscore{}▶\AgdaUnderscore{}}}\AgdaSpace{}%
\AgdaSymbol{:}\AgdaSpace{}%
\AgdaDatatype{Ctx}\AgdaSpace{}%
\AgdaGeneralizable{S}\AgdaSpace{}%
\AgdaSymbol{→}\AgdaSpace{}%
\AgdaDatatype{Term}\AgdaSpace{}%
\AgdaGeneralizable{S}\AgdaSpace{}%
\AgdaSymbol{(}\AgdaFunction{item-of}\AgdaSpace{}%
\AgdaGeneralizable{s}\AgdaSymbol{)}\AgdaSpace{}%
\AgdaSymbol{→}\AgdaSpace{}%
\AgdaDatatype{Ctx}\AgdaSpace{}%
\AgdaSymbol{(}\AgdaGeneralizable{S}\AgdaSpace{}%
\AgdaOperator{\AgdaInductiveConstructor{▷}}\AgdaSpace{}%
\AgdaGeneralizable{s}\AgdaSymbol{)}\<%
\\
%
\\[\AgdaEmptyExtraSkip]%
%
\\[\AgdaEmptyExtraSkip]%
\>[0]\AgdaFunction{lookup}\AgdaSpace{}%
\AgdaSymbol{:}\AgdaSpace{}%
\AgdaDatatype{Ctx}\AgdaSpace{}%
\AgdaGeneralizable{S}\AgdaSpace{}%
\AgdaSymbol{→}\AgdaSpace{}%
\AgdaFunction{Var}\AgdaSpace{}%
\AgdaGeneralizable{S}\AgdaSpace{}%
\AgdaGeneralizable{s}\AgdaSpace{}%
\AgdaSymbol{→}\AgdaSpace{}%
\AgdaDatatype{Term}\AgdaSpace{}%
\AgdaGeneralizable{S}\AgdaSpace{}%
\AgdaSymbol{(}\AgdaFunction{item-of}\AgdaSpace{}%
\AgdaGeneralizable{s}\AgdaSymbol{)}\<%
\\
\>[0]\AgdaFunction{lookup}\AgdaSpace{}%
\AgdaSymbol{(}\AgdaBound{Γ}\AgdaSpace{}%
\AgdaOperator{\AgdaInductiveConstructor{▶}}\AgdaSpace{}%
\AgdaBound{S}\AgdaSymbol{)}\AgdaSpace{}%
\AgdaSymbol{(}\AgdaInductiveConstructor{here}\AgdaSpace{}%
\AgdaInductiveConstructor{refl}\AgdaSymbol{)}\AgdaSpace{}%
\AgdaSymbol{=}\AgdaSpace{}%
\AgdaFunction{wk}\AgdaSpace{}%
\AgdaBound{S}\<%
\\
\>[0]\AgdaFunction{lookup}\AgdaSpace{}%
\AgdaSymbol{(}\AgdaBound{Γ}\AgdaSpace{}%
\AgdaOperator{\AgdaInductiveConstructor{▶}}\AgdaSpace{}%
\AgdaBound{S}\AgdaSymbol{)}\AgdaSpace{}%
\AgdaSymbol{(}\AgdaInductiveConstructor{there}\AgdaSpace{}%
\AgdaBound{x}\AgdaSymbol{)}\AgdaSpace{}%
\AgdaSymbol{=}\AgdaSpace{}%
\AgdaFunction{wk}\AgdaSpace{}%
\AgdaSymbol{(}\AgdaFunction{lookup}\AgdaSpace{}%
\AgdaBound{Γ}\AgdaSpace{}%
\AgdaBound{x}\AgdaSymbol{)}\<%
\\
\>[0]\AgdaFunction{lookup}\AgdaSpace{}%
\AgdaSymbol{(}\AgdaBound{Γ}\AgdaSpace{}%
\AgdaOperator{\AgdaInductiveConstructor{▸}}\AgdaSpace{}%
\AgdaBound{c}\AgdaSymbol{)}\AgdaSpace{}%
\AgdaBound{x}\AgdaSpace{}%
\AgdaSymbol{=}\AgdaSpace{}%
\AgdaFunction{lookup}\AgdaSpace{}%
\AgdaBound{Γ}\AgdaSpace{}%
\AgdaBound{x}\<%
\\
%
\\[\AgdaEmptyExtraSkip]%
\>[0]\AgdaKeyword{variable}\<%
\\
\>[0][@{}l@{\AgdaIndent{0}}]%
\>[2]\AgdaGeneralizable{Γ}\AgdaSpace{}%
\AgdaGeneralizable{Γ'}\AgdaSpace{}%
\AgdaGeneralizable{Γ''}\AgdaSpace{}%
\AgdaGeneralizable{Γ₁}\AgdaSpace{}%
\AgdaGeneralizable{Γ₂}\AgdaSpace{}%
\AgdaSymbol{:}\AgdaSpace{}%
\AgdaDatatype{Ctx}\AgdaSpace{}%
\AgdaGeneralizable{S}\<%
\\
%
\\[\AgdaEmptyExtraSkip]%
\>[0]\AgdaComment{--\ Constraint\ Solving\ -------------------------------------------------------------------}\<%
\end{code}
\newcommand{\FoCstrSolve}[0]{\begin{code}%
\>[0]\AgdaKeyword{data}\AgdaSpace{}%
\AgdaOperator{\AgdaDatatype{[\AgdaUnderscore{}]∈\AgdaUnderscore{}}}\AgdaSpace{}%
\AgdaSymbol{:}\AgdaSpace{}%
\AgdaFunction{Cstr}\AgdaSpace{}%
\AgdaGeneralizable{S}\AgdaSpace{}%
\AgdaSymbol{→}\AgdaSpace{}%
\AgdaDatatype{Ctx}\AgdaSpace{}%
\AgdaGeneralizable{S}\AgdaSpace{}%
\AgdaSymbol{→}\AgdaSpace{}%
\AgdaPrimitive{Set}\AgdaSpace{}%
\AgdaKeyword{where}\<%
\\
\>[0][@{}l@{\AgdaIndent{0}}]%
\>[2]\AgdaInductiveConstructor{here}\AgdaSpace{}%
\AgdaSymbol{:}\AgdaSpace{}%
\AgdaOperator{\AgdaDatatype{[}}\AgdaSpace{}%
\AgdaSymbol{(}\AgdaOperator{\AgdaInductiveConstructor{`}}\AgdaSpace{}%
\AgdaGeneralizable{o}\AgdaSpace{}%
\AgdaOperator{\AgdaInductiveConstructor{∶}}\AgdaSpace{}%
\AgdaGeneralizable{τ}\AgdaSymbol{)}\AgdaSpace{}%
\AgdaOperator{\AgdaDatatype{]∈}}\AgdaSpace{}%
\AgdaSymbol{(}\AgdaGeneralizable{Γ}\AgdaSpace{}%
\AgdaOperator{\AgdaInductiveConstructor{▸}}\AgdaSpace{}%
\AgdaSymbol{(}\AgdaOperator{\AgdaInductiveConstructor{`}}\AgdaSpace{}%
\AgdaGeneralizable{o}\AgdaSpace{}%
\AgdaOperator{\AgdaInductiveConstructor{∶}}\AgdaSpace{}%
\AgdaGeneralizable{τ}\AgdaSymbol{))}\<%
\\
%
\>[2]\AgdaInductiveConstructor{under-bind}\AgdaSpace{}%
\AgdaSymbol{:}\AgdaSpace{}%
\AgdaSymbol{\{}\AgdaBound{I}\AgdaSpace{}%
\AgdaSymbol{:}\AgdaSpace{}%
\AgdaDatatype{Term}\AgdaSpace{}%
\AgdaGeneralizable{S}\AgdaSpace{}%
\AgdaSymbol{(}\AgdaFunction{item-of}\AgdaSpace{}%
\AgdaGeneralizable{s'}\AgdaSymbol{)\}}\AgdaSpace{}%
\AgdaSymbol{→}\<%
\\
\>[2][@{}l@{\AgdaIndent{0}}]%
\>[4]\AgdaOperator{\AgdaDatatype{[}}\AgdaSpace{}%
\AgdaSymbol{(}\AgdaOperator{\AgdaInductiveConstructor{`}}\AgdaSpace{}%
\AgdaGeneralizable{o}\AgdaSpace{}%
\AgdaOperator{\AgdaInductiveConstructor{∶}}\AgdaSpace{}%
\AgdaGeneralizable{τ}\AgdaSymbol{)}\AgdaSpace{}%
\AgdaOperator{\AgdaDatatype{]∈}}\AgdaSpace{}%
\AgdaGeneralizable{Γ}\AgdaSpace{}%
\AgdaSymbol{→}\AgdaSpace{}%
\AgdaOperator{\AgdaDatatype{[}}\AgdaSpace{}%
\AgdaSymbol{(}\AgdaOperator{\AgdaInductiveConstructor{`}}\AgdaSpace{}%
\AgdaInductiveConstructor{there}\AgdaSpace{}%
\AgdaGeneralizable{o}\AgdaSpace{}%
\AgdaOperator{\AgdaInductiveConstructor{∶}}\AgdaSpace{}%
\AgdaFunction{wk}\AgdaSpace{}%
\AgdaGeneralizable{τ}\AgdaSymbol{)}\AgdaSpace{}%
\AgdaOperator{\AgdaDatatype{]∈}}\AgdaSpace{}%
\AgdaSymbol{(}\AgdaGeneralizable{Γ}\AgdaSpace{}%
\AgdaOperator{\AgdaInductiveConstructor{▶}}\AgdaSpace{}%
\AgdaBound{I}\AgdaSymbol{)}\<%
\\
%
\>[2]\AgdaInductiveConstructor{under-cstr}\AgdaSpace{}%
\AgdaSymbol{:}\AgdaSpace{}%
\AgdaOperator{\AgdaDatatype{[}}\AgdaSpace{}%
\AgdaGeneralizable{c}\AgdaSpace{}%
\AgdaOperator{\AgdaDatatype{]∈}}\AgdaSpace{}%
\AgdaGeneralizable{Γ}\AgdaSpace{}%
\AgdaSymbol{→}\AgdaSpace{}%
\AgdaOperator{\AgdaDatatype{[}}\AgdaSpace{}%
\AgdaGeneralizable{c}\AgdaSpace{}%
\AgdaOperator{\AgdaDatatype{]∈}}\AgdaSpace{}%
\AgdaSymbol{(}\AgdaGeneralizable{Γ}\AgdaSpace{}%
\AgdaOperator{\AgdaInductiveConstructor{▸}}\AgdaSpace{}%
\AgdaGeneralizable{c'}\AgdaSymbol{)}\<%
\end{code}}
\begin{code}[hide]%
\>[0]\AgdaComment{--\ Typing\ -------------------------------------------------------------------------------}\<%
\\
%
\\[\AgdaEmptyExtraSkip]%
\>[0]\AgdaFunction{kind-Bindable}\AgdaSpace{}%
\AgdaSymbol{:}\AgdaSpace{}%
\AgdaDatatype{Sort}\AgdaSpace{}%
\AgdaInductiveConstructor{⊤ᴮ}\AgdaSpace{}%
\AgdaSymbol{→}\AgdaSpace{}%
\AgdaDatatype{Bindable}\<%
\\
\>[0]\AgdaFunction{kind-Bindable}\AgdaSpace{}%
\AgdaInductiveConstructor{eₛ}\AgdaSpace{}%
\AgdaSymbol{=}\AgdaSpace{}%
\AgdaInductiveConstructor{⊤ᴮ}\<%
\\
\>[0]\AgdaFunction{kind-Bindable}\AgdaSpace{}%
\AgdaInductiveConstructor{τₛ}\AgdaSpace{}%
\AgdaSymbol{=}\AgdaSpace{}%
\AgdaInductiveConstructor{⊥ᴮ}\<%
\\
\>[0]\AgdaFunction{kind-Bindable}\AgdaSpace{}%
\AgdaInductiveConstructor{oₛ}\AgdaSpace{}%
\AgdaSymbol{=}\AgdaSpace{}%
\AgdaInductiveConstructor{⊤ᴮ}\<%
\\
%
\\[\AgdaEmptyExtraSkip]%
\>[0]\AgdaFunction{type-of}\AgdaSpace{}%
\AgdaSymbol{:}\AgdaSpace{}%
\AgdaSymbol{(}\AgdaBound{s}\AgdaSpace{}%
\AgdaSymbol{:}\AgdaSpace{}%
\AgdaDatatype{Sort}\AgdaSpace{}%
\AgdaInductiveConstructor{⊤ᴮ}\AgdaSymbol{)}\AgdaSpace{}%
\AgdaSymbol{→}\AgdaSpace{}%
\AgdaDatatype{Sort}\AgdaSpace{}%
\AgdaSymbol{(}\AgdaFunction{kind-Bindable}\AgdaSpace{}%
\AgdaBound{s}\AgdaSymbol{)}\<%
\end{code}
\newcommand{\Fokind}[0]{\begin{code}%
\>[0]\AgdaFunction{type-of}\AgdaSpace{}%
\AgdaInductiveConstructor{eₛ}\AgdaSpace{}%
\AgdaSymbol{=}\AgdaSpace{}%
\AgdaInductiveConstructor{τₛ}\<%
\\
\>[0]\AgdaFunction{type-of}\AgdaSpace{}%
\AgdaInductiveConstructor{τₛ}\AgdaSpace{}%
\AgdaSymbol{=}\AgdaSpace{}%
\AgdaInductiveConstructor{κₛ}\<%
\\
\>[0]\AgdaFunction{type-of}\AgdaSpace{}%
\AgdaInductiveConstructor{oₛ}\AgdaSpace{}%
\AgdaSymbol{=}\AgdaSpace{}%
\AgdaInductiveConstructor{τₛ}\<%
\end{code}}
\begin{code}[hide]%
\>[0]\AgdaKeyword{variable}\<%
\\
\>[0][@{}l@{\AgdaIndent{0}}]%
\>[2]\AgdaGeneralizable{T}\AgdaSpace{}%
\AgdaGeneralizable{T'}\AgdaSpace{}%
\AgdaGeneralizable{T''}\AgdaSpace{}%
\AgdaGeneralizable{T₁}\AgdaSpace{}%
\AgdaGeneralizable{T₂}\AgdaSpace{}%
\AgdaSymbol{:}\AgdaSpace{}%
\AgdaDatatype{Term}\AgdaSpace{}%
\AgdaGeneralizable{S}\AgdaSpace{}%
\AgdaSymbol{(}\AgdaFunction{type-of}\AgdaSpace{}%
\AgdaGeneralizable{s}\AgdaSymbol{)}\<%
\\
%
\\[\AgdaEmptyExtraSkip]%
%
\\[\AgdaEmptyExtraSkip]%
\>[0]\AgdaKeyword{infix}\AgdaSpace{}%
\AgdaNumber{3}\AgdaSpace{}%
\AgdaOperator{\AgdaDatatype{\AgdaUnderscore{}⊢\AgdaUnderscore{}∶\AgdaUnderscore{}}}\<%
\end{code}
\newcommand{\FoTyping}[0]{\begin{code}%
\>[0]\AgdaKeyword{data}\AgdaSpace{}%
\AgdaOperator{\AgdaDatatype{\AgdaUnderscore{}⊢\AgdaUnderscore{}∶\AgdaUnderscore{}}}\AgdaSpace{}%
\AgdaSymbol{:}\AgdaSpace{}%
\AgdaDatatype{Ctx}\AgdaSpace{}%
\AgdaGeneralizable{S}\AgdaSpace{}%
\AgdaSymbol{→}\AgdaSpace{}%
\AgdaDatatype{Term}\AgdaSpace{}%
\AgdaGeneralizable{S}\AgdaSpace{}%
\AgdaGeneralizable{s}\AgdaSpace{}%
\AgdaSymbol{→}\AgdaSpace{}%
\AgdaDatatype{Term}\AgdaSpace{}%
\AgdaGeneralizable{S}\AgdaSpace{}%
\AgdaSymbol{(}\AgdaFunction{type-of}\AgdaSpace{}%
\AgdaGeneralizable{s}\AgdaSymbol{)}\AgdaSpace{}%
\AgdaSymbol{→}\AgdaSpace{}%
\AgdaPrimitive{Set}\AgdaSpace{}%
\AgdaKeyword{where}\<%
\\
\>[0][@{}l@{\AgdaIndent{0}}]%
\>[2]\AgdaInductiveConstructor{⊢`o}\AgdaSpace{}%
\AgdaSymbol{:}\<%
\\
\>[2][@{}l@{\AgdaIndent{0}}]%
\>[4]\AgdaOperator{\AgdaDatatype{[}}\AgdaSpace{}%
\AgdaOperator{\AgdaInductiveConstructor{`}}\AgdaSpace{}%
\AgdaGeneralizable{o}\AgdaSpace{}%
\AgdaOperator{\AgdaInductiveConstructor{∶}}\AgdaSpace{}%
\AgdaGeneralizable{τ}\AgdaSpace{}%
\AgdaOperator{\AgdaDatatype{]∈}}\AgdaSpace{}%
\AgdaGeneralizable{Γ}\AgdaSpace{}%
\AgdaSymbol{→}\<%
\\
%
\>[4]\AgdaGeneralizable{Γ}\AgdaSpace{}%
\AgdaOperator{\AgdaDatatype{⊢}}\AgdaSpace{}%
\AgdaOperator{\AgdaInductiveConstructor{`}}\AgdaSpace{}%
\AgdaGeneralizable{o}\AgdaSpace{}%
\AgdaOperator{\AgdaDatatype{∶}}\AgdaSpace{}%
\AgdaGeneralizable{τ}\<%
\\
%
\>[2]\AgdaInductiveConstructor{⊢ƛ}\AgdaSpace{}%
\AgdaSymbol{:}\<%
\\
\>[2][@{}l@{\AgdaIndent{0}}]%
\>[4]\AgdaGeneralizable{Γ}\AgdaSpace{}%
\AgdaOperator{\AgdaInductiveConstructor{▸}}\AgdaSpace{}%
\AgdaGeneralizable{c}\AgdaSpace{}%
\AgdaOperator{\AgdaDatatype{⊢}}\AgdaSpace{}%
\AgdaGeneralizable{e}\AgdaSpace{}%
\AgdaOperator{\AgdaDatatype{∶}}\AgdaSpace{}%
\AgdaGeneralizable{τ}\AgdaSpace{}%
\AgdaSymbol{→}\<%
\\
%
\>[4]\AgdaGeneralizable{Γ}\AgdaSpace{}%
\AgdaOperator{\AgdaDatatype{⊢}}\AgdaSpace{}%
\AgdaOperator{\AgdaInductiveConstructor{ƛ}}\AgdaSpace{}%
\AgdaGeneralizable{c}\AgdaSpace{}%
\AgdaOperator{\AgdaInductiveConstructor{⇒}}\AgdaSpace{}%
\AgdaGeneralizable{e}\AgdaSpace{}%
\AgdaOperator{\AgdaDatatype{∶}}\AgdaSpace{}%
\AgdaOperator{\AgdaInductiveConstructor{[}}\AgdaSpace{}%
\AgdaGeneralizable{c}\AgdaSpace{}%
\AgdaOperator{\AgdaInductiveConstructor{]⇒}}\AgdaSpace{}%
\AgdaGeneralizable{τ}\<%
\\
%
\>[2]\AgdaInductiveConstructor{⊢⊘}\AgdaSpace{}%
\AgdaSymbol{:}\<%
\\
\>[2][@{}l@{\AgdaIndent{0}}]%
\>[4]\AgdaGeneralizable{Γ}\AgdaSpace{}%
\AgdaOperator{\AgdaDatatype{⊢}}\AgdaSpace{}%
\AgdaGeneralizable{e}\AgdaSpace{}%
\AgdaOperator{\AgdaDatatype{∶}}\AgdaSpace{}%
\AgdaOperator{\AgdaInductiveConstructor{[}}\AgdaSpace{}%
\AgdaOperator{\AgdaInductiveConstructor{`}}\AgdaSpace{}%
\AgdaGeneralizable{o}\AgdaSpace{}%
\AgdaOperator{\AgdaInductiveConstructor{∶}}\AgdaSpace{}%
\AgdaGeneralizable{τ}\AgdaSpace{}%
\AgdaOperator{\AgdaInductiveConstructor{]⇒}}\AgdaSpace{}%
\AgdaGeneralizable{τ'}\AgdaSpace{}%
\AgdaSymbol{→}\<%
\\
%
\>[4]\AgdaOperator{\AgdaDatatype{[}}\AgdaSpace{}%
\AgdaOperator{\AgdaInductiveConstructor{`}}\AgdaSpace{}%
\AgdaGeneralizable{o}\AgdaSpace{}%
\AgdaOperator{\AgdaInductiveConstructor{∶}}\AgdaSpace{}%
\AgdaGeneralizable{τ}\AgdaSpace{}%
\AgdaOperator{\AgdaDatatype{]∈}}\AgdaSpace{}%
\AgdaGeneralizable{Γ}\AgdaSpace{}%
\AgdaSymbol{→}\<%
\\
%
\>[4]\AgdaGeneralizable{Γ}\AgdaSpace{}%
\AgdaOperator{\AgdaDatatype{⊢}}\AgdaSpace{}%
\AgdaGeneralizable{e}\AgdaSpace{}%
\AgdaOperator{\AgdaDatatype{∶}}\AgdaSpace{}%
\AgdaGeneralizable{τ'}\<%
\\
%
\>[2]\AgdaInductiveConstructor{⊢decl}\AgdaSpace{}%
\AgdaSymbol{:}\<%
\\
\>[2][@{}l@{\AgdaIndent{0}}]%
\>[4]\AgdaGeneralizable{Γ}\AgdaSpace{}%
\AgdaOperator{\AgdaInductiveConstructor{▶}}\AgdaSpace{}%
\AgdaInductiveConstructor{⋆}\AgdaSpace{}%
\AgdaOperator{\AgdaDatatype{⊢}}\AgdaSpace{}%
\AgdaGeneralizable{e}\AgdaSpace{}%
\AgdaOperator{\AgdaDatatype{∶}}\AgdaSpace{}%
\AgdaFunction{wk}\AgdaSpace{}%
\AgdaGeneralizable{τ}\AgdaSpace{}%
\AgdaSymbol{→}\<%
\\
%
\>[4]\AgdaGeneralizable{Γ}\AgdaSpace{}%
\AgdaOperator{\AgdaDatatype{⊢}}\AgdaSpace{}%
\AgdaOperator{\AgdaInductiveConstructor{decl`o`in}}\AgdaSpace{}%
\AgdaGeneralizable{e}\AgdaSpace{}%
\AgdaOperator{\AgdaDatatype{∶}}\AgdaSpace{}%
\AgdaGeneralizable{τ}\<%
\\
%
\>[2]\AgdaInductiveConstructor{⊢inst}\AgdaSpace{}%
\AgdaSymbol{:}\<%
\\
\>[2][@{}l@{\AgdaIndent{0}}]%
\>[4]\AgdaGeneralizable{Γ}\AgdaSpace{}%
\AgdaOperator{\AgdaDatatype{⊢}}\AgdaSpace{}%
\AgdaGeneralizable{e₂}\AgdaSpace{}%
\AgdaOperator{\AgdaDatatype{∶}}\AgdaSpace{}%
\AgdaGeneralizable{τ}\AgdaSpace{}%
\AgdaSymbol{→}\<%
\\
%
\>[4]\AgdaGeneralizable{Γ}\AgdaSpace{}%
\AgdaOperator{\AgdaInductiveConstructor{▸}}\AgdaSpace{}%
\AgdaSymbol{(}\AgdaOperator{\AgdaInductiveConstructor{`}}\AgdaSpace{}%
\AgdaGeneralizable{o}\AgdaSpace{}%
\AgdaOperator{\AgdaInductiveConstructor{∶}}\AgdaSpace{}%
\AgdaGeneralizable{τ}\AgdaSymbol{)}\AgdaSpace{}%
\AgdaOperator{\AgdaDatatype{⊢}}\AgdaSpace{}%
\AgdaGeneralizable{e₁}\AgdaSpace{}%
\AgdaOperator{\AgdaDatatype{∶}}\AgdaSpace{}%
\AgdaGeneralizable{τ'}\AgdaSpace{}%
\AgdaSymbol{→}\<%
\\
%
\>[4]\AgdaGeneralizable{Γ}\AgdaSpace{}%
\AgdaOperator{\AgdaDatatype{⊢}}\AgdaSpace{}%
\AgdaOperator{\AgdaInductiveConstructor{inst`}}\AgdaSpace{}%
\AgdaOperator{\AgdaInductiveConstructor{`}}\AgdaSpace{}%
\AgdaGeneralizable{o}\AgdaSpace{}%
\AgdaOperator{\AgdaInductiveConstructor{`=}}\AgdaSpace{}%
\AgdaGeneralizable{e₂}\AgdaSpace{}%
\AgdaOperator{\AgdaInductiveConstructor{`in}}\AgdaSpace{}%
\AgdaGeneralizable{e₁}\AgdaSpace{}%
\AgdaOperator{\AgdaDatatype{∶}}\AgdaSpace{}%
\AgdaGeneralizable{τ'}\<%
\\
%
\>[2]\AgdaComment{--\ ...}\<%
\end{code}}
\begin{code}[hide]%
%
\>[2]\AgdaInductiveConstructor{⊢`x}\AgdaSpace{}%
\AgdaSymbol{:}\<%
\\
\>[2][@{}l@{\AgdaIndent{0}}]%
\>[4]\AgdaFunction{lookup}\AgdaSpace{}%
\AgdaGeneralizable{Γ}\AgdaSpace{}%
\AgdaGeneralizable{x}\AgdaSpace{}%
\AgdaOperator{\AgdaDatatype{≡}}\AgdaSpace{}%
\AgdaGeneralizable{τ}\AgdaSpace{}%
\AgdaSymbol{→}\<%
\\
%
\>[4]\AgdaComment{----------------}\<%
\\
%
\>[4]\AgdaGeneralizable{Γ}\AgdaSpace{}%
\AgdaOperator{\AgdaDatatype{⊢}}\AgdaSpace{}%
\AgdaSymbol{(}\AgdaOperator{\AgdaInductiveConstructor{`}}\AgdaSpace{}%
\AgdaGeneralizable{x}\AgdaSymbol{)}\AgdaSpace{}%
\AgdaOperator{\AgdaDatatype{∶}}\AgdaSpace{}%
\AgdaGeneralizable{τ}\<%
\\
%
\>[2]\AgdaInductiveConstructor{⊢⊤}\AgdaSpace{}%
\AgdaSymbol{:}\<%
\\
\>[2][@{}l@{\AgdaIndent{0}}]%
\>[4]\AgdaComment{-----------}\<%
\\
%
\>[4]\AgdaGeneralizable{Γ}\AgdaSpace{}%
\AgdaOperator{\AgdaDatatype{⊢}}\AgdaSpace{}%
\AgdaInductiveConstructor{tt}\AgdaSpace{}%
\AgdaOperator{\AgdaDatatype{∶}}\AgdaSpace{}%
\AgdaInductiveConstructor{`⊤}\<%
\\
%
\>[2]\AgdaInductiveConstructor{⊢λ}\AgdaSpace{}%
\AgdaSymbol{:}\<%
\\
\>[2][@{}l@{\AgdaIndent{0}}]%
\>[4]\AgdaGeneralizable{Γ}\AgdaSpace{}%
\AgdaOperator{\AgdaInductiveConstructor{▶}}\AgdaSpace{}%
\AgdaGeneralizable{τ}\AgdaSpace{}%
\AgdaOperator{\AgdaDatatype{⊢}}\AgdaSpace{}%
\AgdaGeneralizable{e}\AgdaSpace{}%
\AgdaOperator{\AgdaDatatype{∶}}\AgdaSpace{}%
\AgdaFunction{wk}\AgdaSpace{}%
\AgdaGeneralizable{τ'}\AgdaSpace{}%
\AgdaSymbol{→}\<%
\\
%
\>[4]\AgdaComment{------------------}\<%
\\
%
\>[4]\AgdaGeneralizable{Γ}\AgdaSpace{}%
\AgdaOperator{\AgdaDatatype{⊢}}\AgdaSpace{}%
\AgdaOperator{\AgdaInductiveConstructor{λ`x→}}\AgdaSpace{}%
\AgdaGeneralizable{e}\AgdaSpace{}%
\AgdaOperator{\AgdaDatatype{∶}}\AgdaSpace{}%
\AgdaGeneralizable{τ}\AgdaSpace{}%
\AgdaOperator{\AgdaInductiveConstructor{⇒}}\AgdaSpace{}%
\AgdaGeneralizable{τ'}\<%
\\
%
\>[2]\AgdaInductiveConstructor{⊢Λ}\AgdaSpace{}%
\AgdaSymbol{:}\<%
\\
\>[2][@{}l@{\AgdaIndent{0}}]%
\>[4]\AgdaGeneralizable{Γ}\AgdaSpace{}%
\AgdaOperator{\AgdaInductiveConstructor{▶}}\AgdaSpace{}%
\AgdaInductiveConstructor{⋆}\AgdaSpace{}%
\AgdaOperator{\AgdaDatatype{⊢}}\AgdaSpace{}%
\AgdaGeneralizable{e}\AgdaSpace{}%
\AgdaOperator{\AgdaDatatype{∶}}\AgdaSpace{}%
\AgdaGeneralizable{τ}\AgdaSpace{}%
\AgdaSymbol{→}\<%
\\
%
\>[4]\AgdaComment{-------------------}\<%
\\
%
\>[4]\AgdaGeneralizable{Γ}\AgdaSpace{}%
\AgdaOperator{\AgdaDatatype{⊢}}\AgdaSpace{}%
\AgdaOperator{\AgdaInductiveConstructor{Λ`α→}}\AgdaSpace{}%
\AgdaGeneralizable{e}\AgdaSpace{}%
\AgdaOperator{\AgdaDatatype{∶}}\AgdaSpace{}%
\AgdaOperator{\AgdaInductiveConstructor{∀`α}}\AgdaSpace{}%
\AgdaGeneralizable{τ}\<%
\\
%
\>[2]\AgdaInductiveConstructor{⊢·}\AgdaSpace{}%
\AgdaSymbol{:}\<%
\\
\>[2][@{}l@{\AgdaIndent{0}}]%
\>[4]\AgdaGeneralizable{Γ}\AgdaSpace{}%
\AgdaOperator{\AgdaDatatype{⊢}}\AgdaSpace{}%
\AgdaGeneralizable{e₁}\AgdaSpace{}%
\AgdaOperator{\AgdaDatatype{∶}}\AgdaSpace{}%
\AgdaGeneralizable{τ₁}\AgdaSpace{}%
\AgdaOperator{\AgdaInductiveConstructor{⇒}}\AgdaSpace{}%
\AgdaGeneralizable{τ₂}\AgdaSpace{}%
\AgdaSymbol{→}\<%
\\
%
\>[4]\AgdaGeneralizable{Γ}\AgdaSpace{}%
\AgdaOperator{\AgdaDatatype{⊢}}\AgdaSpace{}%
\AgdaGeneralizable{e₂}\AgdaSpace{}%
\AgdaOperator{\AgdaDatatype{∶}}\AgdaSpace{}%
\AgdaGeneralizable{τ₁}\AgdaSpace{}%
\AgdaSymbol{→}\<%
\\
%
\>[4]\AgdaComment{------------------}\<%
\\
%
\>[4]\AgdaGeneralizable{Γ}\AgdaSpace{}%
\AgdaOperator{\AgdaDatatype{⊢}}\AgdaSpace{}%
\AgdaGeneralizable{e₁}\AgdaSpace{}%
\AgdaOperator{\AgdaInductiveConstructor{·}}\AgdaSpace{}%
\AgdaGeneralizable{e₂}\AgdaSpace{}%
\AgdaOperator{\AgdaDatatype{∶}}\AgdaSpace{}%
\AgdaGeneralizable{τ₂}\<%
\\
%
\>[2]\AgdaInductiveConstructor{⊢•}\AgdaSpace{}%
\AgdaSymbol{:}\<%
\\
\>[2][@{}l@{\AgdaIndent{0}}]%
\>[4]\AgdaGeneralizable{Γ}\AgdaSpace{}%
\AgdaOperator{\AgdaDatatype{⊢}}\AgdaSpace{}%
\AgdaGeneralizable{e}\AgdaSpace{}%
\AgdaOperator{\AgdaDatatype{∶}}\AgdaSpace{}%
\AgdaOperator{\AgdaInductiveConstructor{∀`α}}\AgdaSpace{}%
\AgdaGeneralizable{τ}\AgdaSpace{}%
\AgdaSymbol{→}\<%
\\
%
\>[4]\AgdaGeneralizable{Γ}\AgdaSpace{}%
\AgdaOperator{\AgdaDatatype{⊢}}\AgdaSpace{}%
\AgdaGeneralizable{e}\AgdaSpace{}%
\AgdaOperator{\AgdaInductiveConstructor{•}}\AgdaSpace{}%
\AgdaGeneralizable{τ'}\AgdaSpace{}%
\AgdaOperator{\AgdaDatatype{∶}}\AgdaSpace{}%
\AgdaGeneralizable{τ}\AgdaSpace{}%
\AgdaOperator{\AgdaFunction{[}}\AgdaSpace{}%
\AgdaGeneralizable{τ'}\AgdaSpace{}%
\AgdaOperator{\AgdaFunction{]}}\<%
\\
%
\>[2]\AgdaInductiveConstructor{⊢let}\AgdaSpace{}%
\AgdaSymbol{:}\<%
\\
\>[2][@{}l@{\AgdaIndent{0}}]%
\>[4]\AgdaGeneralizable{Γ}\AgdaSpace{}%
\AgdaOperator{\AgdaDatatype{⊢}}\AgdaSpace{}%
\AgdaGeneralizable{e₂}\AgdaSpace{}%
\AgdaOperator{\AgdaDatatype{∶}}\AgdaSpace{}%
\AgdaGeneralizable{τ}\AgdaSpace{}%
\AgdaSymbol{→}\<%
\\
%
\>[4]\AgdaGeneralizable{Γ}\AgdaSpace{}%
\AgdaOperator{\AgdaInductiveConstructor{▶}}\AgdaSpace{}%
\AgdaGeneralizable{τ}\AgdaSpace{}%
\AgdaOperator{\AgdaDatatype{⊢}}\AgdaSpace{}%
\AgdaGeneralizable{e₁}\AgdaSpace{}%
\AgdaOperator{\AgdaDatatype{∶}}\AgdaSpace{}%
\AgdaFunction{wk}\AgdaSpace{}%
\AgdaGeneralizable{τ'}\AgdaSpace{}%
\AgdaSymbol{→}\<%
\\
%
\>[4]\AgdaComment{--------------------------}\<%
\\
%
\>[4]\AgdaGeneralizable{Γ}\AgdaSpace{}%
\AgdaOperator{\AgdaDatatype{⊢}}\AgdaSpace{}%
\AgdaOperator{\AgdaInductiveConstructor{let`x=}}\AgdaSpace{}%
\AgdaGeneralizable{e₂}\AgdaSpace{}%
\AgdaOperator{\AgdaInductiveConstructor{`in}}\AgdaSpace{}%
\AgdaGeneralizable{e₁}\AgdaSpace{}%
\AgdaOperator{\AgdaDatatype{∶}}\AgdaSpace{}%
\AgdaGeneralizable{τ'}\<%
\\
%
\\[\AgdaEmptyExtraSkip]%
%
\\[\AgdaEmptyExtraSkip]%
\>[0]\AgdaComment{--\ Renaming\ Typing}\<%
\\
%
\\[\AgdaEmptyExtraSkip]%
%
\\[\AgdaEmptyExtraSkip]%
\>[0]\AgdaKeyword{infix}\AgdaSpace{}%
\AgdaNumber{3}\AgdaSpace{}%
\AgdaOperator{\AgdaDatatype{\AgdaUnderscore{}∶\AgdaUnderscore{}⇒ᵣ\AgdaUnderscore{}}}\<%
\end{code}
\newcommand{\FoRenTyping}[0]{\begin{code}%
\>[0]\AgdaKeyword{data}\AgdaSpace{}%
\AgdaOperator{\AgdaDatatype{\AgdaUnderscore{}∶\AgdaUnderscore{}⇒ᵣ\AgdaUnderscore{}}}\AgdaSpace{}%
\AgdaSymbol{:}\AgdaSpace{}%
\AgdaFunction{Ren}\AgdaSpace{}%
\AgdaGeneralizable{S₁}\AgdaSpace{}%
\AgdaGeneralizable{S₂}\AgdaSpace{}%
\AgdaSymbol{→}\AgdaSpace{}%
\AgdaDatatype{Ctx}\AgdaSpace{}%
\AgdaGeneralizable{S₁}\AgdaSpace{}%
\AgdaSymbol{→}\AgdaSpace{}%
\AgdaDatatype{Ctx}\AgdaSpace{}%
\AgdaGeneralizable{S₂}\AgdaSpace{}%
\AgdaSymbol{→}\AgdaSpace{}%
\AgdaPrimitive{Set}\AgdaSpace{}%
\AgdaKeyword{where}\<%
\\
\>[0][@{}l@{\AgdaIndent{0}}]%
\>[2]\AgdaInductiveConstructor{⊢drop-cstrᵣ}\AgdaSpace{}%
\AgdaSymbol{:}\AgdaSpace{}%
\AgdaSymbol{∀}\AgdaSpace{}%
\AgdaSymbol{\{}\AgdaBound{Γ₁}\AgdaSpace{}%
\AgdaSymbol{:}\AgdaSpace{}%
\AgdaDatatype{Ctx}\AgdaSpace{}%
\AgdaGeneralizable{S₁}\AgdaSymbol{\}}\AgdaSpace{}%
\AgdaSymbol{\{}\AgdaBound{Γ₂}\AgdaSpace{}%
\AgdaSymbol{:}\AgdaSpace{}%
\AgdaDatatype{Ctx}\AgdaSpace{}%
\AgdaGeneralizable{S₂}\AgdaSymbol{\}}\AgdaSpace{}%
\AgdaSymbol{\{}\AgdaBound{τ}\AgdaSymbol{\}}\AgdaSpace{}%
\AgdaSymbol{\{}\AgdaBound{o}\AgdaSymbol{\}}\AgdaSpace{}%
\AgdaSymbol{→}\<%
\\
\>[2][@{}l@{\AgdaIndent{0}}]%
\>[4]\AgdaGeneralizable{ρ}\AgdaSpace{}%
\AgdaOperator{\AgdaDatatype{∶}}\AgdaSpace{}%
\AgdaBound{Γ₁}\AgdaSpace{}%
\AgdaOperator{\AgdaDatatype{⇒ᵣ}}\AgdaSpace{}%
\AgdaBound{Γ₂}\AgdaSpace{}%
\AgdaSymbol{→}\<%
\\
%
\>[4]\AgdaGeneralizable{ρ}\AgdaSpace{}%
\AgdaOperator{\AgdaDatatype{∶}}\AgdaSpace{}%
\AgdaBound{Γ₁}\AgdaSpace{}%
\AgdaOperator{\AgdaDatatype{⇒ᵣ}}\AgdaSpace{}%
\AgdaSymbol{(}\AgdaBound{Γ₂}\AgdaSpace{}%
\AgdaOperator{\AgdaInductiveConstructor{▸}}\AgdaSpace{}%
\AgdaSymbol{(}\AgdaBound{o}\AgdaSpace{}%
\AgdaOperator{\AgdaInductiveConstructor{∶}}\AgdaSpace{}%
\AgdaBound{τ}\AgdaSymbol{))}\<%
\\
%
\>[2]\AgdaComment{--\ ...}\<%
\end{code}}
\begin{code}[hide]%
%
\>[2]\AgdaInductiveConstructor{⊢idᵣ}\AgdaSpace{}%
\AgdaSymbol{:}\AgdaSpace{}%
\AgdaSymbol{∀}\AgdaSpace{}%
\AgdaSymbol{\{}\AgdaBound{Γ}\AgdaSymbol{\}}\AgdaSpace{}%
\AgdaSymbol{→}\AgdaSpace{}%
\AgdaOperator{\AgdaDatatype{\AgdaUnderscore{}∶\AgdaUnderscore{}⇒ᵣ\AgdaUnderscore{}}}\AgdaSpace{}%
\AgdaSymbol{\{}\AgdaArgument{S₁}\AgdaSpace{}%
\AgdaSymbol{=}\AgdaSpace{}%
\AgdaGeneralizable{S}\AgdaSymbol{\}}\AgdaSpace{}%
\AgdaSymbol{\{}\AgdaArgument{S₂}\AgdaSpace{}%
\AgdaSymbol{=}\AgdaSpace{}%
\AgdaGeneralizable{S}\AgdaSymbol{\}}\AgdaSpace{}%
\AgdaFunction{idᵣ}\AgdaSpace{}%
\AgdaBound{Γ}\AgdaSpace{}%
\AgdaBound{Γ}\<%
\\
%
\>[2]\AgdaInductiveConstructor{⊢extᵣ}\AgdaSpace{}%
\AgdaSymbol{:}\AgdaSpace{}%
\AgdaSymbol{∀}\AgdaSpace{}%
\AgdaSymbol{\{}\AgdaBound{Γ₁}\AgdaSpace{}%
\AgdaSymbol{:}\AgdaSpace{}%
\AgdaDatatype{Ctx}\AgdaSpace{}%
\AgdaGeneralizable{S₁}\AgdaSymbol{\}}\AgdaSpace{}%
\AgdaSymbol{\{}\AgdaBound{Γ₂}\AgdaSpace{}%
\AgdaSymbol{:}\AgdaSpace{}%
\AgdaDatatype{Ctx}\AgdaSpace{}%
\AgdaGeneralizable{S₂}\AgdaSymbol{\}}\AgdaSpace{}%
\AgdaSymbol{\{}\AgdaBound{I}\AgdaSpace{}%
\AgdaSymbol{:}\AgdaSpace{}%
\AgdaDatatype{Term}\AgdaSpace{}%
\AgdaGeneralizable{S₁}\AgdaSpace{}%
\AgdaSymbol{(}\AgdaFunction{item-of}\AgdaSpace{}%
\AgdaGeneralizable{s}\AgdaSymbol{)\}}\AgdaSpace{}%
\AgdaSymbol{→}\<%
\\
\>[2][@{}l@{\AgdaIndent{0}}]%
\>[4]\AgdaGeneralizable{ρ}\AgdaSpace{}%
\AgdaOperator{\AgdaDatatype{∶}}\AgdaSpace{}%
\AgdaBound{Γ₁}\AgdaSpace{}%
\AgdaOperator{\AgdaDatatype{⇒ᵣ}}\AgdaSpace{}%
\AgdaBound{Γ₂}\AgdaSpace{}%
\AgdaSymbol{→}\<%
\\
%
\>[4]\AgdaComment{--------------------------------------}\<%
\\
%
\>[4]\AgdaFunction{extᵣ}\AgdaSpace{}%
\AgdaGeneralizable{ρ}\AgdaSpace{}%
\AgdaOperator{\AgdaDatatype{∶}}\AgdaSpace{}%
\AgdaBound{Γ₁}\AgdaSpace{}%
\AgdaOperator{\AgdaInductiveConstructor{▶}}\AgdaSpace{}%
\AgdaBound{I}\AgdaSpace{}%
\AgdaOperator{\AgdaDatatype{⇒ᵣ}}\AgdaSpace{}%
\AgdaBound{Γ₂}\AgdaSpace{}%
\AgdaOperator{\AgdaInductiveConstructor{▶}}\AgdaSpace{}%
\AgdaFunction{ren}\AgdaSpace{}%
\AgdaGeneralizable{ρ}\AgdaSpace{}%
\AgdaBound{I}\<%
\\
%
\>[2]\AgdaInductiveConstructor{⊢dropᵣ}\AgdaSpace{}%
\AgdaSymbol{:}\AgdaSpace{}%
\AgdaSymbol{∀}\AgdaSpace{}%
\AgdaSymbol{\{}\AgdaBound{Γ₁}\AgdaSpace{}%
\AgdaSymbol{:}\AgdaSpace{}%
\AgdaDatatype{Ctx}\AgdaSpace{}%
\AgdaGeneralizable{S₁}\AgdaSymbol{\}}\AgdaSpace{}%
\AgdaSymbol{\{}\AgdaBound{Γ₂}\AgdaSpace{}%
\AgdaSymbol{:}\AgdaSpace{}%
\AgdaDatatype{Ctx}\AgdaSpace{}%
\AgdaGeneralizable{S₂}\AgdaSymbol{\}}\AgdaSpace{}%
\AgdaSymbol{\{}\AgdaBound{I}\AgdaSpace{}%
\AgdaSymbol{:}\AgdaSpace{}%
\AgdaDatatype{Term}\AgdaSpace{}%
\AgdaGeneralizable{S₂}\AgdaSpace{}%
\AgdaSymbol{(}\AgdaFunction{item-of}\AgdaSpace{}%
\AgdaGeneralizable{s}\AgdaSymbol{)\}}\AgdaSpace{}%
\AgdaSymbol{→}\<%
\\
\>[2][@{}l@{\AgdaIndent{0}}]%
\>[4]\AgdaGeneralizable{ρ}\AgdaSpace{}%
\AgdaOperator{\AgdaDatatype{∶}}\AgdaSpace{}%
\AgdaBound{Γ₁}\AgdaSpace{}%
\AgdaOperator{\AgdaDatatype{⇒ᵣ}}\AgdaSpace{}%
\AgdaBound{Γ₂}\AgdaSpace{}%
\AgdaSymbol{→}\<%
\\
%
\>[4]\AgdaComment{-------------}\<%
\\
%
\>[4]\AgdaFunction{dropᵣ}\AgdaSpace{}%
\AgdaGeneralizable{ρ}\AgdaSpace{}%
\AgdaOperator{\AgdaDatatype{∶}}\AgdaSpace{}%
\AgdaBound{Γ₁}\AgdaSpace{}%
\AgdaOperator{\AgdaDatatype{⇒ᵣ}}\AgdaSpace{}%
\AgdaBound{Γ₂}\AgdaSpace{}%
\AgdaOperator{\AgdaInductiveConstructor{▶}}\AgdaSpace{}%
\AgdaBound{I}\<%
\\
%
\\[\AgdaEmptyExtraSkip]%
\>[0]\AgdaFunction{⊢wkᵣ}\AgdaSpace{}%
\AgdaSymbol{:}\AgdaSpace{}%
\AgdaSymbol{∀}\AgdaSpace{}%
\AgdaSymbol{\{}\AgdaBound{I}\AgdaSpace{}%
\AgdaSymbol{:}\AgdaSpace{}%
\AgdaDatatype{Term}\AgdaSpace{}%
\AgdaGeneralizable{S}\AgdaSpace{}%
\AgdaSymbol{(}\AgdaFunction{item-of}\AgdaSpace{}%
\AgdaGeneralizable{s}\AgdaSymbol{)\}}\AgdaSpace{}%
\AgdaSymbol{→}\AgdaSpace{}%
\AgdaSymbol{(}\AgdaFunction{dropᵣ}\AgdaSpace{}%
\AgdaFunction{idᵣ}\AgdaSymbol{)}\AgdaSpace{}%
\AgdaOperator{\AgdaDatatype{∶}}\AgdaSpace{}%
\AgdaGeneralizable{Γ}\AgdaSpace{}%
\AgdaOperator{\AgdaDatatype{⇒ᵣ}}\AgdaSpace{}%
\AgdaSymbol{(}\AgdaGeneralizable{Γ}\AgdaSpace{}%
\AgdaOperator{\AgdaInductiveConstructor{▶}}\AgdaSpace{}%
\AgdaBound{I}\AgdaSymbol{)}\<%
\\
\>[0]\AgdaFunction{⊢wkᵣ}\AgdaSpace{}%
\AgdaSymbol{=}\AgdaSpace{}%
\AgdaInductiveConstructor{⊢dropᵣ}\AgdaSpace{}%
\AgdaInductiveConstructor{⊢idᵣ}\<%
\\
%
\\[\AgdaEmptyExtraSkip]%
\>[0]\AgdaFunction{⊢wk-instᵣ}\AgdaSpace{}%
\AgdaSymbol{:}\AgdaSpace{}%
\AgdaSymbol{∀}\AgdaSpace{}%
\AgdaSymbol{\{}\AgdaBound{o}\AgdaSymbol{\}}\AgdaSpace{}%
\AgdaSymbol{→}\AgdaSpace{}%
\AgdaFunction{idᵣ}\AgdaSpace{}%
\AgdaOperator{\AgdaDatatype{∶}}\AgdaSpace{}%
\AgdaGeneralizable{Γ}\AgdaSpace{}%
\AgdaOperator{\AgdaDatatype{⇒ᵣ}}\AgdaSpace{}%
\AgdaSymbol{(}\AgdaGeneralizable{Γ}\AgdaSpace{}%
\AgdaOperator{\AgdaInductiveConstructor{▸}}\AgdaSpace{}%
\AgdaSymbol{(}\AgdaBound{o}\AgdaSpace{}%
\AgdaOperator{\AgdaInductiveConstructor{∶}}\AgdaSpace{}%
\AgdaGeneralizable{τ}\AgdaSymbol{))}\<%
\\
\>[0]\AgdaFunction{⊢wk-instᵣ}\AgdaSpace{}%
\AgdaSymbol{=}\AgdaSpace{}%
\AgdaInductiveConstructor{⊢drop-cstrᵣ}\AgdaSpace{}%
\AgdaInductiveConstructor{⊢idᵣ}\<%
\\
%
\\[\AgdaEmptyExtraSkip]%
\>[0]\AgdaFunction{extᵣidᵣ≡idᵣ}\AgdaSpace{}%
\AgdaSymbol{:}\AgdaSpace{}%
\AgdaSymbol{∀}\AgdaSpace{}%
\AgdaSymbol{(}\AgdaBound{x}\AgdaSpace{}%
\AgdaSymbol{:}\AgdaSpace{}%
\AgdaFunction{Var}\AgdaSpace{}%
\AgdaSymbol{(}\AgdaGeneralizable{S}\AgdaSpace{}%
\AgdaOperator{\AgdaInductiveConstructor{▷}}\AgdaSpace{}%
\AgdaGeneralizable{s'}\AgdaSymbol{)}\AgdaSpace{}%
\AgdaGeneralizable{s}\AgdaSymbol{)}\AgdaSpace{}%
\AgdaSymbol{→}\AgdaSpace{}%
\AgdaFunction{extᵣ}\AgdaSpace{}%
\AgdaFunction{idᵣ}\AgdaSpace{}%
\AgdaBound{x}\AgdaSpace{}%
\AgdaOperator{\AgdaDatatype{≡}}\AgdaSpace{}%
\AgdaFunction{idᵣ}\AgdaSpace{}%
\AgdaBound{x}\<%
\\
\>[0]\AgdaFunction{extᵣidᵣ≡idᵣ}\AgdaSpace{}%
\AgdaSymbol{(}\AgdaInductiveConstructor{here}\AgdaSpace{}%
\AgdaInductiveConstructor{refl}\AgdaSymbol{)}\AgdaSpace{}%
\AgdaSymbol{=}\AgdaSpace{}%
\AgdaInductiveConstructor{refl}\<%
\\
\>[0]\AgdaFunction{extᵣidᵣ≡idᵣ}\AgdaSpace{}%
\AgdaSymbol{(}\AgdaInductiveConstructor{there}\AgdaSpace{}%
\AgdaBound{x}\AgdaSymbol{)}\AgdaSpace{}%
\AgdaSymbol{=}\AgdaSpace{}%
\AgdaInductiveConstructor{refl}\<%
\\
%
\\[\AgdaEmptyExtraSkip]%
\>[0]\AgdaFunction{⊢ext-ρ₁≡ext-ρ₂}\AgdaSpace{}%
\AgdaSymbol{:}\AgdaSpace{}%
\AgdaSymbol{∀}\AgdaSpace{}%
\AgdaSymbol{\{}\AgdaBound{ρ₁}\AgdaSpace{}%
\AgdaBound{ρ₂}\AgdaSpace{}%
\AgdaSymbol{:}\AgdaSpace{}%
\AgdaFunction{Ren}\AgdaSpace{}%
\AgdaGeneralizable{S₁}\AgdaSpace{}%
\AgdaGeneralizable{S₂}\AgdaSymbol{\}}\AgdaSpace{}%
\AgdaSymbol{→}\<%
\\
\>[0][@{}l@{\AgdaIndent{0}}]%
\>[1]\AgdaSymbol{(∀}\AgdaSpace{}%
\AgdaSymbol{\{}\AgdaBound{s}\AgdaSymbol{\}}\AgdaSpace{}%
\AgdaSymbol{(}\AgdaBound{x}\AgdaSpace{}%
\AgdaSymbol{:}\AgdaSpace{}%
\AgdaFunction{Var}\AgdaSpace{}%
\AgdaGeneralizable{S₁}\AgdaSpace{}%
\AgdaBound{s}\AgdaSymbol{)}\AgdaSpace{}%
\AgdaSymbol{→}\AgdaSpace{}%
\AgdaBound{ρ₁}\AgdaSpace{}%
\AgdaBound{x}\AgdaSpace{}%
\AgdaOperator{\AgdaDatatype{≡}}\AgdaSpace{}%
\AgdaBound{ρ₂}\AgdaSpace{}%
\AgdaBound{x}\AgdaSymbol{)}\AgdaSpace{}%
\AgdaSymbol{→}\<%
\\
%
\>[1]\AgdaSymbol{(∀}\AgdaSpace{}%
\AgdaSymbol{\{}\AgdaBound{s}\AgdaSymbol{\}}\AgdaSpace{}%
\AgdaSymbol{(}\AgdaBound{x}\AgdaSpace{}%
\AgdaSymbol{:}\AgdaSpace{}%
\AgdaFunction{Var}\AgdaSpace{}%
\AgdaSymbol{(}\AgdaGeneralizable{S₁}\AgdaSpace{}%
\AgdaOperator{\AgdaInductiveConstructor{▷}}\AgdaSpace{}%
\AgdaGeneralizable{s'}\AgdaSymbol{)}\AgdaSpace{}%
\AgdaBound{s}\AgdaSymbol{)}\AgdaSpace{}%
\AgdaSymbol{→}\AgdaSpace{}%
\AgdaSymbol{(}\AgdaFunction{extᵣ}\AgdaSpace{}%
\AgdaBound{ρ₁}\AgdaSymbol{)}\AgdaSpace{}%
\AgdaBound{x}\AgdaSpace{}%
\AgdaOperator{\AgdaDatatype{≡}}\AgdaSpace{}%
\AgdaSymbol{(}\AgdaFunction{extᵣ}\AgdaSpace{}%
\AgdaBound{ρ₂}\AgdaSymbol{)}\AgdaSpace{}%
\AgdaBound{x}\AgdaSymbol{)}\<%
\\
\>[0]\AgdaFunction{⊢ext-ρ₁≡ext-ρ₂}\AgdaSpace{}%
\AgdaBound{ρ₁≡ρ₂}\AgdaSpace{}%
\AgdaSymbol{(}\AgdaInductiveConstructor{here}\AgdaSpace{}%
\AgdaInductiveConstructor{refl}\AgdaSymbol{)}\AgdaSpace{}%
\AgdaSymbol{=}\AgdaSpace{}%
\AgdaInductiveConstructor{refl}\<%
\\
\>[0]\AgdaFunction{⊢ext-ρ₁≡ext-ρ₂}\AgdaSpace{}%
\AgdaBound{ρ₁≡ρ₂}\AgdaSpace{}%
\AgdaSymbol{(}\AgdaInductiveConstructor{there}\AgdaSpace{}%
\AgdaBound{x}\AgdaSymbol{)}\AgdaSpace{}%
\AgdaSymbol{=}\AgdaSpace{}%
\AgdaFunction{cong}\AgdaSpace{}%
\AgdaInductiveConstructor{there}\AgdaSpace{}%
\AgdaSymbol{(}\AgdaBound{ρ₁≡ρ₂}\AgdaSpace{}%
\AgdaBound{x}\AgdaSymbol{)}\<%
\\
%
\\[\AgdaEmptyExtraSkip]%
\>[0]\AgdaFunction{ρ₁≡ρ₂→ρ₁τ≡ρ₂τ}\AgdaSpace{}%
\AgdaSymbol{:}\AgdaSpace{}%
\AgdaSymbol{∀}\AgdaSpace{}%
\AgdaSymbol{\{}\AgdaBound{ρ₁}\AgdaSpace{}%
\AgdaBound{ρ₂}\AgdaSpace{}%
\AgdaSymbol{:}\AgdaSpace{}%
\AgdaFunction{Ren}\AgdaSpace{}%
\AgdaGeneralizable{S₁}\AgdaSpace{}%
\AgdaGeneralizable{S₂}\AgdaSymbol{\}}\AgdaSpace{}%
\AgdaSymbol{\{}\AgdaBound{τ}\AgdaSpace{}%
\AgdaSymbol{:}\AgdaSpace{}%
\AgdaFunction{Type}\AgdaSpace{}%
\AgdaGeneralizable{S₁}\AgdaSymbol{\}}\AgdaSpace{}%
\AgdaSymbol{→}\<%
\\
\>[0][@{}l@{\AgdaIndent{0}}]%
\>[2]\AgdaSymbol{(∀}\AgdaSpace{}%
\AgdaSymbol{\{}\AgdaBound{s}\AgdaSymbol{\}}\AgdaSpace{}%
\AgdaSymbol{(}\AgdaBound{x}\AgdaSpace{}%
\AgdaSymbol{:}\AgdaSpace{}%
\AgdaFunction{Var}\AgdaSpace{}%
\AgdaGeneralizable{S₁}\AgdaSpace{}%
\AgdaBound{s}\AgdaSymbol{)}\AgdaSpace{}%
\AgdaSymbol{→}\AgdaSpace{}%
\AgdaBound{ρ₁}\AgdaSpace{}%
\AgdaBound{x}\AgdaSpace{}%
\AgdaOperator{\AgdaDatatype{≡}}\AgdaSpace{}%
\AgdaBound{ρ₂}\AgdaSpace{}%
\AgdaBound{x}\AgdaSymbol{)}\AgdaSpace{}%
\AgdaSymbol{→}\<%
\\
%
\>[2]\AgdaFunction{ren}\AgdaSpace{}%
\AgdaBound{ρ₁}\AgdaSpace{}%
\AgdaBound{τ}\AgdaSpace{}%
\AgdaOperator{\AgdaDatatype{≡}}\AgdaSpace{}%
\AgdaFunction{ren}\AgdaSpace{}%
\AgdaBound{ρ₂}\AgdaSpace{}%
\AgdaBound{τ}\<%
\\
\>[0]\AgdaFunction{ρ₁≡ρ₂→ρ₁τ≡ρ₂τ}\AgdaSpace{}%
\AgdaSymbol{\{}\AgdaArgument{τ}\AgdaSpace{}%
\AgdaSymbol{=}\AgdaSpace{}%
\AgdaOperator{\AgdaInductiveConstructor{`}}\AgdaSpace{}%
\AgdaBound{x}\AgdaSymbol{\}}\AgdaSpace{}%
\AgdaBound{ρ₁≡ρ₂}\AgdaSpace{}%
\AgdaSymbol{=}\AgdaSpace{}%
\AgdaFunction{cong}\AgdaSpace{}%
\AgdaOperator{\AgdaInductiveConstructor{`\AgdaUnderscore{}}}\AgdaSpace{}%
\AgdaSymbol{(}\AgdaBound{ρ₁≡ρ₂}\AgdaSpace{}%
\AgdaBound{x}\AgdaSymbol{)}\<%
\\
\>[0]\AgdaFunction{ρ₁≡ρ₂→ρ₁τ≡ρ₂τ}\AgdaSpace{}%
\AgdaSymbol{\{}\AgdaArgument{τ}\AgdaSpace{}%
\AgdaSymbol{=}\AgdaSpace{}%
\AgdaInductiveConstructor{`⊤}\AgdaSymbol{\}}\AgdaSpace{}%
\AgdaBound{ρ₁≡ρ₂}\AgdaSpace{}%
\AgdaSymbol{=}\AgdaSpace{}%
\AgdaInductiveConstructor{refl}\<%
\\
\>[0]\AgdaFunction{ρ₁≡ρ₂→ρ₁τ≡ρ₂τ}\AgdaSpace{}%
\AgdaSymbol{\{}\AgdaArgument{τ}\AgdaSpace{}%
\AgdaSymbol{=}\AgdaSpace{}%
\AgdaBound{τ₁}\AgdaSpace{}%
\AgdaOperator{\AgdaInductiveConstructor{⇒}}\AgdaSpace{}%
\AgdaBound{τ₂}\AgdaSymbol{\}}\AgdaSpace{}%
\AgdaBound{ρ₁≡ρ₂}\AgdaSpace{}%
\AgdaSymbol{=}\AgdaSpace{}%
\AgdaFunction{cong₂}\AgdaSpace{}%
\AgdaOperator{\AgdaInductiveConstructor{\AgdaUnderscore{}⇒\AgdaUnderscore{}}}\AgdaSpace{}%
\AgdaSymbol{(}\AgdaFunction{ρ₁≡ρ₂→ρ₁τ≡ρ₂τ}\AgdaSpace{}%
\AgdaBound{ρ₁≡ρ₂}\AgdaSymbol{)}\AgdaSpace{}%
\AgdaSymbol{(}\AgdaFunction{ρ₁≡ρ₂→ρ₁τ≡ρ₂τ}\AgdaSpace{}%
\AgdaBound{ρ₁≡ρ₂}\AgdaSymbol{)}\<%
\\
\>[0]\AgdaFunction{ρ₁≡ρ₂→ρ₁τ≡ρ₂τ}\AgdaSpace{}%
\AgdaSymbol{\{}\AgdaArgument{τ}\AgdaSpace{}%
\AgdaSymbol{=}\AgdaSpace{}%
\AgdaOperator{\AgdaInductiveConstructor{∀`α}}\AgdaSpace{}%
\AgdaBound{τ}\AgdaSymbol{\}}\AgdaSpace{}%
\AgdaBound{ρ₁≡ρ₂}\AgdaSpace{}%
\AgdaSymbol{=}\AgdaSpace{}%
\AgdaFunction{cong}\AgdaSpace{}%
\AgdaOperator{\AgdaInductiveConstructor{∀`α\AgdaUnderscore{}}}\AgdaSpace{}%
\AgdaSymbol{(}\AgdaFunction{ρ₁≡ρ₂→ρ₁τ≡ρ₂τ}\AgdaSpace{}%
\AgdaSymbol{(}\AgdaFunction{⊢ext-ρ₁≡ext-ρ₂}\AgdaSpace{}%
\AgdaBound{ρ₁≡ρ₂}\AgdaSymbol{))}\<%
\\
\>[0]\AgdaFunction{ρ₁≡ρ₂→ρ₁τ≡ρ₂τ}\AgdaSpace{}%
\AgdaSymbol{\{}\AgdaArgument{τ}\AgdaSpace{}%
\AgdaSymbol{=}\AgdaSpace{}%
\AgdaOperator{\AgdaInductiveConstructor{[}}\AgdaSpace{}%
\AgdaOperator{\AgdaInductiveConstructor{`}}\AgdaSpace{}%
\AgdaBound{o}\AgdaSpace{}%
\AgdaOperator{\AgdaInductiveConstructor{∶}}\AgdaSpace{}%
\AgdaBound{τ}\AgdaSpace{}%
\AgdaOperator{\AgdaInductiveConstructor{]⇒}}\AgdaSpace{}%
\AgdaBound{τ'}\AgdaSymbol{\}}\AgdaSpace{}%
\AgdaBound{ρ₁≡ρ₂}\AgdaSpace{}%
\AgdaSymbol{=}\AgdaSpace{}%
\AgdaFunction{cong₂}\AgdaSpace{}%
\AgdaOperator{\AgdaInductiveConstructor{[\AgdaUnderscore{}]⇒\AgdaUnderscore{}}}\AgdaSpace{}%
\AgdaSymbol{(}\AgdaFunction{cong₂}\AgdaSpace{}%
\AgdaOperator{\AgdaInductiveConstructor{\AgdaUnderscore{}∶\AgdaUnderscore{}}}\AgdaSpace{}%
\AgdaSymbol{(}\AgdaFunction{cong}\AgdaSpace{}%
\AgdaOperator{\AgdaInductiveConstructor{`\AgdaUnderscore{}}}\AgdaSpace{}%
\AgdaSymbol{(}\AgdaBound{ρ₁≡ρ₂}\AgdaSpace{}%
\AgdaBound{o}\AgdaSymbol{))}\AgdaSpace{}%
\AgdaSymbol{(}\AgdaFunction{ρ₁≡ρ₂→ρ₁τ≡ρ₂τ}\AgdaSpace{}%
\AgdaBound{ρ₁≡ρ₂}\AgdaSymbol{))}\AgdaSpace{}%
\AgdaSymbol{(}\AgdaFunction{ρ₁≡ρ₂→ρ₁τ≡ρ₂τ}\AgdaSpace{}%
\AgdaBound{ρ₁≡ρ₂}\AgdaSymbol{)}\<%
\end{code}
\newcommand{\FoRenIdEq}[0]{\begin{code}%
\>[0]\AgdaFunction{idᵣτ≡τ}\AgdaSpace{}%
\AgdaSymbol{:}\AgdaSpace{}%
\AgdaSymbol{(}\AgdaBound{τ}\AgdaSpace{}%
\AgdaSymbol{:}\AgdaSpace{}%
\AgdaFunction{Type}\AgdaSpace{}%
\AgdaGeneralizable{S}\AgdaSymbol{)}\AgdaSpace{}%
\AgdaSymbol{→}\<%
\\
\>[0][@{}l@{\AgdaIndent{0}}]%
\>[2]\AgdaFunction{ren}\AgdaSpace{}%
\AgdaFunction{idᵣ}\AgdaSpace{}%
\AgdaBound{τ}\AgdaSpace{}%
\AgdaOperator{\AgdaDatatype{≡}}\AgdaSpace{}%
\AgdaBound{τ}\<%
\\
\>[0]\AgdaFunction{idᵣτ≡τ}\AgdaSpace{}%
\AgdaSymbol{(}\AgdaOperator{\AgdaInductiveConstructor{`}}\AgdaSpace{}%
\AgdaBound{x}\AgdaSymbol{)}\AgdaSpace{}%
\AgdaSymbol{=}\AgdaSpace{}%
\AgdaInductiveConstructor{refl}\<%
\\
\>[0]\AgdaFunction{idᵣτ≡τ}\AgdaSpace{}%
\AgdaInductiveConstructor{`⊤}\AgdaSpace{}%
\AgdaSymbol{=}\AgdaSpace{}%
\AgdaInductiveConstructor{refl}\<%
\\
\>[0]\AgdaFunction{idᵣτ≡τ}\AgdaSpace{}%
\AgdaSymbol{(}\AgdaBound{τ₁}\AgdaSpace{}%
\AgdaOperator{\AgdaInductiveConstructor{⇒}}\AgdaSpace{}%
\AgdaBound{τ₂}\AgdaSymbol{)}\AgdaSpace{}%
\AgdaSymbol{=}\AgdaSpace{}%
\AgdaFunction{cong₂}\AgdaSpace{}%
\AgdaOperator{\AgdaInductiveConstructor{\AgdaUnderscore{}⇒\AgdaUnderscore{}}}\AgdaSpace{}%
\AgdaSymbol{(}\AgdaFunction{idᵣτ≡τ}\AgdaSpace{}%
\AgdaBound{τ₁}\AgdaSymbol{)}\AgdaSpace{}%
\AgdaSymbol{(}\AgdaFunction{idᵣτ≡τ}\AgdaSpace{}%
\AgdaBound{τ₂}\AgdaSymbol{)}\<%
\\
\>[0]\AgdaFunction{idᵣτ≡τ}\AgdaSpace{}%
\AgdaSymbol{(}\AgdaOperator{\AgdaInductiveConstructor{∀`α}}\AgdaSpace{}%
\AgdaBound{τ}\AgdaSymbol{)}\AgdaSpace{}%
\AgdaSymbol{=}\AgdaSpace{}%
\AgdaFunction{cong}\AgdaSpace{}%
\AgdaOperator{\AgdaInductiveConstructor{∀`α\AgdaUnderscore{}}}\AgdaSpace{}%
\AgdaSymbol{(}\AgdaFunction{trans}\AgdaSpace{}%
\AgdaSymbol{(}\AgdaFunction{ρ₁≡ρ₂→ρ₁τ≡ρ₂τ}\AgdaSpace{}%
\AgdaFunction{extᵣidᵣ≡idᵣ}\AgdaSymbol{)}\AgdaSpace{}%
\AgdaSymbol{(}\AgdaFunction{idᵣτ≡τ}\AgdaSpace{}%
\AgdaBound{τ}\AgdaSymbol{))}\<%
\\
\>[0]\AgdaFunction{idᵣτ≡τ}\AgdaSpace{}%
\AgdaSymbol{(}\AgdaOperator{\AgdaInductiveConstructor{[}}\AgdaSpace{}%
\AgdaOperator{\AgdaInductiveConstructor{`}}\AgdaSpace{}%
\AgdaBound{o}\AgdaSpace{}%
\AgdaOperator{\AgdaInductiveConstructor{∶}}\AgdaSpace{}%
\AgdaBound{τ}\AgdaSpace{}%
\AgdaOperator{\AgdaInductiveConstructor{]⇒}}\AgdaSpace{}%
\AgdaBound{τ'}\AgdaSymbol{)}\AgdaSpace{}%
\AgdaSymbol{=}\AgdaSpace{}%
\AgdaFunction{cong₂}\AgdaSpace{}%
\AgdaOperator{\AgdaInductiveConstructor{[\AgdaUnderscore{}]⇒\AgdaUnderscore{}}}\AgdaSpace{}%
\AgdaSymbol{(}\AgdaFunction{cong₂}\AgdaSpace{}%
\AgdaOperator{\AgdaInductiveConstructor{\AgdaUnderscore{}∶\AgdaUnderscore{}}}\AgdaSpace{}%
\AgdaInductiveConstructor{refl}\AgdaSpace{}%
\AgdaSymbol{(}\AgdaFunction{idᵣτ≡τ}\AgdaSpace{}%
\AgdaBound{τ}\AgdaSymbol{))}\AgdaSpace{}%
\AgdaSymbol{(}\AgdaFunction{idᵣτ≡τ}\AgdaSpace{}%
\AgdaBound{τ'}\AgdaSymbol{)}\<%
\end{code}}
\begin{code}[hide]%
\>[0]\AgdaComment{--\ Substitution\ Typing\ ------------------------------------------------------------------}\<%
\\
%
\\[\AgdaEmptyExtraSkip]%
%
\\[\AgdaEmptyExtraSkip]%
\>[0]\AgdaKeyword{infix}\AgdaSpace{}%
\AgdaNumber{3}\AgdaSpace{}%
\AgdaOperator{\AgdaDatatype{\AgdaUnderscore{}∶\AgdaUnderscore{}⇒ₛ\AgdaUnderscore{}}}\<%
\end{code}
\newcommand{\FoSubTyping}[0]{\begin{code}%
\>[0]\AgdaKeyword{data}\AgdaSpace{}%
\AgdaOperator{\AgdaDatatype{\AgdaUnderscore{}∶\AgdaUnderscore{}⇒ₛ\AgdaUnderscore{}}}\AgdaSpace{}%
\AgdaSymbol{:}\AgdaSpace{}%
\AgdaFunction{Sub}\AgdaSpace{}%
\AgdaGeneralizable{S₁}\AgdaSpace{}%
\AgdaGeneralizable{S₂}\AgdaSpace{}%
\AgdaSymbol{→}\AgdaSpace{}%
\AgdaDatatype{Ctx}\AgdaSpace{}%
\AgdaGeneralizable{S₁}\AgdaSpace{}%
\AgdaSymbol{→}\AgdaSpace{}%
\AgdaDatatype{Ctx}\AgdaSpace{}%
\AgdaGeneralizable{S₂}\AgdaSpace{}%
\AgdaSymbol{→}\AgdaSpace{}%
\AgdaPrimitive{Set}\AgdaSpace{}%
\AgdaKeyword{where}\<%
\\
\>[0][@{}l@{\AgdaIndent{0}}]%
\>[2]\AgdaInductiveConstructor{⊢typeₛ}\AgdaSpace{}%
\AgdaSymbol{:}\AgdaSpace{}%
\AgdaSymbol{∀}\AgdaSpace{}%
\AgdaSymbol{\{}\AgdaBound{Γ₁}\AgdaSpace{}%
\AgdaSymbol{:}\AgdaSpace{}%
\AgdaDatatype{Ctx}\AgdaSpace{}%
\AgdaGeneralizable{S₁}\AgdaSymbol{\}}\AgdaSpace{}%
\AgdaSymbol{\{}\AgdaBound{Γ₂}\AgdaSpace{}%
\AgdaSymbol{:}\AgdaSpace{}%
\AgdaDatatype{Ctx}\AgdaSpace{}%
\AgdaGeneralizable{S₂}\AgdaSymbol{\}}\AgdaSpace{}%
\AgdaSymbol{\{}\AgdaBound{τ}\AgdaSpace{}%
\AgdaSymbol{:}\AgdaSpace{}%
\AgdaFunction{Type}\AgdaSpace{}%
\AgdaGeneralizable{S₂}\AgdaSymbol{\}}\AgdaSpace{}%
\AgdaSymbol{→}\<%
\\
\>[2][@{}l@{\AgdaIndent{0}}]%
\>[4]\AgdaGeneralizable{σ}\AgdaSpace{}%
\AgdaOperator{\AgdaDatatype{∶}}\AgdaSpace{}%
\AgdaBound{Γ₁}\AgdaSpace{}%
\AgdaOperator{\AgdaDatatype{⇒ₛ}}\AgdaSpace{}%
\AgdaBound{Γ₂}\AgdaSpace{}%
\AgdaSymbol{→}\<%
\\
%
\>[4]\AgdaFunction{single-typeₛ}\AgdaSpace{}%
\AgdaGeneralizable{σ}\AgdaSpace{}%
\AgdaBound{τ}\AgdaSpace{}%
\AgdaOperator{\AgdaDatatype{∶}}\AgdaSpace{}%
\AgdaBound{Γ₁}\AgdaSpace{}%
\AgdaOperator{\AgdaInductiveConstructor{▶}}\AgdaSpace{}%
\AgdaInductiveConstructor{⋆}\AgdaSpace{}%
\AgdaOperator{\AgdaDatatype{⇒ₛ}}\AgdaSpace{}%
\AgdaBound{Γ₂}\<%
\\
%
\>[2]\AgdaComment{--\ ...}\<%
\end{code}}
\begin{code}[hide]%
%
\>[2]\AgdaInductiveConstructor{⊢idₛ}\AgdaSpace{}%
\AgdaSymbol{:}\AgdaSpace{}%
\AgdaSymbol{∀}\AgdaSpace{}%
\AgdaSymbol{\{}\AgdaBound{Γ}\AgdaSymbol{\}}\AgdaSpace{}%
\AgdaSymbol{→}\AgdaSpace{}%
\AgdaOperator{\AgdaDatatype{\AgdaUnderscore{}∶\AgdaUnderscore{}⇒ₛ\AgdaUnderscore{}}}\AgdaSpace{}%
\AgdaSymbol{\{}\AgdaArgument{S₁}\AgdaSpace{}%
\AgdaSymbol{=}\AgdaSpace{}%
\AgdaGeneralizable{S}\AgdaSymbol{\}}\AgdaSpace{}%
\AgdaSymbol{\{}\AgdaArgument{S₂}\AgdaSpace{}%
\AgdaSymbol{=}\AgdaSpace{}%
\AgdaGeneralizable{S}\AgdaSymbol{\}}\AgdaSpace{}%
\AgdaFunction{idₛ}\AgdaSpace{}%
\AgdaBound{Γ}\AgdaSpace{}%
\AgdaBound{Γ}\<%
\\
%
\>[2]\AgdaInductiveConstructor{⊢extₛ}%
\>[9]\AgdaSymbol{:}\AgdaSpace{}%
\AgdaSymbol{∀}\AgdaSpace{}%
\AgdaSymbol{\{}\AgdaBound{Γ₁}\AgdaSpace{}%
\AgdaSymbol{:}\AgdaSpace{}%
\AgdaDatatype{Ctx}\AgdaSpace{}%
\AgdaGeneralizable{S₁}\AgdaSymbol{\}}\AgdaSpace{}%
\AgdaSymbol{\{}\AgdaBound{Γ₂}\AgdaSpace{}%
\AgdaSymbol{:}\AgdaSpace{}%
\AgdaDatatype{Ctx}\AgdaSpace{}%
\AgdaGeneralizable{S₂}\AgdaSymbol{\}}\AgdaSpace{}%
\AgdaSymbol{\{}\AgdaBound{I}\AgdaSpace{}%
\AgdaSymbol{:}\AgdaSpace{}%
\AgdaDatatype{Term}\AgdaSpace{}%
\AgdaGeneralizable{S₁}\AgdaSpace{}%
\AgdaSymbol{(}\AgdaFunction{item-of}\AgdaSpace{}%
\AgdaGeneralizable{s}\AgdaSymbol{)\}}\AgdaSpace{}%
\AgdaSymbol{→}\<%
\\
\>[2][@{}l@{\AgdaIndent{0}}]%
\>[4]\AgdaGeneralizable{σ}\AgdaSpace{}%
\AgdaOperator{\AgdaDatatype{∶}}\AgdaSpace{}%
\AgdaBound{Γ₁}\AgdaSpace{}%
\AgdaOperator{\AgdaDatatype{⇒ₛ}}\AgdaSpace{}%
\AgdaBound{Γ₂}\AgdaSpace{}%
\AgdaSymbol{→}\<%
\\
%
\>[4]\AgdaFunction{extₛ}\AgdaSpace{}%
\AgdaGeneralizable{σ}\AgdaSpace{}%
\AgdaOperator{\AgdaDatatype{∶}}\AgdaSpace{}%
\AgdaBound{Γ₁}\AgdaSpace{}%
\AgdaOperator{\AgdaInductiveConstructor{▶}}\AgdaSpace{}%
\AgdaBound{I}\AgdaSpace{}%
\AgdaOperator{\AgdaDatatype{⇒ₛ}}\AgdaSpace{}%
\AgdaBound{Γ₂}\AgdaSpace{}%
\AgdaOperator{\AgdaInductiveConstructor{▶}}\AgdaSpace{}%
\AgdaFunction{sub}\AgdaSpace{}%
\AgdaGeneralizable{σ}\AgdaSpace{}%
\AgdaBound{I}\<%
\\
%
\>[2]\AgdaInductiveConstructor{⊢dropₛ}\AgdaSpace{}%
\AgdaSymbol{:}\AgdaSpace{}%
\AgdaSymbol{∀}\AgdaSpace{}%
\AgdaSymbol{\{}\AgdaBound{Γ₁}\AgdaSpace{}%
\AgdaSymbol{:}\AgdaSpace{}%
\AgdaDatatype{Ctx}\AgdaSpace{}%
\AgdaGeneralizable{S₁}\AgdaSymbol{\}}\AgdaSpace{}%
\AgdaSymbol{\{}\AgdaBound{Γ₂}\AgdaSpace{}%
\AgdaSymbol{:}\AgdaSpace{}%
\AgdaDatatype{Ctx}\AgdaSpace{}%
\AgdaGeneralizable{S₂}\AgdaSymbol{\}}\AgdaSpace{}%
\AgdaSymbol{\{}\AgdaBound{I}\AgdaSpace{}%
\AgdaSymbol{:}\AgdaSpace{}%
\AgdaDatatype{Term}\AgdaSpace{}%
\AgdaGeneralizable{S₂}\AgdaSpace{}%
\AgdaSymbol{(}\AgdaFunction{item-of}\AgdaSpace{}%
\AgdaGeneralizable{s}\AgdaSymbol{)\}}\AgdaSpace{}%
\AgdaSymbol{→}\<%
\\
\>[2][@{}l@{\AgdaIndent{0}}]%
\>[4]\AgdaGeneralizable{σ}\AgdaSpace{}%
\AgdaOperator{\AgdaDatatype{∶}}\AgdaSpace{}%
\AgdaBound{Γ₁}\AgdaSpace{}%
\AgdaOperator{\AgdaDatatype{⇒ₛ}}\AgdaSpace{}%
\AgdaBound{Γ₂}\AgdaSpace{}%
\AgdaSymbol{→}\<%
\\
%
\>[4]\AgdaFunction{dropₛ}\AgdaSpace{}%
\AgdaGeneralizable{σ}\AgdaSpace{}%
\AgdaOperator{\AgdaDatatype{∶}}\AgdaSpace{}%
\AgdaBound{Γ₁}\AgdaSpace{}%
\AgdaOperator{\AgdaDatatype{⇒ₛ}}\AgdaSpace{}%
\AgdaSymbol{(}\AgdaBound{Γ₂}\AgdaSpace{}%
\AgdaOperator{\AgdaInductiveConstructor{▶}}\AgdaSpace{}%
\AgdaBound{I}\AgdaSymbol{)}\<%
\\
%
\>[2]\AgdaInductiveConstructor{⊢drop-cstrₛ}\AgdaSpace{}%
\AgdaSymbol{:}\AgdaSpace{}%
\AgdaSymbol{∀}\AgdaSpace{}%
\AgdaSymbol{\{}\AgdaBound{Γ₁}\AgdaSpace{}%
\AgdaSymbol{:}\AgdaSpace{}%
\AgdaDatatype{Ctx}\AgdaSpace{}%
\AgdaGeneralizable{S₁}\AgdaSymbol{\}}\AgdaSpace{}%
\AgdaSymbol{\{}\AgdaBound{Γ₂}\AgdaSpace{}%
\AgdaSymbol{:}\AgdaSpace{}%
\AgdaDatatype{Ctx}\AgdaSpace{}%
\AgdaGeneralizable{S₂}\AgdaSymbol{\}}\AgdaSpace{}%
\AgdaSymbol{\{}\AgdaBound{τ}\AgdaSymbol{\}}\AgdaSpace{}%
\AgdaSymbol{\{}\AgdaBound{o}\AgdaSymbol{\}}\AgdaSpace{}%
\AgdaSymbol{→}\<%
\\
\>[2][@{}l@{\AgdaIndent{0}}]%
\>[4]\AgdaGeneralizable{σ}\AgdaSpace{}%
\AgdaOperator{\AgdaDatatype{∶}}\AgdaSpace{}%
\AgdaBound{Γ₁}\AgdaSpace{}%
\AgdaOperator{\AgdaDatatype{⇒ₛ}}\AgdaSpace{}%
\AgdaBound{Γ₂}\AgdaSpace{}%
\AgdaSymbol{→}\<%
\\
%
\>[4]\AgdaGeneralizable{σ}\AgdaSpace{}%
\AgdaOperator{\AgdaDatatype{∶}}\AgdaSpace{}%
\AgdaBound{Γ₁}\AgdaSpace{}%
\AgdaOperator{\AgdaDatatype{⇒ₛ}}\AgdaSpace{}%
\AgdaSymbol{(}\AgdaBound{Γ₂}\AgdaSpace{}%
\AgdaOperator{\AgdaInductiveConstructor{▸}}\AgdaSpace{}%
\AgdaSymbol{(}\AgdaBound{o}\AgdaSpace{}%
\AgdaOperator{\AgdaInductiveConstructor{∶}}\AgdaSpace{}%
\AgdaBound{τ}\AgdaSymbol{))}\<%
\end{code}
\newcommand{\FoSubTypingSingle}[0]{\begin{code}%
\>[0]\AgdaFunction{⊢single-typeₛ}\AgdaSpace{}%
\AgdaSymbol{:}\AgdaSpace{}%
\AgdaFunction{single-typeₛ}\AgdaSpace{}%
\AgdaFunction{idₛ}\AgdaSpace{}%
\AgdaGeneralizable{τ}\AgdaSpace{}%
\AgdaOperator{\AgdaDatatype{∶}}\AgdaSpace{}%
\AgdaSymbol{(}\AgdaGeneralizable{Γ}\AgdaSpace{}%
\AgdaOperator{\AgdaInductiveConstructor{▶}}\AgdaSpace{}%
\AgdaInductiveConstructor{⋆}\AgdaSymbol{)}%
\>[46]\AgdaOperator{\AgdaDatatype{⇒ₛ}}\AgdaSpace{}%
\AgdaGeneralizable{Γ}\<%
\\
\>[0]\AgdaFunction{⊢single-typeₛ}\AgdaSpace{}%
\AgdaSymbol{=}\AgdaSpace{}%
\AgdaInductiveConstructor{⊢typeₛ}\AgdaSpace{}%
\AgdaInductiveConstructor{⊢idₛ}\<%
\end{code}}

\begin{code}[hide]%
\>[0]\AgdaKeyword{open}\AgdaSpace{}%
\AgdaKeyword{import}\AgdaSpace{}%
\AgdaModule{SystemF}\<%
\\
\>[0]\AgdaKeyword{open}\AgdaSpace{}%
\AgdaKeyword{import}\AgdaSpace{}%
\AgdaModule{SystemFo}\<%
\\
\>[0]\AgdaKeyword{open}\AgdaSpace{}%
\AgdaKeyword{import}\AgdaSpace{}%
\AgdaModule{Data.List}\AgdaSpace{}%
\AgdaKeyword{using}\AgdaSpace{}%
\AgdaSymbol{(}\AgdaDatatype{List}\AgdaSymbol{;}\AgdaSpace{}%
\AgdaInductiveConstructor{[]}\AgdaSymbol{)}\<%
\\
\>[0]\AgdaKeyword{open}\AgdaSpace{}%
\AgdaKeyword{import}\AgdaSpace{}%
\AgdaModule{Data.Product}\AgdaSpace{}%
\AgdaKeyword{using}\AgdaSpace{}%
\AgdaSymbol{(}\AgdaOperator{\AgdaFunction{\AgdaUnderscore{}×\AgdaUnderscore{}}}\AgdaSymbol{;}\AgdaSpace{}%
\AgdaOperator{\AgdaInductiveConstructor{\AgdaUnderscore{},\AgdaUnderscore{}}}\AgdaSymbol{;}\AgdaSpace{}%
\AgdaFunction{Σ-syntax}\AgdaSymbol{;}\AgdaSpace{}%
\AgdaFunction{∃-syntax}\AgdaSymbol{)}\<%
\\
\>[0]\AgdaKeyword{open}\AgdaSpace{}%
\AgdaKeyword{import}\AgdaSpace{}%
\AgdaModule{Relation.Binary.PropositionalEquality}\AgdaSpace{}%
\AgdaKeyword{using}\AgdaSpace{}%
\AgdaSymbol{(}\AgdaOperator{\AgdaDatatype{\AgdaUnderscore{}≡\AgdaUnderscore{}}}\AgdaSymbol{;}\AgdaSpace{}%
\AgdaInductiveConstructor{refl}\AgdaSymbol{;}\AgdaSpace{}%
\AgdaFunction{cong₂}\AgdaSymbol{;}\AgdaSpace{}%
\AgdaFunction{cong}\AgdaSymbol{;}\AgdaSpace{}%
\AgdaFunction{trans}\AgdaSymbol{;}\AgdaSpace{}%
\AgdaFunction{subst}\AgdaSymbol{;}\AgdaSpace{}%
\AgdaFunction{sym}\AgdaSymbol{;}\AgdaSpace{}%
\AgdaFunction{subst₂}\AgdaSymbol{;}\AgdaSpace{}%
\AgdaKeyword{module}\AgdaSpace{}%
\AgdaModule{≡-Reasoning}\AgdaSymbol{)}\<%
\\
\>[0]\AgdaKeyword{open}\AgdaSpace{}%
\AgdaKeyword{import}\AgdaSpace{}%
\AgdaModule{Data.List.Relation.Unary.Any}\AgdaSpace{}%
\AgdaKeyword{using}\AgdaSpace{}%
\AgdaSymbol{(}\AgdaInductiveConstructor{here}\AgdaSymbol{;}\AgdaSpace{}%
\AgdaInductiveConstructor{there}\AgdaSymbol{)}\<%
\\
\>[0]\AgdaKeyword{open}\AgdaSpace{}%
\AgdaKeyword{import}\AgdaSpace{}%
\AgdaModule{Data.Unit}\AgdaSpace{}%
\AgdaKeyword{using}\AgdaSpace{}%
\AgdaSymbol{(}\AgdaRecord{⊤}\AgdaSymbol{;}\AgdaSpace{}%
\AgdaInductiveConstructor{tt}\AgdaSymbol{)}\<%
\\
\>[0]\AgdaKeyword{open}\AgdaSpace{}%
\AgdaKeyword{import}\AgdaSpace{}%
\AgdaModule{Function}\AgdaSpace{}%
\AgdaKeyword{using}\AgdaSpace{}%
\AgdaSymbol{(}\AgdaFunction{id}\AgdaSymbol{)}\<%
\\
\>[0]\AgdaKeyword{open}\AgdaSpace{}%
\AgdaModule{≡-Reasoning}\<%
\\
%
\\[\AgdaEmptyExtraSkip]%
\>[0]\AgdaKeyword{module}\AgdaSpace{}%
\AgdaModule{DictionaryPassingTransform}\AgdaSpace{}%
\AgdaKeyword{where}\<%
\\
%
\\[\AgdaEmptyExtraSkip]%
\>[0]\AgdaKeyword{module}\AgdaSpace{}%
\AgdaModule{Fᴼ}\AgdaSpace{}%
\AgdaSymbol{=}\AgdaSpace{}%
\AgdaModule{SystemFo}\<%
\\
\>[0]\AgdaKeyword{module}\AgdaSpace{}%
\AgdaModule{F}\AgdaSpace{}%
\AgdaSymbol{=}\AgdaSpace{}%
\AgdaModule{SystemF}\<%
\\
%
\\[\AgdaEmptyExtraSkip]%
\>[0]\AgdaComment{--\ Translation\ --------------------------------------------------------------------------}\<%
\\
%
\\[\AgdaEmptyExtraSkip]%
\>[0]\AgdaComment{--\ Sorts\ }\<%
\end{code}
\newcommand{\DPTSort}[0]{\begin{code}%
\>[0]\AgdaFunction{s⇝s}\AgdaSpace{}%
\AgdaSymbol{:}\AgdaSpace{}%
\AgdaDatatype{Fᴼ.Sort}\AgdaSpace{}%
\AgdaInductiveConstructor{var}\AgdaSpace{}%
\AgdaSymbol{→}\AgdaSpace{}%
\AgdaDatatype{F.Sort}\AgdaSpace{}%
\AgdaInductiveConstructor{var}\<%
\\
\>[0]\AgdaFunction{s⇝s}\AgdaSpace{}%
\AgdaInductiveConstructor{eₛ}\AgdaSpace{}%
\AgdaSymbol{=}\AgdaSpace{}%
\AgdaInductiveConstructor{eₛ}\<%
\\
\>[0]\AgdaFunction{s⇝s}\AgdaSpace{}%
\AgdaInductiveConstructor{oₛ}\AgdaSpace{}%
\AgdaSymbol{=}\AgdaSpace{}%
\AgdaInductiveConstructor{eₛ}\<%
\\
\>[0]\AgdaFunction{s⇝s}\AgdaSpace{}%
\AgdaInductiveConstructor{τₛ}\AgdaSpace{}%
\AgdaSymbol{=}\AgdaSpace{}%
\AgdaInductiveConstructor{τₛ}\<%
\end{code}}
\newcommand{\DPTSorts}[0]{\begin{code}%
\>[0]\AgdaFunction{Γ⇝S}\AgdaSpace{}%
\AgdaSymbol{:}\AgdaSpace{}%
\AgdaDatatype{Fᴼ.Ctx}\AgdaSpace{}%
\AgdaGeneralizable{Fᴼ.S}\AgdaSpace{}%
\AgdaSymbol{→}\AgdaSpace{}%
\AgdaFunction{F.Sorts}\<%
\\
\>[0]\AgdaFunction{Γ⇝S}%
\>[5]\AgdaInductiveConstructor{∅}\AgdaSpace{}%
\AgdaSymbol{=}\AgdaSpace{}%
\AgdaInductiveConstructor{[]}\<%
\\
\>[0]\AgdaFunction{Γ⇝S}\AgdaSpace{}%
\AgdaSymbol{(}\AgdaBound{Γ}\AgdaSpace{}%
\AgdaOperator{\AgdaInductiveConstructor{▸}}\AgdaSpace{}%
\AgdaBound{c}\AgdaSymbol{)}\AgdaSpace{}%
\AgdaSymbol{=}\AgdaSpace{}%
\AgdaFunction{Γ⇝S}\AgdaSpace{}%
\AgdaBound{Γ}\AgdaSpace{}%
\AgdaOperator{\AgdaInductiveConstructor{▷}}\AgdaSpace{}%
\AgdaInductiveConstructor{F.eₛ}\<%
\\
\>[0]\AgdaFunction{Γ⇝S}\AgdaSpace{}%
\AgdaSymbol{\{}\AgdaBound{S}\AgdaSpace{}%
\AgdaOperator{\AgdaInductiveConstructor{▷}}\AgdaSpace{}%
\AgdaBound{s}\AgdaSymbol{\}}\AgdaSpace{}%
\AgdaSymbol{(}\AgdaBound{Γ}\AgdaSpace{}%
\AgdaOperator{\AgdaInductiveConstructor{▶}}\AgdaSpace{}%
\AgdaBound{x}\AgdaSymbol{)}\AgdaSpace{}%
\AgdaSymbol{=}\AgdaSpace{}%
\AgdaFunction{Γ⇝S}\AgdaSpace{}%
\AgdaBound{Γ}\AgdaSpace{}%
\AgdaOperator{\AgdaInductiveConstructor{▷}}\AgdaSpace{}%
\AgdaFunction{s⇝s}\AgdaSpace{}%
\AgdaBound{s}\<%
\end{code}}
\begin{code}[hide]%
\>[0]\AgdaComment{--\ Variables}\<%
\end{code}
\newcommand{\DPTVar}[0]{\begin{code}%
\>[0]\AgdaFunction{x⇝x}\AgdaSpace{}%
\AgdaSymbol{:}%
\>[7]\AgdaSymbol{∀}\AgdaSpace{}%
\AgdaSymbol{\{}\AgdaBound{Γ}\AgdaSpace{}%
\AgdaSymbol{:}\AgdaSpace{}%
\AgdaDatatype{Fᴼ.Ctx}\AgdaSpace{}%
\AgdaGeneralizable{Fᴼ.S}\AgdaSymbol{\}}\AgdaSpace{}%
\AgdaSymbol{→}\<%
\\
\>[0][@{}l@{\AgdaIndent{0}}]%
\>[2]\AgdaFunction{Fᴼ.Var}\AgdaSpace{}%
\AgdaGeneralizable{Fᴼ.S}\AgdaSpace{}%
\AgdaGeneralizable{Fᴼ.s}\AgdaSpace{}%
\AgdaSymbol{→}\AgdaSpace{}%
\AgdaFunction{F.Var}\AgdaSpace{}%
\AgdaSymbol{(}\AgdaFunction{Γ⇝S}\AgdaSpace{}%
\AgdaBound{Γ}\AgdaSymbol{)}\AgdaSpace{}%
\AgdaSymbol{(}\AgdaFunction{s⇝s}\AgdaSpace{}%
\AgdaGeneralizable{Fᴼ.s}\AgdaSymbol{)}\<%
\\
\>[0]\AgdaFunction{x⇝x}\AgdaSpace{}%
\AgdaSymbol{\{}\AgdaArgument{Γ}\AgdaSpace{}%
\AgdaSymbol{=}\AgdaSpace{}%
\AgdaBound{Γ}\AgdaSpace{}%
\AgdaOperator{\AgdaInductiveConstructor{▶}}\AgdaSpace{}%
\AgdaBound{τ}\AgdaSymbol{\}}\AgdaSpace{}%
\AgdaSymbol{(}\AgdaInductiveConstructor{here}\AgdaSpace{}%
\AgdaInductiveConstructor{refl}\AgdaSymbol{)}\AgdaSpace{}%
\AgdaSymbol{=}\AgdaSpace{}%
\AgdaInductiveConstructor{here}\AgdaSpace{}%
\AgdaInductiveConstructor{refl}\<%
\\
\>[0]\AgdaFunction{x⇝x}\AgdaSpace{}%
\AgdaSymbol{\{}\AgdaArgument{Γ}\AgdaSpace{}%
\AgdaSymbol{=}\AgdaSpace{}%
\AgdaBound{Γ}\AgdaSpace{}%
\AgdaOperator{\AgdaInductiveConstructor{▶}}\AgdaSpace{}%
\AgdaBound{τ}\AgdaSymbol{\}}\AgdaSpace{}%
\AgdaSymbol{(}\AgdaInductiveConstructor{there}\AgdaSpace{}%
\AgdaBound{x}\AgdaSymbol{)}\AgdaSpace{}%
\AgdaSymbol{=}\AgdaSpace{}%
\AgdaInductiveConstructor{there}\AgdaSpace{}%
\AgdaSymbol{(}\AgdaFunction{x⇝x}\AgdaSpace{}%
\AgdaBound{x}\AgdaSymbol{)}\<%
\\
\>[0]\AgdaFunction{x⇝x}\AgdaSpace{}%
\AgdaSymbol{\{}\AgdaArgument{Γ}\AgdaSpace{}%
\AgdaSymbol{=}\AgdaSpace{}%
\AgdaBound{Γ}\AgdaSpace{}%
\AgdaOperator{\AgdaInductiveConstructor{▸}}\AgdaSpace{}%
\AgdaBound{c}\AgdaSymbol{\}}\AgdaSpace{}%
\AgdaBound{x}\AgdaSpace{}%
\AgdaSymbol{=}\AgdaSpace{}%
\AgdaInductiveConstructor{there}\AgdaSpace{}%
\AgdaSymbol{(}\AgdaFunction{x⇝x}\AgdaSpace{}%
\AgdaBound{x}\AgdaSymbol{)}\<%
\end{code}}
\begin{code}[hide]%
\>[0]\AgdaComment{--\ Overloaded\ Variables}\<%
\end{code}
\newcommand{\DPTOVar}[0]{\begin{code}%
\>[0]\AgdaFunction{o⇝x}\AgdaSpace{}%
\AgdaSymbol{:}\AgdaSpace{}%
\AgdaSymbol{∀}\AgdaSpace{}%
\AgdaSymbol{\{}\AgdaBound{Γ}\AgdaSpace{}%
\AgdaSymbol{:}\AgdaSpace{}%
\AgdaDatatype{Fᴼ.Ctx}\AgdaSpace{}%
\AgdaGeneralizable{Fᴼ.S}\AgdaSymbol{\}}\AgdaSpace{}%
\AgdaSymbol{→}\<%
\\
\>[0][@{}l@{\AgdaIndent{0}}]%
\>[2]\AgdaOperator{\AgdaDatatype{[}}\AgdaSpace{}%
\AgdaOperator{\AgdaInductiveConstructor{`}}\AgdaSpace{}%
\AgdaGeneralizable{Fᴼ.o}\AgdaSpace{}%
\AgdaOperator{\AgdaInductiveConstructor{∶}}\AgdaSpace{}%
\AgdaGeneralizable{Fᴼ.τ}\AgdaSpace{}%
\AgdaOperator{\AgdaDatatype{]∈}}\AgdaSpace{}%
\AgdaBound{Γ}\AgdaSpace{}%
\AgdaSymbol{→}\AgdaSpace{}%
\AgdaFunction{F.Var}\AgdaSpace{}%
\AgdaSymbol{(}\AgdaFunction{Γ⇝S}\AgdaSpace{}%
\AgdaBound{Γ}\AgdaSymbol{)}\AgdaSpace{}%
\AgdaInductiveConstructor{F.eₛ}\<%
\\
\>[0]\AgdaFunction{o⇝x}\AgdaSpace{}%
\AgdaInductiveConstructor{here}\AgdaSpace{}%
\AgdaSymbol{=}\AgdaSpace{}%
\AgdaInductiveConstructor{here}\AgdaSpace{}%
\AgdaInductiveConstructor{refl}\<%
\\
\>[0]\AgdaFunction{o⇝x}\AgdaSpace{}%
\AgdaSymbol{(}\AgdaInductiveConstructor{under-bind}\AgdaSpace{}%
\AgdaBound{o∶τ∈Γ}\AgdaSymbol{)}\AgdaSpace{}%
\AgdaSymbol{=}\AgdaSpace{}%
\AgdaInductiveConstructor{there}\AgdaSpace{}%
\AgdaSymbol{(}\AgdaFunction{o⇝x}\AgdaSpace{}%
\AgdaBound{o∶τ∈Γ}\AgdaSymbol{)}\<%
\\
\>[0]\AgdaFunction{o⇝x}\AgdaSpace{}%
\AgdaSymbol{(}\AgdaInductiveConstructor{under-cstr}\AgdaSpace{}%
\AgdaBound{o∶τ∈Γ}\AgdaSymbol{)}\AgdaSpace{}%
\AgdaSymbol{=}\AgdaSpace{}%
\AgdaInductiveConstructor{there}\AgdaSpace{}%
\AgdaSymbol{(}\AgdaFunction{o⇝x}\AgdaSpace{}%
\AgdaBound{o∶τ∈Γ}\AgdaSymbol{)}\<%
\end{code}}
\begin{code}[hide]%
\>[0]\AgdaComment{--\ Types\ \ }\<%
\end{code}
\newcommand{\DPTType}[0]{\begin{code}%
\>[0]\AgdaFunction{τ⇝τ}\AgdaSpace{}%
\AgdaSymbol{:}\AgdaSpace{}%
\AgdaSymbol{∀}\AgdaSpace{}%
\AgdaSymbol{\{}\AgdaBound{Γ}\AgdaSpace{}%
\AgdaSymbol{:}\AgdaSpace{}%
\AgdaDatatype{Fᴼ.Ctx}\AgdaSpace{}%
\AgdaGeneralizable{Fᴼ.S}\AgdaSymbol{\}}\AgdaSpace{}%
\AgdaSymbol{→}\<%
\\
\>[0][@{}l@{\AgdaIndent{0}}]%
\>[2]\AgdaFunction{Fᴼ.Type}\AgdaSpace{}%
\AgdaGeneralizable{Fᴼ.S}\AgdaSpace{}%
\AgdaSymbol{→}\<%
\\
%
\>[2]\AgdaFunction{F.Type}\AgdaSpace{}%
\AgdaSymbol{(}\AgdaFunction{Γ⇝S}\AgdaSpace{}%
\AgdaBound{Γ}\AgdaSymbol{)}\<%
\\
\>[0]\AgdaFunction{τ⇝τ}\AgdaSpace{}%
\AgdaSymbol{(}\AgdaOperator{\AgdaInductiveConstructor{[}}\AgdaSpace{}%
\AgdaBound{o}\AgdaSpace{}%
\AgdaOperator{\AgdaInductiveConstructor{∶}}\AgdaSpace{}%
\AgdaBound{τ}\AgdaSpace{}%
\AgdaOperator{\AgdaInductiveConstructor{]⇒}}\AgdaSpace{}%
\AgdaBound{τ'}\AgdaSymbol{)}\AgdaSpace{}%
\AgdaSymbol{=}\AgdaSpace{}%
\AgdaFunction{τ⇝τ}\AgdaSpace{}%
\AgdaBound{τ}\AgdaSpace{}%
\AgdaOperator{\AgdaInductiveConstructor{⇒}}\AgdaSpace{}%
\AgdaFunction{τ⇝τ}\AgdaSpace{}%
\AgdaBound{τ'}\<%
\\
\>[0]\AgdaComment{--\ ...}\<%
\end{code}}
\begin{code}[hide]%
\>[0]\AgdaFunction{τ⇝τ}\AgdaSpace{}%
\AgdaSymbol{(}\AgdaOperator{\AgdaInductiveConstructor{`}}\AgdaSpace{}%
\AgdaBound{x}\AgdaSymbol{)}\AgdaSpace{}%
\AgdaSymbol{=}\AgdaSpace{}%
\AgdaOperator{\AgdaInductiveConstructor{`}}\AgdaSpace{}%
\AgdaFunction{x⇝x}\AgdaSpace{}%
\AgdaBound{x}\<%
\\
\>[0]\AgdaFunction{τ⇝τ}\AgdaSpace{}%
\AgdaInductiveConstructor{`⊤}\AgdaSpace{}%
\AgdaSymbol{=}\AgdaSpace{}%
\AgdaInductiveConstructor{`⊤}\<%
\\
\>[0]\AgdaFunction{τ⇝τ}\AgdaSpace{}%
\AgdaSymbol{(}\AgdaBound{τ₁}\AgdaSpace{}%
\AgdaOperator{\AgdaInductiveConstructor{⇒}}\AgdaSpace{}%
\AgdaBound{τ₂}\AgdaSymbol{)}\AgdaSpace{}%
\AgdaSymbol{=}\AgdaSpace{}%
\AgdaFunction{τ⇝τ}\AgdaSpace{}%
\AgdaBound{τ₁}\AgdaSpace{}%
\AgdaOperator{\AgdaInductiveConstructor{⇒}}\AgdaSpace{}%
\AgdaFunction{τ⇝τ}\AgdaSpace{}%
\AgdaBound{τ₂}\<%
\\
\>[0]\AgdaFunction{τ⇝τ}\AgdaSpace{}%
\AgdaSymbol{\{}\AgdaArgument{Γ}\AgdaSpace{}%
\AgdaSymbol{=}\AgdaSpace{}%
\AgdaBound{Γ}\AgdaSymbol{\}}\AgdaSpace{}%
\AgdaSymbol{(}\AgdaOperator{\AgdaInductiveConstructor{Fᴼ.∀`α}}\AgdaSpace{}%
\AgdaBound{τ}\AgdaSymbol{)}\AgdaSpace{}%
\AgdaSymbol{=}\AgdaSpace{}%
\AgdaOperator{\AgdaInductiveConstructor{F.∀`α}}\AgdaSpace{}%
\AgdaFunction{τ⇝τ}\AgdaSpace{}%
\AgdaSymbol{\{}\AgdaArgument{Γ}\AgdaSpace{}%
\AgdaSymbol{=}\AgdaSpace{}%
\AgdaBound{Γ}\AgdaSpace{}%
\AgdaOperator{\AgdaInductiveConstructor{▶}}\AgdaSpace{}%
\AgdaInductiveConstructor{⋆}\AgdaSymbol{\}}\AgdaSpace{}%
\AgdaBound{τ}\<%
\end{code}
\newcommand{\DPTKind}[0]{\begin{code}%
\>[0]\AgdaFunction{T⇝T}\AgdaSpace{}%
\AgdaSymbol{:}\AgdaSpace{}%
\AgdaSymbol{∀}\AgdaSpace{}%
\AgdaSymbol{\{}\AgdaBound{Γ}\AgdaSpace{}%
\AgdaSymbol{:}\AgdaSpace{}%
\AgdaDatatype{Fᴼ.Ctx}\AgdaSpace{}%
\AgdaGeneralizable{Fᴼ.S}\AgdaSymbol{\}}\AgdaSpace{}%
\AgdaSymbol{→}\<%
\\
\>[0][@{}l@{\AgdaIndent{0}}]%
\>[2]\AgdaDatatype{Fᴼ.Term}\AgdaSpace{}%
\AgdaGeneralizable{Fᴼ.S}\AgdaSpace{}%
\AgdaSymbol{(}\AgdaFunction{Fᴼ.type-of}\AgdaSpace{}%
\AgdaGeneralizable{Fᴼ.s}\AgdaSymbol{)}\AgdaSpace{}%
\AgdaSymbol{→}\<%
\\
%
\>[2]\AgdaDatatype{F.Term}\AgdaSpace{}%
\AgdaSymbol{(}\AgdaFunction{Γ⇝S}\AgdaSpace{}%
\AgdaBound{Γ}\AgdaSymbol{)}\AgdaSpace{}%
\AgdaSymbol{(}\AgdaFunction{F.type-of}\AgdaSpace{}%
\AgdaSymbol{(}\AgdaFunction{s⇝s}\AgdaSpace{}%
\AgdaGeneralizable{Fᴼ.s}\AgdaSymbol{))}\<%
\\
\>[0]\AgdaFunction{T⇝T}\AgdaSpace{}%
\AgdaSymbol{\{}\AgdaArgument{s}\AgdaSpace{}%
\AgdaSymbol{=}\AgdaSpace{}%
\AgdaInductiveConstructor{eₛ}\AgdaSymbol{\}}\AgdaSpace{}%
\AgdaBound{τ}\AgdaSpace{}%
\AgdaSymbol{=}\AgdaSpace{}%
\AgdaFunction{τ⇝τ}\AgdaSpace{}%
\AgdaBound{τ}\<%
\\
\>[0]\AgdaFunction{T⇝T}\AgdaSpace{}%
\AgdaSymbol{\{}\AgdaArgument{s}\AgdaSpace{}%
\AgdaSymbol{=}\AgdaSpace{}%
\AgdaInductiveConstructor{oₛ}\AgdaSymbol{\}}\AgdaSpace{}%
\AgdaBound{τ}\AgdaSpace{}%
\AgdaSymbol{=}\AgdaSpace{}%
\AgdaFunction{τ⇝τ}\AgdaSpace{}%
\AgdaBound{τ}\<%
\\
\>[0]\AgdaFunction{T⇝T}\AgdaSpace{}%
\AgdaSymbol{\{}\AgdaArgument{s}\AgdaSpace{}%
\AgdaSymbol{=}\AgdaSpace{}%
\AgdaInductiveConstructor{τₛ}\AgdaSymbol{\}}\AgdaSpace{}%
\AgdaSymbol{\AgdaUnderscore{}}\AgdaSpace{}%
\AgdaSymbol{=}\AgdaSpace{}%
\AgdaInductiveConstructor{⋆}\<%
\end{code}}
\begin{code}[hide]%
\>[0]\AgdaComment{--\ Context\ }\<%
\\
%
\\[\AgdaEmptyExtraSkip]%
\>[0]\AgdaFunction{I⇝T}\AgdaSpace{}%
\AgdaSymbol{:}\AgdaSpace{}%
\AgdaSymbol{∀}\AgdaSpace{}%
\AgdaSymbol{\{}\AgdaBound{Γ}\AgdaSpace{}%
\AgdaSymbol{:}\AgdaSpace{}%
\AgdaDatatype{Fᴼ.Ctx}\AgdaSpace{}%
\AgdaGeneralizable{Fᴼ.S}\AgdaSymbol{\}}\AgdaSpace{}%
\AgdaSymbol{→}\AgdaSpace{}%
\AgdaDatatype{Fᴼ.Term}\AgdaSpace{}%
\AgdaGeneralizable{Fᴼ.S}\AgdaSpace{}%
\AgdaSymbol{(}\AgdaFunction{item-of}\AgdaSpace{}%
\AgdaGeneralizable{Fᴼ.s}\AgdaSymbol{)}\AgdaSpace{}%
\AgdaSymbol{→}\AgdaSpace{}%
\AgdaDatatype{F.Term}\AgdaSpace{}%
\AgdaSymbol{(}\AgdaFunction{Γ⇝S}\AgdaSpace{}%
\AgdaBound{Γ}\AgdaSymbol{)}\AgdaSpace{}%
\AgdaSymbol{(}\AgdaFunction{F.type-of}\AgdaSpace{}%
\AgdaSymbol{(}\AgdaFunction{s⇝s}\AgdaSpace{}%
\AgdaGeneralizable{Fᴼ.s}\AgdaSymbol{))}\<%
\\
\>[0]\AgdaFunction{I⇝T}\AgdaSpace{}%
\AgdaSymbol{\{}\AgdaArgument{s}\AgdaSpace{}%
\AgdaSymbol{=}\AgdaSpace{}%
\AgdaInductiveConstructor{eₛ}\AgdaSymbol{\}}\AgdaSpace{}%
\AgdaBound{τ}\AgdaSpace{}%
\AgdaSymbol{=}\AgdaSpace{}%
\AgdaFunction{τ⇝τ}\AgdaSpace{}%
\AgdaBound{τ}\<%
\\
\>[0]\AgdaFunction{I⇝T}\AgdaSpace{}%
\AgdaSymbol{\{}\AgdaArgument{s}\AgdaSpace{}%
\AgdaSymbol{=}\AgdaSpace{}%
\AgdaInductiveConstructor{oₛ}\AgdaSymbol{\}}\AgdaSpace{}%
\AgdaInductiveConstructor{⋆}\AgdaSpace{}%
\AgdaSymbol{=}\AgdaSpace{}%
\AgdaInductiveConstructor{`⊤}\<%
\\
\>[0]\AgdaFunction{I⇝T}\AgdaSpace{}%
\AgdaSymbol{\{}\AgdaArgument{s}\AgdaSpace{}%
\AgdaSymbol{=}\AgdaSpace{}%
\AgdaInductiveConstructor{τₛ}\AgdaSymbol{\}}\AgdaSpace{}%
\AgdaInductiveConstructor{⋆}\AgdaSpace{}%
\AgdaSymbol{=}\AgdaSpace{}%
\AgdaInductiveConstructor{⋆}\<%
\end{code}
\newcommand{\DPTCtx}[0]{\begin{code}%
\>[0]\AgdaFunction{Γ⇝Γ}\AgdaSpace{}%
\AgdaSymbol{:}\AgdaSpace{}%
\AgdaSymbol{(}\AgdaBound{Γ}\AgdaSpace{}%
\AgdaSymbol{:}\AgdaSpace{}%
\AgdaDatatype{Fᴼ.Ctx}\AgdaSpace{}%
\AgdaGeneralizable{Fᴼ.S}\AgdaSymbol{)}\AgdaSpace{}%
\AgdaSymbol{→}\AgdaSpace{}%
\AgdaDatatype{F.Ctx}\AgdaSpace{}%
\AgdaSymbol{(}\AgdaFunction{Γ⇝S}\AgdaSpace{}%
\AgdaBound{Γ}\AgdaSymbol{)}\<%
\\
\>[0]\AgdaFunction{Γ⇝Γ}\AgdaSpace{}%
\AgdaSymbol{(}\AgdaBound{Γ}\AgdaSpace{}%
\AgdaOperator{\AgdaInductiveConstructor{▸}}\AgdaSpace{}%
\AgdaSymbol{(}\AgdaOperator{\AgdaInductiveConstructor{`}}\AgdaSpace{}%
\AgdaBound{o}\AgdaSpace{}%
\AgdaOperator{\AgdaInductiveConstructor{∶}}\AgdaSpace{}%
\AgdaBound{τ}\AgdaSymbol{))}\AgdaSpace{}%
\AgdaSymbol{=}\AgdaSpace{}%
\AgdaSymbol{(}\AgdaFunction{Γ⇝Γ}\AgdaSpace{}%
\AgdaBound{Γ}\AgdaSymbol{)}\AgdaSpace{}%
\AgdaOperator{\AgdaInductiveConstructor{▶}}\AgdaSpace{}%
\AgdaFunction{τ⇝τ}\AgdaSpace{}%
\AgdaBound{τ}\<%
\\
\>[0]\AgdaComment{--\ ...}\<%
\end{code}}
\begin{code}[hide]%
\>[0]\AgdaFunction{Γ⇝Γ}\AgdaSpace{}%
\AgdaInductiveConstructor{∅}\AgdaSpace{}%
\AgdaSymbol{=}\AgdaSpace{}%
\AgdaInductiveConstructor{∅}\<%
\\
\>[0]\AgdaFunction{Γ⇝Γ}\AgdaSpace{}%
\AgdaSymbol{(}\AgdaBound{Γ}\AgdaSpace{}%
\AgdaOperator{\AgdaInductiveConstructor{▶}}\AgdaSpace{}%
\AgdaBound{I}\AgdaSymbol{)}\AgdaSpace{}%
\AgdaSymbol{=}\AgdaSpace{}%
\AgdaSymbol{(}\AgdaFunction{Γ⇝Γ}\AgdaSpace{}%
\AgdaBound{Γ}\AgdaSymbol{)}\AgdaSpace{}%
\AgdaOperator{\AgdaInductiveConstructor{▶}}\AgdaSpace{}%
\AgdaFunction{I⇝T}\AgdaSpace{}%
\AgdaBound{I}\<%
\\
%
\\[\AgdaEmptyExtraSkip]%
\>[0]\AgdaComment{--\ Terms}\<%
\end{code}
\newcommand{\DPTTerms}[0]{\begin{code}%
\>[0]\AgdaFunction{⊢t⇝t}\AgdaSpace{}%
\AgdaSymbol{:}\AgdaSpace{}%
\AgdaSymbol{∀}%
\>[336I]\AgdaSymbol{\{}\AgdaBound{Γ}\AgdaSpace{}%
\AgdaSymbol{:}\AgdaSpace{}%
\AgdaDatatype{Fᴼ.Ctx}\AgdaSpace{}%
\AgdaGeneralizable{Fᴼ.S}\AgdaSymbol{\}}\AgdaSpace{}%
\AgdaSymbol{\{}\AgdaBound{t}\AgdaSpace{}%
\AgdaSymbol{:}\AgdaSpace{}%
\AgdaDatatype{Fᴼ.Term}\AgdaSpace{}%
\AgdaGeneralizable{Fᴼ.S}\AgdaSpace{}%
\AgdaGeneralizable{Fᴼ.s}\AgdaSymbol{\}}\<%
\\
\>[.][@{}l@{}]\<[336I]%
\>[9]\AgdaSymbol{\{}\AgdaBound{T}\AgdaSpace{}%
\AgdaSymbol{:}\AgdaSpace{}%
\AgdaDatatype{Fᴼ.Term}\AgdaSpace{}%
\AgdaGeneralizable{Fᴼ.S}\AgdaSpace{}%
\AgdaSymbol{(}\AgdaFunction{Fᴼ.type-of}\AgdaSpace{}%
\AgdaGeneralizable{Fᴼ.s}\AgdaSymbol{)\}}\AgdaSpace{}%
\AgdaSymbol{→}\<%
\\
\>[0][@{}l@{\AgdaIndent{0}}]%
\>[2]\AgdaBound{Γ}\AgdaSpace{}%
\AgdaOperator{\AgdaDatatype{Fᴼ.⊢}}\AgdaSpace{}%
\AgdaBound{t}\AgdaSpace{}%
\AgdaOperator{\AgdaDatatype{∶}}\AgdaSpace{}%
\AgdaBound{T}\AgdaSpace{}%
\AgdaSymbol{→}\<%
\\
%
\>[2]\AgdaDatatype{F.Term}\AgdaSpace{}%
\AgdaSymbol{(}\AgdaFunction{Γ⇝S}\AgdaSpace{}%
\AgdaBound{Γ}\AgdaSymbol{)}\AgdaSpace{}%
\AgdaSymbol{(}\AgdaFunction{s⇝s}\AgdaSpace{}%
\AgdaGeneralizable{Fᴼ.s}\AgdaSymbol{)}\<%
\\
\>[0]\AgdaFunction{⊢t⇝t}\AgdaSpace{}%
\AgdaSymbol{(}\AgdaInductiveConstructor{⊢`o}\AgdaSpace{}%
\AgdaBound{o∶τ∈Γ}\AgdaSymbol{)}\AgdaSpace{}%
\AgdaSymbol{=}\AgdaSpace{}%
\AgdaOperator{\AgdaInductiveConstructor{`}}\AgdaSpace{}%
\AgdaFunction{o⇝x}\AgdaSpace{}%
\AgdaBound{o∶τ∈Γ}\<%
\\
\>[0]\AgdaFunction{⊢t⇝t}\AgdaSpace{}%
\AgdaSymbol{(}\AgdaInductiveConstructor{⊢ƛ}\AgdaSpace{}%
\AgdaBound{⊢e}\AgdaSymbol{)}\AgdaSpace{}%
\AgdaSymbol{=}\AgdaSpace{}%
\AgdaOperator{\AgdaInductiveConstructor{λ`x→}}\AgdaSpace{}%
\AgdaSymbol{(}\AgdaFunction{⊢t⇝t}\AgdaSpace{}%
\AgdaBound{⊢e}\AgdaSymbol{)}\<%
\\
\>[0]\AgdaFunction{⊢t⇝t}\AgdaSpace{}%
\AgdaSymbol{(}\AgdaInductiveConstructor{⊢⊘}\AgdaSpace{}%
\AgdaBound{⊢e}\AgdaSpace{}%
\AgdaBound{o∶τ∈Γ}\AgdaSymbol{)}\AgdaSpace{}%
\AgdaSymbol{=}\AgdaSpace{}%
\AgdaFunction{⊢t⇝t}\AgdaSpace{}%
\AgdaBound{⊢e}\AgdaSpace{}%
\AgdaOperator{\AgdaInductiveConstructor{·}}\AgdaSpace{}%
\AgdaOperator{\AgdaInductiveConstructor{`}}\AgdaSpace{}%
\AgdaFunction{o⇝x}\AgdaSpace{}%
\AgdaBound{o∶τ∈Γ}\<%
\\
\>[0]\AgdaFunction{⊢t⇝t}\AgdaSpace{}%
\AgdaSymbol{(}\AgdaInductiveConstructor{⊢decl}\AgdaSpace{}%
\AgdaBound{⊢e}\AgdaSymbol{)}\AgdaSpace{}%
\AgdaSymbol{=}\AgdaSpace{}%
\AgdaOperator{\AgdaInductiveConstructor{let`x=}}\AgdaSpace{}%
\AgdaInductiveConstructor{tt}\AgdaSpace{}%
\AgdaOperator{\AgdaInductiveConstructor{`in}}%
\>[33]\AgdaFunction{⊢t⇝t}\AgdaSpace{}%
\AgdaBound{⊢e}\<%
\\
\>[0]\AgdaFunction{⊢t⇝t}\AgdaSpace{}%
\AgdaSymbol{(}\AgdaInductiveConstructor{⊢inst}\AgdaSpace{}%
\AgdaBound{⊢e₂}\AgdaSpace{}%
\AgdaBound{⊢e₁}\AgdaSymbol{)}\AgdaSpace{}%
\AgdaSymbol{=}\AgdaSpace{}%
\AgdaOperator{\AgdaInductiveConstructor{let`x=}}\AgdaSpace{}%
\AgdaFunction{⊢t⇝t}\AgdaSpace{}%
\AgdaBound{⊢e₂}\AgdaSpace{}%
\AgdaOperator{\AgdaInductiveConstructor{`in}}\AgdaSpace{}%
\AgdaFunction{⊢t⇝t}\AgdaSpace{}%
\AgdaBound{⊢e₁}\<%
\\
\>[0]\AgdaComment{--\ ...}\<%
\end{code}}
\begin{code}[hide]%
\>[0]\AgdaFunction{⊢t⇝t}\AgdaSpace{}%
\AgdaSymbol{(}\AgdaInductiveConstructor{⊢`x}\AgdaSpace{}%
\AgdaSymbol{\{}\AgdaArgument{x}\AgdaSpace{}%
\AgdaSymbol{=}\AgdaSpace{}%
\AgdaBound{x}\AgdaSymbol{\}}\AgdaSpace{}%
\AgdaBound{Γx≡τ}\AgdaSymbol{)}\AgdaSpace{}%
\AgdaSymbol{=}\AgdaSpace{}%
\AgdaOperator{\AgdaInductiveConstructor{`}}\AgdaSpace{}%
\AgdaFunction{x⇝x}\AgdaSpace{}%
\AgdaBound{x}\<%
\\
\>[0]\AgdaFunction{⊢t⇝t}\AgdaSpace{}%
\AgdaInductiveConstructor{⊢⊤}\AgdaSpace{}%
\AgdaSymbol{=}\AgdaSpace{}%
\AgdaInductiveConstructor{tt}\<%
\\
\>[0]\AgdaFunction{⊢t⇝t}\AgdaSpace{}%
\AgdaSymbol{(}\AgdaInductiveConstructor{⊢λ}\AgdaSpace{}%
\AgdaBound{⊢e}\AgdaSymbol{)}\AgdaSpace{}%
\AgdaSymbol{=}\AgdaSpace{}%
\AgdaOperator{\AgdaInductiveConstructor{λ`x→}}\AgdaSpace{}%
\AgdaSymbol{(}\AgdaFunction{⊢t⇝t}\AgdaSpace{}%
\AgdaBound{⊢e}\AgdaSymbol{)}\<%
\\
\>[0]\AgdaFunction{⊢t⇝t}\AgdaSpace{}%
\AgdaSymbol{(}\AgdaInductiveConstructor{⊢Λ}\AgdaSpace{}%
\AgdaBound{⊢e}\AgdaSymbol{)}\AgdaSpace{}%
\AgdaSymbol{=}\AgdaSpace{}%
\AgdaOperator{\AgdaInductiveConstructor{Λ`α→}}\AgdaSpace{}%
\AgdaSymbol{(}\AgdaFunction{⊢t⇝t}\AgdaSpace{}%
\AgdaBound{⊢e}\AgdaSymbol{)}\<%
\\
\>[0]\AgdaFunction{⊢t⇝t}\AgdaSpace{}%
\AgdaSymbol{(}\AgdaInductiveConstructor{⊢·}\AgdaSpace{}%
\AgdaBound{⊢e₁}\AgdaSpace{}%
\AgdaBound{⊢e₂}\AgdaSymbol{)}\AgdaSpace{}%
\AgdaSymbol{=}\AgdaSpace{}%
\AgdaFunction{⊢t⇝t}\AgdaSpace{}%
\AgdaBound{⊢e₁}\AgdaSpace{}%
\AgdaOperator{\AgdaInductiveConstructor{·}}\AgdaSpace{}%
\AgdaFunction{⊢t⇝t}\AgdaSpace{}%
\AgdaBound{⊢e₂}\<%
\\
\>[0]\AgdaFunction{⊢t⇝t}\AgdaSpace{}%
\AgdaSymbol{(}\AgdaInductiveConstructor{⊢•}\AgdaSpace{}%
\AgdaSymbol{\{}\AgdaArgument{τ'}\AgdaSpace{}%
\AgdaSymbol{=}\AgdaSpace{}%
\AgdaBound{τ'}\AgdaSymbol{\}}\AgdaSpace{}%
\AgdaBound{⊢e}\AgdaSymbol{)}\AgdaSpace{}%
\AgdaSymbol{=}\AgdaSpace{}%
\AgdaFunction{⊢t⇝t}\AgdaSpace{}%
\AgdaBound{⊢e}\AgdaSpace{}%
\AgdaOperator{\AgdaInductiveConstructor{•}}\AgdaSpace{}%
\AgdaSymbol{(}\AgdaFunction{τ⇝τ}\AgdaSpace{}%
\AgdaBound{τ'}\AgdaSymbol{)}\<%
\\
\>[0]\AgdaFunction{⊢t⇝t}\AgdaSpace{}%
\AgdaSymbol{(}\AgdaInductiveConstructor{⊢let}\AgdaSpace{}%
\AgdaBound{⊢e₂}\AgdaSpace{}%
\AgdaBound{⊢e₁}\AgdaSymbol{)}\AgdaSpace{}%
\AgdaSymbol{=}\AgdaSpace{}%
\AgdaOperator{\AgdaInductiveConstructor{let`x=}}\AgdaSpace{}%
\AgdaFunction{⊢t⇝t}\AgdaSpace{}%
\AgdaBound{⊢e₂}\AgdaSpace{}%
\AgdaOperator{\AgdaInductiveConstructor{`in}}\AgdaSpace{}%
\AgdaFunction{⊢t⇝t}\AgdaSpace{}%
\AgdaBound{⊢e₁}\<%
\\
%
\\[\AgdaEmptyExtraSkip]%
\>[0]\AgdaComment{--\ Renaming\ }\<%
\end{code}
\newcommand{\DPTRen}[0]{\begin{code}%
\>[0]\AgdaFunction{⊢ρ⇝ρ}\AgdaSpace{}%
\AgdaSymbol{:}\AgdaSpace{}%
\AgdaSymbol{∀}\AgdaSpace{}%
\AgdaSymbol{\{}\AgdaBound{ρ}\AgdaSpace{}%
\AgdaSymbol{:}\AgdaSpace{}%
\AgdaFunction{Fᴼ.Ren}\AgdaSpace{}%
\AgdaGeneralizable{Fᴼ.S₁}\AgdaSpace{}%
\AgdaGeneralizable{Fᴼ.S₂}\AgdaSymbol{\}}\AgdaSpace{}%
\AgdaSymbol{\{}\AgdaBound{Γ₁}\AgdaSpace{}%
\AgdaSymbol{:}\AgdaSpace{}%
\AgdaDatatype{Fᴼ.Ctx}\AgdaSpace{}%
\AgdaGeneralizable{Fᴼ.S₁}\AgdaSymbol{\}}\AgdaSpace{}%
\AgdaSymbol{\{}\AgdaBound{Γ₂}\AgdaSpace{}%
\AgdaSymbol{:}\AgdaSpace{}%
\AgdaDatatype{Fᴼ.Ctx}\AgdaSpace{}%
\AgdaGeneralizable{Fᴼ.S₂}\AgdaSymbol{\}}\AgdaSpace{}%
\AgdaSymbol{→}\<%
\\
\>[0][@{}l@{\AgdaIndent{0}}]%
\>[2]\AgdaBound{ρ}\AgdaSpace{}%
\AgdaOperator{\AgdaDatatype{Fᴼ.∶}}\AgdaSpace{}%
\AgdaBound{Γ₁}\AgdaSpace{}%
\AgdaOperator{\AgdaDatatype{⇒ᵣ}}\AgdaSpace{}%
\AgdaBound{Γ₂}\AgdaSpace{}%
\AgdaSymbol{→}\<%
\\
%
\>[2]\AgdaFunction{F.Ren}\AgdaSpace{}%
\AgdaSymbol{(}\AgdaFunction{Γ⇝S}\AgdaSpace{}%
\AgdaBound{Γ₁}\AgdaSymbol{)}\AgdaSpace{}%
\AgdaSymbol{(}\AgdaFunction{Γ⇝S}\AgdaSpace{}%
\AgdaBound{Γ₂}\AgdaSymbol{)}\<%
\\
\>[0]\AgdaFunction{⊢ρ⇝ρ}\AgdaSpace{}%
\AgdaSymbol{(}\AgdaInductiveConstructor{⊢drop-cstrᵣ}\AgdaSpace{}%
\AgdaBound{⊢ρ}\AgdaSymbol{)}\AgdaSpace{}%
\AgdaSymbol{=}\AgdaSpace{}%
\AgdaFunction{F.dropᵣ}\AgdaSpace{}%
\AgdaSymbol{(}\AgdaFunction{⊢ρ⇝ρ}\AgdaSpace{}%
\AgdaBound{⊢ρ}\AgdaSymbol{)}\<%
\\
\>[0]\AgdaComment{--\ ...}\<%
\end{code}}
\begin{code}[hide]%
\>[0]\AgdaFunction{⊢ρ⇝ρ}\AgdaSpace{}%
\AgdaInductiveConstructor{⊢idᵣ}\AgdaSpace{}%
\AgdaSymbol{\AgdaUnderscore{}}\AgdaSpace{}%
\AgdaSymbol{=}\AgdaSpace{}%
\AgdaFunction{id}\<%
\\
\>[0]\AgdaFunction{⊢ρ⇝ρ}\AgdaSpace{}%
\AgdaSymbol{(}\AgdaInductiveConstructor{⊢extᵣ}\AgdaSpace{}%
\AgdaBound{⊢ρ}\AgdaSymbol{)}\AgdaSpace{}%
\AgdaSymbol{=}\AgdaSpace{}%
\AgdaFunction{F.extᵣ}\AgdaSpace{}%
\AgdaSymbol{(}\AgdaFunction{⊢ρ⇝ρ}\AgdaSpace{}%
\AgdaBound{⊢ρ}\AgdaSymbol{)}\AgdaSpace{}%
\AgdaSymbol{\AgdaUnderscore{}}\<%
\\
\>[0]\AgdaFunction{⊢ρ⇝ρ}\AgdaSpace{}%
\AgdaSymbol{(}\AgdaInductiveConstructor{⊢dropᵣ}\AgdaSpace{}%
\AgdaBound{⊢ρ}\AgdaSymbol{)}\AgdaSpace{}%
\AgdaSymbol{=}\AgdaSpace{}%
\AgdaFunction{F.dropᵣ}\AgdaSpace{}%
\AgdaSymbol{(}\AgdaFunction{⊢ρ⇝ρ}\AgdaSpace{}%
\AgdaBound{⊢ρ}\AgdaSymbol{)}\<%
\end{code}
\begin{code}[hide]%
\>[0]\AgdaComment{--\ Substititution}\<%
\end{code}
\newcommand{\DPTSub}[0]{\begin{code}%
\>[0]\AgdaFunction{⊢σ⇝σ}\AgdaSpace{}%
\AgdaSymbol{:}\AgdaSpace{}%
\AgdaSymbol{∀}\AgdaSpace{}%
\AgdaSymbol{\{}\AgdaBound{σ}\AgdaSpace{}%
\AgdaSymbol{:}\AgdaSpace{}%
\AgdaFunction{Fᴼ.Sub}\AgdaSpace{}%
\AgdaGeneralizable{Fᴼ.S₁}\AgdaSpace{}%
\AgdaGeneralizable{Fᴼ.S₂}\AgdaSymbol{\}}\AgdaSpace{}%
\AgdaSymbol{\{}\AgdaBound{Γ₁}\AgdaSpace{}%
\AgdaSymbol{:}\AgdaSpace{}%
\AgdaDatatype{Fᴼ.Ctx}\AgdaSpace{}%
\AgdaGeneralizable{Fᴼ.S₁}\AgdaSymbol{\}}\AgdaSpace{}%
\AgdaSymbol{\{}\AgdaBound{Γ₂}\AgdaSpace{}%
\AgdaSymbol{:}\AgdaSpace{}%
\AgdaDatatype{Fᴼ.Ctx}\AgdaSpace{}%
\AgdaGeneralizable{Fᴼ.S₂}\AgdaSymbol{\}}\AgdaSpace{}%
\AgdaSymbol{→}\<%
\\
\>[0][@{}l@{\AgdaIndent{0}}]%
\>[2]\AgdaBound{σ}\AgdaSpace{}%
\AgdaOperator{\AgdaDatatype{Fᴼ.∶}}\AgdaSpace{}%
\AgdaBound{Γ₁}\AgdaSpace{}%
\AgdaOperator{\AgdaDatatype{⇒ₛ}}\AgdaSpace{}%
\AgdaBound{Γ₂}\AgdaSpace{}%
\AgdaSymbol{→}\<%
\\
%
\>[2]\AgdaFunction{F.Sub}\AgdaSpace{}%
\AgdaSymbol{(}\AgdaFunction{Γ⇝S}\AgdaSpace{}%
\AgdaBound{Γ₁}\AgdaSymbol{)}\AgdaSpace{}%
\AgdaSymbol{(}\AgdaFunction{Γ⇝S}\AgdaSpace{}%
\AgdaBound{Γ₂}\AgdaSymbol{)}\<%
\\
\>[0]\AgdaFunction{⊢σ⇝σ}\AgdaSpace{}%
\AgdaSymbol{(}\AgdaInductiveConstructor{⊢single-typeₛ}\AgdaSpace{}%
\AgdaSymbol{\{}\AgdaArgument{τ}\AgdaSpace{}%
\AgdaSymbol{=}\AgdaSpace{}%
\AgdaBound{τ}\AgdaSymbol{\}}\AgdaSpace{}%
\AgdaBound{⊢σ}\AgdaSymbol{)}\AgdaSpace{}%
\AgdaSymbol{=}\AgdaSpace{}%
\AgdaFunction{F.singleₛ}\AgdaSpace{}%
\AgdaSymbol{(}\AgdaFunction{⊢σ⇝σ}\AgdaSpace{}%
\AgdaBound{⊢σ}\AgdaSymbol{)}\AgdaSpace{}%
\AgdaSymbol{(}\AgdaFunction{τ⇝τ}\AgdaSpace{}%
\AgdaBound{τ}\AgdaSymbol{)}\<%
\\
\>[0]\AgdaComment{--\ ...}\<%
\end{code}}
\begin{code}[hide]%
\>[0]\AgdaFunction{⊢σ⇝σ}\AgdaSpace{}%
\AgdaInductiveConstructor{⊢idₛ}\AgdaSpace{}%
\AgdaSymbol{=}\AgdaSpace{}%
\AgdaFunction{F.idₛ}\<%
\\
\>[0]\AgdaFunction{⊢σ⇝σ}\AgdaSpace{}%
\AgdaSymbol{(}\AgdaInductiveConstructor{⊢extₛ}\AgdaSpace{}%
\AgdaBound{⊢σ}\AgdaSymbol{)}\AgdaSpace{}%
\AgdaSymbol{=}\AgdaSpace{}%
\AgdaFunction{F.extₛ}\AgdaSpace{}%
\AgdaSymbol{(}\AgdaFunction{⊢σ⇝σ}\AgdaSpace{}%
\AgdaBound{⊢σ}\AgdaSymbol{)}\AgdaSpace{}%
\AgdaSymbol{\AgdaUnderscore{}}\<%
\\
\>[0]\AgdaFunction{⊢σ⇝σ}\AgdaSpace{}%
\AgdaSymbol{(}\AgdaInductiveConstructor{⊢dropₛ}\AgdaSpace{}%
\AgdaBound{⊢σ}\AgdaSymbol{)}\AgdaSpace{}%
\AgdaSymbol{=}\AgdaSpace{}%
\AgdaFunction{F.dropₛ}\AgdaSpace{}%
\AgdaSymbol{(}\AgdaFunction{⊢σ⇝σ}\AgdaSpace{}%
\AgdaBound{⊢σ}\AgdaSymbol{)}\<%
\\
\>[0]\AgdaComment{--\ ⊢σ⇝σ\ (⊢ext-cstrₛ\ ⊢σ)\ =\ F.extₛ\ (⊢σ⇝σ\ ⊢σ)}\<%
\\
\>[0]\AgdaFunction{⊢σ⇝σ}\AgdaSpace{}%
\AgdaSymbol{(}\AgdaInductiveConstructor{⊢drop-cstrₛ}\AgdaSpace{}%
\AgdaBound{⊢σ}\AgdaSymbol{)}\AgdaSpace{}%
\AgdaSymbol{=}\AgdaSpace{}%
\AgdaFunction{F.dropₛ}\AgdaSpace{}%
\AgdaSymbol{(}\AgdaFunction{⊢σ⇝σ}\AgdaSpace{}%
\AgdaBound{⊢σ}\AgdaSymbol{)}\<%
\end{code}
\begin{code}[hide]%
\>[0]\AgdaComment{--\ Type\ Preservation\ --------------------------------------------------------------------}\<%
\\
%
\\[\AgdaEmptyExtraSkip]%
\>[0]\AgdaComment{--\ Renaming}\<%
\\
\>[0]\AgdaFunction{⇝-dist-ren-var}\AgdaSpace{}%
\AgdaSymbol{:}\AgdaSpace{}%
\AgdaSymbol{\{}\AgdaBound{ρ}\AgdaSpace{}%
\AgdaSymbol{:}\AgdaSpace{}%
\AgdaFunction{Fᴼ.Ren}\AgdaSpace{}%
\AgdaGeneralizable{Fᴼ.S₁}\AgdaSpace{}%
\AgdaGeneralizable{Fᴼ.S₂}\AgdaSymbol{\}}\AgdaSpace{}%
\AgdaSymbol{\{}\AgdaBound{Γ₁}\AgdaSpace{}%
\AgdaSymbol{:}\AgdaSpace{}%
\AgdaDatatype{Fᴼ.Ctx}\AgdaSpace{}%
\AgdaGeneralizable{Fᴼ.S₁}\AgdaSymbol{\}}\AgdaSpace{}%
\AgdaSymbol{\{}\AgdaBound{Γ₂}\AgdaSpace{}%
\AgdaSymbol{:}\AgdaSpace{}%
\AgdaDatatype{Fᴼ.Ctx}\AgdaSpace{}%
\AgdaGeneralizable{Fᴼ.S₂}\AgdaSymbol{\}}\AgdaSpace{}%
\AgdaSymbol{→}\<%
\\
\>[0][@{}l@{\AgdaIndent{0}}]%
\>[2]\AgdaSymbol{(}\AgdaBound{⊢ρ}\AgdaSpace{}%
\AgdaSymbol{:}\AgdaSpace{}%
\AgdaBound{ρ}\AgdaSpace{}%
\AgdaOperator{\AgdaDatatype{Fᴼ.∶}}\AgdaSpace{}%
\AgdaBound{Γ₁}\AgdaSpace{}%
\AgdaOperator{\AgdaDatatype{⇒ᵣ}}\AgdaSpace{}%
\AgdaBound{Γ₂}\AgdaSymbol{)}\AgdaSpace{}%
\AgdaSymbol{→}\<%
\\
%
\>[2]\AgdaSymbol{(}\AgdaBound{x}\AgdaSpace{}%
\AgdaSymbol{:}\AgdaSpace{}%
\AgdaFunction{Fᴼ.Var}\AgdaSpace{}%
\AgdaGeneralizable{Fᴼ.S₁}\AgdaSpace{}%
\AgdaGeneralizable{Fᴼ.s}\AgdaSymbol{)}\AgdaSpace{}%
\AgdaSymbol{→}\<%
\end{code}
\newcommand{\DPTVarPresRen}[0]{\begin{code}[inline]%
%
\>[2]\AgdaSymbol{(}\AgdaFunction{⊢ρ⇝ρ}\AgdaSpace{}%
\AgdaBound{⊢ρ}\AgdaSymbol{)}\AgdaSpace{}%
\AgdaSymbol{\AgdaUnderscore{}}\AgdaSpace{}%
\AgdaSymbol{(}\AgdaFunction{x⇝x}\AgdaSpace{}%
\AgdaBound{x}\AgdaSymbol{)}\AgdaSpace{}%
\AgdaOperator{\AgdaDatatype{≡}}\AgdaSpace{}%
\AgdaFunction{x⇝x}\AgdaSpace{}%
\AgdaSymbol{(}\AgdaBound{ρ}\AgdaSpace{}%
\AgdaBound{x}\AgdaSymbol{)}\<%
\end{code}}
\begin{code}[hide]%
\>[0]\AgdaFunction{⇝-dist-ren-var}\AgdaSpace{}%
\AgdaInductiveConstructor{⊢idᵣ}\AgdaSpace{}%
\AgdaBound{x}\AgdaSpace{}%
\AgdaSymbol{=}\AgdaSpace{}%
\AgdaInductiveConstructor{refl}\<%
\\
\>[0]\AgdaFunction{⇝-dist-ren-var}\AgdaSpace{}%
\AgdaSymbol{(}\AgdaInductiveConstructor{⊢extᵣ}\AgdaSpace{}%
\AgdaBound{⊢ρ}\AgdaSymbol{)}\AgdaSpace{}%
\AgdaSymbol{(}\AgdaInductiveConstructor{here}\AgdaSpace{}%
\AgdaInductiveConstructor{refl}\AgdaSymbol{)}\AgdaSpace{}%
\AgdaSymbol{=}\AgdaSpace{}%
\AgdaInductiveConstructor{refl}\<%
\\
\>[0]\AgdaFunction{⇝-dist-ren-var}\AgdaSpace{}%
\AgdaSymbol{(}\AgdaInductiveConstructor{⊢extᵣ}\AgdaSpace{}%
\AgdaBound{⊢ρ}\AgdaSymbol{)}\AgdaSpace{}%
\AgdaSymbol{(}\AgdaInductiveConstructor{there}\AgdaSpace{}%
\AgdaBound{x}\AgdaSymbol{)}\AgdaSpace{}%
\AgdaSymbol{=}\AgdaSpace{}%
\AgdaFunction{cong}\AgdaSpace{}%
\AgdaInductiveConstructor{there}\AgdaSpace{}%
\AgdaSymbol{(}\AgdaFunction{⇝-dist-ren-var}\AgdaSpace{}%
\AgdaBound{⊢ρ}\AgdaSpace{}%
\AgdaBound{x}\AgdaSymbol{)}\<%
\\
\>[0]\AgdaFunction{⇝-dist-ren-var}\AgdaSpace{}%
\AgdaSymbol{(}\AgdaInductiveConstructor{⊢dropᵣ}\AgdaSpace{}%
\AgdaBound{⊢ρ}\AgdaSymbol{)}\AgdaSpace{}%
\AgdaBound{x}\AgdaSpace{}%
\AgdaSymbol{=}\AgdaSpace{}%
\AgdaFunction{cong}\AgdaSpace{}%
\AgdaInductiveConstructor{there}\AgdaSpace{}%
\AgdaSymbol{(}\AgdaFunction{⇝-dist-ren-var}\AgdaSpace{}%
\AgdaBound{⊢ρ}\AgdaSpace{}%
\AgdaBound{x}\AgdaSymbol{)}\<%
\\
\>[0]\AgdaComment{--\ ⇝-dist-ren-var\ (⊢ext-cstrᵣ\ ⊢ρ)\ x\ =\ cong\ there\ (⇝-dist-ren-var\ ⊢ρ\ x)}\<%
\\
\>[0]\AgdaFunction{⇝-dist-ren-var}\AgdaSpace{}%
\AgdaSymbol{(}\AgdaInductiveConstructor{⊢drop-cstrᵣ}\AgdaSpace{}%
\AgdaBound{⊢ρ}\AgdaSymbol{)}\AgdaSpace{}%
\AgdaBound{x}\AgdaSpace{}%
\AgdaSymbol{=}\AgdaSpace{}%
\AgdaFunction{cong}\AgdaSpace{}%
\AgdaInductiveConstructor{there}\AgdaSpace{}%
\AgdaSymbol{(}\AgdaFunction{⇝-dist-ren-var}\AgdaSpace{}%
\AgdaBound{⊢ρ}\AgdaSpace{}%
\AgdaBound{x}\AgdaSymbol{)}\<%
\end{code}
\newcommand{\DPTTypePresRen}[0]{\begin{code}%
\>[0]\AgdaFunction{⇝-dist-ren-type}%
\>[632I]\AgdaSymbol{:}%
\>[19]\AgdaSymbol{\{}\AgdaBound{ρ}\AgdaSpace{}%
\AgdaSymbol{:}\AgdaSpace{}%
\AgdaFunction{Fᴼ.Ren}\AgdaSpace{}%
\AgdaGeneralizable{Fᴼ.S₁}\AgdaSpace{}%
\AgdaGeneralizable{Fᴼ.S₂}\AgdaSymbol{\}}\<%
\\
\>[632I][@{}l@{\AgdaIndent{0}}]%
\>[18]\AgdaSymbol{\{}\AgdaBound{Γ₁}\AgdaSpace{}%
\AgdaSymbol{:}\AgdaSpace{}%
\AgdaDatatype{Fᴼ.Ctx}\AgdaSpace{}%
\AgdaGeneralizable{Fᴼ.S₁}\AgdaSymbol{\}}\AgdaSpace{}%
\AgdaSymbol{\{}\AgdaBound{Γ₂}\AgdaSpace{}%
\AgdaSymbol{:}\AgdaSpace{}%
\AgdaDatatype{Fᴼ.Ctx}\AgdaSpace{}%
\AgdaGeneralizable{Fᴼ.S₂}\AgdaSymbol{\}}\AgdaSpace{}%
\AgdaSymbol{→}\<%
\\
\>[0][@{}l@{\AgdaIndent{0}}]%
\>[2]\AgdaSymbol{(}\AgdaBound{⊢ρ}\AgdaSpace{}%
\AgdaSymbol{:}\AgdaSpace{}%
\AgdaBound{ρ}\AgdaSpace{}%
\AgdaOperator{\AgdaDatatype{Fᴼ.∶}}\AgdaSpace{}%
\AgdaBound{Γ₁}\AgdaSpace{}%
\AgdaOperator{\AgdaDatatype{⇒ᵣ}}\AgdaSpace{}%
\AgdaBound{Γ₂}\AgdaSymbol{)}\AgdaSpace{}%
\AgdaSymbol{→}\<%
\\
%
\>[2]\AgdaSymbol{(}\AgdaBound{τ}\AgdaSpace{}%
\AgdaSymbol{:}\AgdaSpace{}%
\AgdaFunction{Fᴼ.Type}\AgdaSpace{}%
\AgdaGeneralizable{Fᴼ.S₁}\AgdaSymbol{)}\AgdaSpace{}%
\AgdaSymbol{→}\<%
\\
%
\>[2]\AgdaFunction{F.ren}\AgdaSpace{}%
\AgdaSymbol{(}\AgdaFunction{⊢ρ⇝ρ}\AgdaSpace{}%
\AgdaBound{⊢ρ}\AgdaSymbol{)}\AgdaSpace{}%
\AgdaSymbol{(}\AgdaFunction{τ⇝τ}\AgdaSpace{}%
\AgdaBound{τ}\AgdaSymbol{)}\AgdaSpace{}%
\AgdaOperator{\AgdaDatatype{≡}}\AgdaSpace{}%
\AgdaFunction{τ⇝τ}\AgdaSpace{}%
\AgdaSymbol{(}\AgdaFunction{Fᴼ.ren}\AgdaSpace{}%
\AgdaBound{ρ}\AgdaSpace{}%
\AgdaBound{τ}\AgdaSymbol{)}\<%
\\
\>[0]\AgdaFunction{⇝-dist-ren-type}\AgdaSpace{}%
\AgdaBound{⊢ρ}\AgdaSpace{}%
\AgdaSymbol{(}\AgdaOperator{\AgdaInductiveConstructor{`}}\AgdaSpace{}%
\AgdaBound{x}\AgdaSymbol{)}\AgdaSpace{}%
\AgdaSymbol{=}\AgdaSpace{}%
\AgdaFunction{cong}\AgdaSpace{}%
\AgdaOperator{\AgdaInductiveConstructor{`\AgdaUnderscore{}}}\AgdaSpace{}%
\AgdaSymbol{(}\AgdaFunction{⇝-dist-ren-var}%
\>[52]\AgdaBound{⊢ρ}\AgdaSpace{}%
\AgdaBound{x}\AgdaSymbol{)}\<%
\\
\>[0]\AgdaFunction{⇝-dist-ren-type}\AgdaSpace{}%
\AgdaBound{⊢ρ}\AgdaSpace{}%
\AgdaSymbol{(}\AgdaOperator{\AgdaInductiveConstructor{[}}\AgdaSpace{}%
\AgdaOperator{\AgdaInductiveConstructor{`}}\AgdaSpace{}%
\AgdaBound{o}\AgdaSpace{}%
\AgdaOperator{\AgdaInductiveConstructor{∶}}\AgdaSpace{}%
\AgdaBound{τ}\AgdaSpace{}%
\AgdaOperator{\AgdaInductiveConstructor{]⇒}}\AgdaSpace{}%
\AgdaBound{τ'}\AgdaSymbol{)}\AgdaSpace{}%
\AgdaSymbol{=}\AgdaSpace{}%
\AgdaFunction{cong₂}\AgdaSpace{}%
\AgdaOperator{\AgdaInductiveConstructor{\AgdaUnderscore{}⇒\AgdaUnderscore{}}}\<%
\\
\>[0][@{}l@{\AgdaIndent{0}}]%
\>[2]\AgdaSymbol{(}\AgdaFunction{⇝-dist-ren-type}\AgdaSpace{}%
\AgdaBound{⊢ρ}\AgdaSpace{}%
\AgdaBound{τ}\AgdaSymbol{)}\AgdaSpace{}%
\AgdaSymbol{(}\AgdaFunction{⇝-dist-ren-type}\AgdaSpace{}%
\AgdaBound{⊢ρ}\AgdaSpace{}%
\AgdaBound{τ'}\AgdaSymbol{)}\<%
\\
\>[0]\AgdaComment{--\ ...}\<%
\end{code}}
\begin{code}[hide]%
\>[0]\AgdaFunction{⇝-dist-ren-type}\AgdaSpace{}%
\AgdaBound{⊢ρ}\AgdaSpace{}%
\AgdaInductiveConstructor{`⊤}\AgdaSpace{}%
\AgdaSymbol{=}\AgdaSpace{}%
\AgdaInductiveConstructor{refl}\<%
\\
\>[0]\AgdaFunction{⇝-dist-ren-type}\AgdaSpace{}%
\AgdaBound{⊢ρ}\AgdaSpace{}%
\AgdaSymbol{(}\AgdaBound{τ₁}\AgdaSpace{}%
\AgdaOperator{\AgdaInductiveConstructor{⇒}}\AgdaSpace{}%
\AgdaBound{τ₂}\AgdaSymbol{)}\AgdaSpace{}%
\AgdaSymbol{=}\AgdaSpace{}%
\AgdaFunction{cong₂}\AgdaSpace{}%
\AgdaOperator{\AgdaInductiveConstructor{\AgdaUnderscore{}⇒\AgdaUnderscore{}}}\AgdaSpace{}%
\AgdaSymbol{(}\AgdaFunction{⇝-dist-ren-type}\AgdaSpace{}%
\AgdaBound{⊢ρ}\AgdaSpace{}%
\AgdaBound{τ₁}\AgdaSymbol{)}\AgdaSpace{}%
\AgdaSymbol{(}\AgdaFunction{⇝-dist-ren-type}\AgdaSpace{}%
\AgdaBound{⊢ρ}\AgdaSpace{}%
\AgdaBound{τ₂}\AgdaSymbol{)}\<%
\\
\>[0]\AgdaFunction{⇝-dist-ren-type}\AgdaSpace{}%
\AgdaBound{⊢ρ}\AgdaSpace{}%
\AgdaSymbol{(}\AgdaOperator{\AgdaInductiveConstructor{∀`α}}\AgdaSpace{}%
\AgdaBound{τ}\AgdaSymbol{)}\AgdaSpace{}%
\AgdaSymbol{=}\AgdaSpace{}%
\AgdaFunction{cong}\AgdaSpace{}%
\AgdaOperator{\AgdaInductiveConstructor{F.∀`α\AgdaUnderscore{}}}\AgdaSpace{}%
\AgdaSymbol{(}\AgdaFunction{⇝-dist-ren-type}\AgdaSpace{}%
\AgdaSymbol{(}\AgdaInductiveConstructor{⊢extᵣ}\AgdaSpace{}%
\AgdaBound{⊢ρ}\AgdaSymbol{)}\AgdaSpace{}%
\AgdaBound{τ}\AgdaSymbol{)}\<%
\\
%
\\[\AgdaEmptyExtraSkip]%
%
\\[\AgdaEmptyExtraSkip]%
\>[0]\AgdaFunction{⇝-dist-wk-type}\AgdaSpace{}%
\AgdaSymbol{:}\AgdaSpace{}%
\AgdaSymbol{\{}\AgdaBound{Γ}\AgdaSpace{}%
\AgdaSymbol{:}\AgdaSpace{}%
\AgdaDatatype{Fᴼ.Ctx}\AgdaSpace{}%
\AgdaGeneralizable{Fᴼ.S}\AgdaSymbol{\}}\AgdaSpace{}%
\AgdaSymbol{\{}\AgdaBound{τ'}\AgdaSpace{}%
\AgdaSymbol{:}\AgdaSpace{}%
\AgdaFunction{Fᴼ.Type}\AgdaSpace{}%
\AgdaGeneralizable{Fᴼ.S}\AgdaSymbol{\}}\AgdaSpace{}%
\AgdaSymbol{\{}\AgdaBound{T}\AgdaSpace{}%
\AgdaSymbol{:}\AgdaSpace{}%
\AgdaDatatype{Fᴼ.Term}\AgdaSpace{}%
\AgdaGeneralizable{Fᴼ.S}\AgdaSpace{}%
\AgdaSymbol{(}\AgdaFunction{item-of}\AgdaSpace{}%
\AgdaGeneralizable{Fᴼ.s}\AgdaSymbol{)\}}\AgdaSpace{}%
\AgdaSymbol{→}\<%
\end{code}
\newcommand{\DPTTypePresWk}[0]{\begin{code}[inline]%
\>[0][@{}l@{\AgdaIndent{1}}]%
\>[2]\AgdaFunction{τ⇝τ}\AgdaSpace{}%
\AgdaSymbol{\{}\AgdaArgument{Γ}\AgdaSpace{}%
\AgdaSymbol{=}\AgdaSpace{}%
\AgdaBound{Γ}\AgdaSpace{}%
\AgdaOperator{\AgdaInductiveConstructor{▶}}\AgdaSpace{}%
\AgdaBound{T}\AgdaSymbol{\}}\AgdaSpace{}%
\AgdaSymbol{(}\AgdaFunction{Fᴼ.wk}\AgdaSpace{}%
\AgdaBound{τ'}\AgdaSymbol{)}\AgdaSpace{}%
\AgdaOperator{\AgdaDatatype{≡}}\AgdaSpace{}%
\AgdaFunction{F.wk}\AgdaSpace{}%
\AgdaSymbol{(}\AgdaFunction{τ⇝τ}\AgdaSpace{}%
\AgdaBound{τ'}\AgdaSymbol{)}\<%
\end{code}}
\begin{code}[hide]%
\>[0]\AgdaFunction{⇝-dist-wk-type}\AgdaSymbol{\{}\AgdaArgument{τ'}\AgdaSpace{}%
\AgdaSymbol{=}\AgdaSpace{}%
\AgdaBound{τ'}\AgdaSymbol{\}}\AgdaSpace{}%
\AgdaSymbol{=}\AgdaSpace{}%
\AgdaFunction{sym}\AgdaSpace{}%
\AgdaSymbol{(}\AgdaFunction{⇝-dist-ren-type}\AgdaSpace{}%
\AgdaFunction{Fᴼ.⊢wkᵣ}\AgdaSpace{}%
\AgdaBound{τ'}\AgdaSymbol{)}\<%
\\
%
\\[\AgdaEmptyExtraSkip]%
\>[0]\AgdaFunction{⇝-dist-wk-inst-type}\AgdaSpace{}%
\AgdaSymbol{:}\AgdaSpace{}%
\AgdaSymbol{∀}\AgdaSpace{}%
\AgdaSymbol{\{}\AgdaBound{Γ}\AgdaSpace{}%
\AgdaSymbol{:}\AgdaSpace{}%
\AgdaDatatype{Fᴼ.Ctx}\AgdaSpace{}%
\AgdaGeneralizable{Fᴼ.S}\AgdaSymbol{\}}\AgdaSpace{}%
\AgdaSymbol{\{}\AgdaBound{τ}\AgdaSpace{}%
\AgdaSymbol{:}\AgdaSpace{}%
\AgdaFunction{Fᴼ.Type}\AgdaSpace{}%
\AgdaGeneralizable{Fᴼ.S}\AgdaSymbol{\}}\AgdaSpace{}%
\AgdaSymbol{\{}\AgdaBound{τ'}\AgdaSpace{}%
\AgdaSymbol{:}\AgdaSpace{}%
\AgdaFunction{Fᴼ.Type}\AgdaSpace{}%
\AgdaGeneralizable{Fᴼ.S}\AgdaSymbol{\}}\AgdaSpace{}%
\AgdaSymbol{\{}\AgdaBound{o}\AgdaSymbol{\}}\AgdaSpace{}%
\AgdaSymbol{→}\<%
\end{code}
\newcommand{\DPTTypePresWkInst}[0]{\begin{code}[inline]%
\>[0][@{}l@{\AgdaIndent{1}}]%
\>[2]\AgdaFunction{τ⇝τ}\AgdaSpace{}%
\AgdaSymbol{\{}\AgdaArgument{Γ}\AgdaSpace{}%
\AgdaSymbol{=}\AgdaSpace{}%
\AgdaBound{Γ}\AgdaSpace{}%
\AgdaOperator{\AgdaInductiveConstructor{▸}}\AgdaSpace{}%
\AgdaSymbol{(}\AgdaOperator{\AgdaInductiveConstructor{`}}\AgdaSpace{}%
\AgdaBound{o}\AgdaSpace{}%
\AgdaOperator{\AgdaInductiveConstructor{∶}}\AgdaSpace{}%
\AgdaBound{τ'}\AgdaSymbol{)\}}\AgdaSpace{}%
\AgdaBound{τ}\AgdaSpace{}%
\AgdaOperator{\AgdaDatatype{≡}}\AgdaSpace{}%
\AgdaFunction{F.wk}\AgdaSpace{}%
\AgdaSymbol{(}\AgdaFunction{τ⇝τ}\AgdaSpace{}%
\AgdaBound{τ}\AgdaSymbol{)}\<%
\end{code}}
\begin{code}[hide]%
\>[0]\AgdaFunction{⇝-dist-wk-inst-type}%
\>[21]\AgdaSymbol{\{}\AgdaArgument{τ}\AgdaSpace{}%
\AgdaSymbol{=}\AgdaSpace{}%
\AgdaBound{τ}\AgdaSymbol{\}}\AgdaSpace{}%
\AgdaSymbol{=}\<%
\\
\>[0][@{}l@{\AgdaIndent{0}}]%
\>[2]\AgdaOperator{\AgdaFunction{begin}}\<%
\\
\>[2][@{}l@{\AgdaIndent{0}}]%
\>[4]\AgdaFunction{τ⇝τ}\AgdaSpace{}%
\AgdaBound{τ}\<%
\\
%
\>[2]\AgdaFunction{≡⟨}\AgdaSpace{}%
\AgdaFunction{cong}\AgdaSpace{}%
\AgdaFunction{τ⇝τ}\AgdaSpace{}%
\AgdaSymbol{(}\AgdaFunction{sym}\AgdaSpace{}%
\AgdaSymbol{(}\AgdaFunction{Fᴼ.idᵣτ≡τ}\AgdaSpace{}%
\AgdaBound{τ}\AgdaSymbol{))}\AgdaSpace{}%
\AgdaFunction{⟩}\<%
\\
\>[2][@{}l@{\AgdaIndent{0}}]%
\>[4]\AgdaFunction{τ⇝τ}\AgdaSpace{}%
\AgdaSymbol{(}\AgdaFunction{Fᴼ.ren}\AgdaSpace{}%
\AgdaFunction{Fᴼ.idᵣ}\AgdaSpace{}%
\AgdaBound{τ}\AgdaSymbol{)}\<%
\\
%
\>[2]\AgdaFunction{≡⟨}\AgdaSpace{}%
\AgdaFunction{sym}\AgdaSpace{}%
\AgdaSymbol{(}\AgdaFunction{⇝-dist-ren-type}\AgdaSpace{}%
\AgdaFunction{⊢wk-instᵣ}\AgdaSpace{}%
\AgdaBound{τ}\AgdaSymbol{)}\AgdaSpace{}%
\AgdaFunction{⟩}\<%
\\
\>[2][@{}l@{\AgdaIndent{0}}]%
\>[4]\AgdaFunction{F.wk}\AgdaSpace{}%
\AgdaSymbol{(}\AgdaFunction{τ⇝τ}\AgdaSpace{}%
\AgdaBound{τ}\AgdaSymbol{)}\<%
\\
%
\>[2]\AgdaOperator{\AgdaFunction{∎}}\<%
\\
%
\\[\AgdaEmptyExtraSkip]%
\>[0]\AgdaComment{--\ Substititution}\<%
\end{code}
\newcommand{\DPTVarPresSub}[0]{\begin{code}%
\>[0]\AgdaFunction{⇝-dist-sub-var}\AgdaSpace{}%
\AgdaSymbol{:}\AgdaSpace{}%
\AgdaSymbol{\{}\AgdaBound{σ}\AgdaSpace{}%
\AgdaSymbol{:}%
\>[802I]\AgdaFunction{Fᴼ.Sub}\AgdaSpace{}%
\AgdaGeneralizable{Fᴼ.S₁}\AgdaSpace{}%
\AgdaGeneralizable{Fᴼ.S₂}\AgdaSymbol{\}}\<%
\\
\>[.][@{}l@{}]\<[802I]%
\>[22]\AgdaSymbol{\{}\AgdaBound{Γ₁}\AgdaSpace{}%
\AgdaSymbol{:}\AgdaSpace{}%
\AgdaDatatype{Fᴼ.Ctx}\AgdaSpace{}%
\AgdaGeneralizable{Fᴼ.S₁}\AgdaSymbol{\}}\AgdaSpace{}%
\AgdaSymbol{\{}\AgdaBound{Γ₂}\AgdaSpace{}%
\AgdaSymbol{:}\AgdaSpace{}%
\AgdaDatatype{Fᴼ.Ctx}\AgdaSpace{}%
\AgdaGeneralizable{Fᴼ.S₂}\AgdaSymbol{\}}\AgdaSpace{}%
\AgdaSymbol{→}\<%
\\
\>[0][@{}l@{\AgdaIndent{0}}]%
\>[2]\AgdaSymbol{(}\AgdaBound{⊢σ}\AgdaSpace{}%
\AgdaSymbol{:}\AgdaSpace{}%
\AgdaBound{σ}\AgdaSpace{}%
\AgdaOperator{\AgdaDatatype{Fᴼ.∶}}\AgdaSpace{}%
\AgdaBound{Γ₁}\AgdaSpace{}%
\AgdaOperator{\AgdaDatatype{⇒ₛ}}\AgdaSpace{}%
\AgdaBound{Γ₂}\AgdaSymbol{)}\AgdaSpace{}%
\AgdaSymbol{→}\<%
\\
%
\>[2]\AgdaSymbol{(}\AgdaBound{x}\AgdaSpace{}%
\AgdaSymbol{:}\AgdaSpace{}%
\AgdaFunction{Fᴼ.Var}\AgdaSpace{}%
\AgdaGeneralizable{Fᴼ.S₁}\AgdaSpace{}%
\AgdaInductiveConstructor{τₛ}\AgdaSymbol{)}\AgdaSpace{}%
\AgdaSymbol{→}\<%
\\
%
\>[2]\AgdaFunction{F.sub}\AgdaSpace{}%
\AgdaSymbol{(}\AgdaFunction{⊢σ⇝σ}\AgdaSpace{}%
\AgdaBound{⊢σ}\AgdaSymbol{)}\AgdaSpace{}%
\AgdaSymbol{(}\AgdaOperator{\AgdaInductiveConstructor{`}}\AgdaSpace{}%
\AgdaFunction{x⇝x}\AgdaSpace{}%
\AgdaBound{x}\AgdaSymbol{)}\AgdaSpace{}%
\AgdaOperator{\AgdaDatatype{≡}}\AgdaSpace{}%
\AgdaFunction{τ⇝τ}\AgdaSpace{}%
\AgdaSymbol{(}\AgdaFunction{Fᴼ.sub}\AgdaSpace{}%
\AgdaBound{σ}\AgdaSpace{}%
\AgdaSymbol{(}\AgdaOperator{\AgdaInductiveConstructor{`}}\AgdaSpace{}%
\AgdaBound{x}\AgdaSymbol{))}\<%
\\
\>[0]\AgdaFunction{⇝-dist-sub-var}\AgdaSpace{}%
\AgdaSymbol{(}\AgdaInductiveConstructor{⊢extₛ}\AgdaSpace{}%
\AgdaBound{⊢σ}\AgdaSymbol{)}\AgdaSpace{}%
\AgdaSymbol{(}\AgdaInductiveConstructor{here}\AgdaSpace{}%
\AgdaInductiveConstructor{refl}\AgdaSymbol{)}\AgdaSpace{}%
\AgdaSymbol{=}\AgdaSpace{}%
\AgdaInductiveConstructor{refl}\<%
\\
\>[0]\AgdaFunction{⇝-dist-sub-var}\AgdaSpace{}%
\AgdaSymbol{(}\AgdaInductiveConstructor{⊢extₛ}\AgdaSpace{}%
\AgdaSymbol{\{}\AgdaArgument{σ}\AgdaSpace{}%
\AgdaSymbol{=}\AgdaSpace{}%
\AgdaBound{σ}\AgdaSymbol{\}}\AgdaSpace{}%
\AgdaBound{⊢σ}\AgdaSymbol{)}\AgdaSpace{}%
\AgdaSymbol{(}\AgdaInductiveConstructor{there}\AgdaSpace{}%
\AgdaBound{x}\AgdaSymbol{)}\AgdaSpace{}%
\AgdaSymbol{=}\AgdaSpace{}%
\AgdaFunction{trans}\<%
\\
\>[0][@{}l@{\AgdaIndent{0}}]%
\>[2]\AgdaSymbol{(}\AgdaFunction{cong}\AgdaSpace{}%
\AgdaFunction{F.wk}\AgdaSpace{}%
\AgdaSymbol{(}\AgdaFunction{⇝-dist-sub-var}\AgdaSpace{}%
\AgdaBound{⊢σ}\AgdaSpace{}%
\AgdaBound{x}\AgdaSymbol{))}\AgdaSpace{}%
\AgdaSymbol{(}\AgdaFunction{⇝-dist-ren-type}\AgdaSpace{}%
\AgdaFunction{Fᴼ.⊢wkᵣ}\AgdaSpace{}%
\AgdaSymbol{(}\AgdaBound{σ}\AgdaSpace{}%
\AgdaBound{x}\AgdaSymbol{))}\<%
\end{code}}
\begin{code}[hide]%
\>[0]\AgdaFunction{⇝-dist-sub-var}\AgdaSpace{}%
\AgdaInductiveConstructor{⊢idₛ}\AgdaSpace{}%
\AgdaBound{x}\AgdaSpace{}%
\AgdaSymbol{=}\AgdaSpace{}%
\AgdaInductiveConstructor{refl}\<%
\\
\>[0]\AgdaFunction{⇝-dist-sub-var}\AgdaSpace{}%
\AgdaSymbol{(}\AgdaInductiveConstructor{⊢dropₛ}\AgdaSpace{}%
\AgdaSymbol{\{}\AgdaArgument{σ}\AgdaSpace{}%
\AgdaSymbol{=}\AgdaSpace{}%
\AgdaBound{σ}\AgdaSymbol{\}}\AgdaSpace{}%
\AgdaBound{⊢σ}\AgdaSymbol{)}\AgdaSpace{}%
\AgdaBound{x}%
\>[38]\AgdaSymbol{=}\AgdaSpace{}%
\AgdaFunction{trans}\<%
\\
\>[0][@{}l@{\AgdaIndent{0}}]%
\>[2]\AgdaSymbol{(}\AgdaFunction{cong}\AgdaSpace{}%
\AgdaFunction{F.wk}\AgdaSpace{}%
\AgdaSymbol{(}\AgdaFunction{⇝-dist-sub-var}\AgdaSpace{}%
\AgdaBound{⊢σ}\AgdaSpace{}%
\AgdaBound{x}\AgdaSymbol{))}\AgdaSpace{}%
\AgdaSymbol{(}\AgdaFunction{⇝-dist-ren-type}\AgdaSpace{}%
\AgdaFunction{Fᴼ.⊢wkᵣ}\AgdaSpace{}%
\AgdaSymbol{(}\AgdaBound{σ}\AgdaSpace{}%
\AgdaBound{x}\AgdaSymbol{))}\<%
\\
\>[0]\AgdaFunction{⇝-dist-sub-var}\AgdaSpace{}%
\AgdaSymbol{(}\AgdaInductiveConstructor{⊢single-typeₛ}\AgdaSpace{}%
\AgdaBound{⊢σ}\AgdaSymbol{)}\AgdaSpace{}%
\AgdaSymbol{(}\AgdaInductiveConstructor{here}\AgdaSpace{}%
\AgdaInductiveConstructor{refl}\AgdaSymbol{)}\AgdaSpace{}%
\AgdaSymbol{=}\AgdaSpace{}%
\AgdaInductiveConstructor{refl}\<%
\\
\>[0]\AgdaFunction{⇝-dist-sub-var}\AgdaSpace{}%
\AgdaSymbol{(}\AgdaInductiveConstructor{⊢single-typeₛ}\AgdaSpace{}%
\AgdaBound{⊢σ}\AgdaSymbol{)}\AgdaSpace{}%
\AgdaSymbol{(}\AgdaInductiveConstructor{there}\AgdaSpace{}%
\AgdaBound{x}\AgdaSymbol{)}\AgdaSpace{}%
\AgdaSymbol{=}\AgdaSpace{}%
\AgdaFunction{⇝-dist-sub-var}\AgdaSpace{}%
\AgdaBound{⊢σ}\AgdaSpace{}%
\AgdaBound{x}\<%
\\
\>[0]\AgdaFunction{⇝-dist-sub-var}\AgdaSpace{}%
\AgdaSymbol{(}\AgdaInductiveConstructor{⊢drop-cstrₛ}\AgdaSpace{}%
\AgdaSymbol{\{}\AgdaArgument{σ}\AgdaSpace{}%
\AgdaSymbol{=}\AgdaSpace{}%
\AgdaBound{σ}\AgdaSymbol{\}}\AgdaSpace{}%
\AgdaBound{⊢σ}\AgdaSymbol{)}\AgdaSpace{}%
\AgdaBound{x}\AgdaSpace{}%
\AgdaSymbol{=}\AgdaSpace{}%
\AgdaFunction{trans}\AgdaSpace{}%
\AgdaSymbol{(}\AgdaFunction{cong}\AgdaSpace{}%
\AgdaFunction{F.wk}\AgdaSpace{}%
\AgdaSymbol{(}\AgdaFunction{⇝-dist-sub-var}\AgdaSpace{}%
\AgdaBound{⊢σ}\AgdaSpace{}%
\AgdaBound{x}\AgdaSymbol{))}\AgdaSpace{}%
\AgdaSymbol{(}\<%
\\
\>[0][@{}l@{\AgdaIndent{0}}]%
\>[3]\AgdaOperator{\AgdaFunction{begin}}\<%
\\
\>[3][@{}l@{\AgdaIndent{0}}]%
\>[4]\AgdaFunction{F.wk}\AgdaSpace{}%
\AgdaSymbol{(}\AgdaFunction{τ⇝τ}\AgdaSpace{}%
\AgdaSymbol{(}\AgdaBound{σ}\AgdaSpace{}%
\AgdaBound{x}\AgdaSymbol{))}\<%
\\
\>[0][@{}l@{\AgdaIndent{0}}]%
\>[2]\AgdaFunction{≡⟨}\AgdaSpace{}%
\AgdaFunction{⇝-dist-ren-type}\AgdaSpace{}%
\AgdaFunction{⊢wk-instᵣ}\AgdaSpace{}%
\AgdaSymbol{(}\AgdaBound{σ}\AgdaSpace{}%
\AgdaBound{x}\AgdaSymbol{)}\AgdaSpace{}%
\AgdaFunction{⟩}\<%
\\
\>[2][@{}l@{\AgdaIndent{0}}]%
\>[4]\AgdaFunction{τ⇝τ}\AgdaSpace{}%
\AgdaSymbol{(}\AgdaFunction{Fᴼ.ren}\AgdaSpace{}%
\AgdaFunction{Fᴼ.idᵣ}\AgdaSpace{}%
\AgdaSymbol{(}\AgdaBound{σ}\AgdaSpace{}%
\AgdaBound{x}\AgdaSymbol{))}\<%
\\
%
\>[2]\AgdaFunction{≡⟨}\AgdaSpace{}%
\AgdaFunction{cong}\AgdaSpace{}%
\AgdaFunction{τ⇝τ}\AgdaSpace{}%
\AgdaSymbol{(}\AgdaFunction{Fᴼ.idᵣτ≡τ}\AgdaSpace{}%
\AgdaSymbol{(}\AgdaBound{σ}\AgdaSpace{}%
\AgdaBound{x}\AgdaSymbol{))}\AgdaSpace{}%
\AgdaFunction{⟩}\<%
\\
\>[2][@{}l@{\AgdaIndent{0}}]%
\>[4]\AgdaFunction{τ⇝τ}\AgdaSpace{}%
\AgdaSymbol{(}\AgdaBound{σ}\AgdaSpace{}%
\AgdaBound{x}\AgdaSymbol{)}\<%
\\
%
\>[2]\AgdaOperator{\AgdaFunction{∎}}\AgdaSymbol{)}\<%
\\
%
\\[\AgdaEmptyExtraSkip]%
\>[0]\AgdaFunction{⇝-dist-sub-type}%
\>[17]\AgdaSymbol{:}\AgdaSpace{}%
\AgdaSymbol{∀}\AgdaSpace{}%
\AgdaSymbol{\{}\AgdaBound{σ}\AgdaSpace{}%
\AgdaSymbol{:}\AgdaSpace{}%
\AgdaFunction{Fᴼ.Sub}\AgdaSpace{}%
\AgdaGeneralizable{Fᴼ.S₁}\AgdaSpace{}%
\AgdaGeneralizable{Fᴼ.S₂}\AgdaSymbol{\}}\<%
\\
\>[0][@{}l@{\AgdaIndent{0}}]%
\>[15]\AgdaSymbol{\{}\AgdaBound{Γ₁}\AgdaSpace{}%
\AgdaSymbol{:}\AgdaSpace{}%
\AgdaDatatype{Fᴼ.Ctx}\AgdaSpace{}%
\AgdaGeneralizable{Fᴼ.S₁}\AgdaSymbol{\}}\AgdaSpace{}%
\AgdaSymbol{\{}\AgdaBound{Γ₂}\AgdaSpace{}%
\AgdaSymbol{:}\AgdaSpace{}%
\AgdaDatatype{Fᴼ.Ctx}\AgdaSpace{}%
\AgdaGeneralizable{Fᴼ.S₂}\AgdaSymbol{\}}\AgdaSpace{}%
\AgdaSymbol{→}\<%
\\
\>[0][@{}l@{\AgdaIndent{0}}]%
\>[2]\AgdaSymbol{(}\AgdaBound{⊢σ}\AgdaSpace{}%
\AgdaSymbol{:}\AgdaSpace{}%
\AgdaBound{σ}\AgdaSpace{}%
\AgdaOperator{\AgdaDatatype{Fᴼ.∶}}\AgdaSpace{}%
\AgdaBound{Γ₁}\AgdaSpace{}%
\AgdaOperator{\AgdaDatatype{⇒ₛ}}\AgdaSpace{}%
\AgdaBound{Γ₂}\AgdaSymbol{)}\AgdaSpace{}%
\AgdaSymbol{→}\<%
\\
%
\>[2]\AgdaSymbol{(}\AgdaBound{τ}\AgdaSpace{}%
\AgdaSymbol{:}\AgdaSpace{}%
\AgdaFunction{Fᴼ.Type}\AgdaSpace{}%
\AgdaGeneralizable{Fᴼ.S₁}\AgdaSymbol{)}\AgdaSpace{}%
\AgdaSymbol{→}\<%
\end{code}
\newcommand{\DPTTypePresSub}[0]{\begin{code}[inline]%
%
\>[2]\AgdaFunction{F.sub}\AgdaSpace{}%
\AgdaSymbol{(}\AgdaFunction{⊢σ⇝σ}\AgdaSpace{}%
\AgdaBound{⊢σ}\AgdaSymbol{)}\AgdaSpace{}%
\AgdaSymbol{(}\AgdaFunction{τ⇝τ}\AgdaSpace{}%
\AgdaBound{τ}\AgdaSymbol{)}\AgdaSpace{}%
\AgdaOperator{\AgdaDatatype{≡}}\AgdaSpace{}%
\AgdaFunction{τ⇝τ}\AgdaSpace{}%
\AgdaSymbol{(}\AgdaFunction{Fᴼ.sub}\AgdaSpace{}%
\AgdaBound{σ}\AgdaSpace{}%
\AgdaBound{τ}\AgdaSymbol{)}\<%
\end{code}}
\begin{code}[hide]%
\>[0]\AgdaFunction{⇝-dist-sub-type}%
\>[17]\AgdaBound{⊢σ}\AgdaSpace{}%
\AgdaSymbol{(}\AgdaOperator{\AgdaInductiveConstructor{`}}\AgdaSpace{}%
\AgdaBound{x}\AgdaSymbol{)}\AgdaSpace{}%
\AgdaSymbol{=}\AgdaSpace{}%
\AgdaFunction{⇝-dist-sub-var}\AgdaSpace{}%
\AgdaBound{⊢σ}\AgdaSpace{}%
\AgdaBound{x}\<%
\\
\>[0]\AgdaFunction{⇝-dist-sub-type}%
\>[17]\AgdaBound{⊢σ}\AgdaSpace{}%
\AgdaInductiveConstructor{`⊤}\AgdaSpace{}%
\AgdaSymbol{=}\AgdaSpace{}%
\AgdaInductiveConstructor{refl}\<%
\\
\>[0]\AgdaFunction{⇝-dist-sub-type}%
\>[17]\AgdaBound{⊢σ}\AgdaSpace{}%
\AgdaSymbol{(}\AgdaBound{τ₁}\AgdaSpace{}%
\AgdaOperator{\AgdaInductiveConstructor{⇒}}\AgdaSpace{}%
\AgdaBound{τ₂}\AgdaSymbol{)}\AgdaSpace{}%
\AgdaSymbol{=}\AgdaSpace{}%
\AgdaFunction{cong₂}\AgdaSpace{}%
\AgdaOperator{\AgdaInductiveConstructor{\AgdaUnderscore{}⇒\AgdaUnderscore{}}}\AgdaSpace{}%
\AgdaSymbol{(}\AgdaFunction{⇝-dist-sub-type}%
\>[60]\AgdaBound{⊢σ}\AgdaSpace{}%
\AgdaBound{τ₁}\AgdaSymbol{)}\AgdaSpace{}%
\AgdaSymbol{(}\AgdaFunction{⇝-dist-sub-type}%
\>[86]\AgdaBound{⊢σ}\AgdaSpace{}%
\AgdaBound{τ₂}\AgdaSymbol{)}\<%
\\
\>[0]\AgdaFunction{⇝-dist-sub-type}%
\>[17]\AgdaBound{⊢σ}\AgdaSpace{}%
\AgdaSymbol{(}\AgdaOperator{\AgdaInductiveConstructor{∀`α}}\AgdaSpace{}%
\AgdaBound{τ}\AgdaSymbol{)}\AgdaSpace{}%
\AgdaSymbol{=}\AgdaSpace{}%
\AgdaFunction{cong}\AgdaSpace{}%
\AgdaOperator{\AgdaInductiveConstructor{F.∀`α\AgdaUnderscore{}}}\AgdaSpace{}%
\AgdaSymbol{(}\AgdaFunction{⇝-dist-sub-type}%
\>[60]\AgdaSymbol{(}\AgdaInductiveConstructor{Fᴼ.⊢extₛ}\AgdaSpace{}%
\AgdaBound{⊢σ}\AgdaSymbol{)}\AgdaSpace{}%
\AgdaBound{τ}\AgdaSymbol{)}\<%
\\
\>[0]\AgdaFunction{⇝-dist-sub-type}%
\>[17]\AgdaBound{⊢σ}\AgdaSpace{}%
\AgdaSymbol{(}\AgdaOperator{\AgdaInductiveConstructor{[}}\AgdaSpace{}%
\AgdaOperator{\AgdaInductiveConstructor{`}}\AgdaSpace{}%
\AgdaBound{o}\AgdaSpace{}%
\AgdaOperator{\AgdaInductiveConstructor{∶}}\AgdaSpace{}%
\AgdaBound{τ}\AgdaSpace{}%
\AgdaOperator{\AgdaInductiveConstructor{]⇒}}\AgdaSpace{}%
\AgdaBound{τ'}\AgdaSymbol{)}\AgdaSpace{}%
\AgdaSymbol{=}\AgdaSpace{}%
\AgdaFunction{cong₂}\AgdaSpace{}%
\AgdaOperator{\AgdaInductiveConstructor{\AgdaUnderscore{}⇒\AgdaUnderscore{}}}\AgdaSpace{}%
\AgdaSymbol{(}\AgdaFunction{⇝-dist-sub-type}%
\>[68]\AgdaBound{⊢σ}\AgdaSpace{}%
\AgdaBound{τ}\AgdaSymbol{)}\AgdaSpace{}%
\AgdaSymbol{(}\AgdaFunction{⇝-dist-sub-type}%
\>[92]\AgdaBound{⊢σ}\AgdaSpace{}%
\AgdaBound{τ'}\AgdaSymbol{)}\<%
\\
%
\\[\AgdaEmptyExtraSkip]%
\>[0]\AgdaFunction{⇝-dist-τ[τ']}\AgdaSpace{}%
\AgdaSymbol{:}\AgdaSpace{}%
\AgdaSymbol{\{}\AgdaBound{Γ}\AgdaSpace{}%
\AgdaSymbol{:}\AgdaSpace{}%
\AgdaDatatype{Fᴼ.Ctx}\AgdaSpace{}%
\AgdaGeneralizable{Fᴼ.S₁}\AgdaSymbol{\}}\AgdaSpace{}%
\AgdaSymbol{(}\AgdaBound{τ}\AgdaSpace{}%
\AgdaSymbol{:}\AgdaSpace{}%
\AgdaFunction{Fᴼ.Type}\AgdaSpace{}%
\AgdaGeneralizable{Fᴼ.S₁}\AgdaSymbol{)}\AgdaSpace{}%
\AgdaSymbol{(}\AgdaBound{τ'}\AgdaSpace{}%
\AgdaSymbol{:}\AgdaSpace{}%
\AgdaFunction{Fᴼ.Type}\AgdaSpace{}%
\AgdaSymbol{(}\AgdaGeneralizable{Fᴼ.S₁}\AgdaSpace{}%
\AgdaOperator{\AgdaInductiveConstructor{▷}}\AgdaSpace{}%
\AgdaInductiveConstructor{τₛ}\AgdaSymbol{))}\AgdaSpace{}%
\AgdaSymbol{→}\<%
\end{code}
\newcommand{\DPTTypeDistSingleSub}[0]{\begin{code}[inline]%
\>[0][@{}l@{\AgdaIndent{1}}]%
\>[2]\AgdaSymbol{(}\AgdaFunction{τ⇝τ}\AgdaSpace{}%
\AgdaSymbol{\{}\AgdaArgument{Γ}\AgdaSpace{}%
\AgdaSymbol{=}\AgdaSpace{}%
\AgdaBound{Γ}\AgdaSpace{}%
\AgdaOperator{\AgdaInductiveConstructor{▶}}\AgdaSpace{}%
\AgdaInductiveConstructor{⋆}\AgdaSymbol{\}}\AgdaSpace{}%
\AgdaBound{τ'}\AgdaSpace{}%
\AgdaOperator{\AgdaFunction{F.[}}\AgdaSpace{}%
\AgdaFunction{τ⇝τ}\AgdaSpace{}%
\AgdaBound{τ}\AgdaSpace{}%
\AgdaOperator{\AgdaFunction{]}}\AgdaSymbol{)}\AgdaSpace{}%
\AgdaOperator{\AgdaDatatype{≡}}\AgdaSpace{}%
\AgdaFunction{τ⇝τ}\AgdaSpace{}%
\AgdaSymbol{(}\AgdaBound{τ'}\AgdaSpace{}%
\AgdaOperator{\AgdaFunction{Fᴼ.[}}\AgdaSpace{}%
\AgdaBound{τ}\AgdaSpace{}%
\AgdaOperator{\AgdaFunction{]}}\AgdaSymbol{)}\<%
\end{code}}
\newcommand{\DPTTypePresSingleSub}[0]{\begin{code}[inline]%
\>[0]\AgdaFunction{⇝-dist-τ[τ']}\AgdaSpace{}%
\AgdaBound{τ}\AgdaSpace{}%
\AgdaBound{τ'}\AgdaSpace{}%
\AgdaSymbol{=}\AgdaSpace{}%
\AgdaFunction{⇝-dist-sub-type}%
\>[37]\AgdaFunction{⊢[]}\AgdaSpace{}%
\AgdaBound{τ'}\<%
\end{code}}
\begin{code}[hide]%
\>[0]\AgdaComment{--\ Type\ Preserving\ Translation\ ----------------------------------------------------------}\<%
\\
%
\\[\AgdaEmptyExtraSkip]%
\>[0]\AgdaComment{--\ Variables}\<%
\end{code}
\newcommand{\DPTVarPresLookup}[0]{\begin{code}%
\>[0]\AgdaFunction{⇝-pres-lookup}\AgdaSpace{}%
\AgdaSymbol{:}\AgdaSpace{}%
\AgdaSymbol{∀}\AgdaSpace{}%
\AgdaSymbol{\{}\AgdaBound{Γ}\AgdaSpace{}%
\AgdaSymbol{:}\AgdaSpace{}%
\AgdaDatatype{Fᴼ.Ctx}\AgdaSpace{}%
\AgdaGeneralizable{Fᴼ.S}\AgdaSymbol{\}}\AgdaSpace{}%
\AgdaSymbol{\{}\AgdaBound{τ}\AgdaSpace{}%
\AgdaSymbol{:}\AgdaSpace{}%
\AgdaFunction{Fᴼ.Type}\AgdaSpace{}%
\AgdaGeneralizable{Fᴼ.S}\AgdaSymbol{\}}\AgdaSpace{}%
\AgdaSymbol{(}\AgdaBound{x}\AgdaSpace{}%
\AgdaSymbol{:}\AgdaSpace{}%
\AgdaFunction{Fᴼ.Var}\AgdaSpace{}%
\AgdaGeneralizable{Fᴼ.S}\AgdaSpace{}%
\AgdaInductiveConstructor{eₛ}\AgdaSymbol{)}\AgdaSpace{}%
\AgdaSymbol{→}\<%
\\
\>[0][@{}l@{\AgdaIndent{0}}]%
\>[2]\AgdaFunction{Fᴼ.lookup}\AgdaSpace{}%
\AgdaBound{Γ}\AgdaSpace{}%
\AgdaBound{x}\AgdaSpace{}%
\AgdaOperator{\AgdaDatatype{≡}}\AgdaSpace{}%
\AgdaBound{τ}\AgdaSpace{}%
\AgdaSymbol{→}\<%
\\
%
\>[2]\AgdaFunction{F.lookup}\AgdaSpace{}%
\AgdaSymbol{(}\AgdaFunction{Γ⇝Γ}\AgdaSpace{}%
\AgdaBound{Γ}\AgdaSymbol{)}\AgdaSpace{}%
\AgdaSymbol{(}\AgdaFunction{x⇝x}\AgdaSpace{}%
\AgdaBound{x}\AgdaSymbol{)}\AgdaSpace{}%
\AgdaOperator{\AgdaDatatype{≡}}\AgdaSpace{}%
\AgdaSymbol{(}\AgdaFunction{τ⇝τ}\AgdaSpace{}%
\AgdaBound{τ}\AgdaSymbol{)}\<%
\\
\>[0]\AgdaFunction{⇝-pres-lookup}\AgdaSpace{}%
\AgdaSymbol{\{}\AgdaArgument{Γ}\AgdaSpace{}%
\AgdaSymbol{=}\AgdaSpace{}%
\AgdaBound{Γ}\AgdaSpace{}%
\AgdaOperator{\AgdaInductiveConstructor{▶}}\AgdaSpace{}%
\AgdaBound{τ}\AgdaSymbol{\}}\AgdaSpace{}%
\AgdaSymbol{(}\AgdaInductiveConstructor{here}\AgdaSpace{}%
\AgdaInductiveConstructor{refl}\AgdaSymbol{)}\AgdaSpace{}%
\AgdaInductiveConstructor{refl}\AgdaSpace{}%
\AgdaSymbol{=}\AgdaSpace{}%
\AgdaFunction{⇝-dist-ren-type}\AgdaSpace{}%
\AgdaFunction{Fᴼ.⊢wkᵣ}\AgdaSpace{}%
\AgdaBound{τ}\<%
\\
\>[0]\AgdaFunction{⇝-pres-lookup}\AgdaSpace{}%
\AgdaSymbol{\{}\AgdaArgument{Γ}\AgdaSpace{}%
\AgdaSymbol{=}\AgdaSpace{}%
\AgdaBound{Γ}\AgdaSpace{}%
\AgdaOperator{\AgdaInductiveConstructor{▶}}\AgdaSpace{}%
\AgdaSymbol{\AgdaUnderscore{}\}}\AgdaSpace{}%
\AgdaSymbol{\{}\AgdaBound{τ'}\AgdaSymbol{\}}\AgdaSpace{}%
\AgdaSymbol{(}\AgdaInductiveConstructor{there}\AgdaSpace{}%
\AgdaBound{x}\AgdaSymbol{)}\AgdaSpace{}%
\AgdaInductiveConstructor{refl}\AgdaSpace{}%
\AgdaSymbol{=}\AgdaSpace{}%
\AgdaFunction{trans}\<%
\\
\>[0][@{}l@{\AgdaIndent{0}}]%
\>[2]\AgdaSymbol{(}\AgdaFunction{cong}\AgdaSpace{}%
\AgdaFunction{F.wk}\AgdaSpace{}%
\AgdaSymbol{(}\AgdaFunction{⇝-pres-lookup}\AgdaSpace{}%
\AgdaBound{x}\AgdaSpace{}%
\AgdaInductiveConstructor{refl}\AgdaSymbol{))}\<%
\\
%
\>[2]\AgdaSymbol{(}\AgdaFunction{⇝-dist-ren-type}\AgdaSpace{}%
\AgdaFunction{Fᴼ.⊢wkᵣ}\AgdaSpace{}%
\AgdaSymbol{(}\AgdaFunction{Fᴼ.lookup}\AgdaSpace{}%
\AgdaBound{Γ}\AgdaSpace{}%
\AgdaBound{x}\AgdaSymbol{))}\<%
\\
\>[0]\AgdaComment{--\ ...}\<%
\end{code}}
\begin{code}[hide]%
\>[0]\AgdaFunction{⇝-pres-lookup}\AgdaSpace{}%
\AgdaSymbol{\{}\AgdaArgument{Γ}\AgdaSpace{}%
\AgdaSymbol{=}\AgdaSpace{}%
\AgdaBound{Γ}\AgdaSpace{}%
\AgdaOperator{\AgdaInductiveConstructor{▸}}\AgdaSpace{}%
\AgdaBound{c}\AgdaSymbol{@(}\AgdaOperator{\AgdaInductiveConstructor{`}}\AgdaSpace{}%
\AgdaBound{o}\AgdaSpace{}%
\AgdaOperator{\AgdaInductiveConstructor{∶}}\AgdaSpace{}%
\AgdaBound{τ'}\AgdaSymbol{)\}}\AgdaSpace{}%
\AgdaSymbol{\{}\AgdaBound{τ}\AgdaSymbol{\}}\AgdaSpace{}%
\AgdaBound{x}\AgdaSpace{}%
\AgdaInductiveConstructor{refl}\AgdaSpace{}%
\AgdaSymbol{=}%
\>[51]\AgdaSymbol{(}\<%
\\
\>[0][@{}l@{\AgdaIndent{0}}]%
\>[2]\AgdaOperator{\AgdaFunction{begin}}\<%
\\
\>[2][@{}l@{\AgdaIndent{0}}]%
\>[4]\AgdaFunction{F.wk}\AgdaSpace{}%
\AgdaSymbol{(}\AgdaFunction{F.lookup}\AgdaSpace{}%
\AgdaSymbol{(}\AgdaFunction{Γ⇝Γ}\AgdaSpace{}%
\AgdaBound{Γ}\AgdaSymbol{)}\AgdaSpace{}%
\AgdaSymbol{(}\AgdaFunction{x⇝x}\AgdaSpace{}%
\AgdaBound{x}\AgdaSymbol{))}\<%
\\
%
\>[2]\AgdaFunction{≡⟨}\AgdaSpace{}%
\AgdaFunction{cong}\AgdaSpace{}%
\AgdaFunction{F.wk}\AgdaSpace{}%
\AgdaSymbol{(}\AgdaFunction{⇝-pres-lookup}\AgdaSpace{}%
\AgdaBound{x}\AgdaSpace{}%
\AgdaInductiveConstructor{refl}\AgdaSymbol{)}\AgdaSpace{}%
\AgdaFunction{⟩}\<%
\\
\>[2][@{}l@{\AgdaIndent{0}}]%
\>[4]\AgdaFunction{F.wk}\AgdaSpace{}%
\AgdaSymbol{(}\AgdaFunction{τ⇝τ}\AgdaSpace{}%
\AgdaBound{τ}\AgdaSymbol{)}\<%
\\
%
\>[2]\AgdaFunction{≡⟨}\AgdaSpace{}%
\AgdaFunction{⇝-dist-ren-type}\AgdaSpace{}%
\AgdaFunction{⊢wk-instᵣ}\AgdaSpace{}%
\AgdaBound{τ}\AgdaSpace{}%
\AgdaFunction{⟩}\<%
\\
\>[2][@{}l@{\AgdaIndent{0}}]%
\>[4]\AgdaFunction{τ⇝τ}\AgdaSpace{}%
\AgdaSymbol{(}\AgdaFunction{Fᴼ.ren}\AgdaSpace{}%
\AgdaFunction{Fᴼ.idᵣ}\AgdaSpace{}%
\AgdaBound{τ}\AgdaSymbol{)}\<%
\\
%
\>[2]\AgdaFunction{≡⟨}\AgdaSpace{}%
\AgdaFunction{cong}\AgdaSpace{}%
\AgdaFunction{τ⇝τ}\AgdaSpace{}%
\AgdaSymbol{(}\AgdaFunction{Fᴼ.idᵣτ≡τ}\AgdaSpace{}%
\AgdaBound{τ}\AgdaSymbol{)}\AgdaSpace{}%
\AgdaFunction{⟩}\<%
\\
\>[2][@{}l@{\AgdaIndent{0}}]%
\>[4]\AgdaFunction{τ⇝τ}\AgdaSpace{}%
\AgdaBound{τ}\<%
\\
%
\>[2]\AgdaOperator{\AgdaFunction{∎}}\AgdaSymbol{)}\<%
\end{code}
\newcommand{\DPTOVarPresLookup}[0]{\begin{code}%
\>[0]\AgdaFunction{⇝-pres-cstr-solve}\AgdaSpace{}%
\AgdaSymbol{:}\AgdaSpace{}%
\AgdaSymbol{∀}\AgdaSpace{}%
\AgdaSymbol{\{}\AgdaBound{Γ}\AgdaSpace{}%
\AgdaSymbol{:}\AgdaSpace{}%
\AgdaDatatype{Fᴼ.Ctx}\AgdaSpace{}%
\AgdaGeneralizable{Fᴼ.S}\AgdaSymbol{\}}\AgdaSpace{}%
\AgdaSymbol{→}\<%
\\
\>[0][@{}l@{\AgdaIndent{0}}]%
\>[2]\AgdaSymbol{(}\AgdaBound{o∶τ∈Γ}\AgdaSpace{}%
\AgdaSymbol{:}\AgdaSpace{}%
\AgdaOperator{\AgdaDatatype{[}}\AgdaSpace{}%
\AgdaOperator{\AgdaInductiveConstructor{`}}\AgdaSpace{}%
\AgdaGeneralizable{Fᴼ.o}\AgdaSpace{}%
\AgdaOperator{\AgdaInductiveConstructor{∶}}\AgdaSpace{}%
\AgdaGeneralizable{Fᴼ.τ}\AgdaSpace{}%
\AgdaOperator{\AgdaDatatype{]∈}}\AgdaSpace{}%
\AgdaBound{Γ}\AgdaSymbol{)}\AgdaSpace{}%
\AgdaSymbol{→}\<%
\\
%
\>[2]\AgdaFunction{F.lookup}\AgdaSpace{}%
\AgdaSymbol{(}\AgdaFunction{Γ⇝Γ}\AgdaSpace{}%
\AgdaBound{Γ}\AgdaSymbol{)}\AgdaSpace{}%
\AgdaSymbol{(}\AgdaFunction{o⇝x}\AgdaSpace{}%
\AgdaBound{o∶τ∈Γ}\AgdaSymbol{)}\AgdaSpace{}%
\AgdaOperator{\AgdaDatatype{≡}}\AgdaSpace{}%
\AgdaSymbol{(}\AgdaFunction{τ⇝τ}\AgdaSpace{}%
\AgdaGeneralizable{Fᴼ.τ}\AgdaSymbol{)}\<%
\\
\>[0]\AgdaComment{--\ ...}\<%
\end{code}}
\begin{code}[hide]%
\>[0]\AgdaFunction{⇝-pres-cstr-solve}\AgdaSpace{}%
\AgdaSymbol{\{}\AgdaArgument{τ}\AgdaSpace{}%
\AgdaSymbol{=}\AgdaSpace{}%
\AgdaBound{τ}\AgdaSymbol{\}}\AgdaSpace{}%
\AgdaSymbol{\{}\AgdaArgument{Γ}\AgdaSpace{}%
\AgdaSymbol{=}\AgdaSpace{}%
\AgdaBound{Γ}\AgdaSpace{}%
\AgdaOperator{\AgdaInductiveConstructor{Fᴼ.▸}}\AgdaSpace{}%
\AgdaBound{c}\AgdaSymbol{@(}\AgdaOperator{\AgdaInductiveConstructor{`}}\AgdaSpace{}%
\AgdaBound{o}\AgdaSpace{}%
\AgdaOperator{\AgdaInductiveConstructor{∶}}\AgdaSpace{}%
\AgdaBound{τ}\AgdaSymbol{)\}}\AgdaSpace{}%
\AgdaSymbol{(}\AgdaInductiveConstructor{here}\AgdaSpace{}%
\AgdaSymbol{\{}\AgdaArgument{Γ}\AgdaSpace{}%
\AgdaSymbol{=}\AgdaSpace{}%
\AgdaBound{Γ}\AgdaSymbol{\})}\AgdaSpace{}%
\AgdaSymbol{=}\<%
\\
\>[0][@{}l@{\AgdaIndent{0}}]%
\>[2]\AgdaOperator{\AgdaFunction{begin}}\<%
\\
\>[2][@{}l@{\AgdaIndent{0}}]%
\>[4]\AgdaFunction{F.lookup}\AgdaSpace{}%
\AgdaSymbol{(}\AgdaFunction{Γ⇝Γ}\AgdaSpace{}%
\AgdaBound{Γ}\AgdaSpace{}%
\AgdaOperator{\AgdaInductiveConstructor{▶}}\AgdaSpace{}%
\AgdaFunction{τ⇝τ}\AgdaSpace{}%
\AgdaBound{τ}\AgdaSymbol{)}\AgdaSpace{}%
\AgdaSymbol{(}\AgdaInductiveConstructor{here}\AgdaSpace{}%
\AgdaInductiveConstructor{refl}\AgdaSymbol{)}\<%
\\
%
\>[2]\AgdaFunction{≡⟨}\AgdaSpace{}%
\AgdaFunction{⇝-dist-ren-type}\AgdaSpace{}%
\AgdaFunction{⊢wk-instᵣ}\AgdaSpace{}%
\AgdaBound{τ}\AgdaSpace{}%
\AgdaFunction{⟩}\<%
\\
\>[2][@{}l@{\AgdaIndent{0}}]%
\>[4]\AgdaFunction{τ⇝τ}\AgdaSpace{}%
\AgdaSymbol{(}\AgdaFunction{Fᴼ.ren}\AgdaSpace{}%
\AgdaFunction{Fᴼ.idᵣ}\AgdaSpace{}%
\AgdaBound{τ}\AgdaSymbol{)}\<%
\\
%
\>[2]\AgdaFunction{≡⟨}\AgdaSpace{}%
\AgdaFunction{cong}\AgdaSpace{}%
\AgdaFunction{τ⇝τ}\AgdaSpace{}%
\AgdaSymbol{(}\AgdaFunction{Fᴼ.idᵣτ≡τ}\AgdaSpace{}%
\AgdaBound{τ}\AgdaSymbol{)}\AgdaSpace{}%
\AgdaFunction{⟩}\<%
\\
\>[2][@{}l@{\AgdaIndent{0}}]%
\>[4]\AgdaFunction{τ⇝τ}\AgdaSpace{}%
\AgdaBound{τ}\<%
\\
%
\>[2]\AgdaOperator{\AgdaFunction{∎}}\<%
\\
\>[0]\AgdaFunction{⇝-pres-cstr-solve}\AgdaSpace{}%
\AgdaSymbol{\{}\AgdaArgument{Γ}\AgdaSpace{}%
\AgdaSymbol{=}\AgdaSpace{}%
\AgdaBound{Γ}\AgdaSpace{}%
\AgdaOperator{\AgdaInductiveConstructor{▶}}\AgdaSpace{}%
\AgdaSymbol{\AgdaUnderscore{}\}}\AgdaSpace{}%
\AgdaSymbol{(}\AgdaInductiveConstructor{under-bind}\AgdaSpace{}%
\AgdaSymbol{\{}\AgdaArgument{τ}\AgdaSpace{}%
\AgdaSymbol{=}\AgdaSpace{}%
\AgdaBound{τ}\AgdaSymbol{\}}\AgdaSpace{}%
\AgdaBound{x}\AgdaSymbol{)}\AgdaSpace{}%
\AgdaSymbol{=}\AgdaSpace{}%
\AgdaFunction{trans}\<%
\\
\>[0][@{}l@{\AgdaIndent{0}}]%
\>[2]\AgdaSymbol{(}\AgdaFunction{cong}\AgdaSpace{}%
\AgdaFunction{F.wk}\AgdaSpace{}%
\AgdaSymbol{(}\AgdaFunction{⇝-pres-cstr-solve}\AgdaSpace{}%
\AgdaBound{x}\AgdaSymbol{))}\<%
\\
%
\>[2]\AgdaSymbol{(}\AgdaFunction{⇝-dist-ren-type}\AgdaSpace{}%
\AgdaFunction{Fᴼ.⊢wkᵣ}\AgdaSpace{}%
\AgdaBound{τ}\AgdaSymbol{)}\<%
\\
\>[0]\AgdaFunction{⇝-pres-cstr-solve}\AgdaSpace{}%
\AgdaSymbol{\{}\AgdaArgument{τ}\AgdaSpace{}%
\AgdaSymbol{=}\AgdaSpace{}%
\AgdaBound{τ}\AgdaSymbol{\}}\AgdaSpace{}%
\AgdaSymbol{\{}\AgdaArgument{Γ}\AgdaSpace{}%
\AgdaSymbol{=}\AgdaSpace{}%
\AgdaBound{Γ}\AgdaSpace{}%
\AgdaOperator{\AgdaInductiveConstructor{▸}}\AgdaSpace{}%
\AgdaBound{c}\AgdaSymbol{@(}\AgdaOperator{\AgdaInductiveConstructor{`}}\AgdaSpace{}%
\AgdaBound{o}\AgdaSpace{}%
\AgdaOperator{\AgdaInductiveConstructor{∶}}\AgdaSpace{}%
\AgdaBound{τ'}\AgdaSymbol{)\}}\AgdaSpace{}%
\AgdaSymbol{(}\AgdaInductiveConstructor{under-cstr}\AgdaSpace{}%
\AgdaSymbol{\{}\AgdaArgument{c'}\AgdaSpace{}%
\AgdaSymbol{=}\AgdaSpace{}%
\AgdaSymbol{\AgdaUnderscore{}}\AgdaSpace{}%
\AgdaOperator{\AgdaInductiveConstructor{∶}}\AgdaSpace{}%
\AgdaBound{τ'}\AgdaSymbol{\}}\AgdaSpace{}%
\AgdaBound{o∶τ∈Γ}\AgdaSymbol{)}\AgdaSpace{}%
\AgdaSymbol{=}\<%
\\
\>[0][@{}l@{\AgdaIndent{0}}]%
\>[2]\AgdaOperator{\AgdaFunction{begin}}\<%
\\
\>[2][@{}l@{\AgdaIndent{0}}]%
\>[4]\AgdaFunction{F.wk}\AgdaSpace{}%
\AgdaSymbol{(}\AgdaFunction{F.lookup}\AgdaSpace{}%
\AgdaSymbol{(}\AgdaFunction{Γ⇝Γ}\AgdaSpace{}%
\AgdaBound{Γ}\AgdaSymbol{)}\AgdaSpace{}%
\AgdaSymbol{(}\AgdaFunction{o⇝x}\AgdaSpace{}%
\AgdaBound{o∶τ∈Γ}\AgdaSymbol{))}\<%
\\
%
\>[2]\AgdaFunction{≡⟨}\AgdaSpace{}%
\AgdaFunction{cong}\AgdaSpace{}%
\AgdaFunction{F.wk}\AgdaSpace{}%
\AgdaSymbol{(}\AgdaFunction{⇝-pres-cstr-solve}\AgdaSpace{}%
\AgdaBound{o∶τ∈Γ}\AgdaSymbol{)}\AgdaSpace{}%
\AgdaFunction{⟩}\<%
\\
\>[2][@{}l@{\AgdaIndent{0}}]%
\>[4]\AgdaFunction{F.wk}\AgdaSpace{}%
\AgdaSymbol{(}\AgdaFunction{τ⇝τ}\AgdaSpace{}%
\AgdaBound{τ}\AgdaSymbol{)}\<%
\\
%
\>[2]\AgdaFunction{≡⟨}\AgdaSpace{}%
\AgdaFunction{⇝-dist-ren-type}\AgdaSpace{}%
\AgdaFunction{⊢wk-instᵣ}\AgdaSpace{}%
\AgdaBound{τ}\AgdaSpace{}%
\AgdaFunction{⟩}\<%
\\
\>[2][@{}l@{\AgdaIndent{0}}]%
\>[4]\AgdaFunction{τ⇝τ}\AgdaSpace{}%
\AgdaSymbol{(}\AgdaFunction{Fᴼ.ren}\AgdaSpace{}%
\AgdaFunction{Fᴼ.idᵣ}\AgdaSpace{}%
\AgdaBound{τ}\AgdaSymbol{)}\<%
\\
%
\>[2]\AgdaFunction{≡⟨}\AgdaSpace{}%
\AgdaFunction{cong}\AgdaSpace{}%
\AgdaFunction{τ⇝τ}\AgdaSpace{}%
\AgdaSymbol{(}\AgdaFunction{Fᴼ.idᵣτ≡τ}\AgdaSpace{}%
\AgdaBound{τ}\AgdaSymbol{)}\AgdaSpace{}%
\AgdaFunction{⟩}\<%
\\
\>[2][@{}l@{\AgdaIndent{0}}]%
\>[4]\AgdaFunction{τ⇝τ}\AgdaSpace{}%
\AgdaBound{τ}\<%
\\
%
\>[2]\AgdaOperator{\AgdaFunction{∎}}\<%
\\
%
\\[\AgdaEmptyExtraSkip]%
\>[0]\AgdaComment{--\ Terms}\<%
\end{code}
\newcommand{\DPTTermPres}[0]{\begin{code}%
\>[0]\AgdaFunction{⇝-pres-⊢}\AgdaSpace{}%
\AgdaSymbol{:}\AgdaSpace{}%
\AgdaSymbol{\{}\AgdaBound{Γ}\AgdaSpace{}%
\AgdaSymbol{:}\AgdaSpace{}%
\AgdaDatatype{Fᴼ.Ctx}\AgdaSpace{}%
\AgdaGeneralizable{Fᴼ.S}\AgdaSymbol{\}}\AgdaSpace{}%
\AgdaSymbol{\{}\AgdaBound{t}\AgdaSpace{}%
\AgdaSymbol{:}\AgdaSpace{}%
\AgdaDatatype{Fᴼ.Term}\AgdaSpace{}%
\AgdaGeneralizable{Fᴼ.S}\AgdaSpace{}%
\AgdaGeneralizable{Fᴼ.s}\AgdaSymbol{\}}\<%
\\
\>[0][@{}l@{\AgdaIndent{0}}]%
\>[8]\AgdaSymbol{\{}\AgdaBound{T}\AgdaSpace{}%
\AgdaSymbol{:}\AgdaSpace{}%
\AgdaDatatype{Fᴼ.Term}\AgdaSpace{}%
\AgdaGeneralizable{Fᴼ.S}\AgdaSpace{}%
\AgdaSymbol{(}\AgdaFunction{Fᴼ.type-of}\AgdaSpace{}%
\AgdaGeneralizable{Fᴼ.s}\AgdaSymbol{)\}}\AgdaSpace{}%
\AgdaSymbol{→}\<%
\\
\>[0][@{}l@{\AgdaIndent{0}}]%
\>[2]\AgdaSymbol{(}\AgdaBound{⊢t}\AgdaSpace{}%
\AgdaSymbol{:}\AgdaSpace{}%
\AgdaBound{Γ}\AgdaSpace{}%
\AgdaOperator{\AgdaDatatype{Fᴼ.⊢}}\AgdaSpace{}%
\AgdaBound{t}\AgdaSpace{}%
\AgdaOperator{\AgdaDatatype{∶}}\AgdaSpace{}%
\AgdaBound{T}\AgdaSymbol{)}\AgdaSpace{}%
\AgdaSymbol{→}\<%
\\
%
\>[2]\AgdaSymbol{(}\AgdaFunction{Γ⇝Γ}\AgdaSpace{}%
\AgdaBound{Γ}\AgdaSymbol{)}\AgdaSpace{}%
\AgdaOperator{\AgdaDatatype{F.⊢}}\AgdaSpace{}%
\AgdaSymbol{(}\AgdaFunction{⊢t⇝t}\AgdaSpace{}%
\AgdaBound{⊢t}\AgdaSymbol{)}\AgdaSpace{}%
\AgdaOperator{\AgdaDatatype{∶}}\AgdaSpace{}%
\AgdaSymbol{(}\AgdaFunction{T⇝T}\AgdaSpace{}%
\AgdaBound{T}\AgdaSymbol{)}\<%
\\
\>[0]\AgdaFunction{⇝-pres-⊢}\AgdaSpace{}%
\AgdaSymbol{(}\AgdaInductiveConstructor{⊢`x}\AgdaSpace{}%
\AgdaSymbol{\{}\AgdaArgument{x}\AgdaSpace{}%
\AgdaSymbol{=}\AgdaSpace{}%
\AgdaBound{x}\AgdaSymbol{\}}\AgdaSpace{}%
\AgdaBound{Γx≡τ}\AgdaSymbol{)}\AgdaSpace{}%
\AgdaSymbol{=}\AgdaSpace{}%
\AgdaInductiveConstructor{⊢`x}%
\>[35]\AgdaSymbol{(}\AgdaFunction{⇝-pres-lookup}\AgdaSpace{}%
\AgdaBound{x}\AgdaSpace{}%
\AgdaBound{Γx≡τ}\AgdaSymbol{)}\<%
\\
\>[0]\AgdaFunction{⇝-pres-⊢}\AgdaSpace{}%
\AgdaSymbol{(}\AgdaInductiveConstructor{⊢`o}\AgdaSpace{}%
\AgdaBound{o∶τ∈Γ}\AgdaSymbol{)}\AgdaSpace{}%
\AgdaSymbol{=}\AgdaSpace{}%
\AgdaInductiveConstructor{⊢`x}\AgdaSpace{}%
\AgdaSymbol{(}\AgdaFunction{⇝-pres-cstr-solve}\AgdaSpace{}%
\AgdaBound{o∶τ∈Γ}\AgdaSymbol{)}\<%
\\
\>[0]\AgdaFunction{⇝-pres-⊢}\AgdaSpace{}%
\AgdaSymbol{(}\AgdaInductiveConstructor{⊢let}\AgdaSpace{}%
\AgdaBound{⊢e₂}\AgdaSpace{}%
\AgdaBound{⊢e₁}\AgdaSymbol{)}\AgdaSpace{}%
\AgdaSymbol{=}\AgdaSpace{}%
\AgdaInductiveConstructor{⊢let}\AgdaSpace{}%
\AgdaSymbol{(}\AgdaFunction{⇝-pres-⊢}\AgdaSpace{}%
\AgdaBound{⊢e₂}\AgdaSymbol{)}\<%
\\
\>[0][@{}l@{\AgdaIndent{0}}]%
\>[2]\AgdaSymbol{(}\AgdaFunction{subst}\AgdaSpace{}%
\AgdaSymbol{(\AgdaUnderscore{}}\AgdaSpace{}%
\AgdaOperator{\AgdaDatatype{F.⊢}}\AgdaSpace{}%
\AgdaFunction{⊢t⇝t}\AgdaSpace{}%
\AgdaBound{⊢e₁}\AgdaSpace{}%
\AgdaOperator{\AgdaDatatype{∶\AgdaUnderscore{}}}\AgdaSymbol{)}\AgdaSpace{}%
\AgdaFunction{⇝-dist-wk-type}\AgdaSymbol{(}\AgdaFunction{⇝-pres-⊢}\AgdaSpace{}%
\AgdaBound{⊢e₁}\AgdaSymbol{))}\<%
\\
\>[0]\AgdaFunction{⇝-pres-⊢}\AgdaSpace{}%
\AgdaSymbol{(}\AgdaInductiveConstructor{⊢ƛ}\AgdaSpace{}%
\AgdaSymbol{\{}\AgdaArgument{c}\AgdaSpace{}%
\AgdaSymbol{=}\AgdaSpace{}%
\AgdaSymbol{(}\AgdaOperator{\AgdaInductiveConstructor{`}}\AgdaSpace{}%
\AgdaBound{o}\AgdaSpace{}%
\AgdaOperator{\AgdaInductiveConstructor{∶}}\AgdaSpace{}%
\AgdaBound{τ}\AgdaSymbol{)\}}\AgdaSpace{}%
\AgdaBound{⊢e}\AgdaSymbol{)}\AgdaSpace{}%
\AgdaSymbol{=}\AgdaSpace{}%
\AgdaInductiveConstructor{⊢λ}\<%
\\
\>[0][@{}l@{\AgdaIndent{0}}]%
\>[2]\AgdaSymbol{(}\AgdaFunction{subst}\AgdaSpace{}%
\AgdaSymbol{(\AgdaUnderscore{}}\AgdaSpace{}%
\AgdaOperator{\AgdaDatatype{F.⊢}}\AgdaSpace{}%
\AgdaFunction{⊢t⇝t}\AgdaSpace{}%
\AgdaBound{⊢e}\AgdaSpace{}%
\AgdaOperator{\AgdaDatatype{∶\AgdaUnderscore{}}}\AgdaSymbol{)}\AgdaSpace{}%
\AgdaFunction{⇝-dist-wk-inst-type}\AgdaSpace{}%
\AgdaSymbol{(}\AgdaFunction{⇝-pres-⊢}\AgdaSpace{}%
\AgdaBound{⊢e}\AgdaSymbol{))}\<%
\\
\>[0]\AgdaFunction{⇝-pres-⊢}\AgdaSpace{}%
\AgdaSymbol{(}\AgdaInductiveConstructor{⊢⊘}\AgdaSpace{}%
\AgdaBound{⊢e}\AgdaSpace{}%
\AgdaBound{o∶τ∈Γ}\AgdaSymbol{)}\AgdaSpace{}%
\AgdaSymbol{=}\AgdaSpace{}%
\AgdaInductiveConstructor{⊢·}\AgdaSpace{}%
\AgdaSymbol{(}\AgdaFunction{⇝-pres-⊢}\AgdaSpace{}%
\AgdaBound{⊢e}\AgdaSymbol{)}\AgdaSpace{}%
\AgdaSymbol{(}\AgdaInductiveConstructor{⊢`x}\AgdaSpace{}%
\AgdaSymbol{(}\AgdaFunction{⇝-pres-cstr-solve}\AgdaSpace{}%
\AgdaBound{o∶τ∈Γ}\AgdaSymbol{))}\<%
\\
\>[0]\AgdaFunction{⇝-pres-⊢}\AgdaSpace{}%
\AgdaSymbol{(}\AgdaInductiveConstructor{⊢•}\AgdaSpace{}%
\AgdaSymbol{\{}\AgdaArgument{τ}\AgdaSpace{}%
\AgdaSymbol{=}\AgdaSpace{}%
\AgdaBound{τ}\AgdaSymbol{\}}\AgdaSpace{}%
\AgdaSymbol{\{}\AgdaArgument{τ'}\AgdaSpace{}%
\AgdaSymbol{=}\AgdaSpace{}%
\AgdaBound{τ'}\AgdaSymbol{\}}\AgdaSpace{}%
\AgdaBound{⊢e}\AgdaSymbol{)}\AgdaSpace{}%
\AgdaSymbol{=}\AgdaSpace{}%
\AgdaFunction{subst}\AgdaSpace{}%
\AgdaSymbol{(\AgdaUnderscore{}}\AgdaSpace{}%
\AgdaOperator{\AgdaDatatype{F.⊢}}%
\>[51]\AgdaFunction{⊢t⇝t}\AgdaSpace{}%
\AgdaBound{⊢e}\AgdaSpace{}%
\AgdaOperator{\AgdaInductiveConstructor{•}}\AgdaSpace{}%
\AgdaFunction{τ⇝τ}\AgdaSpace{}%
\AgdaBound{τ'}%
\>[69]\AgdaOperator{\AgdaDatatype{∶\AgdaUnderscore{}}}\AgdaSymbol{)}\<%
\\
\>[0][@{}l@{\AgdaIndent{0}}]%
\>[2]\AgdaSymbol{(}\AgdaFunction{⇝-dist-τ[τ']}\AgdaSpace{}%
\AgdaBound{τ'}\AgdaSpace{}%
\AgdaBound{τ}\AgdaSymbol{)}\AgdaSpace{}%
\AgdaSymbol{(}\AgdaInductiveConstructor{⊢•}\AgdaSpace{}%
\AgdaSymbol{(}\AgdaFunction{⇝-pres-⊢}\AgdaSpace{}%
\AgdaBound{⊢e}\AgdaSymbol{))}\<%
\\
\>[0]\AgdaComment{--\ ...}\<%
\end{code}}
\begin{code}[hide]%
\>[0]\AgdaFunction{⇝-pres-⊢}\AgdaSpace{}%
\AgdaInductiveConstructor{⊢⊤}\AgdaSpace{}%
\AgdaSymbol{=}\AgdaSpace{}%
\AgdaInductiveConstructor{⊢⊤}\<%
\\
\>[0]\AgdaFunction{⇝-pres-⊢}\AgdaSpace{}%
\AgdaSymbol{(}\AgdaInductiveConstructor{⊢λ}\AgdaSpace{}%
\AgdaSymbol{\{}\AgdaArgument{τ'}\AgdaSpace{}%
\AgdaSymbol{=}\AgdaSpace{}%
\AgdaBound{τ'}\AgdaSymbol{\}}\AgdaSpace{}%
\AgdaBound{⊢e}\AgdaSymbol{)}\AgdaSpace{}%
\AgdaSymbol{=}\AgdaSpace{}%
\AgdaInductiveConstructor{⊢λ}\AgdaSpace{}%
\AgdaSymbol{(}\AgdaFunction{subst}\AgdaSpace{}%
\AgdaSymbol{(\AgdaUnderscore{}}\AgdaSpace{}%
\AgdaOperator{\AgdaDatatype{F.⊢}}\AgdaSpace{}%
\AgdaFunction{⊢t⇝t}\AgdaSpace{}%
\AgdaBound{⊢e}\AgdaSpace{}%
\AgdaOperator{\AgdaDatatype{∶\AgdaUnderscore{}}}\AgdaSymbol{)}\<%
\\
\>[0][@{}l@{\AgdaIndent{0}}]%
\>[2]\AgdaFunction{⇝-dist-wk-type}\AgdaSymbol{(}\AgdaFunction{⇝-pres-⊢}\AgdaSpace{}%
\AgdaBound{⊢e}\AgdaSymbol{))}\<%
\\
\>[0]\AgdaFunction{⇝-pres-⊢}\AgdaSpace{}%
\AgdaSymbol{(}\AgdaInductiveConstructor{⊢Λ}\AgdaSpace{}%
\AgdaBound{⊢e}\AgdaSymbol{)}\AgdaSpace{}%
\AgdaSymbol{=}\AgdaSpace{}%
\AgdaInductiveConstructor{⊢Λ}\AgdaSpace{}%
\AgdaSymbol{(}\AgdaFunction{⇝-pres-⊢}\AgdaSpace{}%
\AgdaBound{⊢e}\AgdaSymbol{)}\<%
\\
\>[0]\AgdaFunction{⇝-pres-⊢}\AgdaSpace{}%
\AgdaSymbol{(}\AgdaInductiveConstructor{⊢·}\AgdaSpace{}%
\AgdaBound{⊢e₁}\AgdaSpace{}%
\AgdaBound{⊢e₂}\AgdaSymbol{)}\AgdaSpace{}%
\AgdaSymbol{=}\AgdaSpace{}%
\AgdaInductiveConstructor{⊢·}\AgdaSpace{}%
\AgdaSymbol{(}\AgdaFunction{⇝-pres-⊢}\AgdaSpace{}%
\AgdaBound{⊢e₁}\AgdaSymbol{)}\AgdaSpace{}%
\AgdaSymbol{(}\AgdaFunction{⇝-pres-⊢}\AgdaSpace{}%
\AgdaBound{⊢e₂}\AgdaSymbol{)}\<%
\\
\>[0]\AgdaFunction{⇝-pres-⊢}\AgdaSpace{}%
\AgdaSymbol{(}\AgdaInductiveConstructor{⊢decl}\AgdaSpace{}%
\AgdaBound{⊢e}\AgdaSymbol{)}\AgdaSpace{}%
\AgdaSymbol{=}\AgdaSpace{}%
\AgdaInductiveConstructor{⊢let}\AgdaSpace{}%
\AgdaInductiveConstructor{⊢⊤}\AgdaSpace{}%
\AgdaSymbol{(}\AgdaFunction{subst}\AgdaSpace{}%
\AgdaSymbol{(\AgdaUnderscore{}}\AgdaSpace{}%
\AgdaOperator{\AgdaDatatype{F.⊢}}\AgdaSpace{}%
\AgdaFunction{⊢t⇝t}\AgdaSpace{}%
\AgdaBound{⊢e}\AgdaSpace{}%
\AgdaOperator{\AgdaDatatype{∶\AgdaUnderscore{}}}\AgdaSymbol{)}\<%
\\
\>[0][@{}l@{\AgdaIndent{0}}]%
\>[2]\AgdaFunction{⇝-dist-wk-type}\AgdaSymbol{(}\AgdaFunction{⇝-pres-⊢}\AgdaSpace{}%
\AgdaBound{⊢e}\AgdaSymbol{))}\<%
\\
\>[0]\AgdaFunction{⇝-pres-⊢}\AgdaSpace{}%
\AgdaSymbol{(}\AgdaInductiveConstructor{⊢inst}\AgdaSpace{}%
\AgdaSymbol{\{}\AgdaArgument{o}\AgdaSpace{}%
\AgdaSymbol{=}\AgdaSpace{}%
\AgdaBound{o}\AgdaSymbol{\}}\AgdaSpace{}%
\AgdaBound{⊢e₂}\AgdaSpace{}%
\AgdaBound{⊢e₁}\AgdaSymbol{)}\AgdaSpace{}%
\AgdaSymbol{=}\AgdaSpace{}%
\AgdaInductiveConstructor{⊢let}\AgdaSpace{}%
\AgdaSymbol{(}\AgdaFunction{⇝-pres-⊢}\AgdaSpace{}%
\AgdaBound{⊢e₂}\AgdaSymbol{)}\<%
\\
\>[0][@{}l@{\AgdaIndent{0}}]%
\>[1]\AgdaSymbol{(}\AgdaFunction{subst}\AgdaSpace{}%
\AgdaSymbol{(\AgdaUnderscore{}}\AgdaSpace{}%
\AgdaOperator{\AgdaDatatype{F.⊢}}\AgdaSpace{}%
\AgdaFunction{⊢t⇝t}\AgdaSpace{}%
\AgdaBound{⊢e₁}\AgdaSpace{}%
\AgdaOperator{\AgdaDatatype{∶\AgdaUnderscore{}}}\AgdaSymbol{)}\AgdaSpace{}%
\AgdaFunction{⇝-dist-wk-inst-type}\AgdaSpace{}%
\AgdaSymbol{(}\AgdaFunction{⇝-pres-⊢}\AgdaSpace{}%
\AgdaBound{⊢e₁}\AgdaSymbol{))}\<%
\end{code}



\begin{frame}
  \titlepage
\end{frame}

\section{Introduction}

\begin{frame}[fragile]
  \frametitle{Type Classes in Haskell}
  \begin{block}{Overloading Equality in Haskell}
    \begin{center}
      ~
      \begin{minted}[escapeinside=||]{haskell}
 class Eq α where
   eq :: α → α → Bool
 
 instance Eq Nat where
   eq x y = x ≐ y
 instance Eq α ⇒ Eq [α] where
   eq []       []       = True
   eq (x : xs) (y : ys) = eq x y && eq xs ys 
     
 .. eq 42 0 .. eq [42, 0] [42, 0] ..
      \end{minted}
      ~
    \end{center}
  \end{block}
\end{frame}

\begin{frame}[fragile]
  \frametitle{Desugaring Type Classes}
  \begin{block}{Overloading Equality in System \Fo}
    \begin{center}
      ~
      \begin{minted}[escapeinside=||]{haskell}
 |\Decl| eq in

 |\Inst| eq : Nat → Nat → Bool 
   = λx. λy. .. in
 |\Inst| eq : ∀α. [eq : α → α → Bool] ⇒ [α] → [α] → Bool 
   = Λα. ƛ(eq : α → α → Bool). λxs. λys. .. in
    
 .. eq 42 0 .. eq Nat [42, 0] [42, 0] .. 
      \end{minted}
      ~
    \end{center}
  \end{block}
\end{frame}

\begin{frame}[fragile]
  \frametitle{Dictionary Passing Transform}
  \begin{block}{Overloading Equality in System \Fo\ }
    \begin{center}
      \begin{minted}[escapeinside=||]{haskell}
 |\Decl| eq in
 |\Inst| eq : Nat → Nat → Bool 
   = λx. λy. .. in
 |\Inst| eq : ∀α. [eq : α → α → Bool] ⇒ [α] → [α] → Bool 
   = Λα. ƛ(eq : α → α → Bool). λxs. λys. .. in
 .. eq 42 0 .. eq Nat [42, 0] [42, 0] .. 
      \end{minted}
    \end{center}
  \end{block}
  \begin{block}{System \Fo\ Transformed to System F}
    \begin{center}
      \begin{minted}[escapeinside=||]{haskell}
 let eq₁ : Nat → Nat → Bool 
   = λx. λy. .. in
 let eq₂ : ∀α. (α → α → Bool) → [α] → [α] → Bool 
   = Λα. λeq₁. λxs. λys. .. in
  
 .. eq₁ 42 0 .. eq₂ Nat eq₁ [42, 0] [42, 0] .. 
      \end{minted}
    \end{center}
  \end{block}
\end{frame}

\section{Agda Formalization}

\begin{frame}[fragile]
  \frametitle{Agda Formalization of System \Fo}
  \begin{block}{Syntax Representation in Agda}
    \begin{small}
      
        \FoTerm
     
    \end{small}
  \end{block}
\end{frame}

\begin{frame}[fragile]
  \frametitle{Agda Formalization of System \Fo}
  \begin{block}{Context}
    \FoCtx
  \end{block}
  \begin{block}{Constraint Solving}
    \FoCstrSolve
  \end{block}
\end{frame}

\begin{frame}[fragile]
  \frametitle{The Dectionary Passing Transform}
\end{frame}

\section{Agda Formalization of the Dictionary Passing Transform}
\begin{frame}[fragile]
  \frametitle{Fun Lemmas on Our Way to Type Preservation}
  
\end{frame}

\begin{frame}[fragile]
  \frametitle{Type Preservation of the Dictionary Passing Transform}
  
\end{frame}


\end{document}