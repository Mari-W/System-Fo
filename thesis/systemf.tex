\subsection{Specification}
\subsubsection{Sorts}\hfill\\\\
The formalization of System F requires three sorts: \Constr{eₛ} for expressions, \Constr{τₛ} for types and \Constr{κₛ} for kinds. 
\FSort
Sorts are indexed by the boolish data type \Data{Bindable}. 
The index \Constr{var} indicates that variables for terms of a sort can be bound. 
In contrast, the index \Constr{no-var} says that variables for terms of a sort cannot be bound. 
In this case, System F supports abstracting over expressions and types, but not over kinds. 
Going forward, we will use the variable \Sym{r} for elements of type \Data{Bindable}, the variable \Sym{s} for elements of type \Data{Sort} and the variable \Sym{S} for lists of bindable sorts with type \FSorts.

\subsubsection{Syntax}\hfill\\\\
The syntax of System F is represented in a single data type \Data{Term} that is indexed by sorts \Sym{S} and sort \Sym{s}. 
The index \Sym{S} is inspired by Debruijn indices. 
Debruijn indices reference variables using a number that counts the amount of binders that are in scope between the binding of the variable and the position it is used at. 
In Agda, terms are often indexed by the amount of bound variables. 
The variable constructor then only accepts Debruijn indices, instead of variable names, that are smaller than the current amount of bound variables.
As a result, unbound variables cannot be referenced by definition. 
This technique is also referred to as intrinsically scoped.
But indexing \Data{Term} with a number is not sufficient because System F has both expression variables and type variables that need to be distinguished. 
To solve this problem, we need to extend the idea of Debruijn indices and store the corresponding sort for each variable. Thus, we let \Sym{S} be a list of bindable sorts, instead of a number.
The length of \Sym{S} represents the amount of bound variables and the elements \Sym{s$_i$} of the list represent the sort of the variable bound at Debruijn index \Sym{i}. 

\noindent The index \Sym{s} represents the sort of the term itself.
\newpage
\FTerm
Variables \Constr{`} \Sym{x} are represented as membership proofs of type \Sym{s} \Prim{∈} \Sym{S}.
Membership proofs are inductively defined, similar to the definition of natural numbers. 
Membership proofs can be constructed using the constructor \Constr{here refl}, where \Constr{refl} is proof that the last element in a list \Sym{S} is the element we searched for. 
Alternatively, membership proofs for a list \Sym{S} can be constructed via the constructor \Constr{there} \Sym{x}, where \Sym{x} is a membership proof for the sublist \Sym{S'} of \Sym{S} that has one element less. 
As discussed, the Debruijn representation of variables has the advantage that only already bound variables can be referenced by the variable constructor by definition. 

\noindent The unit element \Constr{tt} and type \Constr{`⊤} represent base expressions and types respectively. 

\noindent Lambda abstractions \Constr{λ`x→} \Sym{e} result in function types \Sym{τ₁} \Constr{⇒} \Sym{τ₂} and type abstractions \Constr{Λ`α→} \Sym{e} result in forall types \Constr{∀`α} \Sym{τ}. 
Both abstractions and forall types introduce an additional sort \Constr{eₛ}, or \Constr{τₛ} respectively, to the index \Sym{S} of their corresponding body to account for the additional new binding.

\noindent The application constructor \Sym{e₁} \Constr{·} \Sym{e₂} applies the argument \Sym{e₂} to the function \Sym{e₁}.

\noindent Similarly, type application \Sym{e} \Constr{•} \Sym{τ} eliminates type abstractions. 

\noindent Let bindings \Constr{let`x=} \Sym{e₂} \Constr{`in} \Sym{e₁} combine abstraction and application. 

\noindent The kind \Constr{⋆} is kind of all types.

\noindent We use abbreviations \FVar, \FExpr, \FType\ and variables \Sym{x}, \Sym{e} and \Sym{τ} respectively. Furthermore, we use the variable \Sym{t} for an arbitrary \Data{Term} \Sym{S} \Sym{s}.

\subsubsection{Renaming}\hfill\\\\
Renamings \Sym{ρ} of type \Data{Ren} \Sym{S₁} \Sym{S₂} are defined as total functions that map variables \Data{Var} \Sym{S₁} \Sym{s} to variables \Data{Var} \Sym{S₂} \Sym{s}. 
Renamings preserve the sort \Sym{s} of the variable.
\FRen
Applying a renaming \Data{Ren} \Sym{S₁} \Sym{S₂} to a term \Data{Term} \Sym{S₁} \Sym{s} yields a new term \Data{Term} \Sym{S₂} \Sym{s}, where variables are represented as references to elements in \Sym{S₂} instead of \Sym{S₁}. The function \Data{ren} applies a renaming to a term.
\Fren
In the first case, the renaming is applied to all variables \Sym{x}.

\noindent When we encounter a binder for a term of sort \Sym{s}, as seen in the second case, the renaming is extended using function \Data{extᵣ}. 
If we want to use a renaming as a function or use the function \Data{extᵣ}, the sort argument \Sym{s} can usually be inferred by Agda. 
Inferring a function argument is denoted with \Sym{\_}. 
\Frenext
The extension of a renaming introduces an additional variable binding of sort \Sym{s}. Thus, if we encounter the new binding \Constr{here refl} in the extended renaming, then we return the variable for the new binding \Constr{here refl}. 
The variables \Sym{x} of the original renaming \Sym{ρ} are weakened by wrapping them in an additional \Constr{there} constructor. The sort arguments are ignored inside the function body of \Data{extᵣ} by using the wildcard pattern \Sym{\_}.

\noindent Similar to variables, terms can be weakened using the function \Data{wk} that shifts all variables present in the term by one recursively. 
\Fwk 
The function \Data{wkᵣ} generates a weakening by wrapping all variables in an additional \Constr{there} constructor.
\Frenwk

\subsubsection{Substitution}\hfill\\\\
The definition of substitutions \Sym{σ} with type \Data{Sub} \Sym{S₁} \Sym{S₂} is similar to the definition of renamings. 
But rather than mapping variables to variables, substitutions map variables to terms.
\FSub
Applying a substitution using the \Data{sub} function is analogous to applying a renaming using \Data{ren}. 
If we encounter a binder in \Data{sub}, the substitution must be extended using function \Data{extₛ}.
\Fext
For the newly bound variable \Constr{here refl}, we return the variable term \Constr{` here refl}.  
Furthermore, all terms \Sym{σ \_ x} originally present in the substitution \Sym{σ} are weakened using the function \Data{wk}.

\noindent The substitution operator \Sym{t} \Data{[} \Sym{t'} \Data{]} substitutes the last bound variable in \Sym{t} with \Sym{t'}, given that the sort of the last binder corresponds to the sort of \Sym{t'}.
\Fsubs
A single substitution \Data{singleₛ} takes an existing substitution \Sym{σ'} and term \Sym{t'}. The term \Sym{t'} is then introduced for an additional new binding \Constr{here refl}. 
\Fsinglesub
In the case of \Data{\_[\_]}, we let \Sym{σ'} be the identity substitution \Fidsub.

\subsubsection{Context}\hfill\\\\
Similar to terms, typing contexts \Sym{Γ} are indexed by a list of bound variables.
In consequence, only types and kinds for bound variables can be stored in \Sym{Γ} by definition.
\FCtx
Contexts are inductively defined and can either be empty \Constr{∅} or extended with one element \Sym{T}, using the constructor \Sym{Γ} \Constr{▶} \Sym{T}.
The variable \Sym{T} represents terms of the sort \Data{type-of} \Sym{s}. 
\noindent The function \Data{type-of} maps bindable sorts \Sym{s} to the sort of the term that is stored in \Sym{Γ} for variables of the sort \Sym{s}. Thus, if we bind a new variable for a term of the sort \Sym{s}, then \Sym{Γ} needs to be extended by a term of the sort \Data{type-of} \Sym{s}.
\Fkind
Expression variables require \Sym{Γ} to store the corresponding type. 
For type variables, the context \Sym{Γ} stores the corresponding kind. 

\noindent The \Data{lookup} function resolves the type or kind \Sym{T} for a variable in the context \Sym{Γ}.
\Flookup
Inside both cases of the case split on variables, we wrap the looked up \Sym{T} in a weakening. 
As a result, \Sym{T} always has the index \Sym{S} that aligns with the current required amount of bound variables. 
The \Data{lookup} function cannot fail by definition because we only allow to lookup bound variables that must have an entry in \Sym{Γ} by definition.

\subsubsection{Typing}\hfill\\\\
The typing relation \Sym{Γ} \Data{⊢} \Sym{t} \Data{:} \Sym{T} relates a term \Sym{t} to its typing result \Sym{T} in a context \Sym{Γ}.

\newpage

\FTyping
The rule \Constr{⊢`x} says that a variable \Constr{`} \Sym{x} has type \Sym{τ}, if the type for \Sym{x} in \Sym{Γ} is \Sym{τ}. 

\noindent All unit expressions \Constr{tt} have type \Constr{`⊤}. This is expressed by the rule \Constr{⊢⊤}.

\noindent The rule for abstractions \Constr{⊢λ} introduces an expression variable of type \Sym{τ} to the body \Sym{e}. 
Because the resulting body type \Sym{τ'} cannot use the newly introduced expression variable, we let \Sym{τ'} have one variable bound less and weaken it to align in the list of bound variables with the context \Sym{Γ} \Constr{▶} \Sym{τ}. 
As a result, \Sym{τ'} aligns with \Sym{τ} in the list of bound variables to form the resulting function type \Sym{τ} \Constr{⇒} \Sym{τ’}. 

\noindent The type abstraction rule \Constr{⊢Λ} introduces a type variable to the body \Sym{e} and results in the forall type \Constr{∀`α} \Sym{τ}, where \Sym{τ} is the type of \Sym{e}. 
The type variable in \Sym{e} is introduced by extending \Sym{Γ} with the kind \Constr{⋆}. 

\noindent Application is handled by the rule \Constr{⊢·}. 
The rule says that if \Sym{e₁} is a function from \Sym{τ₁} to \Sym{τ₂} and \Sym{e₂} has type \Sym{τ₁}, then \Sym{e₁} \Constr{·} \Sym{e₂} has type \Sym{τ₂}. 

\noindent Similarly, the type application rule \Constr{⊢•} states that if \Sym{e} has type \Constr{∀`α} \Sym{τ}, then \Sym{\alpha} can be substituted with another type \Sym{τ'} inside \Sym{τ}. 

\noindent The rule \Constr{⊢let} combines the abstraction and application rule.

\noindent Regarding the typing of types, the rule \Constr{⊢τ} indicates that all types \Sym{τ} are well formed and have kind \Constr{⋆}. Type variables are correctly typed per definition and type constructors \Constr{∀`α} and \Constr{⇒} accept arbitrary types as their arguments. Hence, all types are well typed.

\subsubsection{Typing of Renaming \& Substitution}\hfill\\\\
Because of extrinsic typing, both renamings and substitutions need to have typed counterparts.

\noindent We formalize typed renamings \Sym{⊢ρ} inductively as order preserving embeddings. 
Thus, if a variable \Sym{x₁} of type \Sym{s₁} \Data{∈} \Sym{S₁} references an element with an index smaller than some other variable \Sym{x₂} in \Sym{S₁}, then renamed \Sym{x₁} must still reference an element with a smaller index than renamed \Sym{x₂} in \Sym{S₂}.
Arbitrary renamings would allow swapping items and potentially violate the telescoping. 
Telescoping allows types stored in the context to depend on type variables bound inside the context before them. 

\noindent Interestingly, because of the intrinsically scoped definition of terms, all renamings must be order preserving embeddings by definition. 
Thus, it should be possible to prove order preservation in the form of lemmas.
Instead we choose to represent the rules for order preserving embeddings as constructors of a data type, such that we can access the property of order preservation by matching on the data type.
\FRenTyping
The identity renaming \Constr{⊢idᵣ} is typed by definition.

\noindent The typed extension of a renaming \Constr{⊢extᵣ} allows to extend both \Sym{Γ₁} and \Sym{Γ₂} by \Sym{T'} and renamed \Sym{T'} respectively. 
The constructor \Constr{⊢extᵣ} corresponds to the typed version of the function \Data{extᵣ} that is used when a binder is encountered. 

\noindent The constructor \Constr{⊢dropᵣ} allows to introduce \Sym{T'} only in \Sym{Γ₂}. 
Hence, \Constr{⊢dropᵣ ⊢idᵣ} corresponds to the typed weakening \Data{⊢wkᵣ} of a term.

\noindent The absence of a constructor that allows to introduce a term only to \Sym{Γ₁} is exactly the restriction needed for typed renamings to be order preserving embeddings.

\noindent Typed Substitutions are defined as total functions, similar to untyped substitutions.
\FSubTyping
Typed substitutions \Sym{⊢σ} map variables \Sym{x} \Constr{∈} \Sym{S₁} to the corresponding typing of the term \Sym{σ \_ x} in \Sym{Γ₂}. 
The type of the term \Sym{σ \_ x} must be the original type of \Sym{x} in \Sym{Γ₁} applied to the substitution \Sym{σ}.
\subsubsection{Semantics}\hfill\\\\
The semantics of System F are formalized as call-by-value, that is, there is no reduction under binders. 

\noindent Values are indexed by their corresponding irreducible expression.
\FVal
System F has three values. 
The two closure values \Constr{v-λ} and \Constr{v-Λ} and the unit value \Constr{v-tt}.
We formalize small step semantics, where each constructor represents a single reduction step \Sym{e} \Constr{↪} \Sym{e'}.
Small step semantics distinguish between \Sym{β} and \Sym{ξ} rules. 
Meaningful computation in the form of substitution is done by \Sym{β} rules while \Sym{ξ} rules only reduce subexpressions.
\FSemantics
The rules \Constr{β-λ} and \Constr{β-Λ} give meaning to application and type application by substituting the applied expression, or type respectively, into the abstraction body. 
In both cases, we make sure that the abstraction and the applied argument are values.

\noindent The reduction rule \Constr{β-let} functions similar to the rule \Constr{β-λ} and substitutes the value \Sym{e₂} into \Sym{e₁}. 

\noindent The rules \Constr{ξ-·$_i$} and \Constr{ξ-•} evaluate subexpressions of applications until \Sym{e₁} and \Sym{e₂}, or \Sym{e} respectively, are values. 

\noindent The rule \Constr{ξ-let} reduces the bound expression \Sym{e₂} until \Sym{e₂} is a value and \Constr{β-let} can be applied. 

\subsection{Soundness}

\subsubsection{Progress}\hfill\\\\
We prove \Data{progress} by showing that a typed expression \Sym{e} can either be further reduced to another expression \Sym{e'} or \Sym{e} is a value. 
The proof follows by induction over the typing rules. 
\newpage
\FProgress
The cases \Constr{⊢⊤}, \Constr{⊢λ} and \Constr{⊢Λ} result in values. 
The application cases \Constr{⊢·}, \Constr{⊢•} and \Constr{⊢let} follow directly from the induction hypothesis. 
\subsubsection{Subject Reduction}\hfill\\\\
We prove \Data{subject-reduction}, that is, reduction preserves typing. 
More specifically, an expression \Sym{e} with type \Sym{τ} still has type \Sym{τ} after being reduced to \Sym{e'}. 
We prove subject reduction by induction over the reduction rules. 
\FSubjectReduction
The \Sym{ξ} reduction cases \Constr{ξ-·₁}, \Constr{ξ-·₂}, \Constr{ξ-•} and \Constr{ξ-let} follow directly from the induction hypothesis. 

\noindent For the \Sym{β} reduction cases \Constr{β-λ}, \Constr{β–Λ} and \Constr{β-let}, we need to prove that substitutions preserve the typing. 
We have two different types of substitution present inside the reduction rules: \Sym{e} \Data{[} \Sym{e} \Data{]} and \Sym{e} \Data{[} \Sym{τ} \Data{]}.
Both \Data{e[e]-preserves} and \Data{e[τ]-preserves} follow from a more general lemma \Data{⊢σ-preserves}. 
The lemma \Data{⊢σ-preserves} proves that applying a typed substitution preserves the typing.
\newpage
\Fpreserves
The lemma \Data{⊢σ-preserves} follows by induction over the typing rules and lemmas about the interaction of substitutions and renamings. 
More specifically, we also need to prove that all operations on substitutions preserve the typing. 
For instance, we need to prove the lemma \Data{⊢σ↑} that says that the typed extension of a substitution \Constr{⊢extₛ} is type preserving. 
Because \Data{extₛ} uses renaming under the hood, we also need to prove the lemma \Data{⊢ρ-preserves} that says that applying a typed renaming preserves the typing. 
Furthermore, we need to prove the lemmas \Data{assoc-sub-ren}, \Data{assoc-ren-ren}, \Data{assoc-ren-sub} and \Data{assoc-sub-sub} that prove the operations of applying a renaming and substitution to be associative in all combinations.
~\footnote{Considering the fact that the soundness proof for System F is not the main part of this work and resources can be found online~\cite{fp}, the overview of the proof itself is rather short.
The full proof can be found as Agda code file: \url{https://github.com/Mari-W/System-Fo/blob/main/proofs/SystemF.agda}}.

\noindent The soundness property of System F follows as a consequence of \Data{progress} and \Data{subject-reduction}. 