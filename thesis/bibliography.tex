% \bibitem{system-o}Odersky, M., Wadler, P. \& Wehr, M. A Second Look at Overloading. 
% {\em Proceedings Of The Seventh International Conference On Functional Programming Languages And Computer Architecture}. 
% pp. 135-146 (1995), https://doi.org/10.1145/224164.224195
% \bibitem{hindley-milner}Milner, R. A theory of type polymorphism in programming. 
% {\em Journal Of Computer And System Sciences}. 
% \textbf{17}, 348-375 (1978), https://www.sciencedirect.com/science/article/pii/0022000078900144
% - https://dl.acm.org/doi/pdf/10.1145/227699.227700
% - https://link.springer.com/chapter/10.1007/3-540-19027-9_9
% - https://www.ioc.ee/~cneste/files/system-f-fun-and-profit.pdf
% - https://dl.acm.org/doi/pdf/10.1145/582153.582176
% - https://www.microsoft.com/en-us/research/wp-content/uploads/1997/01/henk.pdf
% - https://dl.acm.org/doi/10.5555/207528
% - https://www4.di.uminho.pt/~mjf/pub/PSVC-Lecture7.pdf

\bibitem{logrel}Abel, A., Allais, G., Hameer, A., Pientka, B., Momigliano, A., Schäfer, S. \& Stark, K. POPLMark reloaded: Mechanizing proofs by logical relations. {\em Journal Of Functional Programming}. \textbf{29} pp. e19 (2019), \url{http://dx.doi.org/10.1017/S0956796819000170}

\bibitem{pts} Barendregt, H. Introduction to generalized type systems. {\em Journal Of Functional Programming}. \textbf{1}, 125-154 (1991), \url{https://doi.org/10.1017%2Fs0956796800020025}

\bibitem{agda} Bove, A., Dybjer, P., Norell, U. (2009). A Brief Overview of Agda – A Functional Language with Dependent Types. In: Berghofer, S., Nipkow, T., Urban, C., Wenzel, M. (eds) Theorem Proving in Higher Order Logics. TPHOLs 2009. Lecture Notes in Computer Science, vol 5674. Springer, Berlin, Heidelberg. \url{https://doi.org/10.1007/978-3-642-03359-9_6}

\bibitem{tc} Cordelia V. Hall, Kevin Hammond, Simon L. Peyton Jones, and Philip L. Wadler. 1996. Type classes in Haskell. ACM Trans. Program. Lang. Syst. 18, 2 (March 1996), 109–138. \url{https://doi.org/10.1145/227699.227700}

\bibitem{qt} Jones, M.P. (1992). A theory of qualified types. In: Krieg-Brückner, B. (eds) ESOP '92. ESOP 1992. Lecture Notes in Computer Science, vol 582. Springer, Berlin, Heidelberg. \url{https://doi.org/10.1007/3-540-55253-7_17}

\bibitem{itt} Martin-Löf, P. \& Sambin, G. Intuitionistic Type Theory. (Bibliopolis, 1984), \url{https://www.cse.chalmers.se/~peterd/papers/MartinL%C3%B6f1984.pdf}

\bibitem{hm} Milner, R. A theory of type polymorphism in programming. 
{\em Journal Of Computer And System Sciences}. 
\textbf{17}, 348-375 (1978), \url{https://www.sciencedirect.com/science/article/pii/0022000078900144}

\bibitem{syso} Odersky, M., Wadler, P. \& Wehr, M. A Second Look at Overloading. 
{\em Proceedings Of The Seventh International Conference On Functional Programming Languages And Computer Architecture}. 
pp. 135-146 (1995), \url{https://doi.org/10.1145/224164.224195}

\bibitem{ahp} P. Wadler and S. Blott. 1989. How to make ad-hoc polymorphism less ad hoc. In Proceedings of the 16th ACM SIGPLAN-SIGACT symposium on Principles of programming languages (POPL '89). Association for Computing Machinery, New York, NY, USA, 60–76. \url{https://doi.org/10.1145/75277.75283}
