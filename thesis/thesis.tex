\documentclass[runningheads]{llncs} 
\usepackage[utf8x]{inputenc}
\usepackage[greek,english]{babel}
\usepackage{graphicx}
\usepackage{amssymb}
\usepackage{amsmath}
\usepackage{amsfonts}
\usepackage{stmaryrd}
\usepackage{agda}
\usepackage{bbm}
\usepackage{newunicodechar}
\usepackage{ucs}
\usepackage{autofe}

\newunicodechar{λ}{\ensuremath{\mathnormal\lambda}}
\newunicodechar{Λ}{\ensuremath{\mathnormal\Lambda}}
\newunicodechar{ƛ}{\ensuremath{\lambdabar}}
\newunicodechar{τ}{\ensuremath{\mathnormal\tau}}
\newunicodechar{ℕ}{\ensuremath{\mathbb{N}}}
\newunicodechar{∶}{\ensuremath{:}}
\newunicodechar{≡}{\ensuremath{\equiv}}
\newunicodechar{∀}{\ensuremath{\forall}}
\newunicodechar{⊤}{\ensuremath{\top}}
\newunicodechar{⊥}{\ensuremath{\bot}}
\newunicodechar{₁}{\ensuremath{_1}}
\newunicodechar{₂}{\ensuremath{_2}}
\newunicodechar{∈}{\ensuremath{\in}}
\newunicodechar{′}{\ensuremath{'}}
\newunicodechar{·}{\ensuremath{\cdot}}
\newunicodechar{⊎}{\ensuremath{\uplus}}
\newunicodechar{∷}{\ensuremath{::}}
\newunicodechar{▷}{\ensuremath{\triangleright}}
\newunicodechar{ᶜ}{\ensuremath{^c}}
\newunicodechar{⊎}{\ensuremath{\uplus}}
\newunicodechar{×}{\ensuremath{\times}}
\newunicodechar{Σ}{\ensuremath{\Sigma}}
\newunicodechar{∃}{\ensuremath{\exists}}
\newunicodechar{≢}{\ensuremath{\not\equiv}}
\newunicodechar{∘}{\ensuremath{\circ}}
\newunicodechar{α}{\ensuremath{\mathnormal\alpha}}
\newunicodechar{⇒}{\ensuremath{\Rightarrow}}
\newunicodechar{→}{\ensuremath{\parbox{0.15cm}{\tikz{\draw[->](0,0)--(0.15,0);}}}}
\newunicodechar{·}{\ensuremath{\cdot}}
\newunicodechar{•}{\ensuremath{\bullet}}
\newunicodechar{∶}{\ensuremath{:}}
\newunicodechar{∀}{\ensuremath{\forall}}
\newunicodechar{ρ}{\ensuremath{\mathnormal\rho}}
\newunicodechar{σ}{\ensuremath{\mathnormal\sigma}}
\newunicodechar{∅}{\ensuremath{\emptyset}}
\newunicodechar{▶}{\ensuremath{\blacktriangleright}}
\newunicodechar{▸}{\ensuremath{\smallblacktriangleright}}
\newunicodechar{⊘}{\ensuremath{\oslash}}
\newunicodechar{Γ}{\ensuremath{\mathnormal\Gamma}}
\newunicodechar{⊢}{\ensuremath{\vdash}}
\newunicodechar{ᵣ}{\ensuremath{_r}}
\newunicodechar{ₛ}{\ensuremath{_s}}
\newunicodechar{ᴼ}{\ensuremath{^\text{O}}}
\newunicodechar{⇝}{\ensuremath{\rightsquigarrow}}
\newunicodechar{≐}{\ensuremath{\doteq}}


\title{ Formal Proof of Type Preservation of the Dictionary Passing Transform for System F}
\institute{Chair of Programming Languages, University of Freiburg \\ \email{weidner@cs.uni-freiburg.de}}
\author{Marius Weidner}

\begin{document}

\maketitle

%\noindent\makebox[\linewidth]{\small\today}
%\\\\
%\noindent\makebox[\linewidth]{Examiner: Prof. Dr. Peter Thiemann}
%\noindent\makebox[\linewidth]{Advisor: Hannes Saffrich}

\begin{abstract}
  Abstract.
\end{abstract}

\newpage
\subsubsection{Declaration}\hfill\\\\
\noindent I hereby declare, that I am the sole author and composer of my thesis and that no other sources or learning aids, other than those listed, have been used. 
Furthermore, I declare that I have acknowledged the work of others by providing detailed references of said work.  \newline
I also hereby declare that my thesis has not been prepared for another examination or assignment, either in its entirety or excerpts thereof. 
\\
\\
\\
\begin{tabular}{p{\textwidth/2} l}
  Freiburg i. Br, 27.03.2023   &   \includegraphics[width=0.2\textwidth]{signature.png} \\
  \rule{\textwidth/3}{0.4pt}   &   \rule{\textwidth/3}{0.4pt} \\
  Place, Date                  &   Signature
\end{tabular}
\newpage
\setcounter{tocdepth}{2}
\tableofcontents
\newpage

\section{Introduction}
\subsection{Motivation}
A common use cases for function overloading is operator overloading. Without overloading, code becomes less readable, since we would need to define a unique name for each operator on each type.
Haskell, for example, solves this problem using type classes.
Essentially, type classes allow to declare function names with multiple meanings. We can give one or more meanings to a type class by instantiating the type class on concrete types. 
When we invoke an overloaded function name, we determine the correct instance based on the types of the supplied arguments.
\subsubsection{Example: Overloading Equality in Haskell}\hfill\\\\
Our goal is to overload the function \inl{|\Blk eq| : α → α → Bool} with different meanings for different types substituted for \mono{α}.
We want to be able to call \mono{eq} on both \inl{Nat} and \inl{[Nat]} respectively. In Haskell we would solve the problem as follows:
\begin{minted}[escapeinside=||]{haskell}
    class Eq α where
      eq :: α → α → Bool 

    instance Eq Nat where
      eq x y = x ≐ y
    instance Eq α ⇒ Eq [α] where
      eq []       []       = True
      eq (x : xs) (y : ys) = eq x y && eq xs ys 

    .. eq 42 0 .. eq [42, 0] [42, 0] ..
\end{minted}
First, type class \inl{Eq} is declared and instantiated for \inl{Nat}. 
Next, \inl{Eq} is instantiated for \inl{[α]}, given that an instance \inl{Eq} exists for type \mono{α}. 
Finally, we can call \mono{eq} on elements of type \inl{[Nat]}, since the constraint \inl{Eq α ⇒ ..} in the second instance resolves to the first instance.
\subsection{Introducing System \Fo}
In our minimal language extension to System F we give up high level language constructs like Haskell's type classes. 
Instead, System \Fo\ desugars type class functionality to just overloaded variables. 
Using the \inl{|\Decl o| in e'} expression we can introduce an new overloaded variable \mono{o}. 
If declared as overloaded, \mono{o} can be instantiated for type \mono{τ} of expression \mono{e} using the \inl{|\Inst o| = e in e'} expression.
In contrast to Haskell, it is allowed to overload \mono{o} with arbitrary types. 
Shadowing other instances of the same type is allowed.
Constraints can be introduced using the constraint abstraction \inl{|\lbar| (o : τ). e'} resulting in a expression of constraint type \inl{[o : τ] ⇒ τ'}. Constraints are eliminated implicitly by the typing rules.

\subsubsection{Example: Overloading Equality in System \Fo}\hfill\\\\
Recall the Haskell example from above. The same functionality can be expressed in System \Fo\ as follows: 
\begin{minted}[escapeinside=||]{haskell}
    |\Decl| eq in

    |\Inst| eq : Nat → Nat → Bool 
      = λx. λy. .. in
    |\Inst| eq : ∀α. [eq : α → α → Bool] ⇒ [α] → [α] → Bool 
      = Λα. ƛ(eq : α → α → Bool). λxs. λys. .. in

    .. eq 42 0 .. eq Nat [42, 0] [42, 0] .. 
\end{minted}
First, we declare \mono{eq} to be an overloaded identifier and instantiate \mono{eq} for \inl{Nat}. 
Next, we instantiate \mono{eq} for \inl{[α]}, given the constraint introduced by the constraint abstraction \mono{ƛ} is satisfied.  
For convenience type annotations for instances are given. 
The actual implementations of the instances are omitted.
Because System \Fo\ is based on System F, we are required to bind type variables using type abstractions \inl{Λ} and eliminate type variables using type application. 

\noindent A little caveat: the second instance needs to recursively call instance \mono{eq} for sublists but System \Fo's formalization does not actually support recursive instances or recursive let bindings. Extending System \Fo\ with recursive instances and let bindings should be straight forward but is subject to further work.
\subsection{Translating between System \Fo\ and System F}
The Dictionary Passing Transform translates well typed System \Fo\ expressions to well typed System F expressions. 
The overall goal will be to formally show that the Dictionary Passing Transform is in fact correct.
The translation drops \inl{|\Decl o| in} expressions and replaces \inl{|\Inst o| = e in e'} expressions with \inl{let o|$_τ$| = e in e'} expressions, where \mono{o$_τ$} is an unique name with respect to type \mono{τ} of \mono{e}. 
Constraint abstractions \inl{|\lbar|(o : τ). e'} translate to normal lambda bindings \mono{λo$_τ$. e'}. Similarly constraint types \inl{[o : τ] ⇒ τ'} are translated to function types \mono{τ → τ'}. 
Invocations of overloaded function names are translated to the let binding they would have resolved to.
Implicitly resolved constraints in System \Fo\ must be explicitly applied in System F.
\subsubsection{Example: Dicitionary Passing Transform}\hfill\\\\
Recall the System \Fo\ example from above. We use indices to ensure unique names.
Applying the Dictionary Passing Transform results in the following well typed System F expression:
\begin{minted}[escapeinside=||]{haskell}
    let eq₁ : Nat → Nat → Bool 
      = λx. λy. .. in
    let eq₂ : ∀α. (α → α → Bool) → [α] → [α] → Bool 
      = Λα. λeq₁. λxs. λys. .. in
    
    .. eq₁ 42 0 .. eq₂ Nat eq₁ [42, 0] [42, 0] .. 
\end{minted}
First we drop the \inl{|\Decl|}expression and replace \inl{|\Inst|}definitions with \inl{let} bindings. Inside the second instance the constraint abstraction is translated into a normal function. Invocations of \mono{eq} are translated to the correct unique names \mono{eq$_i$}. When invoking \mono{eq$_2$} the correct instance to resolve the former constraint must be eliminated explicitly by applying \mono{eq$_1$}.
\subsection{Related Work}
- SystemO
- SystemFc
- ..?
\section{Preliminary}
\subsection{Dependently Typed Programming in Agda}
Agda is a dependently typed programming language and proof assistant. [CITE]
Agdas type system is based on intuitionistic type theory [CITE] and allows to construct proofs based on the Curry Howard correspondence [CITE]. 
The Curry Howard correspondence is an isomorphic relationship between programs written in dependently typed languages and mathematical proofs written in first order logic. 
Because of the Curry Howard correspondence, programs in Agda correspond to proofs and formulae correspond to types. 
Thus, type checked Agda programs imply the correctness of the corresponding proofs, given we do not use unsafe Agda features and assuming Agda is implemented correctly. 

\subsection{Design Decisions for the Agda Formalization}
To formalize System F and System \Fo\ in Agda we use a single data type \Data{Term} indexed by sorts \Sym{s} to represent the syntax. 
Sorts distinguish between different categories of terms. 
For example, sort \Constr{eₛ} represents expressions \Sym{e}, \Constr{τₛ} represents \Sym{τ} and \Constr{κₛ} represents the only existing kind \Constr{⋆}. 
Using a single data type to formalize the syntax yields more elegant proofs involving contexts, substitutions and renamings. 
In consequence we must use extrinsic typing, because intrinsically typed terms \Data{Term} \Constr{eₛ} \Constr{⊢} \Data{Term} \Constr{τₛ} would need to be indexed by themselves and Agda does not support self indexed data types. 
In the actual implementation \Data{Term} has another index \Sym{S}, that we will ignore for now.

\subsection{Overview of the Type Preservation Proof}
Our goal will be to prove that the Dictionary Passing Transform is type preserving. Let \Sym{⊢t} be any well formed System \Fo\ term \Sym{Γ} \Constr{⊢$_{F_O}$} \Sym{t} \Constr{:} \Sym{T}, where \Sym{t} is a \Data{Term$_{F_O}$} \Sym{s}, \Sym{T} is a \Data{Term$_{F_O}$} \Sym{s'} and s' is the sort of the typing result for terms of sort \Sym{s}. There exist two cases for typings: \Sym{Γ} \Constr{⊢} \Sym{e} \Constr{:} \Sym{τ} and \Sym{Γ} \Constr{⊢} \Sym{τ} \Constr{:} \Constr{⋆}. Let \Data{⇝} : $($\Sym{Γ} \Constr{⊢$_{F_O}$} \Sym{t} \Constr{:} \Sym{T}$)$ \Sym{→} \Data{Term$_F$} \Sym{s} be the Dictionary Passing Transform that translates well typed System \Fo\ terms to untyped System F terms. Further let \Data{⇝$_\Gamma$} : \Data{Ctx$_{F_O}$} \Sym{→} \Data{Ctx$_F$} be the transform of contexts and \Data{⇝$_T$} : \Data{Term$_{F_O}$} \Sym{s'} \Sym{→} \Data{Term$_F$} \Sym{s'} be the transform of untyped types and kinds. We show that for all well typed System \Fo\  terms \Sym{⊢t} the Dictionary Passing Transform results in a well typed System F term (\Data{⇝$_\Gamma$} \Sym{Γ)} \Constr{⊢$_{F}$} (\Data{⇝} \Sym{⊢t}) \Constr{:} (\Data{⇝$_T$} \Sym{T}). 

\noindent We begin by formalizing System F and prove its soundness [\ref{sec:sysf}]. Then System \Fo\ is formalized, although without semantics and soundness proof [\ref{sec:sysfo}]. In the end, we formalize the translation of the Dictionary Passing Transform and prove it to be type preserving [\ref{sec:dpt}].
\section{System F}
\subsection{Specification}
\subsubsection{Syntax.}
\input{proofs/SystemF/Syntax.lagda}
\subsubsection{Renaming.}
\subsubsection{Substitution.}
\subsubsection{Context.}
\subsubsection{Typing.}
\subsubsection{Semantic.}
\subsection{Soundness}
\subsubsection{Progress.}
\subsubsection{Subject Reduction.}
\section{System F with Overloading}
\subsection{Specification}
\subsubsection{Syntax.} 
\subsubsection{Renaming.}
\subsubsection{Substitution.}
\subsubsection{Context.}
\subsubsection{Typing.}
\section{Dictionary Passing Transform}
\subsection{Translation}
\subsection{Type Preservation}
\section{Conclusion and Further Work}
\subsection{Hindley Milner}
\subsection{Semantic Preservation}

\begin{thebibliography}{8}

% \bibitem{system-o}Odersky, M., Wadler, P. \& Wehr, M. A Second Look at Overloading. 
% {\em Proceedings Of The Seventh International Conference On Functional Programming Languages And Computer Architecture}. 
% pp. 135-146 (1995), https://doi.org/10.1145/224164.224195
% \bibitem{hindley-milner}Milner, R. A theory of type polymorphism in programming. 
% {\em Journal Of Computer And System Sciences}. 
% \textbf{17}, 348-375 (1978), https://www.sciencedirect.com/science/article/pii/0022000078900144
% - https://dl.acm.org/doi/pdf/10.1145/227699.227700
% - https://link.springer.com/chapter/10.1007/3-540-19027-9_9
% - https://www.ioc.ee/~cneste/files/system-f-fun-and-profit.pdf
% - https://dl.acm.org/doi/pdf/10.1145/582153.582176
% - https://www.microsoft.com/en-us/research/wp-content/uploads/1997/01/henk.pdf
% - https://dl.acm.org/doi/10.5555/207528


\end{thebibliography}

\end{document}