\subsection{Translation}
\subsubsection{Sorts}\hfill\\\\
The translation of System \Fo\ sorts to System F sorts only considers sorts that are contextable. 
The two missing non-contextable sorts \Constr{cₛ} and \Constr{κₛ} do not need to be translated for our purpose. 
Intuitively there does not even exist a sensible translation for \Constr{cₛ}.
\DPTSort
Sort \Constr{eₛ} and \Constr{τₛ} translate to their corresponding counterparts in System F. 
Overloaded variables in System \Fo\ are translate to normal variables in System F. 
Thus sort \Constr{oₛ} translates to \Constr{eₛ}. 

\noindent Translating lists \Sym{S} directly is not possible, because there might appear additional sorts inside the list after the translation. 
New sorts must be added for variable bindings introduced by the translation. 
For example, a \Constr{inst`} \Constr{`} \Sym{o} \Constr{=} \Sym{e₂} \Constr{`in} \Sym{e₁} expression does not bind a new variable in \Sym{e₁}, but translates to a \Constr{let`x=} \Sym{e₂} \Constr{`in} \Sym{e₁} binding. 
Hence \Sym{S} must have a new entry \Constr{eₛ} at the corresponding position to further function as valid index for the translated \Sym{e₁}. 
To solve this problem the System \Fo\ context \Sym{Γ} is used to build the translated \Sym{S}. 
The context stores the relevant information about introduced constraints and thus where new bindings will occur, that were not present in System \Fo. 

\DPTSorts
The empty context \Constr{∅} corresponds to the empty list \Constr{[]}.
For each constraint in \Sym{Γ} an additional sort \Constr{eₛ} is appended to \Sym{S}, representing the new binder introduced by the translation. 
If we find that a normal item is stored in the context, \Sym{s} is directly translated to \Data{s⇝s} \Sym{s}.

\subsubsection{Variables}\hfill\\\\
Similar to lists \Sym{S}, the translation for variables \Sym{x} needs context information.  
\DPTVar
If an item is stored in the context we can translate the variable directly. 
Whenever a constraint is encountered, \Sym{x} is wrapped in an additional \Constr{there}. 
This is because, the expression that introduced the constraint will translate to an expression with an additional new binding, that needs to be respected in System F.

\noindent Furthermore, resolved constraints translate to the correct unique expression variable. 
\DPTOVar
The idea is the same as before, we wrap the variable in an additional \Constr{there}, for each constraint in the context.

\subsubsection{Context}\hfill\\\\
The translation of contexts is mostly a direct translation. 
We only look at the translation of constraints stored in the context.
\DPTCtx
Following the idea from above, constraints \Sym{o} \Constr{:} \Sym{τ} stored inside \Sym{Γ} translate to normal items in the translated \Sym{Γ}. 
The item introduced is the translated type \Data{τ⇝τ} \Sym{τ} required by the constraint. Again, whenever we pick up a constraint in System \Fo\, there will be a new binder in System F, that accepts the constraint as higher order function. 
Thus, the corresponding type for that binding is expected in \Sym{Γ} at that position.

\subsubsection{Renaming \& Substitution}\hfill\\\\
Typed renamings in System \Fo\ get translated to untyped renamings in System F.
\DPTRen
Typed renamings \Constr{⊢idᵣ}, \Constr{⊢extᵣ} and \Constr{⊢dropᵣ} translate to their untyped counterparts. 
Because constraints in contexts translate to actual bindings, both \Constr{⊢ext-cstrᵣ} and \Constr{⊢drop-cstrᵣ} translate to normal \Constr{⊢extᵣ} and \Constr{⊢dropᵣ} in System F.

\noindent The translation of typed substitutions to untyped substitutions follows the same idea.

\DPTSub 

Cases \Constr{⊢idₛ}, \Constr{⊢extₛ}, \Constr{⊢dropₛ}, \Constr{⊢ext-cstrₛ} and \Constr{⊢drop-cstrₛ} are analogous to the cases for renamings. 
The typed introduction of a type \Constr{⊢typeₛ} translated to the untyped introduction of a term \Data{singleₛ}.

\subsubsection{Terms}\hfill\\\\
Types and kinds can be translated without typing information. Kind \Constr{⋆} translates to direct counterpart in System F. 
Furthermore, all System \Fo\ types translate to their direct counterparts in System F, except the constraint type \Constr{[} \Sym{o} \Constr{:} \Sym{τ} \Constr{]⇒} \Sym{τ'}.
\DPTType
Constraint types  \Constr{[} \Sym{o} \Constr{:} \Sym{τ} \Constr{]⇒} \Sym{τ'} translate to function types \Sym{τ} \Constr{⇒} \Sym{τ'}. 
The translation from constraint types to function types corresponds directly to the translation of constraint abstractions to normal abstractions. 
The implicitly resolved constraint will now be taken as higher order function argument.

\noindent Arbitrary terms can only be translated using typing information. 
The typing carries information about the instances that were resolved, for all usages of overloaded variables. 
The unique variable name for the resolved instance can then be substituted for the overloaded variable. 
We only look at the translation of System \Fo\ expressions that do not have a direct counterpart in System F.
\DPTTerms
Typed overloaded variables \Constr{⊢`o} carry information about the instance that was resolved for \Sym{o}.
We translate the resolved instance to the unique variable in System F, that points to the former instance, now let binding. 
Constraint abstractions translate to normal abstractions. 
An implicitly resolved constraint translates to a explicit application, that passes the resolved instance as argument. 
The \Constr{decl} expressions could be translated to nothing, as seen in the example at the beginning. 
Instead \Constr{decl} expressions are translated to useless let bindings, binding a unit value.
Because \Constr{decl} expressions bind a new overloaded variable in System \Fo, removing them would result in a variable binding less in System F and hence, more complex proofs.
As already discussed, \Constr{inst} expressions translate to \Constr{let} bindings.

\subsection{Type Preservation}
\subsubsection{Terms}\hfill\\\\
We first look at the final proof of type preservation for the Dictionary Passing Transform to motivate all necessary lemmas. 
Type preservation is proven by induction over the typing rules of System \Fo. 
Given a typed System \Fo\ term \Sym{⊢t}, the function \Data{⊢t⇝⊢t} produces a typed System F term. 
The translated untyped System F term \Data{⊢t⇝t} \Sym{t} gets typed in translated context \Data{Γ⇝Γ} \Sym{Γ} and has typing result \Data{T⇝T} \Sym{T}. 
The function \Data{T⇝T} translates untyped types and kinds from System \Fo\ to System F.
\DPTTermPres
Proof that a variable \Sym{x} has type \Sym{τ} in \Sym{Γ} translates to proof that \Sym{x⇝x x} has type \Data{τ⇝τ} \Sym{τ} in \Data{Γ⇝Γ} \Sym{Γ} using lemma \Data{Γx≡τ⇝Γx≡τ}. 
With lemma \Data{Γx≡τ⇝Γx≡τ} the typing rule \Constr{⊢`x} can be translated to the type rule for variables in System F. 

\noindent Similarly, Lemma \Data{o∶τ∈Γ⇝Γx≡τ} translates proof that an instance \Sym{o} \Constr{:} \Sym{τ} was resolved for a overloaded variable \Sym{o} to proof that \Sym{o:τ∈Γ⇝x} has type \Data{τ⇝τ} \Sym{τ} in \Sym{Γ}.  
Using lemma \Data{o∶τ∈Γ⇝Γx≡τ} the typing rule for overloaded variables \Constr{⊢`o} can be translated to the typing rule for normal variables \Constr{⊢`x}.

\noindent Typed let bindings \Constr{⊢let} \Sym{⊢e₂} \Sym{⊢e₁} translate to typed let bindings in System F. 
Rule \Sym{⊢e₂} is translated directly using the induction hypothesis. 
Because the typing for \Sym{e₁} in \Sym{⊢e₁} results in \Data{wk} \Sym{τ'}, proof is needed that \Sym{τ'} weakened in System \Fo\ and translated to System F is equivalent to the weakening of translated \Sym{τ'} in System F. 
Lemma \Data{τ⇝wk·τ≡wk·τ⇝τ} is used to substitute the required equivalence into the typing rule \Data{⊢t⇝⊢t} \Sym{⊢e₁}.

\noindent Typed constraint abstractions \Constr{⊢ƛ} translate to normal abstractions in System F.
Inside the typing for \Sym{⊢e}, the result type \Sym{τ} for body \Sym{e} does not need to be weakened, because the constraint abstraction only introduced a constraint to context \Sym{Γ} and no actual binding. 
After the translation, the introduced constraint will be bound by a binding and thus a new entry in translated \Sym{Γ} will exist. To ignore the binding, \Sym{τ} is weakened in the abstraction rule \Constr{⊢λ}.
Lemma \Data{τ⇝wk·τ≡wk-inst·τ⇝τ} proves that, the constraint picked up by the constraint abstraction can be ignored by weakening \Sym{τ} in System F. The weakening in System F then ignores the actual binder, that the constraint got translated to.

\noindent Typed implicitly resolved constraints \Constr{⊢⊘} carry the information about the instance resolved. In System F the constraint is now explicitly passed as variable pointing to the correct translated instance. Thus, \Constr{⊢⊘} results in typed application \Constr{⊢·}. We apply the correct instance using lemma \Data{o∶τ∈Γ⇝Γx≡τ} to resolve the correct unique variable for the resolved constraint.

\noindent Type application rule \Constr{⊢•} contains type in type substitution. Hence, we need proof that it is irrelevant, if \Sym{τ'} is substituted into \Sym{τ} and then translated or both \Sym{τ} and \Sym{τ'} are translated and then substituted into. 
Using lemma \Data{τ'⇝τ'[τ⇝τ]≡τ⇝τ'[τ]} we can substitute the equivalence into the translated typing rule \Data{⊢t⇝⊢t} \Sym{⊢e}.

\noindent The translation of \Constr{⊢⊤}, \Constr{⊢λ}, \Constr{⊢·}, \Constr{⊢decl} and \Constr{⊢inst} is either a direct translation or does not use other lemmas than the ones discussed.

\subsubsection{Variables}\hfill\\\\
\DPTVarPresLookup
\DPTOVarPresLookup
\subsubsection{Renaming}\hfill\\\\
\DPTVarPresRen
\DPTTypePresRen
\DPTTypePresWk
\DPTTypePresWkInst
\subsubsection{Substitution}\hfill\\\\
\DPTVarPresSub
\DPTTypePresSub
\DPTTypePresSingleSub