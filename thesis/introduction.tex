\subsection{Overloading in Programming Languages}
Overloading function names is a practical technique to overcome verbosity in real world programming languages. 
In every language there exist commonly used function names and operators that are defined for a variety of type combinations.
Overloading the meaning of a function name helps to solve the problem of having to define similar but differing names and operators for different type combinations. 
Overloading is also sometimes referred to as ad-hoc polymorphism. 
An ad-hoc polymorphic function is allowed to have multiple type-specific meanings for all types that it is defined for. 
In contrast, a parametric polymorphic function only defines abstract behavior that must work for all types.

\noindent Python, for example, uses magic methods to overload commonly used operators on user defined classes and Java utilizes method overloading. 
Both Python and Java implement rather restricted forms of overloading. 
Haskell solves the overloading problem with a more general concept, called typeclasses.

\subsection{Typeclasses in Haskell}
Essentially, typeclasses allow to declare function names with generic type signatures.
We can give one of possibly many meanings to a typeclass by instantiating the typeclass for concrete types. 
When we instantiate a typeclass, we must provide an actual implementation for all functions defined by the typeclass based on the concrete types that the typeclass is instantiated for.
When we invoke an overloaded function name defined by a typeclass, we expect the compiler to determine the correct instance based on the types of the arguments that were applied to the overloaded function name. 
Furthermore, Haskell allows to constrain type variables via type constraints to only be substituted by a concrete type if there exists an instance for the concrete type. 
Type constraints allow to abstract over all types that inherit a shared behavior, but the actual implementation of the behavior can differ per type. Type constraints are a powerful formalism in addition to parametric polymorphism.

\subsubsection{Example: Overloading Equality in Haskell}\hfill\\\\
In this example the function \inl{|\Blk eq| : α → α → Bool}\ is overloaded with different meanings for different substitutions $\{α \mapsto τ\}$. 
We want to be able to call \mono{eq} on both $\{$\mono{α} $\mapsto$ \inl{Nat}$\}$ and $\{$\mono{α} $\mapsto$ \inl{[β]}$\}$, where \mono{β} is a type and there exists an instance that gives meaning to \inl{|\Blk eq| : β → β → Bool}. The intuition here is that we want to be able to compare natural numbers \inl{Nat} and lists \mono{[β]}, given the elements of type \mono{β} are known to be comparable.
\newpage
\begin{minted}[escapeinside=||]{haskell}
    class Eq α where
      eq :: α → α → Bool 

    instance Eq Nat where
      eq x y = x ≐ y
    instance Eq β ⇒ Eq [β] where
      eq []       []       = True
      eq (x : xs) (y : ys) = eq x y && eq xs ys 

    .. eq 42 0 .. eq [42, 0] [42, 0] ..
\end{minted}
First, typeclass \inl{Eq} is declared with a single generic function signature \inl{|\Blk{eq}| :: α → α → Bool}. 
We then instantiate \inl{Eq} for $\{$\mono{α} $\mapsto$ \inl{Nat}$\}$. 
After that,  \inl{Eq} is instantiated for $\{$\mono{α} $\mapsto$ \inl{[β]}$\}$, given that an instance \inl{Eq β} can be resolved for some concrete type \mono{β}.
As a result, we can invoke \mono{eq} on expressions with type \inl{Nat} and \inl{[Nat]}. 
In the latter case, the type constraint \inl{Eq β ⇒ ..} in the instance for lists resolves to the instance for natural numbers.

\subsection{Desugaring Typeclass Functionality to System \Fo}
System \Fo\ is a minimal calculus with support for overloading and polymorphism based on System F.  
In System \Fo\ we give up high level language constructs and instead work with simple overloaded identifiers. 

\noindent Using the \inl{|\Decl o| in e} expression we can introduce an new overloaded variable \mono{o}. 
If declared as overloaded, \mono{o} can be instantiated for the type \mono{τ} of the expression \mono{e} using the \inl{|\Inst o| = e in e'} expression.
In Haskell, instances must comply with the generic type signatures defined by the typeclass. Such signatures are not present in System \Fo\ and overloaded variables can be instantiated for arbitrary types. 
Locally shadowing other instances of the same type is allowed. 
Constraints can be introduced on the expression level using constraint abstractions \inl{|\lambdabar| (o : τ). e'}. 
Constraint abstractions result in constraint types \inl{[o : τ] ⇒ τ'} that lift constraints onto the type level. 
We introduce constraints on the expression level because instance expressions do not have an explicit type annotation in System \Fo.
All expressions with constraint types \inl{[o : τ] ⇒ τ'} are implicitly treated as expressions of type \Sym{τ'} by the type system, given that the constraint \inl{|\Blk{o}| : τ} can be resolved.

\subsubsection{Example: Overloading Equality in System \Fo}\hfill\\\\
Recall the Haskell example from above. 
The same functionality can be expressed in System \Fo. 
For convenience, type annotations for instances are given.
\begin{minted}[escapeinside=||]{haskell}
    |\Decl| eq in

    |\Inst| eq : Nat → Nat → Bool 
      = λx. λy. .. in
    |\Inst| eq : ∀β. [eq : β → β → Bool] ⇒ [β] → [β] → Bool 
      = Λβ. ƛ(eq : β → β → Bool). λxs. λys. .. in

    .. eq 42 0 .. eq Nat [42, 0] [42, 0] .. 
\end{minted} 
We first declare \mono{eq} to be an overloaded identifier and instantiate \mono{eq} for equality on \inl{Nat}. 
Next, we instantiate \mono{eq} for equality on lists \inl{[β]}, given that the constraint \inl{|\Blk{eq}| : β → β → Bool} introduced by the constraint abstraction is satisfied. 
Because System \Fo\ is based on System F, a calculus without type inference, we are required to bind type variables using type abstractions \inl{Λα. ..} and eliminate type variables using type application. 

\noindent A little caveat: the instance for lists would potentially need to recursively call \mono{eq} for sublists, but the formalization of System \Fo\ does not actually support recursion. 
Extending System \Fo\ with recursive let bindings and thus recursive instances is known to be straight forward. 

\subsection{Translating System \Fo\ back to System F}
System \Fo\ can be translated back to System F. 
This implies that System \Fo\ is no more expressive or powerful than System F. 
After all, overloading is more of a convenience feature. 

\noindent The Dictionary Passing Transform translates well typed System \Fo\ expressions to well typed System F expressions. 
The translation requires knowledge acquired during type checking. 
More specifically, we need to know the instances that were resolved for invocations of overloaded identifiers and the instances that constraints were implicitly resolved with.

\noindent The translation removes all \inl{|\Decl|}expressions. 
Instance expressions \inl{|\Inst o| = e in e'} are replaced with \inl{let o|$_τ$| = e in e'} expressions, where \mono{o$_τ$} is a unique name with respect to the type \mono{τ} of the expression \mono{e}. 
Implicitly resolved constraints in System \Fo\ will be taken as explicit higher-order function arguments in System F.
As a result, constraint abstractions \inl{|\lambdabar|(o : τ). e'} translate to normal abstractions \mono{λo$_τ$. e'} and constraint types \inl{[o : τ] ⇒ τ'} translate to function types \mono{τ → τ'}. 
A invocation of an overloaded function name \mono{o} translates to the correct unique variable name \mono{o$_τ$} that is bound by the let binding that got introduced for the corresponding resolved instance. 
The translation becomes more intuitive when looking at an example.

\subsubsection{Example: Dicitionary Passing Transform}\hfill\\\\
Recall the System \Fo\ example from above. 
We use indices to represent unique names \mono{o$_τ$}.
Applying the Dictionary Passing Transform to the example above results in a well typed System F expression. Type annotations for let bindings are given for convenience.
\begin{minted}[escapeinside=||]{haskell}
    let eq₁ : Nat → Nat → Bool 
      = λx. λy. .. in
    let eq₂ : ∀β. (β → β → Bool) → [β] → [β] → Bool 
      = Λβ. λeq₁. λxs. λys. .. in
    
    .. eq₁ 42 0 .. eq₂ Nat eq₁ [42, 0] [42, 0] .. 
\end{minted}
We remove the \inl{|\Decl|}expression and transform both \inl{|\Inst|}expressions to \inl{let} bindings with unique names \mono{eq$_i$}. 
Inside the instance for lists, the constraint abstraction translates to a normal lambda abstraction. 
The lambda abstraction takes the constraint that was implicitly resolved in System \Fo\ as an explicit higher-order function argument.
Invocations of \mono{eq} translate to correct unique variables \mono{eq$_i$} that are bound by the let bindings that got introduced for the former resolved instances.
Because \mono{eq$_2$} is invoked for lists of numbers, we must pass the correct instance to eliminate new higher-order function binding by explicitly passing instance \mono{eq$_1$} as argument.
