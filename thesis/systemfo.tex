\subsection{Specification}

\subsubsection{Sorts}\hfill\\\\
In addition to the sorts of System F, System \Fo\ introduces two new sorts: \Constr{oₛ} for overloaded variables and \Constr{cₛ} for constraints.
\FoSort
Terms of sort \Constr{oₛ} can only be constructed using the variable constructor \Constr{`\_}.
Variables for constraints do not exist and thus \Constr{cₛ} is indexed by \Constr{⊥ᶜ}.

\subsubsection{Syntax}\hfill\\\\
We only discuss additions to the syntax of System F.
\FoTerm
Declarations \Constr{decl`o`in} \Sym{e} introduce a new overloaded variable \Sym{o}. 
Hence, \Sym{S} is extended by sort \Constr{oₛ} inside the body \Sym{e}. 
Expression \Constr{inst`} \Constr{`} \Sym{o} \Constr{=} \Sym{e} \Constr{`in} \Sym{e'} gives overloaded variable \Sym{o} an additional meaning \Sym{e} in \Sym{e'}. 
Constraints \Sym{c} can be constructed using constructor \Constr{`} \Sym{o} \Constr{:} \Sym{τ}. 
Constraints are part of both constraint abstractions \Constr{\lambdabar} \Sym{c} \Constr{⇒} \Sym{e} and constraint types \Constr{[} \Sym{c} \Constr{]⇒} \Sym{τ}, used to introduce constraints to the expression and type level respectively.
Going forward, we will use shorthand \FoCstr.

\subsubsection{Renaming \& Substitution}\hfill\\\\
Renamings and substitutions in System \Fo\ are formalized identically to renamings and substitutions in System F. The only difference is that we define single substitution only on types. 
\Fosubs
Because we do not formalize semantics for System \Fo\, only substitutions of types in types are necessary. Type in type substitutions are used in the typing rule for type application.

\subsubsection{Context}\hfill\\\\
In addition to the normal context items, we also store constraints inside the context.
\FoCtx
We write \Sym{Γ} \Constr{▸} \Sym{c} to pick up constraint \Sym{c}. 
Constraints give an additional meaning to a overloaded variable that is already bound. Hence index \Sym{S} is not modified.

\subsubsection{Constraint Solving}\hfill\\\\
The search for constraints in a context is formalized analogously to membership proofs \Sym{s} \Constr{∈} \Sym{S}. The subtle difference is, that we do reference constraints in \Sym{Γ} and not \Sym{S}. 
\FoCstrSolve
The \Constr{here} constructor is analogous to the \Constr{here} constructor of memberships and can be used when the last item in \Sym{Γ} is the constraint \Sym{c}, the constraint that we searched for. 
If the last item in the context is not the constraint \Sym{c}, \Sym{c} can be further inside the context, either behind a item stored in \Sym{Γ} (\Constr{under-bind}) or constraint (\Constr{under-cstr}). 

\subsubsection{Typing}\hfill\\\\
Again, we only discuss typing rules not already discussed in the System F specification. 
\FoTyping
Rule \Constr{⊢`o} for overloaded variables says that, if we can resolve the constraint \Sym{o} \Constr{:} \Sym{τ} in \Sym{Γ}, then \Sym{o} has type \Sym{τ}. 
The rule for constraint abstraction \Constr{⊢ƛ} appends constraint \Sym{c} to \Sym{Γ} and thus assumes \Sym{c} to be valid in body \Sym{e}. 
Expressions \Sym{e} with constraint type \Constr{[} \Sym{c} \Constr{]⇒} \Sym{τ'} have the constraint implicitly eliminated using the \Constr{⊢⊘} rule, given constraint \Sym{c} can be resolve in \Sym{Γ}. 
Finally, the rule \Constr{⊢decl} introduces a new overloaded variable to body \Sym{e}. Similar to the abstraction rule, type \Sym{τ} is weakened to be compatible in \Sym{S} with \Sym{Γ} not extended by \Sym{o} to act as the resulting type of the typing.

\subsubsection{Typing Renaming \& Substitution}\hfill\\\\
Typed renamings are identical to the typed renamings in System F, except there are two new cases for the constraints that can appear inside contexts. 
\FoRenTyping
Constructor \Constr{⊢ext-instᵣ} allows to introduce a constraint \Sym{c} to \Sym{Γ₁} and renamed \Sym{c} to \Sym{Γ₂}, similar to \Constr{⊢extᵣ}. We can also introduce a constraint \Sym{o} \Constr{:} \Sym{τ} only to \Sym{Γ₂} using constructor \Constr{⊢drop-instᵣ}. 
The latter corresponds to a typed weakening, similar to \Constr{⊢extᵣ}, but instead of introducing an unused variable we introduce an unused constraint.

\noindent Other than in System F arbitrary substitutions will not be allowed in System \Fo. 
% TODO
\FoSubTyping

