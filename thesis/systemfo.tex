\subsection{Specification}

\subsubsection{Sorts}\hfill\\\\
In addition to the sorts of System F, System \Fo\ introduces two new sorts: \Constr{oₛ} for overloaded variables and \Constr{cₛ} for constraints.
\FoSort
Terms of sort \Constr{oₛ} can only be constructed using the variable constructor \Constr{`\_}. Thus, terms of sort \Constr{oₛ} are called overloaded variables and sort \Constr{oₛ} is indexed by \Constr{var}.
Variables for constraints do not exist in System \Fo\ and thus \Constr{cₛ} is indexed by \Constr{no-var}. 
We use the variable symbol \Sym{o} for overloaded variables and the variable symbol \Sym{c} for constraints. 

\subsubsection{Syntax}\hfill\\\\
We only discuss additions to the syntax of System F.
\newpage
\FoTerm
Declarations \Constr{decl`o`in} \Sym{e} introduce a new overloaded variable \Sym{o}. 
Hence, \Sym{S} is extended by the sort \Constr{oₛ} inside the body \Sym{e}. 

\noindent The expression \Constr{inst`} \Sym{o} \Constr{=} \Sym{e₂} \Constr{`in} \Sym{e₁} introduces an additional instance for \Sym{o}. 
The actual meaning for the instance is given by \Sym{e₂}.
Instance expressions do not introduce new bindings and thus, the index \Sym{S} is never extended.

\noindent Constraints \Sym{c} can be constructed using constructor \Sym{o} \Constr{:} \Sym{τ}. 

\noindent A constraint \Sym{c} can be part of a constraint abstraction \Constr{\lambdabar} \Sym{c} \Constr{⇒} \Sym{e}. Constraint abstractions assume the constraint \Sym{c} to be valid inside the body \Sym{e} and result in constraint types \Constr{[} \Sym{c} \Constr{]⇒} \Sym{τ}. The constraint type lifts the constraint from the expression level to the type level, where it will be implicitly eliminated by the typing rules.

\noindent Going forward, we will use the abbreviation \FoCstr.

\subsubsection{Renaming \& Substitution}\hfill\\\\
Renamings and substitutions in System \Fo\ are formalized identically to renamings and substitutions in System F. 
The only difference is that we define the substitution operator only on types. 
\Fosubs
The \Data{single-typeₛ} function only introduces a new binding for types and not for arbitrary terms.
Because we do not formalize direct semantics for System \Fo, only substitutions of types in types are necessary. Type in type substitution appears in the typing rule for type application.

\subsubsection{Context}\hfill\\\\
In addition to types and kinds, the existence of overloaded variables is stored inside the context. Overloaded variables act as normal context items. 
Because overloaded variables themselves do not have a type, but rather multiple types that they can take on, we only need to store their existence in \Sym{Γ}. Thus, similar to type variables, we store kind \Constr{⋆} in \Sym{Γ} to denote the existence of an overloaded variable. 

\noindent The types that an overloaded variable can take on are stored in the form of constraints. Constraints can be introduced to the context by both constraint abstractions and instance expressions.
\FoCtx
We write \Sym{Γ} \Constr{▸} \Sym{c} to pick up a constraint \Sym{c}. 
Because constraints give an additional meaning to an overloaded variable that is already bound, the index \Sym{S} is not modified. 

\noindent The \Data{lookup} function in System \Fo\ is defined analogous to the \Data{lookup} function in System F and simply ignores constraints stored in the context.

\subsubsection{Constraint Solving}\hfill\\\\
The search for constraints in a context is inductively formalized, similar to membership proofs \Sym{s} \Constr{∈} \Sym{S}. The subtle difference is that we reference constraints in \Sym{Γ} and not in \Sym{S}. The constraint solving type does need to search in \Sym{Γ} because \Sym{S} does not know about the existence of constraints.
\FoCstrSolve
The \Constr{here} constructor is analogous to the \Constr{here} constructor of memberships and can be used when the last item in \Sym{Γ} is the desired constraint \Sym{c}.

\noindent If the last item in the context is not the desired constraint \Sym{c}, then \Sym{c} must be further inside the context. The constraint can either be behind an item stored in \Sym{Γ} (\Constr{under-bind}) or a constraint (\Constr{under-cstr}). In the case that \Sym{c} is under a binder, the constraint needs to be weakened to align in \Sym{S} with the position that it is resolved for.

\noindent We use the constraint solving type inside the type system to resolve the instance for usages of overloaded variables and to implicitly eliminate constraints.

\subsubsection{Typing}\hfill\\\\
We only discuss typing rules not already discussed in the System F specification. The typing for overloaded variables results in a type. As a result, the \Data{type-of} function returns the sort \Constr{τₛ} for the sort \Constr{oₛ} in the case of typings.
\FoTyping
The rule for overloaded variables \Constr{⊢`o} says that if we can resolve the constraint \Sym{o} \Constr{:} \Sym{τ} in \Sym{Γ}, then \Sym{o} can take on type \Sym{τ}. 

\noindent The rule for constraint abstractions \Constr{⊢ƛ} appends the constraint \Sym{c} to \Sym{Γ} and thus assumes \Sym{c} to be valid inside the body \Sym{e}. Constraint abstractions result in expressions of the corresponding constraint type \Constr{[} \Sym{c} \Constr{]⇒} \Sym{τ} that lifts the constraint onto the type level.

\noindent Expressions \Sym{e} with constraint type \Constr{[} \Sym{c} \Constr{]⇒} \Sym{τ'} have the constraint implicitly eliminated using the rule \Constr{⊢⊘}, given \Sym{c} can be resolved in \Sym{Γ}.

\noindent The rule \Constr{⊢decl} introduces a new overloaded variable \Sym{o} to \Sym{e}. 
To introduce \Sym{o} in \Sym{Γ}, we only need to store the information that \Sym{o} exists as overloaded variable. The existence of \Sym{o} is denoted by extending \Sym{Γ} with kind \Constr{⋆}.
Analogous to the type \Sym{τ'} inside the abstraction rule \Constr{⊢λ}, the resulting type \Sym{τ} cannot use the introduced overloaded variable and thus is weakened to align in \Sym{S} with \Sym{Γ} \Constr{▶} \Constr{⋆}. In consequence, \Sym{τ} can act as the resulting type of the typing.

\noindent An instance for an overloaded variable \Sym{o} is typed using the rule \Constr{⊢inst}. We extend \Sym{Γ} with constraint \Sym{o} \Constr{:} \Sym{τ} inside \Sym{e₁}, where \Sym{τ} is the type of \Sym{e₂}. 

\subsubsection{Typing Renaming \& Substitution}\hfill\\\\
Typed renamings are identical to typed renamings in System F, except there is an additional case for the weakening by a constraint. 
\FoRenTyping
A constraint \Sym{o} \Constr{:} \Sym{τ} can be introduced only to \Sym{Γ₂} using the \Constr{⊢drop-cstrᵣ} constructor. 
Dropping a constraint corresponds to a typed weakening, similar to constructor \Constr{⊢dropᵣ}, but instead of introducing an unused variable we introduce an unused constraint. In consequence, the typed weakening by a constraint \Data{⊢wk-cstrᵣ} is defined by \Constr{⊢drop-cstrᵣ} \Constr{⊢idᵣ}.

\noindent Other than in System F, arbitrary substitutions will not be allowed in System \Fo. 
Similar to the substitution operator we restrict typed substitutions in System \Fo\ to substitutions of types in types. 
\FoSubTyping
\noindent The constructor \Constr{⊢single-typeₛ} allows to introduce an additional new type variable binder that is substituted with type \Sym{τ}.
Thus, the constructor \Constr{⊢single-typeₛ} complements the \Data{single-typeₛ} function.  

\noindent Constructors  \Constr{⊢idₛ}, \Constr{⊢extₛ}, \Constr{⊢dropₛ} and \Constr{⊢drop-cstrₛ} are not shown. All of them function the same way as their counterparts in typed renamings.

\noindent The intuition here is that if we would allow all terms to be introduced using a \Constr{⊢termₛ} constructor, then typed substitutions in System \Fo\ would be arbitrary again. The restriction to type in type substitutions simplifies the type preservation proof for the Dictionary Passing Transform by eliminating cases for non-type terms that would otherwise needed to be proven. In hindsight arbitrary substitutions would not have produced an unreasonable amount of additional work, but the restriction does not effect the type preservation proof in any way, so it remained as is.  