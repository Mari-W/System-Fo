\subsection{Dependently Typed Programming in Agda}
Agda is a dependently typed programming language and proof assistant. [CITE]
Agdas type system is based on Martin Löf's intuitionistic type theory [CITE] and allows to construct proofs based on the Curry Howard correspondence. 
The Curry Howard correspondence is an isomorphic relationship between programs written in dependently typed languages and mathematical proofs written in first order logic. Because of the Curry Howard correspondence, programs in Agda correspond to proofs and formulae correspond to types. 
Hence, if a type checked Agda program implies that our proofs are sound, given we do not use unsafe Agda features and assuming Agda is implemented correctly. 
Agda is appealing to programmers, because proving in Agda is similar to functional programming using common concepts, for example pattern matching, currying and inductive data types.
Further, Agda has a couple useful support features, for example proving with interactive holes and automatic proof search.
\subsection{Design Decisions for the Agda Formalization}
To formalize System F and System \Fo\ in Agda we will use a single data type \Data{Term} indexed by sorts \Sym{s} to represent the syntax. Sorts distinguish between different kind of terms, for example sort \Constr{eₛ} for expressions and \Constr{τₛ} for types. Using only a single data type to formalize the syntax yields more elegant proofs involving contexts, substitutions and renamings. In consequence we must use extrinsic typing, because intrinsically typed terms \Data{Term} eₛ ⊢ \Data{Term} τₛ would need to be indexed by themselves. 